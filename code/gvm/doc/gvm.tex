\documentclass{beamer}
\usepackage{array}
\usepackage{german}
\usepackage{graphicx}
\usepackage{verbatim}
\usepackage{txfonts}
\usepackage{amsmath}
\usepackage{amssymb}
\usepackage{amsfonts}
\usepackage{amsthm}
\usepackage{textcomp}
\usepackage{graphicx}
\usepackage{array}
\usepackage{picins}
\usepackage{ifthen}
\usepackage{multirow}

\mode<beamer>{%
\usetheme[hideothersubsections,hidetitle]{Hannover}
}
\title[]{GOTO virtual machine}
\date[27. Mai 2012]{27. Mai 2012}
\begin{document}

\def\blank{\text{\textvisiblespace}}

\begin{frame}
\titlepage
\end{frame}

\begin{frame}
\section{GOTO}
\frametitle{Sprache GOTO}
Programm zur Berechnung der Summe $5 + 7$:
\verbatiminput{beispiel.goto}

"Ahnlich k"onnen beliebige arithmetische Operationen implementiert werden.

\end{frame}

\begin{frame}
\frametitle{Erweiterte Sprache GOTO}
Grammatik f"ur Ausdr"ucke $E$, arithmetische Operationen
mit Operator $O$
zwischen
Variablen $V$ und Konstanten $C$:
\begin{align*}
E&\rightarrow V \; \text{\tt ':='} \; C\\
E&\rightarrow V \; \text{\tt ':='} \; V\\
 &\rightarrow V \; \text{\tt ':='} \; V \; O \; C\\
 &\rightarrow V \; \text{\tt ':='} \; V \; O \; V\\
O&\rightarrow \text{'+'}\;|\; \text{'-'}\;|\; \text{'*'}\;|\; \text{'/'}\;|\; \text{'\%'}\\
V&\rightarrow \text{\tt 'v'}\; C\\
C&\rightarrow D \; |\; CD\\
D&\rightarrow
\text{\tt '0'}\;|\;
\text{\tt '1'}\;|\;
\text{\tt '2'}\;|\;
\text{\tt '3'}\;|\;
\text{\tt '4'}\;|\;
\text{\tt '5'}\;|\;
\text{\tt '6'}\;|\;
\text{\tt '7'}\;|\;
\text{\tt '8'}\;|\;
\text{\tt '9'}
\end{align*}
\end{frame}

\begin{frame}
\frametitle{Erweiterte Sprache GOTO}

{\bf Ablaufsteuerung:} Zus"atzlicher Befehl {\tt HALT}
\verbatiminput{steuerung.goto}

{\bf Ausgabe:} 
\verbatiminput{print.goto}

{\tt PRINTTM} gibt Werte der ersten drei Variable in spezieller Form aus
(siehe sp"ater: TM Simulator)

\end{frame}

\begin{frame}
\section{virtual machine}
\frametitle{GOTO virtual machine}
\begin{itemize}
\item Scanner (mit {\tt flex} generiert)
\item Parser (mit {\tt bison} generiert)
\item Variablen und Konstanten mit beliebiger L"ange: GNU multiprecision
library (GNU MP)
\item Programm wird als Array von {\tt gvm\_node\_t} gespeichert
\verbatiminput{node.h}
\item Hilfsfunktionen zum Ausgeben und Ausf"uhren des Programms
\end{itemize}
\end{frame}

\begin{frame}
\frametitle{Node Payload}
\verbatiminput{nodes.h}
\end{frame}

\begin{frame}
\frametitle{VM Struktur}
\verbatiminput{vm.h}
\end{frame}

\begin{frame}
\frametitle{Parser generiert {\tt gvm\_node\_t}}
{\bf Zuweisungen:}
\verbatiminput{parser.y}
\end{frame}

\begin{frame}
\frametitle{Parser generiert {\tt gvm\_node\_t}}
{\bf Bedingter Sprung:}
\verbatiminput{parsergoto.y}
\end{frame}

\begin{frame}
\frametitle{VM Execution Engine}
\verbatiminput{engine.h}
\end{frame}

\begin{frame}
\frametitle{Ausf"uhrung: Zuweisung}
{\small
\verbatiminput{assign.h}
}
\end{frame}

\begin{frame}
\frametitle{Ausf"uhrung: Sprung}
\verbatiminput{goto.h}
\end{frame}

\begin{frame}
\frametitle{Ausf"uhrung: bedingter Sprung}
\verbatiminput{condgoto.h}
\end{frame}

\begin{frame}
\section{Turing Vollst"andigkeit}
\frametitle{GOTO ist Turing vollst"andig}

M"ogliche Beweise:
\begin{enumerate}
\item GOTO ist "aquivalent zu WHILE, man zeigt das WHILE Turing vollst"andig ist
\item Man zeigt, dass sich eine bereits bekannt Turing vollst"andige Sprache
nach GOTO kompilieren l"asst
\item Man baut einen Turing-Maschinen-Simulator in GOTO
\end{enumerate}

Wir w"ahlen einem Mischung von 2 und 3: Compiler, der TM-Beschreibungen
in GOTO "ubersetzt
\[
\Rightarrow
\]
ein beliebiges Turing Programm kann auf der GOTO VM ausgef"uhrt werden.
\end{frame}

\begin{frame}
\frametitle{Turing Maschinen Sprache}
Beschreibung von Turing-Maschinen:
{\small
\verbatiminput{../tmc/beaver3}
}
``Mathematisch'':
\[
\text{\tt (}
q
\text{\tt ,}
a
\text{\tt ) -> (}
q'
\text{\tt ,}
a'
\text{\tt )}
m
\quad
\Leftrightarrow
\quad
\delta(a,q)=(q',q',m),
\quad m\in \{L, R, H\}
\]

\end{frame}

\begin{frame}
\frametitle{TM-Programm: Datenstruktur}
\verbatiminput{tm.h}
\end{frame}

\begin{frame}
\frametitle{TM Programm: Band, Zustand, Kopfposition}
\begin{itemize}
\item Band: Variable {\tt v0}, 2 bit pro Zelle, mit folgender Codierung:
\begin{itemize}
\item['\blank'\quad] Code 0 (leeres Band $= 0$, alle Zellen mit '\blank' initialisiert)
\item['0'\quad] Code 1
\item['1'\quad] Code 2
\end{itemize}
\item Zustand: Variable {\tt v1}
\item Kopfposition: Variable {\tt v2}
\end{itemize}
Beispiel:
\begin{align*}
\text{{\tt v0} dezimal} &= 45736175274\\
\text{4-er System}&=222212011222122222\\
\text{Decodiert}&=\text{'11111011100\blank101111'}
\end{align*}

\end{frame}

\begin{frame}
\frametitle{Zelle lesen/schreiben}
$b =$ Band, $k=$ Kopfposition

\bigskip

{\bf $c$ aus Zelle lesen:}
\[
c=(b / 4^k) \mod 4
\]

\bigskip

{\bf Zelle mit $c$ schreiben:}
\begin{align*}
v_{10}&=v_0\mod 4^k\\
v_{11}&=4(v_0/4^{k+1})\\
v_0&=v_{10} + (c + v_{11})4^k
\end{align*}
\end{frame}

\begin{frame}
\section{TM Compiler}
\frametitle{TM Compiler}
Compiler erzeugt ein GOTO Program folgender Struktur
\begin{enumerate}
\item Initialisiere Variablen
\item Isoliere Zelle {\tt v2} aus dem Band ({\tt v0})
\item F"ur jede TM-Anweisung:
\begin{enumerate}
\item Pr"ufe, ob Zustand "ubereinstimmt
\item Pr"ufe, ob Bandzeichen "ubereinstimmt
\item Merke neuen Zustand, neues Bandzeichen, Kopfbewegung
\end{enumerate}
\item Modifiziere Band
\item Gehe zum neuen Zustand "uber
\item Bewege den Kopf
\item Weiter bei 2.
\end{enumerate}
\end{frame}

\begin{frame}
\frametitle{Initialisierung}
\verbatiminput{tmc-initial.goto}
\end{frame}

\begin{frame}
\frametitle{Zelle isolieren}
{\bf $c$ aus Zelle lesen:}
\[
c=(b / 4^k) \mod 4
\]
\verbatiminput{tmc-isolate.goto}
\end{frame}

\begin{frame}
\frametitle{"Uberetzung einer TM Anweisung}
\begin{center}
\tt (1,1) -> (1, 1) Right
\end{center}
wird "ubersetzt in
\verbatiminput{tmc-command.goto}
{\tt v7} = neuer Zustand, {\tt v6} = neues Bandzeichen, {\tt v5} = Kopfbewegung
(1 = Left, 2 = Right, 3 = Halt)
\end{frame}

\begin{frame}
\frametitle{Zelle "uberschreiben}
{\bf Zelle mit $c$ schreiben:}
\[
(v_0\mod 4^k)
+ (c +
4(v_0/4^{k+1})
)4^k
\]
GOTO Code:
{\small
\verbatiminput{tmc-splice.goto}
}
\end{frame}

\begin{frame}
\frametitle{Kopfbewegung}
\verbatiminput{tmc-move.goto}
\end{frame}

\begin{frame}
\frametitle{Zustand}
Zustand aufdatieren und neuer Durchgang:
\verbatiminput{tmc-state.goto}
Damit man die Rechnung verfolgen kann:
\verbatiminput{tmc-printtm.goto}
liefert den Trace-Output:
\verbatiminput{tmc-beaver.tm}
\end{frame}

\begin{frame}
\frametitle{Demo}
\end{frame}

\end{document}

\subsection{Ausf"ullr"atsel}
Im vorangegangenen Abschnitt haben wir \textsl{SUDOKU} auf \textsl{SAT}
reduziert.
Ein \textsl{SUDOKU}-Problem wurde in eine Formel verwandelt, die genau
dann erf"ullbar war, wenn das \textsl{SUDOKU}-R"atsel l"osbar war.
In diesem Abschnitt zeigen wir, dass diese Idee viel allgemeiner ist.

\begin{figure}
\centering
\includegraphics{images/ausfuell-1.pdf}
\qquad
\qquad
\includegraphics{images/ausfuell-2.pdf}
\caption{Ausf"ullr"atsel zum Grapheinf"arbeproblem aus
Abbildung~\ref{vertex-coloring-examples}, links das leere `Spielfeld',
rechts ausgef"ullt mit den Farben der Knoten.
\label{ausfuell:coloring}}
\end{figure}
Sehr viele R"atsel werden auf einem rechteckigen Feld von Zellen
bespielt, in die man etwas eintragen muss.
Sogar das Graphenf"arbeproblem (\textsl{VERTEX-COLORING}) kann man
so auffassen.
Dazu formt man den gegebenen Graphen mit $n$ in eine
$n\times n$-Tabelle um, die Zeilen und Spalten entsprechen den Knoten
des Graphen, die Felder den m"oglichen Kanten.
Alle Felder, die zu im Graphen nicht existierenden Kanten geh"oren,
werden schwarz gef"arbt.
In die weissen Felder m"ussen jetzt Paare von verschiedenen Farben
eingetragen werden, so dass in jeder Zeile die erste Farbe immer
gleich ist, und in jeder Spalte die zweite Farbe.
Ausserdem muss die Farbe einer Spalte und der entsprechenden Zeile
gleich sein.
Der Graph ist einff"arbbar, wenn es m"oglich ist, die Tabelle wie
beschrieben auszuf"ullen (Abbildung~\ref{ausfuell:coloring}).

\begin{definition}
Ein {\em polynomielles Ausf"ullr"atsel} ist eine $n\times m$-Tabelle,
in die Zeichen eines Alphabets $\Sigma$ werden m"ussen, so dass gewisse
Regeln eingehalten werden. 
Die einzuhaltenden Regeln k"onnen durch eine logische Formel beschrieben
werden, die in einer Zeit polynomiell in $nm$ berechnet werden kann.
\end{definition}

Die Bedingung der Berechenbarkeit in polynomieller Zeit ist meistens
offensichtlich.
Oft entsteht die Formel n"amlich durch Anpassung der immer gleichen
Regeln f"ur jedes Feld.
In solchen F"allen gen"ugt es, wenn die resultierende Formel polynomielle
L"ange hat.

\begin{beispiel}
Die Formel, die die Sudoku-Regeln f"ur das $n^2\times n^2$-beschreibt,
besteht aus einer 
Teilformel f"ur jedes Feld, welche wiederum aus einer Teilformel
f"ur Zeilen, Spalten und Unterfelder besteht.
Diese Teilformeln "uberpr"ufen die Werte aller $n^2$ Felder der Zeile,
Spalte oder des Unterfeldes, dabei sind $n^2$ m"ogliche Feldinhalte
zu ber"ucksichtigen.
Im Abschnitt~\ref{subsection:sudoku-und-sat} wurde gezeigt, wie die
Teilformel aufzubauen ist, ihre L"ange ist
\[
O(\underbrace{n^2}_{\text{Vergleichsfelder}}\cdot
\underbrace{n^2}_{\text{Zeichen}}) = O(n^4).
\]
Die gesamte Formel hat damit die L"ange
\[
O(\underbrace{n^4}_{\text{Felder}}\cdot
\underbrace{n^2}_{\text{Vergleichsfelder}}\cdot
\underbrace{n^2}_{\text{Zeichen}}) = O(n^8).
\]
Insbesondere ist die L"ange der Formel polynomiell in der Feldgr"osse $n^4$,
\textsl{SUDOKU}
ist also ein polynomielles Ausf"ullr"atsel im genannten Sinn.
\end{beispiel}

\begin{satz}
Ein polynomielles Ausf"ullr"atsel $A$ l"asst sich polynomiell auf 
\textsl{SAT} reduzieren.
\end{satz}

\begin{proof}[Beweis]
Polynomielle Reduktion auf \textsl{SAT} bedeutet, dass man zu jeder
Probleminstanz eine logische Formel $\varphi$ konstruieren muss,
deren L"ange polynomiell in der Gr"osse des Ausgangsproblems ist.
Die Formel muss genau dann erf"ullbar sein, wenn das urspr"ungliche
Problem l"osbar ist.
Ein polynomielles Ausf"ullr"atsel f"uhrt aber nach Definition auf
eine solche Formel.
\end{proof}

Als Konsequenz dieser Konstruktion k"onnen wir auch schliessen, dass
jedes Ausf"ullr"atsel in NP liegt.


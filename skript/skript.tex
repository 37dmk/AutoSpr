%
% skript.tex -- Skript zur Vorlesung Automaten und Sprachen
%               gehalten an der Hochschule Rapperswil im Wintersemester 09
%
% (c) 2009 Prof. Dr. Andreas Mueller, HSR
% $Id: skript.tex,v 1.34 2008/11/02 22:46:16 afm Exp $
%
\documentclass[a4paper,12pt]{book}
\usepackage[ngerman]{babel}
\usepackage[T1]{fontenc}
\usepackage{csquotes}
\usepackage{float} %added see: https://tex.stackexchange.com/questions/8625/force-figure-placement-in-text
\usepackage{times}
\usepackage{geometry}
\geometry{papersize={210mm,297mm},total={160mm,240mm},top=31mm,bindingoffset=15mm}
\usepackage{alltt}
\usepackage{verbatim}
\usepackage{fancyhdr}
\usepackage{amsmath}
\usepackage{amssymb}
\usepackage{amsfonts}
\usepackage{amsthm}
\usepackage{textcomp}
\usepackage{graphicx}
\usepackage{array}
\usepackage{ifthen}
\usepackage{multirow}
\usepackage{txfonts}
%\usepackage[basic]{circ}
\usepackage[all]{xy}
\usepackage{algorithm}
\usepackage{algorithmic}
\usepackage{makeidx}
\usepackage{paralist}
\usepackage[colorlinks=true]{hyperref}
\usepackage[backend=bibtex]{biblatex}
\addbibresource{references.bib}
\makeindex
\setlength{\headheight}{15pt}
\newcommand\myatop[2]{\genfrac{}{}{0pt}{}{#1}{#2}}
\begin{document}
\pagestyle{fancy}
\lhead{Automaten und Sprachen}
\rhead{}
\frontmatter
\newcommand\HRule{\noindent\rule{\linewidth}{1.5pt}}
\begin{titlepage}
\vspace*{\stretch{1}}
\HRule
\vspace*{10pt}
\begin{flushright}
{\Huge Automaten und Sprachen}
\end{flushright}
\HRule
\begin{flushright}
\vspace{30pt}
\LARGE
Andreas M"uller
\end{flushright}
\vspace*{\stretch{2}}
\begin{center}
Hochschule f"ur Technik, Rapperswil, 2011-2015
\end{center}
\end{titlepage}
\hypersetup{
    linktoc=all,
    linkcolor=blue
}
\rhead{Inhaltsverzeichnis}
\tableofcontents
\newtheorem{satz}{Satz}[chapter]
\newtheorem{hilfssatz}[satz]{Hilfssatz}
\newtheorem{definition}[satz]{Definition}
\newtheorem{annahme}[satz]{Annahme}
\newenvironment{beispiel}[1][Beispiel]{%
\begin{proof}[#1]%
\renewcommand{\qedsymbol}{$\bigcirc$}
}{\end{proof}}
\def\blank{\text{\textvisiblespace}}
\mainmatter
\begin{refsection}
\input einleitung.tex
\input grundlagen.tex
\input sprachen.tex
\input regulaer.tex
\input cfg.tex
\input turing.tex
\input entscheidbarkeit.tex
\input komplexitaet.tex
\input vollstaendig.tex
\appendix
\input algorithmen.tex
\vfill
\pagebreak
\ifodd\value{page}\else\null\clearpage\fi
\lhead{}
\rhead{}
\printbibliography[heading=subbibliography]
\end{refsection}

\input skript.ind
\end{document}

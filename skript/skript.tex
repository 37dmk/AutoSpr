%
% skript.tex -- Skript zur Vorlesung Mathematische Grundlagen der Informatik
%               gehalten an der Hochschule Rapperswil im Wintersemester 09
%
% (c) 2009 Prof. Dr. Andreas Mueller, HSR
% $Id: skript.tex,v 1.34 2008/11/02 22:46:16 afm Exp $
%
\documentclass[a4paper,12pt]{book}
\usepackage{german}
\usepackage{times}
\usepackage{alltt}
\usepackage{verbatim}
\usepackage{fancyhdr}
\usepackage{amsmath}
\usepackage{amssymb}
\usepackage{amsfonts}
\usepackage{amsthm}
\usepackage{textcomp}
\usepackage{graphicx}
\usepackage{array}
\usepackage{ifthen}
\usepackage{multirow}
\usepackage{txfonts}
%\usepackage[basic]{circ}
\usepackage[all]{xy}
\usepackage{algorithm}
\usepackage{algorithmic}
\usepackage{makeidx}
\usepackage{paralist}
\usepackage{hyperref}
\makeindex
\begin{document}
\pagestyle{fancy}
\lhead{Mathematische Grundlagen}
\rhead{}
\frontmatter
\newcommand\HRule{\noindent\rule{\linewidth}{1.5pt}}
\begin{titlepage}
\vspace*{\stretch{1}}
\HRule
\vspace*{2pt}
\begin{flushright}
{\Huge
Mathematische Grundlagen\bigskip\\
der Informatik 2}
\end{flushright}
\HRule
\begin{flushright}
\vspace{30pt}
\LARGE
Andreas M"uller
\end{flushright}
\vspace*{\stretch{2}}
\begin{center}
Hochschule f"ur Technik, Rapperswil, 2011-2013
\end{center}
\end{titlepage}
\hypersetup{
    colorlinks=true,
    linktoc=all,
    linkcolor=blue
}
\rhead{Inhaltsverzeichnis}
\tableofcontents
\newtheorem{satz}{Satz}[chapter]
\newtheorem{hilfssatz}[satz]{Hilfssatz}
\newtheorem{definition}[satz]{Definition}
\newtheorem{annahme}[satz]{Annahme}
\newenvironment{beispiel}[1][Beispiel]{%
\begin{proof}[#1]%
\renewcommand{\qedsymbol}{$\bigcirc$}
}{\end{proof}}
\mainmatter
\input einleitung.tex
\input grundlagen.tex
\input sprachen.tex
\input regulaer.tex
\input cfg.tex
\input turing.tex
\input entscheidbarkeit.tex
\input komplexitaet.tex
\input vollstaendig.tex
\input skript.ind
\end{document}

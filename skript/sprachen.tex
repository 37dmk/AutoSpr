%
% sprachen.tex
%
% (c) 2011 Prof Dr Andreas Mueller, Hochschule Rapperswil
%
\lhead{Sprachen}
\rhead{}
\chapter{Sprachen}
Diese Vorlesung betrachtet Computer in erster Linie als
Maschinen, die Zeichenketten verarbeiten. Nat"urlich wird man
nicht jede Zeichenkette als sinnvollen Input oder Output betrachten,
nur eine Teilmenge aller m"oglichen Zeichenketten w"urden wir
eine ``Sprache'' nennen. Diese intuitive Vorstellung wollen wir
jetzt formalisieren.

\index{Alphabet}
Die Basis einer Sprachdefinition muss die Auswahl eines geeigneten
Alphabetes sein. W"ahrend in der Informatik das Alphabet meist 
durch die Maschine vorgegeben ist (sie verarbeitet zum Bespiel
einzelne Bytes, also Zahlen zwischen 0 und 255), m"ochten wir f"ur
unsere theoretischen "Uberlegungen mehr Freiheit, und akzeptieren
jede beliebige nichtleere endliche Menge als Alphabet. Alphabete
bezeichnen wir h"aufig mit grossen griechischen Buchstaben, zum
Beispiel
\begin{align*}
\Sigma&=\{{\tt 0},{\tt 1}\}\\
\Sigma&=\{{\tt 1}\}\\
\Gamma&=\{{\tt 0},{\tt 1},\sqcup\}\\
\Delta&=\{{\tt a},\dots,{\tt z}\}
\end{align*}
Zeichenketten sind Tupel aus Elementen eines Alphabets.
\begin{definition}\label{def_wort}
\index{Wort}
\index{leeres Wort}
Ein Wort $w$ der L"ange $n$ "uber dem Alphabet $\Sigma$ ist ein $n$-Tupel,
$w\in\Sigma^n$. Es gibt genau ein Wort der L"ange $0$, es heisst das
leere Wort und wird mit $\varepsilon$ bezeichnet, $\Sigma^0=\{\varepsilon\}$.
Die Menge aller W"orter
ist die Vereinigung aller $\Sigma^n$ und wird mit $\Sigma^*$ bezeichnet:
\[
\Sigma^*=\{\varepsilon\}\cup \Sigma^1\cup\Sigma^2\cup\Sigma^3\cup\dots
=\bigcup_{k=0}^\infty\Sigma^k.
\]
\end{definition}

Nat"urlich kann man von jedem Wort $w\in\Sigma^*$ auch wieder bestimmen,
aus welchem $\Sigma^k$ es stammt, indem man die L"ange der Zeichenkette
ausz"ahlt. Wir bezeichnen die L"ange von $w$ mit $|w|$. Es ist $|{\tt 1111}|=4$
und $|\varepsilon|=0$. Manchmal ist es wichtig zu wissen, wie oft ein bestimmtes
Zeichen in einem Wort vorkommt. Wir bezeichnent mit
$|w|_{a}$ die Anzahl Vorkommnisse des Zeichens $a$, also zum Beispiel
$|{\tt 0010111}|_{\tt 0}=3$ und $|{\tt 0110011}|_{\tt 1}=4$.

Eine Sprache muss nicht alle m"oglichen W"orter umfassen, oft wird sogar
nur eine endliche Menge von W"ortern aus der unendlichen Menge $\Sigma^*$
ausgew"ahlt.

\begin{definition}
\index{Sprache}
Eine Sprache $L$ "uber dem Alphabet $\Sigma$ ist eine Teilmenge
von $\Sigma^*$, also $L\subset \Sigma^*$. "Uber jedem Alphabet
gibt es die leere Sprache $\emptyset$ und die Sprache, die nur
aus dem leeren Wort besteht $\{\varepsilon\}$.
\end{definition}

{\parindent0pt Beispiele:}
\begin{enumerate}
\item Sei $\Sigma=\{{\tt 1}\}$, dann ist
\[
\Sigma^*=\{\varepsilon, {\tt 1}, {\tt 11}, {\tt 111}, {\tt 1111},\dots\}
\]
Die W"orter "uber $\Sigma$ sind also durch ihre L"ange charakterisiert.
Es gibt eine Abbildung
\[
\mathbb N\to\Sigma^*\colon n\mapsto \underbrace{1\dots 1}_{\text{$n$ Zeichen}}=1^n
\]
Eine Sprache "uber dem Alphabet $\{{\tt 1}\}$ entspricht unter dieser
Abbildung genau einer Teilmenge der nat"urlichen Zahlen. Die Darstellung
einer Zahl $n$ als Folge von $n$ Zeichen {\tt 1} heisst
{\em un"are Darstellung}
\index{unaere Darstellung@un\"are Darstellung}
von $n$.
\item Sei $\Sigma=\{{\tt (}, {\tt )}\}$. $\Sigma^*$ besteht aus allen
Ketten von Klammern. Die korrekt geschachtelten Klammerausdr"ucke bilden
darin eine Teilmenge, also eine Sprache.
\item Sei $\Sigma=\{{\tt 0},{\tt 1},\dots,{\tt 9}\}$. $\Sigma^*$ besteht
aus allen Ziffernfolgen. Jede solche Ziffernfolge hat nat"urlich auch
einen numerischen Wert, indem man die Ziffernfolge als Zehnersystemdarstellung
einer Zahl interpretiert. Wegen m"oglicher f"uhrender Nullen k"onnen verschiedene
Ziffernfolgen den gleichen Wert haben. Es gibt also eine Abbildung
\[
v\colon\Sigma^*\to \mathbb N,
\]
die den Wert einer Ziffernfolge ermittelt. Damit k"onnen wir die Sprache
der Ziffernfolgen mit Zweierpotenzwerten definieren:
\[
\{w\in\Sigma^*\,|\, \exists k (v(w)=2^k)\}=\{
{\tt 1},
{\tt 2},
{\tt 4},
{\tt 8},\dots
{\tt 01},
{\tt 02},
{\tt 04},
{\tt 08},\dots
{\tt 001},
{\tt 002},
{\tt 004},
{\tt 008},\dots\}
\]
\item Sei $\Sigma=\{{\tt 0}, {\tt 1}\}$. $\Sigma^*$ besteht dann aus den
bin"aren Zeichenketten. Darin k"onnen wir eine Reihe von Sprachen auszeichnen:
\begin{align*}
L_1&=\{ {\tt 0}^n{\tt 1}^n\,|\,n\ge 0\}\\
L_2&=\{ w\in\Sigma^*\,|\, |w|_{\tt 0}=|w|_{\tt 1}\}
L_3&=\{ w\in\Sigma^*\,|\, \text{Zahlenwert von $w$ ist durch 3 teilbar}\}
\end{align*}
\item Sei $\Sigma$ die Menge der ASCII-Zeichen. Dann ist $\Sigma^*$ die
Menge aller ASCII-Texte, wie unsinnig sie auch immer sein m"ogen. Interessant
ist die Sprache
\[
C=\{w\in\Sigma^*\,|\,\text{$w$ wird von GCC akzeptiert}\}.
\]
Die Sprache $C$ heisst GNU-Dialekt von C.
\item Sei $\Sigma$ die Menge der Unicode-Zeichen. $\Sigma^*$ besteht
dann aus allen Unicode-Zeichenfolgen. Darin k"onnen wir zum Beispiel die
Sprache Java ausw"ahlen
\[
J
=\{w\in\Sigma^*\,|\, {\text{$w$ wird von einem Java-Compiler akzeptiert}}\}.
\]
\end{enumerate}

%
% Regulaere Sprachen und endliche Automaten
%
\lhead{Regul"are Sprachen}
\chapter{Endliche Automaten und regul"are Sprachen\label{chapter-regular}}
Endliche Automaten stellen das einfachste Berechnungsmodell
dar, welches in diesem Skript untersucht werden soll. Ausser 
"uber seine Zust"ande verf"ugt ein Automat "uber keinen weiteren
Speicher.
Trotzdem defineren endliche Automaten eine Menge von Sprachen, die 
mathematisch und f"ur die Praxis sehr zweckm"assige Eigenschaften
haben.
Ausserdem gibt es eine besonders einfach Syntax, um
solche Sprachen zu beschreiben: die regul"aren Ausdr"ucke.
Es geht in diesem Kapitel allerdings nicht darum, das Arbeiten
mit regul"aren Ausdr"ucken zu lernen.
Vielmehr geht es vor allem um das Verst"andnis, welche Eigenschaften
eine Sprache hat, die mit einem endlichen Automaten oder einem
regul"aren Ausdruck beschrieben werden kann.

\rhead{Deterministische endliche Automaten}
\section{Deterministische endliche Automaten\label{regulaer:dea}}
Ein endlicher Automat modelliert ein System, welches
in einer endlichen Zahl verschiedener Zust"ande sein kann. Der
"Ubergang zwischen den Zust"ande geschieht determinstisch beim
Eintreffen neuer Daten.

Als Beispiel betrachten wir ein Drehkreuz an einem Skilift.
Es hat zwei m"ogliche Zust"ande: {\it  verriegelt} ($V$)
und  {\it entriegelt} ($E$).
Ausserdem gibt es zwei Inputs: {\tt Fahrkarte} ($F$)
und {\tt Drehen} ($D$). Zu Beginn befindet sich das Drehkreuz
in verriegeltem Zustand. Wenn ein Skifahrer durch will, es zu drehen
versucht, "andert sich daran nichts. Erst wenn er eine Fahrkarte einschiebt,
wird das Drehkreuz entriegelt. Jetzt n"utzt es nichts, noch weitere
Fahrkarten einzuschieben, das Drehkreuz bleibt entriegelt.
Versucht der Skifahrer jetzt, das Drehkreuz zu drehen, wird es
ihn durchlassen, dabei aber wieder in den verriegelten Zustand "ubergehen.
Schematisch k"onnen wird das wie folgt darstellen:
\[
\entrymodifiers={++[o][F]}
\xymatrix{
*+\txt{} \ar[r]
	&{V}\ar@/^/[r]^{F} \ar@(ur,ul)_{D}
		&{E}\ar@/^/[l]^{D} \ar@(ur,ul)_{F}
}
\]
Wir sehen hier bereits eine M"oglichkeit, Inputfolgen zu beobachten, 
aber es kann noch nicht entschieden werden, welche Inputs akzeptabel
sind. Dazu m"ussen die einzelne Zust"ande ausgezeichnet werden.
Zum Beispiel k"onnte man verlangen, dass nur Abl"aufe akzeptabel
sind, bei denen das Drehkreuz am Ende im verriegelten Zustand steht,
also nicht offen stehen bleibt. Symbolisch bezeichnen wir dies durch
einen Doppelkreis:
\[
\entrymodifiers={++[o][F]}
\xymatrix{
*+\txt{} \ar[r]
	&*++[o][F=]{V}\ar@/^/[r]^{F} \ar@(ur,ul)_{D}
		&{E}\ar@/^/[l]^{D} \ar@(ur,ul)_{F}
}
\]
Damit haben wir alle Elemente zusammen, die einen deterministischen
endlichen Automaten ausmachen.
\subsection{Definition\label{regulaer:definition-dea}}
\index{Automat!endlicher}
\index{Automat!deterministischer endlicher}
\index{DEA|see{deterministischer endlicher Automat}}
Eine abstrakte Definition muss die Menge der erlaubten Inputs
festlegen, wir brauchen also ein Alphabet $\Sigma$. Die Zust"ande
bilden eine Menge. Ausserdem m"ussen wir festlegen, in welchem Zustand
sich der Automat zu Beginn befindet, und wie er auf ein Input-Zeichen
reagiert. Dies wird erreicht durch die folgende Definition
\begin{definition}
Ein deterministischer endlicher Automat (DEA) ist ein Quintupel
$(Q,\Sigma,\delta, q_0,F)$ mit
\begin{compactenum}
\item $Q$ ist eine beliebige endliche Menge von Zust"anden
\item $\Sigma$ ist eine endliche Menge, genannt das Alphabet.
\index{Ubergangsfunktion@\"Ubergangsfunktion!eines endlichen Automaten}
\item $\delta\colon Q\times\Sigma\to Q$ heisst "Ubergangsfunktion
\index{Startzustand}
\item $q_0\in Q$ heisst Startzustand
\index{Akzeptierzustand}
\item $F\subset Q$ heisst die Menge der Akzeptierzust"ande.
\end{compactenum}
\end{definition}
Die Abbildung $\delta\colon Q\times \Sigma\to Q$ berechnet
aus einem Ausgangszustand $q\in Q$ und einem Input-Zeichen $a\in\Sigma$
einen neuen Zustand $\delta(q,a)\in Q$, in den der Automat durch
die Verarbeitung von $a$ "ubergeht.

\subsubsection{Gerichteter beschrifteter Graph eines DEA}
\index{Graph!gerichteter!beschrifteter!eines DEA}
Eine Visualisierung ist h"aufig "ubersichtlicher.
Die Zust"ande von $Q$ werden durch Kreise dargestellt. Die
"Ubergangsfunktion wird durch mit Buchstaben des Alphabets $\Sigma$
angeschriebene Kreise symbolisiert. Akzeptierzust"ande werden durch
doppelte Kreise markiert. Der Startzustand wird durch einen
Pfeil angezeigt, der nicht einen Zustand als Ausgangspunkt hat.

Damit haben wir einen gerichteten beschrifteten Graphen definiert.
Die Vertizes des Graphen sind die Zust"ande, also $V=Q$.
Vom Zustand $q$ aus verl"auft eine mit $a\in\Sigma$ beschriftete
Kante zum Zustand $\delta(q,a)$, 
die Kantenmenge
ist also
\[
E=\{(q,\delta(q,a),a)\;|\; q\in Q\wedge a\in\Sigma\}.
\]
Darin nicht enthalten ist der Startzustand, den wir zwar auch als
Pfeil zeichnen, der aber nicht eine Kante des Graphen ist, weil
er nicht zwei Zust"ande verbindet.
Zus"atzlich sind in diesem Graphen jedoch einzelne Zust"ande ausgezeichnet,
was wir in einem gew"ohnlichen gerichteten Graphen nicht vorgesehen
haben. Ein DEA ist also ein gerichteter beschrifteter Graph, aber nicht
nur.

Wichtig: von jedem Zustand aus gibt es genau einen Pfeil, der mit jedem m"oglichen
Zeichen des Alphabets angeschrieben sind. Insbesondere kann man von jedem
Zustand aus mit jedem beliebigen Zeichen auf genau eine Art fortfahren.
\index{deterministisch}
In diesem Sinne ist der Automat {\em deterministisch}.

\subsubsection{Beispiele}
\begin{beispiel}[\bf Drehkreuz]
Der Drehkreuz-DEA besteht aus den Elementen $Q={V,E}$, $q_0=V$,
$F=\{V\}$, $\Sigma=\{F,D\}$.
Die "Ubergangsfunktion definiert zu jedem Zustand, und jedem m"oglichen
Eingabewert, welcher zugeh"orige Zielwert erreicht wird:
\begin{center}
\begin{tabular}{|cc|c|}
\hline
$q$&$a$&$\delta(q,a)$\\
\hline
$V$&$D$&$V$\\
$V$&$F$&$E$\\
$E$&$D$&$V$\\
$E$&$F$&$E$\\
\hline
\end{tabular}
\end{center}
\end{beispiel}
\begin{beispiel}[\bf Gerade Zahlen]
Ein DEA, welcher erkennen kann, ob eine Bitfolge einer geraden Zahl entspricht.
Dieser DEA braucht zwei Zust"ande, denn eine Bitfolge, interpretiert
als Zahl, kann entweder gerade (Zustand $q_0$, Rest $0$ bei Teilung durch 2)
oder ungerade (Zustand $q_1$, Rest $1$) sein. Die leere
Bitfolge entspricht der Zahl $0$, also einer geraden Zahl.
Hinzuf"ugen eines weiteren
Bits "andert den Zustand m"oglicherweise: h"angt man eine $0$ an, entsteht
eine gerade Zahl, h"ange man eine $1$ an, entsteht eine ungerade Zahl.
Wir haben also $\Sigma=\{0,1\}$, $q_0$ als Startzustand,  $F=\{q_0\}$. Die
"Ubergangsfunktion ist
\begin{center}
\begin{tabular}{|cc|c|}
\hline
$q$&$a$&$\delta(q,a)$\\
\hline
$q_0$&$0$&$q_0$\\
$q_0$&$1$&$q_1$\\
$q_1$&$0$&$q_0$\\
$q_1$&$1$&$q_1$\\
\hline
\end{tabular}
\end{center}
Die graphische Form ist wiederum
\[
\entrymodifiers={++[o][F]}
\xymatrix{
*+\txt{}\ar[r]
	&*++[o][F=]{q_0} \ar@/^/[r]^1 \ar@(ur,ul)_0
		&{q_1} \ar@/^/[l]^0 \ar@(ur,ul)_1
}
\]
\end{beispiel}
\begin{beispiel}[\bf Ganzzahlen]
Ein Automat, der Ganzzahlen ohne f"uhrende Nullen erkennt.
Das Alphabet besteht aus den Ziffern und dem optionalen Minus, also
\[
\Sigma=\{
{\tt 0},
{\tt 1},
{\tt 2},
{\tt 3},
{\tt 4},
{\tt 5},
{\tt 6},
{\tt 7},
{\tt 8},
{\tt 9},
{\tt -}
\}
\]
Eine ganze Zahl kann mit einem {\tt -} beginnen, darf
aber nicht mit einer {\tt 0} anfangen. Danach d"urfen beliebige
Ziffern folgen, aber kein {\tt -}. Sobald ein {\tt -} eingegeben
wird, muss der Automat den Input als Illegal erkennen. Offenbar
gibt es also folgende F"alle: Zahl mit Minus, Zahl ohne Minus,
f"uhrende Null oder andere Illegalit"aten. Zusammen mit dem
Startzustand sollten wir also mit vier Zust"anden auskommen.
Sei $Q={q_0, m,p,e}$, dann wird das Zustandsdiagramm:
\[
\entrymodifiers={++[o][F]}
\xymatrix{
*+\txt{}\ar[r]
	&{q_0}  \ar[dl]_{{\tt 1},\dots,{\tt 9}} \ar[d]^{\tt 0} \ar[dr]^{\tt -}
\\
*++[o][F=]{p}\ar[r]^{\tt -} \ar@(ul,dl)_{{\tt 0},\dots,{\tt 9}}
	&{e} \ar@(dr,dl)^{\Sigma}
		&{m}\ar[l]_{{\tt 0},{\tt -}} \ar@/^40pt/[ll]^{{\tt 1},\dots,{\tt 9}}
}
\]
\end{beispiel}
\begin{beispiel}[\bf Ganzzahlen, diesmal richtig]
Der eben gezeigte Automat ist insofern eine w"ortliche Realisierung
der Anforderungen, dass er keine f"uhrende $0$ akzeptiert. Allerdings
m"ochten wir in der Praxis auch eine einzelne $0$ akzeptieren k"onnen.
Dazu ist es n"otig, den Automaten um einen Zustand zu erweitern:
\[
\entrymodifiers={++[o][F]}
\xymatrix{
*+\txt{}\ar[r]
	&{q_0}  \ar[ddl]_{{\tt 1},\dots,{\tt 9}} \ar[d]^{\tt 0} \ar[ddr]^{\tt -}
\\
*+\txt{}
	&*++[o][F=]{0}\ar[d]^{\Sigma}
\\
*++[o][F=]{p}\ar[r]^{\tt -} \ar@(ul,dl)_{{\tt 0},\dots,{\tt 9}}
	&{e} \ar@(dr,dl)^{\Sigma}
		&{m}\ar[l]_{{\tt 0},{\tt -}} \ar@/^40pt/[ll]^{{\tt 1},\dots,{\tt 9}}
}
\]
\end{beispiel}

\subsubsection{Tabellenform}
\index{Tabellenform eines DEA}
Man kann alle Information eines DEA auch in einer einzigen Tabelle 
darstellen. Die Zeilen werden mit den Zust"anden angeschrieben,
die Spalten mit den Zeichen des Alphabetes. Akzeptierzust"ande werden
mit {\tt /E} markiert, der Anfangszustand steht immer auf der ersten
Zeile. Die oben betrachteten Beispiele haben folgende Tabellenform:

\begin{beispiel}[\bf Drehkreuz] $V$ ist Startzustand und gleichzeitig
Akzeptierzustand.

\begin{center}
\begin{tabular}{|c|cc|}
\hline
&$F$&$D$\\
\hline
$V{\tt /E}$&$E$&$V$\\
$E$&$E$&$V$\\
\hline
\end{tabular}
\end{center}

\end{beispiel}

\begin{beispiel}[\bf Gerade Zahlen] Die
Zust"ande $q_0$ und $q_1$ stehen f"ur gerade und ungerade Zahlen:

\begin{center}
\begin{tabular}{|c|cc|}
\hline
&0&1\\
\hline
$q_0{\tt /E}$&$q_0$&$q_1$\\
$q_1$&$q_0$&$q_1$\\
\hline
\end{tabular}
\end{center}

\end{beispiel}

\begin{beispiel}[\bf Ganzzahl-Automat] Dieser Automat erkennt die $0$
nicht korrekt.

\begin{center}
\begin{tabular}{|c|ccccccccccc|}
\hline
&\tt 0&\tt 1&\tt 2&\tt 3&\tt 4&\tt 5&\tt 6&\tt 7&\tt 8&\tt 9&\tt -\\
\hline
$q_0$&$e$&$p$&$p$&$p$&$p$&$p$&$p$&$p$&$p$&$p$&$m$\\
$p{\tt /E}$&$p$&$p$&$p$&$p$&$p$&$p$&$p$&$p$&$p$&$p$&$e$\\
$m$&$e$&$p$&$p$&$p$&$p$&$p$&$p$&$p$&$p$&$p$&$e$\\
$e$&$e$&$e$&$e$&$e$&$e$&$e$&$e$&$e$&$e$&$e$&$e$\\
\hline
\end{tabular}
\end{center}

\end{beispiel}

\begin{beispiel}[\bf Ganzzahl-Automat mit $0$] Um die $0$ richtig zu erkennen,
wird ein zus"atzlicher Zustand ben"otigt.

\begin{center}
\begin{tabular}{|c|ccccccccccc|}
\hline
&\tt 0&\tt 1&\tt 2&\tt 3&\tt 4&\tt 5&\tt 6&\tt 7&\tt 8&\tt 9&\tt -\\
\hline
$q_0$&$0$&$p$&$p$&$p$&$p$&$p$&$p$&$p$&$p$&$p$&$m$\\
$0{\tt /E}$&$e$&$e$&$e$&$e$&$e$&$e$&$e$&$e$&$e$&$e$&$e$\\
$p{\tt /E}$&$p$&$p$&$p$&$p$&$p$&$p$&$p$&$p$&$p$&$p$&$e$\\
$m$&$e$&$p$&$p$&$p$&$p$&$p$&$p$&$p$&$p$&$p$&$e$\\
$e$&$e$&$e$&$e$&$e$&$e$&$e$&$e$&$e$&$e$&$e$&$e$\\
\hline
\end{tabular}
\end{center}

\end{beispiel}

\subsection{Von einem DEA akzeptierte Sprache\label{regulaer:akzeptiertesprache}}
\index{akzeptierte Sprache eines DEA}
Ein DEA definiert auf nat"urliche Weise eine Sprache: Sie besteht aus
allen W"ortern, die als Input f"ur den DEA angewendet, diesen vom
Startzustand in einen Akzeptierzustand "uberf"uhren.

\begin{definition}
Sei $A$ ein endlicher Automat, dann ist $L(A)$ die Sprache
\[
L(A)=\{w\in\Sigma^*\;|\; \text{$w$ "uberf"uhrt $A$ in einen Akzeptierzustand}\}
\]
\end{definition}

\begin{definition}
\label{regulaer:definition:regulaere-sprache}
\index{Sprache!regul\"are}
Eine Sprache heisst regul"ar, wenn es einen DEA gibt, der sie
akzeptiert.
\end{definition}

\subsubsection{Beispiele}
\begin{beispiel}[\bf Durch drei teilbare Zahlen] 
Man finde einen Automaten, der die durch drei teilbaren Zahlen in
Bin"ardarstellung akzeptiert.
Sei also $\Sigma=\{{\tt 0},{\tt 1}\}$,
$L=\{w\in\Sigma^*\;|\; \text{$w$ stellt eine durch drei teilbare Zahl dar}\}.$
Dabei soll das leere Wort f"ur als $0$
interpretiert werden, und damit auch als durch drei teilbar gelten.
Die Sprache $L$ ist regul"ar, weil sie
von folgendem Automaten akzeptiert wird:
\[
\entrymodifiers={++[o][F]}
\xymatrix{
*+\txt{} \ar[r]
	&*++[o][F=]{0} \ar@/^/[rr]^{\tt 1} \ar@(ur,ul)_{\tt 0}
		&*+\txt{}
			&{1} \ar@/^/[dl]^{\tt 0}  \ar@/^/[ll]^{\tt 1}
\\
*+\txt{}
	&*+\txt{}
		&2\ar@/^/[ur]^{\tt 0} \ar@(dl,ul)^{\tt 1}
}
\]
oder in Tabellenform
\begin{center}
\begin{tabular}{|c|cc|}
\hline
&\tt 0&\tt 1\\
\hline
0{\tt /E}&0&1\\
1        &2&0\\
2        &1&2\\
\hline
\end{tabular}
\end{center}
\end{beispiel}

\begin{beispiel}[\bf W"orter mit einer geraden Anzahl $a$]
Sei $\Sigma=\{{\tt a},{\tt b}\}$,
$L=\{w\in\Sigma^*\;|\; |w|_{\tt a}\equiv 0\mod 2\}$. $L$ besteht aus
den W"ortern, die  eine gerade Anzahl von {\tt a}s enthalten. Diese
Sprache ist regul"ar, weil sie von folgendem Automaten akzeptiert wird:
\[
\entrymodifiers={++[o][F]}
\xymatrix{
*+\txt{} \ar[r]
	&*++[o][F=]{0}\ar@/^/[r]^{\tt a} \ar@(dl,dr)_{\tt b}
		&1 \ar@/^/[l]^{\tt a} \ar@(dl,dr)_{\tt b}
}
\]
\end{beispiel}

\begin{beispiel}[\bf Bedingungen an einzelne Zeichen]
Finde einen Automaten, der die Sprache $L$ "uber dem Alphabet
$\Sigma=\{{\tt a},{\tt b}\}$ akzeptiert, deren W"orter mit 
einem {\tt a} beginnen und genau ein {\tt b} enthalten.
\[
\entrymodifiers={++[o][F]}
\xymatrix{
*+\txt{} \ar[r]
	&{q_0} \ar[r]^{\tt a} \ar[dr]^{\tt b}
		&{q_1} \ar@(ul,ur)^{\tt a} \ar[r]^{\tt b}
			&*++[o][F=]{q_2}\ar@(ul,ur)^{\tt a}  \ar[dl]^{\tt b}
\\
*+\txt{}
	&*+\txt{}
		&{e}\ar@(dl,dr)_{{\tt a},{\tt b}}
}
\]
\end{beispiel}

\subsection{Rekonstruktion des Automaten\label{regulaer:rekonstruktion}}
Wenn eine Sprache regul"ar ist, dann muss es einen DEA geben, der
diese Sprache erkennt. Doch wie findet man aus der Sprache, also 
aus der Menge $L$, diesen DEA?

Nehmen wir an, der DEA sei schon bekannt, und versuchen wir
die Zust"ande zu charakterisieren. Befindet sich ein Automat
im Zustand $q$, dann kann durch zus"atzlichen Input $w_e$ ein
akzeptables Wort entstehen. Sei also
$L(q)$ die Menge der W"orter, mit deren Hilfe man vom Zustand $q$
aus einen Akzeptierzustand erreichen kann. Wenn man mit einem Wort
$w_a$ den Zustand $q$ erreicht, dann ist $L(q)$ offenbar auch
\[
L(q)=\{ w_e\;|\; w_aw_e \in L\}= L(w_a)
\]
Die Mengen $L(w)$ k"onnen also ganz allein aus der Kenntnis der
Sprache bestimmt werden, die $L(q)$ sind nicht mehr n"otig.

Wir zeigen jetzt, dass wir mit den Mengen $L(w)$ als Zust"anden einen
DEA bauen k"onnen, der genau die Sprache $L$ akzeptiert.
Wir setzen als f"ur die Menge der Zust"ande 
\[
Q=\{L(w)\;|\;w\in\Sigma^*\}.
\]

Ein Wort muss offenbar genau dann akzeptiert werden, wenn
man es ans leere Wort anh"angen kann, und damit zu einem Akzeptierzustand
kommt, also $L=L(\varepsilon)$. Dies ist der Anfangszustand des Automaten,
also
\[
q_0=L.
\]

Die "Ubergangsfunktion $\delta$ f"uhrt die Menge $L(w)$ bei Input
$a$ in die Menge $L(wa)$ "uber, also
\[
\delta \colon Q\times\Sigma\to Q:(L(w),a)\mapsto L(wa).
\]

Akzeptierzustande sind offenbar diejenigen $L(w)$, von denen
aus man nicht mehr weiter gehen muss, um einen Akzeptierzustand
zu erreichen, also jene $L(w)$ die das leere Wort enthalten:
\[
F=\{L(w)\;|\;\varepsilon\in L(w)\}.
\]

Wir nennen den eben konstruierten Automaten $A$.
Jetzt m"ussen wir nur noch zeigen, dass dieser Automat genau die
W"orter in $L$ akzeptiert.
Sei das Wort $w\in L$ zusammengesetzt aus den
Zeichen $a_1,a_2,\dots,a_n$, also $w=a_1a_2\dots a_n$. Dann f"uhrt
der Input $w$ den Anfangszustand $L$ schrittweise "uber in
\[
L(a_1)\to L(a_1a_2)\to L(a_1a_2a_3)\to \dots \to L(a_1a_2\dots a_n),
\]
da letzterer das leere Wort enth"alt, ist er ein Akzeptierzustand.
Ein Wort in $L$ wird also automatisch vom Automaten $A$ akzeptiert.

Jetzt m"ussen wir noch nachweisen, dass ein von $A$ akzeptiertes Wort $w$
auch in $L$ drin liegt. Dass $A$ das
Wort $w$ akzeptiert besagt, dass $\varepsilon \in L(w)$.
Das ist aber gleichbedeuten damit, dass $w\in L$. Damit ist alles
bewiesen, und wir haben folgenden Satz.

\begin{satz}[Myhill-Nerode]\label{satz_dea_aus_sprache}
\index{Myhill-Nerode!Satz von}
Ist $L$ eine regul"are Sprache, dann wird $L$ von dem 
endlichen Automat $A=(Q,\Sigma,\delta,q_0,F)$ akzeptiert mit
\begin{align*}
Q&=\{L(w)\;|\;w\in\Sigma^*\}\\
q_0&=L\\
F&=\{q\in Q\;|\; \varepsilon\in q\}\\
\delta&\colon Q\times \Sigma\to Q:(L(w),a)\mapsto L(wa)
\end{align*}
\end{satz}

\subsubsection{Beispiel}
Wir rekonstruieren den Automaten f"ur die folgende Sprache "uber
$\Sigma=\{{\tt a},{\tt b}\}$:
\[
L=\{w\in \Sigma^*\;|\; |w|_{\tt a}\equiv 0\mod 2\}.
\]
Dazu m"ussen wir die Menge der m"oglichen $L(w)$ ermitteln:
\begin{center}
\begin{tabular}{|c|l|}
\hline
$w$&$L(w)$\\
\hline
$\varepsilon$&$L$\\
{\tt a}&$L({\tt a})=\{w\in L^*\;|\; \text{$|w|_a$ ungerade}\}$\\
{\tt b}&$L({\tt b})=\{w\in L^*\;|\; \text{$|w|_a$ gerade}\}=L$\\
$|w|_{\tt a}$ ungerade&$L({\tt a})$\\
$|w|_{\tt a}$ gerade&$L$\\
\hline
\end{tabular}
\end{center}
Offenbar gibt es also genau die zwei Zust"ande $L$ und $L'=L({\tt a})$.
Ein Zeichen {\tt a} f"uhrt $L$ in $L({\tt a})$ "uber und umgekehrt,
das Zeichen {\tt b} "andert den Zustand nicht:
\[
\entrymodifiers={++[o][F]}
\xymatrix{
*+\txt{} \ar[r]
	&*++[o][F=]{L}\ar@/^/[r]^{\tt a} \ar@(ul,ur)^{\tt b}
		&{L'} \ar@/^/[l]^{\tt a} \ar@(ul,ur)^{\tt b}
}
\]
Dieser Automat ist identisch mit dem fr"uher gefundenen.

Mit dem Satz von Myhill-Nerode erhalten wir damit auch einen Satz,
der zu entscheiden erlaubt, ob eine Sprache regul"ar ist.

\begin{satz}
Sei $L$ eine Sprache "uber $\Sigma$ und $L(w)=\{ v\in\Sigma^*\;|\; wv\in L\}$.
Falls die Menge 
\[
\{L(w)\;|\; w\in\Sigma^*\}
\]
endlich ist, ist $L$ regul"ar. Falls sie unendlich ist, kann $L$ nicht
regul"ar sein.
\end{satz}

Als Anwendung dieses Satzes zeigen wir, dass die Sprache
$\{ 0^n1^n \;|\; n\in\mathbb N\}$ "uber $\Sigma=\{0,1\}$
nicht regul"ar ist. Wir m"ussen also
die Mengen $L(w)$ f"ur alle denkbaren W"orter $w\in\Sigma^*$
ermitteln. Dazu erstellen wir die folgende Tabelle:
\begin{center}
\begin{tabular}{|c|c|}
\hline
$w$&$L(w)$\\
\hline
$\varepsilon$&$L$\\
$1$&$\emptyset$\\
$1^n$&$\emptyset$\\
$0$&$\{0^n1^{n+1}\;|\; n\in\mathbb N\}$\\
$0^2$&$\{0^n1^{n+2}\;|\; n\in\mathbb N\}$\\
$0^k$&$\{0^n1^{n+k}\;|\; n\in\mathbb N\}$\\
$0^k1^l$&$\{1^{k-l}\}$ falls $k\ge l$\\
$0^k1^l$&$\emptyset $ falls $k<l$\\
\hline
\end{tabular}
\end{center}
An den Zeilen vier bis sieben kann man ablesen, dass es unendlich
viele verschiedene Mengen $L(w)$ gibt, somit kann $L$ nicht von
einem endlichen Automaten erkannt werden.

\subsection{Minimaler Automat\label{regulaer:minimalautomat}}
\index{Automat!minimaler}
Zwei endliche Automaten k"onnen v"ollig verschieden aussehen,
und trotzdem die gleiche Sprache akzeptieren. Kann man auf einfache
Art herausfinden, ob zwei endliche Automaten die gleiche Sprache
akzeptieren?

G"abe es f"ur einen Automaten auf eine Standardform bringen, k"onnte
man diese f"ur einen solchen Vergleich verwenden. Zun"achst m"usste
man beide Automaten in die Standardform bringen. Wenn sie Standardformen
"ubereinstimmen, akzeptieren die Automaten die gleiche Sprache.

Eine Standardform eines Automaten k"onnen wir "uber die Konstruktion
aus Satz \ref{satz_dea_aus_sprache} ermitteln: Zu einem Automaten
$A$ ermitteln wir zun"achst die Sprache $L=L(A)$, f"ur die wir dann
den Automaten aus Satz \ref{satz_dea_aus_sprache} bilden.

\begin{satz}[Minimalautomat]\label{satz_minimalautomat}
Sei $A$ ein DEA, und $A'$ der DEA, der nach
Satz \ref{satz_dea_aus_sprache} gebildet wurde. Dann gibt es
eine surjektive Abbildung der Zust"ande $f\colon Q\to Q'$, so dass
\begin{compactenum}
\item $f(q_0)=q_0'=L$
\item $\delta'(f(q),a)=f(\delta(q,a))$
\item $q\in F\Rightarrow f(q)\in F'$
\end{compactenum}
Insbesondere ist $A'$ der kleinste aller Automaten, der die Sprache
$L$ akzeptieren kann.
\end{satz}

\begin{proof}[Beweis]
Jedes Wort $w\in\Sigma^*$ f"uhrt den Automaten $A$ aus dem
Anfangszustand in einen Zustand $q\in Q$. F"uhren zwei W"orter
$w_1$ und $w_2$ den DEA $A$ in den gleichen Zustand $q$, werden
von dort aus die gleichen W"orter akzeptiert, d.\,h.~$L(w_1)=L(w_2)$.
Die Abbildung
\[
f\colon Q\to Q': q\mapsto L(w)
\]
ist also unabh"angig von dem Wort $w$, mit welchem man den Zustand $q$
erreicht hat. 

Jetzt muss gezeigt werden, dass $f$ die genannten Eigenschaften hat:
\begin{compactenum}
\item
$\varepsilon$ f"uhrt $A$ in den Anfangszustand "uber, also
$f(q_0)=L(\varepsilon)=L=q_0'$.
\item Falls $w$ den DEA in den Zustand $q$ f"uhrt, f"uhrt das Wort
$wa$ den DEA in den Zustand, den man von $q$ mit Input $a$ erreicht.
Also ist
\[
\delta'(f(q),a)=\delta'(L(w), a)=L(wa)=f(\delta(q,a))
\]
\item Sei $q\in F$, und sei $w$ ein Wort, welches den DEA in den Zustand 
$q$ f"uhrt. Dann ist $f(q)=L(w)$, da aber $q$ ein Akzeptierzustand ist,
ist $\varepsilon\in L(w)$, $f(q)$ ist also ein Akzeptierzustand von
$A'$.
\end{compactenum}

\end{proof}

Die Bestimmung des Automaten "uber den Satz \ref{satz_dea_aus_sprache}
ist eher umst"andlich. Der Satz \ref{satz_minimalautomat} zeigt
aber auch, dass wir einfach nach einem DEA suchen m"ussen, in dem
m"oglichst viele Zust"ande zusammengelegt worden sind, bis sich die
Anzahl der Zust"ande nicht mehr weiter verringern l"asst.

Wir m"ussen also herausfinden, welche Zust"ande man zusammenlegen
kann. Das sind nat"urlich genau diejenigen, die das gleiche $L(q)$
haben.
\index{aquivalent Zustande@\"aquivalent!Zust\"ande}
Statt herauszufinden, welche Zust"ande in diesem Sinne "aquivalent
sind, k"onnte man auch herauszufinden versuchen, welche nicht "aquivalent
sind. Als Beispiel nehmen wir den folgenden Automaten
\[
\entrymodifiers={++[o][F]}
\xymatrix{
*+\txt{}
	&*+\txt{}
		&{z_1}\ar@/_/[dd]_{0} \ar[dr]^{1}
\\
*+\txt{} \ar[r]
	&{z_0}\ar[ur]^{0} \ar[dr]_1
		&*+\txt{}
			&*++[o][F=]{z_3}\ar@(ur,dr)^{0,1}
\\
*+\txt{}
	&*+\txt{}
		&{z_2} \ar@/_/[uu]_{0} \ar[ur]_{1}
}
\]
Von den zwei Zust"anden $z_1$ und $z_2$ aus kann man genau die gleichen
akzeptierten W"orter bilden, also $L(z_1)=L(z_2)$, man m"usste diese
also zusammenlegen k"onnen.

\index{Algorithmus!f\"ur den Minimalautomaten}
Um solche Zust"ande zu finden, erstellen wir jetzt eine Tabelle, in der
wir aufzeichnen, welche Zust"ande "aquivalent sind oder auch nicht.
"Aquivalente Zust"ande markieren wir mit dem Zeichen $\equiv$, nicht
"aquivalente mit einem Kreuz $\times$.
Zun"achst wissen wir nur, dass jeder Zustand zu sich selbst "aquivalent
ist, also
\begin{center}
\begin{tabular}{ccccc}
     &$z_0$   &$z_1$   &$z_2$   &$z_3$   \\
$z_0$&$\equiv$&        &        &        \\
$z_1$&        &$\equiv$&        &        \\
$z_2$&        &        &$\equiv$&        \\
$z_3$&        &        &        &$\equiv$
\end{tabular}
\end{center}
Zwei Zust"ande k"onnen nicht "aquivalent sein, wenn der eine
ein Akzeptierzustand ist, der andere hingegen nicht, wir markieren
also alle solchen Paare mit einem $\times$, f"ullen aber nur den Teil unterhalb der
Diagonalen aus.
\begin{center}
\begin{tabular}{ccccc}
     &$z_0$   &$z_1$   &$z_2$   &$z_3$   \\
$z_0$&$\equiv$&        &        &        \\
$z_1$&        &$\equiv$&        &        \\
$z_2$&        &        &$\equiv$&        \\
$z_3$&$\times$&$\times$&$\times$&$\equiv$
\end{tabular}
\end{center}
Falls man von den Zust"anden $p$ und $q$ mit einem "Ubergang
ein bereits mit $\times$ markiertes Feld erreichen kann, k"onnen die
Zust"ande $p$ und $q$ nicht "aquivalent sein, man muss also auch das
Feld $(p,q)$ mit $\times$ markieren.
Vom Paar $(z_0,z_1)$ aus kann man zum Beispiel mit einer $1$ das
Paar $(z_2,z_3)$ erreichen, welche als nicht "aquivalent bekannt
sind. Ebenso kann man von $(z_0,z_2)$ mit einer $1$ das Paar
$(z_2,z_3)$ erreichen. So findet man die Tabelle
\begin{center}
\begin{tabular}{ccccc}
     &$z_0$   &$z_1$   &$z_2$   &$z_3$   \\
$z_0$&$\equiv$&        &        &        \\
$z_1$&$\times$&$\equiv$&        &        \\
$z_2$&$\times$&        &$\equiv$&        \\
$z_3$&$\times$&$\times$&$\times$&$\equiv$
\end{tabular}
\end{center}
Das verbleibende Paar $(z_1,z_2)$ wird aber immer in $(z_1,z_2)$ 
oder $(z_3,z_3)$ "ubergef"uhrt, von beiden Paare k"onnen wir nicht
schliessen, dass sie nicht "aquivalent sind, also m"ussen sie "aquivalent
sein:
\begin{center}
\begin{tabular}{ccccc}
     &$z_0$   &$z_1$   &$z_2$   &$z_3$   \\
$z_0$&$\equiv$&        &        &        \\
$z_1$&$\times$&$\equiv$&        &        \\
$z_2$&$\times$&$\equiv$&$\equiv$&        \\
$z_3$&$\times$&$\times$&$\times$&$\equiv$
\end{tabular}
\end{center}
Man kann also die beiden Zust"ande $z_1$ und $z_2$ zusammenlegen,
und erh"alt den minimalen Automaten:
\[
\entrymodifiers={++[o][F]}
\xymatrix{
*+\txt{} \ar[r]
	&{z_0}\ar[r]^{0,1} 
		&{z_{1,2}}\ar@(ur,ul)_{0}\ar[r]^1
			&*++[o][F=]{z_3}\ar@(ur,dr)^{0,1}
}
\]

Der Vergleich von zwei deterministischen endlichen Automaten ist zwar
mit Hilfe des Minimalautomaten m"oglich, dies ist jedoch nicht der
bestm"ogliche Algorithmus.
Ein Algorithmus mit linearer Laufzeit wurde von Hopcroft und Karp
angegeben.

\subsection{Pumping Lemma f"ur regul"are Sprachen\label{regulaer:pumpinglemma}}
\index{Pumping Lemma!f\"ur regul\"are Sprachen}
\begin{figure}
\begin{center}
\includegraphics{images/reg-6.pdf}
\end{center}
\caption{Pumping Lemma: Zerlegung eines Wortes $w$ in drei Teile
$w=xyz$.\label{regular:pumpinglemma-graph}}
\end{figure}
Eine regul"are Sprache wird von einem DEA akzeptiert. Dadurch wird
die Struktur der W"orter eingeschr"ankt. Ein Wort $w$ beschreibt
einen Pfad durch den gerichteten beschriftenten Graphen des DEA.
Wenn das Wort l"anger ist als die Anzahl der Zust"ande, dann muss
mindestens ein Zustand mehrmals vorkommen. Wir k"onnen sogar
sagen, dass die erste solche Wiederholung innerhalb der ersten
$N = |Q|$ Zeichen des Wortes $w$ auftreten muss. Sei $x$ das Wort,
das den Zustand in den ersten wiederholten Zustand $q$ f"uhrt.
Sei $y$ das k"urzeste Wort, welches von $q$ wieder zu $q$ f"uhrt, und
$z$ der Rest, also $w=xyz$. Dann k"onnen wir die Schleife $y$
auch mehrmals wiederholen, und immer ein Wort erhalten welches
vom Anfangszustand "uber den Zustand $q$ (evlt.~mehrmals) zu einem
Endzustand f"uhrt. Die W"orter $xy^kz$ werden also alle auch von
dem Automaten akzeptiert. Damit haben wir folgendes bewiesen:
\begin{satz}[Pumping Lemma f"ur regul"are Sprachen]
\index{pumping length!f\"ur regul\"are Sprachen}
Ist $L$ eine regul"are Sprache, dann gibt es eine Zahl $N$, die Pumping Length, so dass
jedes Wort $w\in L$ mit $|w|\ge N$ in drei Teile
$w=xyz$ zerlegt werden kann, so dass
\begin{compactenum}
\item $|y| > 0$
\item $|xy|\le N$
\item $xy^kz\in L\quad\forall k\ge 0$
\end{compactenum}
\end{satz}

Das Pumping Lemma dient in erster Linie dazu, von einer Sprache
nachzuweisen, dass sie nicht regul"ar ist. Dazu nimmt man an, dass
die Sprache regul"ar ist, f"ur sie also die Aussage des Pumping Lemma
gilt. Dann leitet man daraus einen Widerspruch ab.

\begin{figure}
\begin{center}
\begin{tabular}{l}
\includegraphics{images/pl-2.pdf}\\
\includegraphics{images/pl-1.pdf}\\
\includegraphics{images/pl-3.pdf}\\
\includegraphics{images/pl-4.pdf}
\end{tabular}
\end{center}
\caption{Anwendung des Pumping-Lemmas. Ein gen"ugend langes Wort
der Sprache $w=0^N1^N$ kann nach dem Pumping Lemma in $w=xyz$ 
zerlegt werden (2.~Zeile). Das Pumping Lemma verspricht, dass
abgepumpte W"orter (3.~Zeile) und aufgepumpte W"orter (4.~Zeile)
auch zur Sprache geh"oren sollen, tun sie aber nicht.
\label{plimage}}
\end{figure}

\begin{beispiel}[\bf Eine nicht regul"are Sprache]
Wir f"uhren dies an dem Beispiel der Sprache $L=\{0^n1^n|n\ge 0\}$
durch, welche wir bereits als nicht regul"ar erkannt haben. Wir nehmen
jetzt also an, dass $L$ regul"ar sei. Dann gibt es nach dem Pumping
Lemma eine Zahl $N$, so dass W"orter mit mindestens dieser L"ange
interessante Eigenschaften haben. Wir w"ahlen das Wort
\[
0^N1^N\in L
\]
Nach dem Pumping Lemma muss $w$ in drei Teile aufgeteilt werden k"onnen
(Abbildung~\ref{plimage}),
$w=xyz$,
wovon die ersten beiden Teile $x$ und $y$ zusammen eine L"ange $\le N$
haben, also aus lauter Nullen bestehen m"ussen. Der Teil $y$ muss
mindestens eine Null enthalten. Wenn man jetzt $xy^kz$ bildet, vermehrt
man die Zahl der Nullen, nicht aber die Zahl der Einsen, man erh"alt
also ein Wort, welches mehr Nullen als Einsen enth"alt. Das Pumping
Lemma sagt, dass $xy^kz\in L$, aber die Definition von $L$ sagt,
dass ein Wort in $L$ gleich viele Nullen wie Einsen haben muss, also
$xy^kz\not\in L$ f"ur $k>1$. Dieser Widerspruch zeigt, dass die urspr"ungliche
Annahme, $L$ sei regul"ar gewesen, nicht haltbar ist. Also ist
die Sprache nicht regul"ar.
\end{beispiel}

\subsubsection{Beispiel}
\index{Palindrom}
Ein Palindrom ist ein Wort, welches ``vorw"arts'' und ``r"uckw"arts''
gleich lautet. Zum Beispiel {\tt otto}, {\tt anna}, {\tt reittier} oder der
Klassiker
\begin{center}
{\tt ein neger mit gazelle zagt im regen nie}
\end{center}
Formal k"onnen wir das so beschreiben: ist $w=a_1a_2\dots a_n\in\Sigma^*$
ein Wort, dann sei $w^t=a_na_{n-1}\dots a_2a_1$ das ``r"uckw"arts
geschriebene Wort:
\[
{\tt esel}^t={\tt lese},\quad {\tt reliefpfeiler}^t={\tt reliefpfeiler}.
\]
Palindrome sind offenbar W"orter, die sich bei Anwendung der
$\mathstrut^t$-Operation nicht "andern: $w^t=w$. Die Menge der Palindrome
ist nat"urlich eine Sprache $L=\{w\in\Sigma^*\;|\;w^t=w\}$. 

Wir zeigen jetzt, dass die Menge der Palindrome keine regul"are Sprache ist.
Dazu verwenden wir das Pumping-Lemma, wir nehmen also an, die Sprache
sei regul"ar, und f"uhren dies mit dem Pumping Lemma zu einem
Widerspruch:
\begin{compactenum}
\item Das Pumping Lemma verspricht, dass es eine Zahl $N$ gibt, die
Pumping Length.
\item Wir bilden jetzt das Wort $0^N10^N$. Dieses Wort ist ein Palindrom,
also in $L$.
\item Das Pumping Lemma garantiert jetzt eine Zerlegung des Wortes
in drei Teile $xyz$, mit den Eigenschaften $|xy|\le N$, $|y|\ge 1$ und
$xy^kz\in L\forall k\in\mathbb N$.
\item Wir pr"ufen dies nach f"ur $k=2$. Da $|xy|\le N$, bestehen
sowohl $x$ als auch $y$ aus lauter Nullen. Die $1$ befindet sich im Teil $z$.
Das Wort $xyyz$ hat daher $N+|y|$ Nullen vor der $1$, aber nur $N$ Nullen
nach der $1$, ist also kein Palindrom mehr. Es ist also $xy^2z\not\in L$,
im Widerspruch zur Ausssage des Pumping Lemma (vorangegangener Schritt).
\end{compactenum}
Der Widerspruch zeigt jetzt, dass die Annahme, $L$ sei regul"ar, nicht
haltbar ist, $L$ ist also nicht regul"ar.

\rhead{Nicht deterministische Automaten}
\section{Nicht deterministische Automaten\label{regulaer:nea}}
Ein DEA sieht jeweils nur ein Zeichen weit, kann sich nicht an "altere
Zeichen erinnern, und kann seine fr"uheren Entscheidungen beim Eintreffen
neuer Zeichen nicht mehr revidieren. Es ist daher nicht unmittelbar klar,
wie eine Bedingung wie ``wenn ein
Wort mit einer {\tt 0} aufh"ort, dann muss es auch mit einer {\tt 0}
beginnen'' implementiert werden m"usste.

\index{Automat!nicht deterministischer}
\index{NEA|see{nicht deterministischer endlicher Automat}}
Nichtdeterminismus erweitert 
die Definition eines endlichen Automaten, um solche F"alle leichter 
zug"anglich zu machen. Es stellt sich daher die Frage, ob die Menge 
der Sprachen, die von solchen nichtdeterministischen endlichen Automaten (NEA)
erkannt werden k"onnen,
gr"osser ist. Es wird sich zeigen, dass jeder NEA
in einen "aquivalenten DEA umgewandelt werden kann, NEAs erkennen also
die gleichen Sprachen wie DEAs.

\subsection{Definition\label{regulaer:definition-nea}}
In einem DEA gibt es zu jedem Zustand $q\in Q$
und jedem Zeichen $a\in\Sigma$ genau
einen "Ubergang in den Zustand $\delta(q,a)$.
Ein nicht deterministischer
Automat l"asst dagegen mehrere M"oglichkeiten zu. Das ``Resultat''
von $\delta$ ist also nicht mehr ein (eindeutiger) neuer Zustand, sondern
eine ganze Menge von Zustanden, also ein Element der Potenzmenge $P(Q)$.

Zudem soll ein
nicht deterministischer Automat auch einen "Ubergang durchf"uhren
k"onnen, wenn gar kein Input anliegt.
Ausser den Zeichen aus $\Sigma$ soll f"ur das zweite Argument
von $\delta$ auch das leere Wort zul"assig sein.

\begin{definition}\label{definition_nea}
Ein nichtdeterministischer endlicher Automat (NEA) ist ein Quintupel
$(Q,\Sigma,\delta, q_0,F)$ mit
\begin{compactenum}
\item $Q$ ist eine beliebige endliche Menge von Zust"anden
\item $\Sigma$ ist eine endliche Menge, genannt das Alphabet.
\item $\delta\colon Q\times(\Sigma\cup\{\varepsilon\})\to P(Q)$ heisst "Ubergangsfunktion
\item $q_0\in Q$ heisst Startzustand
\item $F\subset Q$ heisst die Menge der Akzeptierzust"ande.
\end{compactenum}
\end{definition}
\index{e-Ubergange@$\varepsilon$-\"Uberg"ange}
"Uberg"ange mit $\varepsilon$ als ``Input'' heissen $\varepsilon$-"Uberg"ange.
Ein NEA kann einen solchen "Ubergang durchf"uhren ``wann er Lust hat''.
Wir bezeichnen NEAs mit $\varepsilon$-"Uberg"angen auch als
NEA$\mathstrut_\varepsilon$.

Ein DEA ist ein Spezialfall eines NEA. Ein NEA ist
ein DEA, wenn
\begin{align}
\delta(q,\varepsilon)&=\emptyset&&\forall q\in Q\label{deanea1}\\
|\delta(q,a)|&=1&&\forall q\in Q, a\in\Sigma\label{deanea2}
\end{align}
Die erste Bedingung (\ref{deanea1}) bedeutet, dass keine
$\varepsilon$-"Uberg"ange vorkommen. Die zweite Bedingung (\ref{deanea2})
bedeutet, dass von jedem Zustand zu jedem Zeichen genau ein
neuer Zustand m"oglich ist.

\subsubsection{Gerichteter beschrifteter Graph eines NEA}
\index{Graph!gerichteter!beschrifteter!eines NEA}
Auch ein NEA kann durch einen gerichteten beschrifteten Graphen
visualisiert werden. Die Vertizes sind die Zust"ande $V=Q$. Die
Kanten sind mit Element aus $\Sigma\cup\{\varepsilon\}$ beschriftet.
Von der Ecke $q\in Q$ aus f"uhrt zu jedem Element von $\delta(q,a)\in P(Q)$
eine mit $a$ beschriftete Kante, wobei $a\in \Sigma\cup\{\varepsilon\}$, $a=\varepsilon$ ist also auch m"oglich.

Es ist also durchaus denkbar, dass von einem Zustand $q$ aus f"ur ein
bestimmtes Eingabezeichen $a$ kein "Ubergang m"oglich ist, n"amlich
wenn $\delta(q,a)=\emptyset$. Der Automat ``verklemmt'' sich sozusagen
in dieser Situation, und ist sicher nicht in der Lage, ein Wort, welches
ihn in diese Situation gef"uhrt hat, zu akzeptieren.


\subsection{Beispiele}
\begin{beispiel}[\bf Ganzzahl-Automat]
Im Ganzzahl-Automat f"uhren viele "Uberg"ange zum Zustand $e$, der
alle fehlerhaften Inputs aufsammelt. Wir k"onnten diesen Zustand einfach
weglassen, und erhalten folgenden, wesentlich "ubersichtlicheren NEA:
\[
\entrymodifiers={++[o][F]}
\xymatrix{
*+\txt{}\ar[r]
	&{q_0}  \ar[dl]_{{\tt 1},\dots,{\tt 9}} \ar[dr]^{\tt -}
\\
*++[o][F=]{p}\ar@(ul,dl)_{{\tt 0},\dots,{\tt 9}}
	&*+\txt{}
		&{m} \ar[ll]^{{\tt 1},\dots,{\tt 9}}
}
\]
Aus Zustand $q_0$ ist kein "Ubergang mit dem Zeichen {\tt 0} m"oglich,
dadurch werden f"uhrende Nullen verboten.
Ein {\tt -} kann nur am Anfang
stehen, befindet sich der Automat bereits im Zustand $m$ oder $p$, kann 
er keine weiteren {\tt -} akzeptieren.
Nach einem {\tt -} darf nur ein {\tt 0} folgen, denn vom Zustand $m$
aus gibt es keinen mit {\tt 0} angeschrieben "Ubergang.
\end{beispiel}

\begin{beispiel}[\bf Bedingung an ein einzelnes Zeichen]
Sei $\Sigma=\{{\tt a},{\tt b}\}$, man finde einen NEA, der die 
Sprache akzeptiert, deren W"orter an der drittletzten Stelle
eine {\tt a} haben.
\[
\entrymodifiers={++[o][F]}
\xymatrix{
*+\txt{} \ar[r]
	&{q_0} \ar[r]^{\tt a} \ar@(dr,dl)^{{\tt a},{\tt b}}
		&{q_1}\ar[r]^{{\tt a},{\tt b}}
			&{q_2}\ar[r]^{{\tt a},{\tt b}}
				&*++[o][F=]{q_3}
}
\]
Der Automat ist nicht deterministisch, weil im Zustand $q_0$ zwei verschiedene
"Uberg"ange f"ur das Zeichen {\tt a} m"oglich sind. Der Automat bleibt
sozusagen im Zustand $q_0$ bis er ``weiss'', dass jetzt das drittletzte
Zeichen ansteht.
\end{beispiel}

\begin{beispiel}[\bf Teilbarkeit]
F"ur $\Sigma=\{{\tt 0}\}$ finde einen NEA f"ur die Sprache 
\[
L=\{w\in \Sigma^*\;|\; \text{$|w|$ ist durch 2 oder 3 teilbar}\}.
\]
Das Problem bei der Konstruktion eines DEA ist, dass wird zu
Beginn noch nicht wissen k"onnen, ob wir einen ankommenden
String auf Teilbarkeit durch $2$ oder durch $3$ testen sollen.
Also verwenden wir $\varepsilon$-"Uberg"ange aus dem Startzustand
in zwei verschiedene Automaten, die die Teilbarkeit testen:
\[
\entrymodifiers={++[o][F]}
\xymatrix{
*+\txt{}
	&*+\txt{}
		&*++[o][F=]{q_1} \ar@/^/[r]^{\tt 0}
			&{q_2} \ar@/^/[l]^{\tt 0}
\\
*+\txt{} \ar[r]
	&{q_0}\ar[ur]^{\varepsilon} \ar[dr]^{\varepsilon}
\\
*+\txt{}
	&*+\txt{}
		&*++[o][F=]{q_3}\ar[r]^{\tt 0}
			&{q_4}\ar[dl]^{\tt 0}
\\
*+\txt{}
	&*+\txt{}
		&{q_5}\ar[u]^{\tt 0}
}
\]
\end{beispiel}

\subsection{Erreichbare Zust"ande\label{regulaer:erreichbarezustaende}}
\index{erreichbare Zust"ande}
Im Folgenden werden wir berechnen m"ussen, welche Menge von Zust"anden
durch einen "Ubergang oder mehrere verkettete "Uberg"ange erreicht werden
k"onnen.  Es besteht ein Unterschied, ob der "Ubergang infolge eines
Input-Zeichens $a\in\Sigma$ erfolgt, oder ob er spontan als
$\varepsilon$-"Ubergang ausgef"uhrt werden kann.

\subsubsection{"Ubergang zu Zeichen aus $\Sigma$}
Die Menge $\delta(q,a)\subset Q$ gibt die in einem Schritt mit Inputzeichen $a$
von $q$ aus erreichbaren Zust"ande von $Q$ an. F"ur den Input $a_1a_2$ kann
man von jedem Zustand in $\delta(q,a_1)$ aus weiter Zust"ande mit einem
"Uberang mit Inputzeichen $a_2$ erreichen. Die Menge der erreichbaren
Zust"ande ist
\begin{equation}
\delta(q,a_1a_2)=\bigcup_{q_1\in\delta(q,a_1)}\delta(q_1, a_2)
=\delta(\delta(q,a_1),a_2)
\label{erreichbar}
\end{equation}
Dies kann man wiederholen:
\begin{align*}
\delta(q,a_1a_2a_3)&=
\bigcup_{q_2\in\delta(q, a_1a_2)}\delta(q_2,a_3)
=
\delta(\delta(\delta(q,a_1),a_2),a_3)
\\
\delta(q,a_1\dots a_n)&=\bigcup_{q_{n-1}\in\delta(q,a_1\dots a_{n-1})}\delta(q_{n-1},a_n)
=\delta(\dots \delta(q,a_1)\dots,a_n)
\end{align*}
Die so definierte Menge $\delta(q,a_1\dots a_n)$ umfasst alle von
$q$ aus erreichbaren Zust"ande.

Falls $q'\in\delta(q,a_1\dots a_n)$ gibt
es insbesondere eine Folge von Zwischenzust"anden $q_1,\dots,q_{n-1}$
mit $q_k\in\delta(q_{k-1},a_k)$ f"ur alle $k$. Ausserdem  ist
nat"urlich $q_k\in\delta(q,a_1\dots a_k)$ f"ur alle $k$.

Eine Menge $M$ von Zust"anden stellt man sich am besten vor, indem man die
darin enthaltenen Zust"ande farbig markiert. In der Menge $\delta(M,a)$
befinden sich alle Zust"ande, die von Zust"anden von $M$ aus mit Input
$a$ erreichbar sind.

\subsubsection{$\varepsilon$-"Uberg"ange}
Einzelne Zust"ande kann man auch durch $\varepsilon$-"Uberg"ange
erreichen. Die Menge der Zust"ande, die von $q\in Q$ aus durch
$\varepsilon$-"Uberg"ange erreichbar sind, bezeichnen wir mit
$E(q)$.
Sie setzt sich zusammen aus Zust"anden, die in einem einzigen 
Schritt erreichbar sind, und solchen, die durch wiederholte
Schritte erreichbar werden. 
\[
E(q)=\{q\} \cup \delta(q,\varepsilon) \cup \delta(\delta(q,\varepsilon),\varepsilon)\cup\dots
\]

\subsection{``K"onnte''-Automat\label{Thompson-NEA}}
Ein NEA hat bei jedem "Ubergang eventuell mehrere Zielst"ande zur Auswahl.
Eine Implementation muss im Prinzip alle diese M"oglichkeiten 
durchprobieren.
Alternativ k"onnte er sich aber auch merken,
welche m"oglichen Zust"ande im Laufe der Verarbeitung eines
Wortes erreicht werden konnten.
Die Menge der m"oglichen erreichten Zust"ande kann mit jedem
neuen Zeichen des Inputwortes neu berechnet werden. 
Eine separate Speicherung ist nicht erforderlich, wenn man die
erreichten Zust"ande mit einer Markierung versieht.

\begin{figure}
\begin{center}
\includegraphics{images/reg-1}
\end{center}
\caption{Beispiel eines NEA\label{koenntenea}}
\end{figure}

\begin{figure}
\begin{center}
\begin{tabular}{cccc}
&%
\includegraphics{images/reg-2}&
\raisebox{60pt}{$\overset{\displaystyle\text{\tt b}}\longrightarrow$}&
\includegraphics{images/reg-3}\\
\raisebox{60pt}{$\overset{\displaystyle\text{\tt a}}\longrightarrow$}&
\includegraphics{images/reg-4}&
\raisebox{60pt}{$\overset{\displaystyle\text{\tt a}}\longrightarrow$}&
\includegraphics{images/reg-5}
\end{tabular}
\end{center}
\caption{Verarbeitung des Wortes {\tt baa} durch den NEA von
Abbildung~\ref{koenntenea}. Die m"oglichen Zust"ande sind jeweils
rot markiert.\label{koenntebeispiel}
}
\end{figure}
Abbildung~\ref{koenntenea} zeigt einen NEA mit drei Zust"anden.
Wir verfolgen die Verarbeitung des Wortes {\tt baa} in
Abbildung~\ref{koenntebeispiel}.

Bevor ein Zeichen verarbeitet wird,
ist der NEA im Startzustand $q_0$, also ist genau dieser Zustand
markiert. Die Verarbeitung des Zeichens {\tt b} ist deterministisch,
sie bringt den NEA in den Zustand $q_1$. Das folgende Zeichen {\tt a}
ist jedoch nicht mehr deterministisch, es sind sowohl $q_1$ als auch
$q_2$ als Folgezust"ande m"oglich. Die markierte m"ogliche Menge
von Zustanden nach Verarbeitung von {\tt ba} ist daher
$\{q_1,q_2\}$. Bei der Verarbeitung des zweiten {\tt a} kommt noch
der Zustand $q_0$ hinzu. Man berechnet also schrittweise die
Menge der erreichbaren Zust"ande $\delta(q_0,{\tt baa})$.

Nach Verarbeitung des ganzen Wortes kann der NEA in den Zust"anden
$\{q_0,q_1,q_2\}$ sein. Da der Akzeptierzustand $q_0$ in diese Menge
enthalten ist, gibt es eine Kombination von nichtdeterministischen
Entscheidungen, die zu $q_0$ f"uhren, das Wort {\tt baa} kann also
akzeptiert werden.

\index{Thompson, Ken}
\index{Thompson-NEA}
Diese Implementation eines NEA geht auf Ken Thompson zur"uck, sie wurde in
der Regex-Library der achten Edition von Unix von Rob Pike
\index{Pike, Rob} implementiert, die allerdings nicht sehr weit
verbreitet war. Sp"ater hat Henry Spencer die Regex-Library neu
implementiert, allerdings ohne die Thompson-NEA Konstruktion, sondern
mit Backtracking. Da er sie in den public domain freigab, hat sie sich
rasch verbreitet und bildete die Basis Regex-Bibliotheken in Perl, PCRE,
Python und vielen anderen.

\subsection{Transformation NEA $\rightarrow$ DEA\label{regulaer:nea-dea}}
NEAs f"uhren trotz der betr"achtlichen Erweiterung durch den
Nichtdeterminismus nicht zu einer gr"osseren Klasse von akzeptierbaren
Sprachen. Dazu gen"ugt es zu zeigen, dass sich jeder NEA in einen DEA
transformieren l"asst, der die gleiche Sprache akzeptiert.
Dass dies m"oglich sein sollte, deutet bereits die Konstruktion
des Thompson-NEA in \ref{Thompson-NEA} an. Der Thompson-NEA
wird zu einem DEA, wenn man die Verteilung der ``roten Markierungen''
als Zustand des DEA ansieht. Ziel dieses Abschnittes ist, dies
mathematisch streng zu formalisieren.

Ein NEA kann aus zwei Gr"unden kein DEA sein:
er kann $\varepsilon$-"Uberg"ange haben, und er kann nicht eindeutige
oder nicht existierende "Uberg"ange haben, also $|\delta(q,a)|\ne 1$
f"ur gewisse $q\in Q$ und $a\in\Sigma$.  Wir transformieren einen 
beliebigen NEA in zwei Schritten in einen DEA:
\begin{enumerate}
\item Wir bauen uns einen Algorithmus, mit dem man einen NEA ohne
$\varepsilon$-"Uberg"ange in einen DEA umwandeln kann.
\item Wir modifizieren den Algorithmus, so dass er auch mit NEAs mit
$\varepsilon$-"Uberg"angen umgehen kann.
\end{enumerate}

\subsubsection{Transformation f"ur NEA ohne $\varepsilon$-"Uberg"ange}
Der konstruierte DEA muss dar"uber Buch f"uhren, in welchen Zust"anden
der NEA sein k"onnte. Seine Zust"ande sind als Megen von m"oglichen
Zust"anden des NEA. Als Zustandsmenge des DEA wird man also $P(Q)$ 
erwarten. Wie die "ubrigen Elemente des DEA konstruiert werden m"ussen,
besagt der folgende Satz.

\begin{satz}
\label{satz_neadea_eps}
Sei $A=(Q,\Sigma,\delta,q_0,F)$ ein NEA ohne $\varepsilon$-"Uberg"ange.
Dann gibt es einen DEA $A'$, der die gleiche Sprache akzeptiert: $L(A)=L(A')$.
Der DEA $A'$ setzt sich zusammen aus
$(P(Q), \Sigma, \delta', \{q_0\}, F')$,
die "Ubergangsfunktion ist
\[
\delta'\colon P(Q)\times \Sigma: (M, a)\mapsto \bigcup_{q\in M} \delta(q,a)
=\delta(M,a),
\]
und die Akzeptierzust"ande sind die Teilmengen, die einen Akzeptierzustand
enthalten:
\[
F'=\{M\in P(Q)\;|\;M\cap F\ne \emptyset\}.
\]
\end{satz}
\begin{proof}[Beweis]
Es ist klar, dass mit dieser "Ubergangsfunktion $\delta'$ und den
Akzeptierzust"anden $F'$ ein NEA definiert ist, es muss nur noch
bewiesen werden, dass er die gleiche Sprache akzeptiert, also
$L(A)=L(A')$. Das wird zutreffen, wenn $L(A)\subset L(A')$ und
$L(A')\subset L(A)$, also jedes Wort von $L(A)$ ist auch in $L(A')$,
und umgekehrt.

Wenn $w=a_1\dots a_n\in L(A)$, dann gibt es einen Pfad durch den gerichteten
Graphen von $A$, der zu einem Akzeptierzustand f"uhrt. Also enth"alt
die Menge der von $q_0$ aus erreichbaren Zust"ande einen Akzeptierzustand.
Diese Menge ist aber die Menge der Zust"ande, die man durch Anwendung
von $\delta'$ aus $\{q_0\}$ und den Inputzeichen $a_1,\dots,a_n$ erhalten
wird. Da diese Menge einen Akzeptierzustand enth"alt, ist sie ein
Akzeptierzustand von $A'$, und damit $w\in L(A')$.

Sei jetzt umgekehrt $w\in L(A')$. Wir schreiben wieder $w=a_1\dots a_n$.
Auf Grund der Konstruktion des Automaten $A'$, liefert die wiederholte
Anwendung von $\delta'(\cdot, a_k)$ aus dem Anfangszustand $q_0'=\{q_0\}$
die Menge $\delta(\dots\delta(q_0, a_1)\dots ,a_n)$. Da diese Menge ein
Akzeptierzustand von $A'$ ist, gibt es darin einen Akzeptierzustand
von $A$. Da also ein Akzeptierzustand erreichbar ist, ist $w\in L(A)$,
also auch $L(A')\subset L(A)$.
\end{proof}

\subsubsection{Beispiel}
Der NEA von Abbildung~\ref{koenntenea}
%\[
%\entrymodifiers={++[o][F]}
%\xymatrix{
%*+\txt{}\ar[r]
%	&*++[o][F=]{q_0} \ar[d]_{\tt b} 
%		&{q_2}\ar[l]_{\tt a}
%\\
%*+\txt{}
%	&{q_1}\ar[ur]_{{\tt a},{\tt b}} \ar@(ul,dl)_{\tt a}
%}
%\]
soll in einen DEA umgewandelt werden. Er ist nicht deterministisch,
weil vom Zustand $q_1$ aus zwei "Uberg"ange mit {\tt a} m"oglich sind.

Aus dem Satz ist bekannt, dass die Zust"ande die Potenzmengen von $Q$
sind. Diese entsprechen den dreistelligen Bin"arzahlen, wobei jede
Stelle angibt, ob das zugeh"orige $q_i$ in der Menge dabei ist. Wir
schreiben also
\begin{align*}
       &          &q_{001}&=\{\phantom{q_2q_1}q_0\}&q_{110}&=\{q_2,q_1\phantom{,q_0}\}&&\\
q_{000}&=\emptyset&q_{010}&=\{\phantom{q_2}q_1\phantom{q_0}\}&q_{101}&=\{q_2,\phantom{q_1,}q_0\}&q_{111}&=\{q_2,q_1,q_0\}\\
       &          &q_{100}&=\{q_2\phantom{q_1q_0}\}&q_{011}&=\{\phantom{q_2,}q_1,q_0\}&&\\
\end{align*}
Akzeptierzust"ande sind jene Mengen, die $q_0$ enthalten, also
\[
F'=\{ q_{001}, q_{011}, q_{101}, q_{111}\},
\]
Startzustand ist $q_{001}$.
Damit k"onnen wir jetzt das Zustandsdiagramm zeichnen
\[
\entrymodifiers={++[o][F]}
\xymatrix{
*+\txt{}\ar[r]
	&*++[o][F=]{q_{001}} \ar[d]^{\tt b}\ar[dl]_{\tt a}
		&{q_{110}}\ar[dr]^{\tt a} \ar[ddl]^{\tt b}
\\
{q_{000}}\ar@(ul,dl)_{{\tt a},{\tt b}}
	&{q_{010}} \ar[ur]^{\tt a} \ar[d]^{\tt b}
		&*++[o][F=]{q_{101}} \ar[ul]^{\tt a} \ar[l]^{\tt b}
			&*++[o][F=]{q_{111}}\ar@(ur,dr)^{{\tt a}} \ar@/_20pt/[ul]_{\tt b}
\\
*+\txt{}
	&{q_{100}}\ar@/^20pt/[uu]^{\tt a}  \ar[ul]^{\tt b}
		&*++[o][F=]{q_{011}} \ar@/_20pt/[uu]_{{\tt a},{\tt b}}
}
\]
Man erkennt sofort, dass die Zust"ande $q_{101}$ und $q_{011}$
nicht erreicht werden k"onnen, also weggelassen werden k"onnten:
\begin{equation}
\entrymodifiers={++[o][F]}
\xymatrix{
*+\txt{}\ar[r]
	&*++[o][F=]{q_{001}} \ar[d]^{\tt b}\ar[dl]_{\tt a}
		&{q_{110}}\ar[dr]^{\tt a} \ar[ddl]^{\tt b}
\\
{q_{000}}\ar@(ul,dl)_{{\tt a},{\tt b}}
	&{q_{010}} \ar[ur]^{\tt a} \ar[d]^{\tt b}
		&*+\txt{}
			&*++[o][F=]{q_{111}}\ar@(ur,dr)^{{\tt a}} \ar@/_20pt/[ul]_{\tt b}
\\
*+\txt{}
	&{q_{100}}\ar@/^20pt/[uu]^{\tt a}  \ar[ul]^{\tt b}
		&*+\txt{}
}
\label{nea-eps}
\end{equation}
Man kann ebenfalls verifzieren, dass dies tats"achlich ein DEA ist, 
von jedem Zustand gehen genau zwei Pfeile f"ur die "Uberg"ange mit
den Zeichen ${\tt a}$ und ${\tt b}$ aus. Etwas "ubersichtlicher
geschrieben lautet er
\[
\entrymodifiers={++[o][F]}
\xymatrix{
*+\txt{}\ar[r]
	&*++[o][F=]{1}\ar[r]^{\tt b} \ar[d]_{\tt a}
		&2 \ar[r]^{\tt a} \ar[d]^{\tt b}
			&6\ar@/_/[r]_{\tt a} \ar[dl]^{\tt b}
				&*++[o][F=]{7}\ar@/_/[l]_{\tt b} \ar@(ur,dr)^{\tt a}
\\
*+\txt{}
	&0 \ar@(l,d)_{{\tt a},{\tt b}}
		&4 \ar[ul]_{\tt a} \ar[l]^{\tt b}
}
\]

\subsubsection{NEA mit $\varepsilon$-"Uberg"angen}
Wir erweitern die Konstruktion des DEA aus einem NEA jetzt, um eventuell
vorhandenen $\varepsilon$-"Uberg"angen Rechnung zu tragen.

\begin{satz}
\label{satz_neadea}
Sei $A=(Q,\Sigma,\delta,q_0,F)$ ein NEA, dann gibt es einen
DEA $A'=(Q',\Sigma,\delta',q_0',F')$, der die gleiche Sprache
akzeptiert, $L(A)=L(A')$. Es ist
\begin{align*}
Q'&=P(Q)\\
\delta'(M,a)&=E(\delta(M, a))\quad M\in P(Q)\\
q_0'&=E(q_0)\\
F'&=\{M\in P(Q)\;|\, M\cap Q\ne \emptyset\}
\end{align*}
\end{satz}

\begin{proof}[Beweis]
Gegen"uber Satz \ref{satz_neadea} hat sich ge"andert, dass wir den
Anfangszustand von $\{q_0\}$ auf $E(q_0)$ vergr"ossert haben. Dadurch
erfassen wir alle Pfade durch den Automaten, die mit einem oder
mehreren $\varepsilon$-"Uberg"angen beginnen. Falls $E(q_0)\cap F\ne \emptyset$
wird der Anfangszustand auch zu einem Endzustand.

Zudem erweitern wir in der Definition des Bildes in a um alle Zust"ande,
die durch einen angeh"angten $\varepsilon$-"Ubergang erreicht werden
k"onnen.
\end{proof}

\subsubsection{Beispiel}
Der NEA 
\[
\entrymodifiers={++[o][F]}
\xymatrix{
*+\txt{}\ar[r]
	&*++[o][F=]{q_0} \ar[d]_{\tt b} \ar@/_/[r]_{\varepsilon}
		&{q_2}\ar@/_/[l]_{\tt a}
\\
*+\txt{}
	&{q_1}\ar[ur]_{{\tt a},{\tt b}} \ar@(ul,dl)_{\tt a}
}
\]
soll in einen DEA umgewandelt werden.

Den NEA ohne den $\varepsilon$-"Ubergang haben wir bereits in
\ref{nea-eps} umgewandelt. Jetzt m"ussen wir nur die beiden
im Satz \ref{satz_neadea} erw"ahnten Modifikationen durchf"uhren.

Urspr"unglich war der Startzustand $q_{001}$. Da aus dem Startzustand
des NEA mittels $\varepsilon$-"Ubergang auch der Zustand $q_2$
erreichbar ist, muss der Anfangszustand des DEA um diesen Zustand
erweitert werden, und ist jetzt $q_{101}$:
\[
\entrymodifiers={++[o][F]}
\xymatrix{
*+\txt{}
	&*++[o][F=]{q_{001}} \ar[d]^{\tt b}\ar[dl]_{\tt a}
		&{q_{110}}\ar[dr]^{\tt a} \ar[ddl]^{\tt b}
\\
{q_{000}}\ar@(ul,dl)_{{\tt a},{\tt b}}
	&{q_{010}} \ar[ur]^{\tt a} \ar[d]^{\tt b}
		&*++[o][F=]{q_{101}} \ar[ul]^{\tt a} \ar[l]^{\tt b}
			&*++[o][F=]{q_{111}}\ar@(ur,dr)^{{\tt a}} \ar@/_20pt/[ul]_{\tt b}
\\
*+\txt{}
	&{q_{100}}\ar@/^20pt/[uu]^{\tt a} \ar[ul]^{\tt b}
		&*++[o][F=]{q_{011}} \ar@/_20pt/[uu]_{{\tt a},{\tt b}}
			&*+\txt{}\ar[ul]
}
\]


Zus"atzlich m"ussen wir bei jedem "Ubergang die M"oglichkeit in
Betracht ziehen, dass noch ein $\varepsilon$-"Ubergang angeh"angt
wird. Dies bedeutet im vorliegenden Fall, dass bei jedem "Ubergang,
der zu einem $q_0$ enthaltenden Zustand f"uhrt, auch noch der
Zustand $q_2$ hinzugef"ugt werden muss.
\[
\entrymodifiers={++[o][F]}
\xymatrix{
*+\txt{}
	&*++[o][F=]{q_{001}} \ar[d]^{\tt b}\ar[dl]_{\tt a}
		&{q_{110}}\ar[dr]^{\tt a} \ar[ddl]^{\tt b}
\\
{q_{000}}\ar@(ul,dl)_{{\tt a},{\tt b}}
	&{q_{010}} \ar[ur]^{\tt a} \ar[d]^{\tt b}
		&*++[o][F=]{q_{101}} \ar@(ul,ur)^{\tt a} \ar[l]^{\tt b}
			&*++[o][F=]{q_{111}}\ar@(ur,dr)^{{\tt a}} \ar@/_20pt/[ul]_{\tt b}
\\
*+\txt{}
	&{q_{100}}\ar[ur]^{\tt a} \ar[ul]^{\tt b}
		&*++[o][F=]{q_{011}} \ar@/_20pt/[uu]_{{\tt a},{\tt b}}
			&*+\txt{}\ar[ul]
}
\]
In diesem Automaten sind die Zust"ande $q_{001}$ und $q_{011}$ nicht
mehr erreichbar, man kann sie weglassen.
\[
\entrymodifiers={++[o][F]}
\xymatrix{
*+\txt{}
	&*+\txt{}
		&{q_{110}}\ar[dr]^{\tt a} \ar[ddl]^{\tt b}
\\
{q_{000}}\ar@(ur,dr)^{{\tt a},{\tt b}}
	&{q_{010}} \ar[ur]^{\tt a} \ar[d]^{\tt b}
		&*++[o][F=]{q_{101}} \ar@(ul,ur)^{\tt a} \ar[l]^{\tt b}
			&*++[o][F=]{q_{111}}\ar@(ur,dr)^{{\tt a}} \ar@/_20pt/[ul]_{\tt b}
\\
*+\txt{}
	&{q_{100}}\ar[ur]^{\tt a} \ar[ul]^{\tt b}
		&*+\txt{}
			&*+\txt{}\ar[ul]
}
\]
Wie man mit dem Algorithmus "uber den Minimalautomaten kontrollieren
kann, l"asst sich dieser Automat nicht mehr weiter reduzieren. Etwas
"ubersichtlicher geschrieben lautet er
\[
\entrymodifiers={++[o][F]}
\xymatrix{
*+\txt{}\ar[r]
	&*++[o][F=]{5}\ar@(ul,ur)^{\tt a} \ar[r]^{\tt b}
		&2\ar[r]^{\tt a} \ar[d]^{\tt b}
			&6\ar@/_/[r]_{\tt a} \ar[dl]^{\tt b}
				&*++[o][F=]{7}\ar@/_/[l]_{\tt b} \ar@(ur,dr)^{\tt a}
\\
*+\txt{}
	&0\ar@(ul,dl)_{{\tt a},{\tt b}}
		&4\ar[ul]^{\tt a} \ar[l]^{\tt b}
}
\]


Satz \ref{satz_neadea} ist nicht nur ein theoretisch interessantes
Resultat. Wenn man die endliche vielen Zust"ande mit den Zahlen
$0,\dots,|Q|-1$ identifiziert, kann man die Teilmengen von $Q$ mit
den $|Q|$-stelligen Bin"arzahlen identifizieren.
Der Teilmenge $M\subset Q$ entspricht die Bin"arzahlen, die Einsen an
den Stellen hat, der Nummer
in $M$ enthalten sind.
Die grosse Vereinigungen ist ebenfalls sehr einfach
zu berechnen, sie entspricht der Oder-Verkn"pfung
der einzelnen Mengen in (\ref{erreichbar}).
Somit haben wir nicht nur ein Existenz-Resultat sondern einen
Algorithmus, mit dem ein beliebiger NEA in einen DEA umgewandelt
werden kann.
Daraus k"onnen wir jetzt einen Vergleichsalgorithmus f"ur regul"are
Sprachen ableiten: 

\begin{satz}
Es gibt einen Algorithmus, mit dem entschieden werden kann, ob
zwei Automaten $A$ und $B$ die gleiche Sprache akzeptieren.
\end{satz}

\begin{proof}[Beweis]
Um zu entscheiden, ob $L(A)=L(B)$, wendet man folgenden Algorithmus
an:
\begin{enumerate}
\item Wandle $A$ und $B$
mit dem Algorithmus des Satzes 
\ref{satz_neadea} in einen DEAs $A'$ und $B'$ um.
\item Reduziert anschliessend $A'$ und $B'$  mit dem Algorithmus
von Satz \ref{satz_minimalautomat} in Minimalautomaten
$A''$ und $B''$.
\item Die von $A$ und $B$ akzeptierten Sprachen sind genau dann
gleich, wenn $A''$ und $B''$ identisch sind.
\end{enumerate}
\end{proof}

Der hier vorgeschlagene Beweis ist allerdings oft nicht praktikabel.
Der aus dem NEA erzeugte DEA hat im schlechtesten Fall $2^n$
Zust"ande, wenn der NEA $n$ Zust"ande hatte, die Laufzeit des
Algorithmus ist also exponentiell in der Gr"osse des NEA.
Ein wesentlich schnellerer Algorithmus wurde 2015 von Bonchi und Pous
gefunden \cite{skript:bonchi-pous}.

\subsection{Mengenoperationen\label{regulaer:mengenoperationen}}
\index{Mengenoperationen}
\index{Durchschnitt}
\index{Vereinigung}
\index{Differenz}
Sprachen sind Mengen von W"ortern, also sind auch deren Vereinigung,
Durchschnitt, Differenz usw.~Sprachen. Sind die Sprachen regul"ar,
sind dann auch Vereinigung, Durchschnitt, Differenz etc.~regul"ar? 
NEAs erlauben uns sehr elegant zu zeigen, dass Sie diese Operationen
alle wieder regul"are Sprachen liefern.

\begin{satz}
\index{Vereinigung}
\label{satz_union}
Sind $L_1$ und $L_2$ regul"are Sprachen, dann
ist auch $L_1\cup L_2$ regul"ar.
\end{satz}

\begin{figure}
\begin{center}
\includegraphics{images/nea-5}
\qquad
\qquad
\includegraphics{images/nea-6}
\end{center}
\caption{Konstruktion eines NEA 
f"ur die Sprache $L(A_1)\cup L(A_2)$.\label{regulaer:vereinigung}}
\end{figure}

\begin{proof}[Beweis]
Da die Sprachen regul"ar sind, gibt es endliche Automaten 
\begin{align*}
A_1&=(Q_1,\Sigma_1,\delta_1, q_{01}, F_1)\\
A_2&=(Q_2,\Sigma_2,\delta_2, q_{02}, F_2)
\end{align*}
mit $L_1=L(A_1)$ und $L_2=L(A_2)$. Wir m"ussen
jetzt einen Automaten
\[
A = (Q, \Sigma, \delta, q_0, F)
\]
konstruieren, der $L_1\cup L_2$ akzeptieren
(Abbildung~\ref{regulaer:vereinigung}).

F"ur die Vereinigung muss das Alphabet alle Zeichen enthalten,
also $\Sigma = \Sigma_1\cup\Sigma_2$. Man kann jeden der Automaten 
$A_1$ und $A_2$ auch als Automaten "uber dem Alphabet $\Sigma$
betrachten, indem man setzt
\[
\delta_i(q, x)=\begin{cases}
\delta_i(q,x)&\qquad q\in Q_i\text{ und }  x\in \Sigma_i\\
\emptyset&\qquad q\in Q_i\text{ und } x\in \Sigma_{3-i}\setminus \Sigma_i\\
\end{cases}
\]
Es ist also keine Einschr"ankung, wenn wir annehmen, dass
$\Sigma=\Sigma_1=\Sigma_2$.

Um den Automaten f"ur $L_1\cup L_2$ zu konstruieren, nehmen wir jetzt
weiter an, dass $Q_1$ und $Q_2$ disjunkt sind.
F"ur $A$ verwenden wir einen neuen Startzustand $q_0$.
Ein Wort in $L_1\cup L_2$ muss von einem der beiden Automaten
akzeptiert werden.
Der Automat muss sich also zu Beginn nichtdeterministisch
daf"ur entscheiden, den einen oder anderen Automaten zum
akzeptieren eines Wortes zu verwenden.
Also verwenden wir 
\begin{align*}
Q&=Q_1\cup Q_1\cup \{q_0\}\\
F&=F_1\cup F_2\\
\delta(x,a)&=\begin{cases}
\delta_1(x,a)&\qquad x\in Q_1\\
\delta_2(x,a)&\qquad x\in Q_2\\
\{q_{01}, q_{02}\}&\qquad x=q_0, a=\varepsilon\\
\emptyset&\qquad x=q_0, a\ne\varepsilon
\end{cases}
\end{align*}
\end{proof}

\begin{satz}
\index{Durchschnitt}
\label{satz_intersection}
Sind $L_1$ und $L_2$ regul"are Sprachen, dann
ist auch $L_1\cap L_2$ regul"ar.
\end{satz}

\begin{proof}[Beweis]
Da die Sprachen regul"ar sind, gibt es endliche Automaten 
\begin{align*}
A_1&=(Q_1,\Sigma_1,\delta_1, q_{01}, F_1)\\
A_2&=(Q_2,\Sigma_2,\delta_2, q_{02}, F_2)
\end{align*}
mit $L_1=L(A_1)$ und $L_2=L(A_2)$. Wir m"ussen
jetzt einen Automaten
\[
A = (Q, \Sigma, \delta, q_0, F)
\]
konstruieren, der $L_1\cap L_2$ akzeptieren.

F"ur die Schnittmenge brauchen wir einen Automaten, der die Abl"aufe 
in beiden Teilautomaten gleichzeitig codiert.
Dies ist m"oglich, indem man Zust"ande und "Uberg"ange nebeneinander
simuliert.
Als Zustandsmenge verwenden wir daher $Q=Q_1\times Q_2$. Als "Ubergangsfunktion
verwenden wir
\[
\delta((q',q''),a)=\delta_1(q',a)\times \delta_2(q'',a).
\]
Akzeptiert werden kann ein Wort, wenn beide Komponenten des Paares 
Akzeptierzust"ande ihrer Automaten sind.
Akzeptierzust"ande von $A$ sind also $F=F_1\times F_2$.
\end{proof}

\index{Produktautomat}
Wir nennen diesen mit Hilfe des kartesischen Produktes konstruierten
Automaten auch den kartesischen Produktautomaten.

\subsubsection{Beispiel}
Zur Illustration wollen 
wir einen Automaten f"ur die Schnittmenge von
\begin{align*}
L_1&=\{w\in\Sigma^*\;|\; \text{$|w|_0$ gerade}\}\qquad\text{und}\\
L_2&=\{w\in\Sigma^*\;|\,\text{$w$ ist eine durch 3 teilbare Bin"arzahl}\}.
\end{align*}
konstruieren. Die einzelnen Teilautomaten sind $A_1$ f"ur $L_1$:
\[
\entrymodifiers={++[o][F]}
\xymatrix{
*+\txt{}\ar[r]
	&*++[o][F=]{} \ar@/^/[r]^{\tt 0} \ar@(ul,ur)^{\tt 1}
		&{} \ar@/^/[l]^{\tt 0} \ar@(ul,ur)^{\tt 1}
}
\]
und $A_2$ f"ur $L_2$
\[
\entrymodifiers={++[o][F]}
\xymatrix{
*+\txt{}\ar[r]
	&*++[o][F=]{0} \ar@/^/[r]^{\tt 1} \ar@(ul,ur)^{\tt 0}
		&{1} \ar@/^/[l]^{\tt 1} \ar@/^/[r]^{\tt 0}
			&{2}\ar@(ur,dr)^{\tt 1} \ar@/^/[l]^{\tt 0}
}
\]
Der Produktautomat ist jetzt
\[
\entrymodifiers={++[o][F]}
\xymatrix{
*+\txt{} \ar[r] \ar[d] \ar[dr]
	&*++[o][F=]{0} \ar@/^/[r]_{\tt 1} \ar@(ul,ur)^{\tt 0}
		&{1} \ar@/^/[l] \ar@/^/[r]_{\tt 0}
			&{2}\ar@(ur,dr)^{\tt 1} \ar@/^/[l]
\\
*++[o][F=]{} \ar@/^/[d] \ar@(ul,dl)_{\tt 1}
	&*++[o][F=]{} \ar@/^/[d]{\tt 0} \ar@/^/[r]_{\tt 1}
		&{} \ar@/^/[l] \ar@/^/[dr]_{\tt 0}
			&{}\ar@(ur,dr)^{\tt 1} \ar@/^/[dl]
\\
{} \ar@/^/[u]_{\tt 0} \ar@(ul,dl)_{\tt 1}
	&{} \ar@/^/[u]_{\tt 0} \ar@/^/[r]_{\tt 1}
		&{} \ar@/^/[l] \ar@/^/[ur]
			&{}\ar@(ur,dr)^{\tt 1} \ar@/^/[ul]
}
\]

Man kann den Produktautomaten auch
verwenden, einen Automaten zu bauen, der $L_1\cup L_2$
akzeptiert. Der kartesische Produktautomat simuliert ja sozusagen
beide Teilautomaten in den beiden Komponenten der Zust"ande. 
Aktzeptabel ist ein Wort $w$, wenn es von $A_1$
oder von $A_2$ akzueptiert ist, oder wenn der Produktautomat
sich am Ende des Wortes in einem Zustand befindet, der ein
Akzeptierzustand von $A_1$ ist oder von $A_2$. Verwendet
man also $F_1\times Q_2\cup Q_1\times F_2$ als Akzeptierzust"ande,
erh"alt man einen Automaten, der $L_1\cup L_2$ akzeptiert.
Das zugeh"orige Zustandsdiagramm ist:
\[
\entrymodifiers={++[o][F]}
\xymatrix{
*+\txt{} \ar[r] \ar[d] \ar[dr]
	&*++[o][F=]{0} \ar@/^/[r]_{\tt 1} \ar@(ul,ur)^{\tt 0}
		&{1} \ar@/^/[l] \ar@/^/[r]_{\tt 0}
			&{2}\ar@(ur,dr)^{\tt 1} \ar@/^/[l]
\\
*++[o][F=]{} \ar@/^/[d] \ar@(ul,dl)_{\tt 1}
	&*++[o][F=]{} \ar@/^/[d]{\tt 0} \ar@/^/[r]_{\tt 1}
		&*++[o][F=]{} \ar@/^/[l] \ar@/^/[dr]_{\tt 0}
			&*++[o][F=]{}\ar@(ur,dr)^{\tt 1} \ar@/^/[dl]
\\
{} \ar@/^/[u]_{\tt 0} \ar@(ul,dl)_{\tt 1}
	&*++[o][F=]{} \ar@/^/[u]_{\tt 0} \ar@/^/[r]_{\tt 1}
		&{} \ar@/^/[l] \ar@/^/[ur]
			&{}\ar@(ur,dr)^{\tt 1} \ar@/^/[ul]
}
\]

\begin{satz}
\index{Differenz}
\label{satz_regcomplement}
Ist $L$ eine regul"are Sprache, dann ist auch $\bar L$ regul"ar.
Sind $L_1$ und $L_2$ regul"ar, dann ist auch $L_1\setminus L_2$
regul"ar.
\end{satz}

\begin{proof}[Beweis]
Da $L$ regul"ar ist, gibt es einen DEA, welcher $L=L(A)$ erf"ullt.
Wir ersetzen in diesem DEA die Menge $F$ der Akzeptierzust"ande
durch $Q\setminus F$, und erhalten einen neuen DEA $A'$. Dieser
neue DEA akzeptiert genau diejenigen W"orter, die $A$ nicht
akzeptiert hat, also $L(A')=\overline{L(A)}$. Somit ist $\bar L=L(A')$
regul"ar.

Die zweite Aussage folgt aus $L_1\setminus L_2=L_1\cap\bar L_2$ und
Satz \ref{satz_intersection}
\end{proof}
Man beachte, dass in diesem Beweis unbedingt ein DEA verwendet werden
muss. Die beiden NEAs
\[
\entrymodifiers={++[o][F]}
\xymatrix{
*+\txt{}\ar[r]
	&{}\ar@(ul,ur)^{\Sigma} \ar[r]^{\varepsilon}
		&*++[o][F=]{}
&
*+\txt{}\ar[r]
	&*++[o][F=]{}\ar@(ul,ur)^{\Sigma} \ar[r]^{\varepsilon}
		&{}
}
\]
gehen auseinander durch Ersetzung der Akzeptierzust"ande
$F\leftrightarrow Q\setminus F$ hervor, aber beide akzeptieren
$\Sigma^*$, also nicht das Komplement.

\rhead{Regul"are Ausdr"ucke}
\section{Regul"are Ausdr"ucke\label{regulaer:re}}
\index{regul\"are Ausdr\"ucke}
Endliche Automaten beschreiben regul"are Sprachen, sind aber f"ur die
Anwendung eher unhandlich. In der Praxis haben sich regul"are Ausdr"ucke
durchgesetzt, die mit einer einfachen Syntax ebenfalls Mengen von
W"ortern zu spezifizieren erlauben. Ziel dieses Abschnittes ist
zu zeigen, dass regul"are Ausdr"ucke genau gleich ausdrucksstark sind
wie DEAs.

\subsection{Regul"are Operationen\label{regulaer:regulaere-operationen}}
\index{regul\"are Operationen}
Wir wissen bereits, dass wir die Mengenoperationen auf regul"are
Sprachen anwenden d"urfen, ohne die Klasse der regul"aren Sprachen
zu verlassen.
Es sind jedoch noch zwei weitere Operationen m"oglich, die ebenfalls
nicht aus der Klasse herausf"uhren:

\begin{definition}
\index{Verkettung}
Seien $L_1$ und $L_2$ Sprachen, dann ist die
Verkettung von $L_1$ und $L_2$ die Sprache
\[
L_1L_2=\{w_1w_2\;|\;w_1\in L_1,w_2\in L_2\}.
\]
Die mehrfache Verkettung von $L$ wird mit $L^n$ bezeichnet:
\begin{align*}
L^0&=\{\varepsilon\}\\
L^n&=L^{n-1}L
\end{align*}
\end{definition}

\begin{definition}
\index{*-Operation@$*$-Operation}
Sei $L$ eine Sprache, dann ist die Stern-Operation
von $L$ die Sprache
\[
L^*=\bigcup_{k\in\mathbb N} L^k.
\]
\end{definition}

Wir m"ochten jetzt zeigen, dass diese Operationen nicht aus den
regul"aren Sprachen herausf"uhren. Dazu m"ussen wir zu gegebenen
Automaten $A_1$ und $A_2$ f"ur zwei Sprachen $L_1$ und $L_2$
neue Automaten konstruieren, welche
die Sprachen $L_1L_2$ oder $L_1^*$ akzeptieren. Die Automaten
$A_1$ und $A_2$ m"ussen wir dabei als ``Black Box'' betrachten,
wir d"urfen daran nichts "andern.

\subsubsection{Automat zu einer Verkettung $L_1L_2$}
\index{Verkettung}
F"ur die Verkettung $L_1L_2$ ist es zum Beispiel
nicht zul"assig, die Akzeptierzust"ande von
$A_1$ mit dem Startzustand von $A_2$ zusammenzulegen. Dadurch
w"urde es n"amlich m"oglich, auch Wege wieder zur"uck in
den ersten Automaten zu verwenden.
Im folgenden Beispiel akzeptiert der erste Automat Strings mit
einer ungeraden Anzahl Nullen, der zweite Strings mit einer
ungeraden Anzahl {\tt a}.
\[
\entrymodifiers={++[o][F]}
\xymatrix{
*+\txt{}\ar[r]
	&{} \ar@/^/[r]^{\tt 0}
		&*++[o][F=]{} \ar@/^/[l]^{\tt 0}
			&*\txt{}\ar[r]
				&{} \ar@/^/[r]^{\tt a}
					&*++[o][F=]{} \ar@/^/[l]^{\tt a }
}
\]
Verkettet man die Automaten, indem man den Akzeptierzustand des
ersten mit dem Startzustand des zweiten Automaten zusammenlegt,
erh"alt man
\[
\entrymodifiers={++[o][F]}
\xymatrix{
*+\txt{}\ar[r]
	&{} \ar@/^/[r]^{\tt 0}
		&{} \ar@/^/[l]^{\tt 0}
				 \ar@/^/[r]^{\tt a}
					&*++[o][F=]{} \ar@/^/[l]^{\tt a }
}
\]
Dieser Automat akzeptiert aber auch das Wort {\tt 0aa00a}, welches
gar nicht in $L_1L_2$ ist! Die Verkettung muss also mit Hilfe
einer ``Einbahnstrasse'' erfolgen, ein $\varepsilon$-"Ubergang
ist ideal daf"ur geeignet. Die folgende Verkettung funktioniert:
\[
\entrymodifiers={++[o][F]}
\xymatrix{
*+\txt{}\ar[r]
	&{} \ar@/^/[r]^{\tt 0}
		&{} \ar@/^/[l]^{\tt 0} \ar[r]^{\varepsilon}
			&{} \ar@/^/[r]^{\tt a}
				&*++[o][F=]{} \ar@/^/[l]^{\tt a }
}
\]

F\"ur die allgemeine Konstruktion gehen wir aus
von zwei Automaten $A_1$ und $A_2$:
\begin{center}
\includegraphics{images/nea-1}
\end{center}
Ein Automat f"ur die Verkettung entsteht, indem man
die Akzeptierzust"ande im ersten Automaten "uber
$\varepsilon$-"Uberg"ange mit dem
Startzustand des zweiten verbindet:
\begin{center}
\includegraphics{images/nea-2}
\end{center}
Die $\varepsilon$-"Uberg"ange
fungieren als Einbahnstrassen und stellen sicher, auch innerhalb der
einzelnen Automaten keine neuen akzeptierten W"orter entstehen k"onnen.

Etwas formeller kann man das Resultat im folgenden Satz zusammenfassen.

\begin{satz}
\index{Verkettung}
\label{satz_concat}
Sind $L_1$ und $L_2$ regul"are Sprachen, dann ist auch $L_1L_2$
regul"ar.
\end{satz}

\begin{proof}[Beweis]
Ein Wort $w\in L_1L_2$ besteht aus zwei Teilen, die von $L_1$
bzw.~$L_2$ akzeptiert werden. Ein Automat $A$ f"ur $L=L_1L_2$
kann aus Automaten $A_1$ und $A_2$ konstruiert werden, indem man
die Akzeptierzust"ande von $A_1$ "uber $\varepsilon$-"Uberg"ange
mit dem Startzustand von $A_2$ verbindet. Also verwenden
wir $Q=Q_1\cup Q_2$.
Startzustand ist der Startzustand $q_{01}$ von $L_1$, Akzeptierzust"ande
sind die Akzeptierzust"ande $F_2$ von $A_2$. Also
\begin{align*}
Q&=Q_1\cup Q_2\\
F&=F_2\\
q_0&=q_{01}\\
\delta(q,a)&=\begin{cases}
\delta_1(q,a)\cup\{q_{02}\}&\qquad q\in F_1, a=\varepsilon\\
\delta_1(q,a)&\qquad q\in Q_1\setminus F_1\\
\delta_2(q,a)&\qquad q\in Q_2
\end{cases}
\end{align*}
\end{proof}

\subsubsection{Automat zur *-Operation $L^*$}
\index{*-Operation@$*$-Operation}
Die Stern-Operation verlangt, dass man einen Automaten mehrfach
durchlaufen k"onnen muss.
Dies k"onnte man mit einem $\varepsilon$-"Ubergang
von den Akzeptierzust"anden direkt zum Startzustand realisieren.
Dadurch "andert man aber den Automaten, und es ist a priori nicht
klar, dass man dadurch die Sprache nicht ver"andert. Der folgende
Automat akzeptiert genau die Strings aus einer ungeraden Anzahl Nullen:
\begin{equation}
\entrymodifiers={++[o][F]}
\xymatrix{
*+\txt{}\ar[r]
	&{}\ar@/^/[r]^{\tt 0}
		&*++[o][F=]{}\ar@/^/[l]^{\tt 0}
}
\label{ungeradenullen}
\end{equation}
Verbindet man den Akzeptierzustand mit dem Startzustand mit einem
$\varepsilon$-"Ubergang, dann entsteht:
\[
\entrymodifiers={++[o][F]}
\xymatrix{
*+\txt{}\ar[r]
	&{}\ar@/^/[r]^{\tt 0}
		&*++[o][F=]{}\ar@/^/[l]^{\tt 0}\ar@/^20pt/[l]^{\varepsilon}
}
\]
Dieser Automat akzeptiert Strings beliebiger L"ange $>0$, was
man von $L^*$ zwar erwartet, was aber im ersten Moment doch
etwas "uberraschend ist.

Es fehlt nun aber noch ein Akzeptierzustand, mit dem man
das leere Wort akzeptieren kann. Das leere Wort wird genau
dann akzeptiert, wenn der Startzustand ein Akzeptierzustand ist.
Wir d"urfen aber den Startzustand des Automaten nicht einfach
zu einem Akzeptierzustand machen, weil damit auch alle
Strings akzeptabel w"urden, die zum Startzustand zur"uckf"uhren.
Nehmen wir wieder den Automaten (\ref{ungeradenullen}) f"ur Strings
ungerader L"ange, und verwandeln wir den Startzustand in einen
Akzeptierzustand:
\[
\entrymodifiers={++[o][F]}
\xymatrix{
*+\txt{}\ar[r]
	&*++[o][F=]{}\ar@/^/[r]^{\tt 0}
		&*++[o][F=]{}\ar@/^/[l]^{\tt 0}
}
\]
Da dieser Automat nur Akzeptierzust"ande hat, werden alle W"orter
akzeptiert.

Beide Schwierigkeiten k"onnen umgangen werden, wenn man einen
neuen Startzustand hinzuf"ugt, der auch Akzeptierzustand
ist, und die $\varepsilon$-"Uberg\"ange nicht zur"uck zum
urspr"unglichen Startzustand, sondern zum neuen Startzustand
f"uhrt:
\[
\entrymodifiers={++[o][F]}
\xymatrix{
*+\txt{}\ar[r]
	&*++[o][F=]{}\ar[r]^{\varepsilon}
		&{}\ar@/^/[r]^{\tt 0}
			&{}\ar@/^/[l]^{\tt 0} \ar@/^20pt/[ll]^{\varepsilon}
}
\]

Die allgemeine Konstruktion geht daher aus von einem Automaten $A$:
\begin{center}
\includegraphics[width=0.3\hsize]{images/nea-3}
\end{center}
Die Sprache $L^*$ wird dann akzeptiert von
\begin{center}
\includegraphics[width=0.5\hsize]{images/nea-4}
\end{center}

Formal kann man das Resultat wie folgt zusammenfassen:
\begin{satz}
\index{*-Operation@$*$-Operation}
\label{satz_star}
Ist $L$ eine regul"are Sprache, dann ist auch $L^*$ regul"ar.
\end{satz}

Zum Beweis reicht es nicht, den Satz \ref{satz_concat} zu verwenden
um zu zeigen, dass $L^n$ regul"ar ist, und dann aus wiederholter
Anwendung Satz \ref{satz_union} zu schliessen zu versuchen,
dass die Vereinigung aller $L^n$ auch sei soll. Da dies eine unendliche
Vereinigung ist, w"urde nach der Konstruktion im Beweis von Satz
\ref{satz_union} ein Automat mit unendlich vielen
Zust"anden entstehen, also kein NEA.

\begin{proof}[Beweis]
Aus dem Automaten
\[
A=(Q,\Sigma, \delta,q_0,F)
\]
mit $L(A)=L$ konstruieren wir einen neuen
Automaten
\[
A'=(Q',\Sigma,\delta',q',F')
\]
mit einem neuen Anfangszustand $q'$, der auch
ein Akzeptierzustand ist.
Damit wird sichergestellt, dass der neue Automat das in $L^0$
enthaltene leere Wort akzeptiert.
Zus"atzlich wird der neue Anfangszustand mit einem $\varepsilon$-"Ubergang
mit $q'$ verbunden. 
Ebenso werden alle Akzeptierzust"ande "uber $\varepsilon$-"Uberg"ange
mit dem neuen Startzustand verbunden:
\begin{align*}
Q'&=Q\cup \{q'\}\\
F'&=\{q'\}\\
\delta'(q',a)&=\emptyset\\
\delta'(q',\varepsilon)&= \{q_0\}\\
\delta'(q,a)&= \delta(q,a)\\
\delta'(q,\varepsilon)&=\begin{cases}
\delta(q,\varepsilon)          &\qquad q\not\in F\\
\delta(q,\varepsilon)\cup\{q'\}&\qquad q\in F
\end{cases}
\end{align*}
\end{proof}
\subsubsection{Variante: mindestens eine Wiederholung}
Die Sprache $LL^*$ besteht aus Verkettungen von W"ortern aus $L$,
im Unterschied zu $L^*$ kommt aber mindestens ein Wort vor.
Einen Automaten daf"ur kann man auf verschiedene Arten gewinnen,
sie laufen aber alle im Wesentlichen auf die L"osung
\includegraphics[width=0.5\hsize]{images/nea-4}
hinaus.

\subsubsection{Automat zur Alternative $L_1\cup L_2$}
\index{Alternative}
\index{Vereinigung}
F"ur die Alternative $L_1\cup L_2$ hatten wir bei der Diskussion
des Produkt-Automaten schon einen Automaten gefunden, auch
hierf"ur l"asst sich ein etwas einfacherer Automat im gleichen
Stil wie f"ur Verkettung und *-Operation finden. Ein neuer Startzustand
wird dazu mit $\varepsilon$-"Uberg"angen mit den Startzust"anden
der Automaten $A_1$ und $A_2$ verbunden, wie in Abbildung~\ref{neaalternative}.
\begin{figure}
\begin{center}
\includegraphics[height=0.4\hsize]{images/nea-5}
\qquad
\qquad
\includegraphics[height=0.4\hsize]{images/nea-6}
\end{center}
\caption{Automatenkonstruktion f"ur die Alternative, rechts der Automat
f"ur die Sprache $L(A_1)\cup L(A_2)$\label{neaalternative}}
\end{figure}

\subsubsection{Regul"are Operationen}
Damit haben wir alle drei regul"aren Operationen kennengelernt:

\begin{definition}
Die drei Operationen
\begin{enumerate}
\item Vereinigung $(L_1,L_2)\mapsto L_1\cup L_2$
\item Verkettung $(L_1,L_2)\mapsto L_1L_2$ und
\item Stern-Operation $L\mapsto L^*$
\end{enumerate}
heissen regul"are Operationen.
\end{definition}

Diese Operationen gen"ugen, um alle regul"aren Sprachen aus
einelementigen Teilmengen von $\Sigma$ aufzubauen.
Dazu werden wir im n"achsten Abschnitt eine pr"agnantere Notation,
die regul"aren Ausdr"ucke definieren. Danach werden wir zeigen,
dass sich jeder DEA in einen regul"aren Ausdruck umwandeln l"asst.

\subsection{Regul"are Ausdr"ucke\label{regulaer:regulaere-ausdruecke}}
\index{regul\"are Ausdr\"ucke}
\index{Ausdr\"ucke!regul\"are}
\begin{table}
\begin{center}
\begin{tabular}{|l|l|}
\hline
Ausdruck $r$&Bedeutung\\
\hline
{\tt a}&steht f"ur das Zeichen ${\tt a}\in \Sigma$\\
{\tt .}&steht f"ur ein beliebiges Zeichen aus $\Sigma$\\
{\tt [aeiou]}&steht f"ur ein Zeichen aus $\{{\tt a},{\tt e},{\tt i},{\tt o},{\tt u}\}\subset \Sigma$\\
{\tt [1-9]}&steht f"ur die positiven Ziffern\\
$\varepsilon$&steht f"ur das leere Wort\\
$\emptyset$&steht f"ur die leere Sprache\\
\hline
\end{tabular}
\end{center}
\index{regul\"are Ausdr\"ucke!primitive}
\index{Ausdr\"ucke!regul\"are!primitive}
\caption{Primitive regul"are Ausdr"ucke.\label{regtab1}}
\end{table}
Regul"are Ausdr"ucke sind Zeichenketten, die (regul"are) Sprachen
beschreiben. 
Die einfachsten regul"aren Sprachen sind die Sprachen mit W"ortern,
die nicht l"anger als 1 Zeichen sind. Regul"are Ausdr"ucke f"ur
diese primitiven Sprachen werden in Tabelle~\ref{regtab1} zusammengestellt.

Regul"are Ausdr"ucke bauen regul"are Sprachen aus einzelnen
Zeichen und regul"aren Operationen aus.
Ist $r$ ein regul"arer Ausdruck, dann schreiben wir  $L(r)$ f"ur die
Sprache, die vom regul"aren Ausdruck $r$ beschrieben wird.
F"ur die regul"aren Operation verwenden wir die Notation
gem"ass Tabelle~\ref{regtab2}
\begin{table}
\begin{center}
\begin{tabular}{|l|c|l|}
\hline
&Ausdruck&regul"are Operation\\
\hline
\index{Verkettung}
Verkettung&$r_1r_2$&$L(r_1)L(r_2)$\\
\index{Alternative}
Alternative&$r_1{\tt |}r_2$&$L(r_1)\cup L(r_2)$\\
\index{*-Operation@$*$-Operation}
Stern-Operation&$r{\mathstrut^{\tt *}}$&$L(r)^*$\\
\hline
\end{tabular}
\end{center}
\caption{Notation f"ur regul"are Operationen\label{regtab2}}
\end{table}
Falls n"otig k"onnen Klammern verwendet werden, um anzuzeigen,
welche Gruppen verkn"upft werden. Da mit regul"aren Operationen
verkn"upfte Sprachen wieder regul"ar sind, k"onnen alle mit
regul"aren Ausdr"ucken beschriebenen Sprachen von einem DEA
akzeptiert werden.

\subsubsection{Automaten f"ur die primitiven regul"aren Ausdr"ucke}
Die primitiven regul"aren Ausdr"ucke aus Tabelle~\ref{regtab1} k"onnen
von folgenden Automaten akzeptiert werden:
\begin{enumerate}
\item Ein einzelnes Zeichen {\tt a}:
\[
\entrymodifiers={++[o][F]}
\xymatrix{
*+\txt{}\ar[r]
	&{}\ar[r]^{\tt a}
		&*++[o][F=]{}
}
\]
\item Sprache $L=\Sigma$, regul"arer Ausdruck $r={\tt .}$:
\[
\entrymodifiers={++[o][F]}
\xymatrix{
*+\txt{}\ar[r]
	&{}\ar[r]^{\Sigma}
		&*++[o][F=]{}
}
\]
\item Das leere Wort:
\[
\entrymodifiers={++[o][F]}
\xymatrix{
*+\txt{}\ar[r]
	&*++[o][F=]{}
}
\]
\item Die leere Sprache:
\[
\entrymodifiers={++[o][F]}
\xymatrix{
*+\txt{}\ar[r]
	&{}
}
\]
\end{enumerate}

\subsubsection{Automat eines regul"aren Ausdrucks}
Mit Hilfe der Automaten f"ur die primitven regul"aren Ausdr"ucke
und den regul"aren Operationen kann man jetzt zu jedem regul"aren
Ausdruck einen NEA aufbauen.

\begin{beispiel}[\bf Beispiel 1] NEA des regul"aren Ausdrucks
${\tt ab}|{\tt cd}$.

Zun"achst brauchen wir einen NEA f"ur die Verkettung ${\tt ab}$.
Diesen k"onnen wir mit der Verkettungskonstruktion aus primitiven
NEAs f"ur {\tt a} und {\tt b} bilden:
\[
\entrymodifiers={++[o][F]}
\xymatrix{
*+\txt{}\ar[r]
	&{}\ar[r]^{\tt a}
		&{}\ar[r]^{\varepsilon}
			&{}\ar[r]^{\tt b}
				&*++[o][F=]{}
}
\]
Auf die gleiche Art kann auch ein NEA f"ur ${\tt cd}$ gebildet werden.
Diese m"ussen jetzt mit Hilfe der Konstruktion f"ur die Alternative
zu einem NEA f"ur den Gesamtausdruck zusammengesetzt werden:
\[
\entrymodifiers={++[o][F]}
\xymatrix{
*+\txt{}
	&*+\txt{}
		&{}\ar[r]^{\tt a}
			&{}\ar[r]^{\varepsilon}
				&{}\ar[r]^{\tt b}
					&*++[o][F=]{}
\\
*+\txt{} \ar[r]
	&{}\ar[ur]^{\varepsilon} \ar[dr]^{\varepsilon}
\\
*+\txt{}
	&*+\txt{}
		&{}\ar[r]^{\tt c}
			&{}\ar[r]^{\varepsilon}
				&{}\ar[r]^{\tt d}
					&*++[o][F=]{}
}
\]
\end{beispiel}

\begin{beispiel}[\bf Beispiel 2] NEA f"ur den regul"aren Ausdruck
$({\tt ab})^*{\tt c}$. 

Die einzelnen Teile sind uns schon bekannt, wir m"ussen nur noch
die *-Konstruktion auf den verketten Automaten anwenden:
\[
\entrymodifiers={++[o][F]}
\xymatrix{
*+\txt{}\ar[r]
	&{}\ar[r]^{\varepsilon}
		&{}\ar[r]^{\varepsilon}
			\ar[d]^{\varepsilon}
			&\ar[r]_{\tt a}
				&{}\ar[r]^{\varepsilon}
					&{}\ar[r]_{\tt b}
						&{}\ar@/_20pt/[llll]_{\varepsilon}
\\
*+\txt{}
	&*+\txt{}
		&{}\ar[r]^{\tt c}
			&*++[o][F=]{}
}
\]
\end{beispiel}

\begin{beispiel}[\bf Beispiel 3:] NEA f"ur den regul"aren Ausdruck
$({\tt ab})+$.

Das Zeichen ${\tt+}$ bedeutet bei regul"aren Ausdr"ucken ``mindestens ein''.
Die einfachste Modifikation der *-Konstruktion, die dies leistet, besteht
darin, den Akzeptierzustand ans Ende des Automaten zu verschieben:
\[
\entrymodifiers={++[o][F]}
\xymatrix{
*+\txt{}\ar[r]
	&{}\ar[r]^{\varepsilon}
		&{}\ar[r]^{\tt a}
			&{}\ar[r]^{\varepsilon}
				&{}\ar[r]^{\tt b}
					&*++[o][F=]{}\ar@/^20pt/[llll]^{\varepsilon}
}
\]
Alternativ k"onnte man auch einfach den NEA f"ur die Verkettung
nehmen, und keinen neuen Startzustand hinzuf"ugen. Bei der
*-Konstruktion wurde das ja gemacht, um auch das leere Wort
akzeptieren zu k"onnen, was bei der {\tt +}-Operation nicht
n"otig ist:
\[
\entrymodifiers={++[o][F]}
\xymatrix{
*+\txt{}\ar[r]
	&{}\ar[r]^{\tt a}
		&{}\ar[r]^{\varepsilon}
			&{}\ar[r]^{\tt b}
				&*++[o][F=]{}\ar@/^20pt/[lll]^{\varepsilon}
}
\]
\end{beispiel}
Diese Beispiel illustrieren, dass sich aus jedem regul"aren Ausdruck
ein NEA bauen l"asst. Die Konstruktion hat die Form eines Algorithmus,
l"asst sich also auch mit einem Computer implementieren, sie geht
auf Ken Thompson zur"uck.
\index{Thompson, Ken}

\subsection{Regul"arer Ausdruck eines DEA\label{regulaer:dea-re}}
Zu jedem DEA gibt es einen regul"aren Ausdruck, der angibt,
welche W"orter er akzeptiert.  In diesem Abschnitt geben
wir einen Algorithmus an, der einen DEA in einen "aquivalenten
regul"aren Ausdruck umwandelt. 

\index{NEA!verallgemeinerter}
\index{VNEA}
Wir verwenden dazu den Begriff eines verallgemeinerten NEA (VNEA),
dessen Pfeile nicht mehr nur mit Zeichen des Alphabets oder mit
$\varepsilon$ angeschrieben sei k"onnen, sondern mit regul"aren
Ausdr"ucken. Dann wandelt man einen DEA zun"achst um in einen
VNEA, und reduziert ihn dann auf einen VNEA mit nur zwei Zust"anden,
einem Startzustand und einem Akzeptierzustand:
\[
\entrymodifiers={++[o][F]}
\xymatrix{
*+\txt{}\ar[r]
	&{q_0}\ar[r]^{r}
		&*++[o][F=]{q_1}
}
\]
Offenbar werden genau diejenigen W"orter von diesem Automaten
akzeptiert, welche auf den regul"aren Ausdruck $r$ passen.

Sei jetzt also ein NEA $A$ gegeben. Da er m"oglicherweise viele
Endzust"ande haben kann, wir aber am Schluss nur noch einen
Endzustand haben wollen, f"ugen wir einen neuen Endzustand
$q_{\text{accept}}$ hinzu.
Da der Automat auch Pfeile haben kann, die im Startzustand enden,
der endg"ultige VNEA aber keine solche Pfeile hat, f"ugen wir
auch einen zus"atzlichen Startzustand $q_{\text{start}}$ hinzu.

%Nat"urlich m"ussen wir jetzt auch noch Pfeile erg"anzen. Zus"atzlich
%zu den urspr"unglichen Pfeilen des NEA erg"anzen wir darin
%alle m"oglichen Pfeile, die es bis jetzt noch nicht gab, und
%beschriften Sie mit dem regul"aren Ausdruck $\emptyset$, was bedeutet,
%dass diese Pfeile gar nie ``genommen'' werden.

Ausserdem f"ugen wir vom neuen Startzustand $q_{\text{start}}$
aus einen Pfeil
Pfeil zu urspr"unglichen Startzustand des NEA mit Beschriftung $\varepsilon$
hinzu.

Alle Akzeptierzust"ande des urspr"unglichen NEA werden jetzt noch mit
$q_{\text{accept}}$ verbunden, beschriftet mit $\varepsilon$.

Damit haben wir jetzt einen VNEA. Diesen
m"ussen wir jetzt auf einen VNEA reduzieren, der nur noch
die beiden Zust"ande $q_{\text{start}}$ und $q_{\text{accept}}$
enth"alt. Dazu reissen wir nacheinander alle Zust"ande des urspr"unglichen
NEA heraus.
Die entstehenden L"ocher 
m"ussen wir nat"urlich reparieren,
indem wir die verbleibenden Pfeile mit erweiterten regul"aren Ausdr"ucken
anschreiben, welche die verlorengegangenen Teile ersetzen.

Nehmen wir also an, wir m"ochten den Zustand $q_ {\text{rip}}$
herausreissen. Seien weiter $q_i$ und $q_j$ zwei Zust"ande, die
"Uberg"ange haben, die "uber $q_{\text{rip}}$ f"uhren:
\[
\entrymodifiers={++[o][F]}
\xymatrix{
{q_i}\ar[rr]^{r_4} \ar[dr]_{r_1}
	&*+\txt{}
		&{q_j}
\\
*+\txt{}
	&{q_{\text{rip}}}\ar[ur]_{r_3} \ar@(dl,dr)_{r_2}
}
\]
Der Weg "uber $q_{\text{rip}}$ entspricht einem "Ubergang
von $q_i$ nach $q_j$ mit regul"arem Ausdruck $r_1r_2^*r_3$,
der zus"atzlich zum Ausdruck $r_4$, m"oglich ist. Wenn $q_{\text{rip}}$
entfernt wird, muss also $r_4$ durch 
\[
r_4{\tt |}r_1r_2^{\tt *}r_3
\]
ersetzt werden:
\[
\entrymodifiers={++[o][F]}
\xymatrix{
{q_i}\ar[rr]^{r_4|r_1r_2^*r_3}
	&*+\txt{}
		&{q_j}
}
\]
Die vom VNEA akzeptierte Sprache "andert sich dabei nicht.
Diese Operation wird wiederholt, bis die Zust"ande des urspr"unglichen
NEA vollst"andig entfernt sind.

Damit  haben wir jetzt folgenden Satz bewiesen.
\begin{satz}
Ist $A$ ein NEA, dann gibt es einen regul"aren Ausdruck $r$ mit
$L(A)=L(r)$.
\end{satz}

\subsubsection{Beispiel}
Wir wollen den DEA
\[
\entrymodifiers={++[o][F]}
\xymatrix{
*+\txt{}\ar[r]
	&{1}\ar@/^/[r]^{\tt a} \ar@/^/[d]^{\tt b}
		&*++[o][F=]{2} \ar@(ul,ur)^{\tt b} \ar@/^/[l]^{\tt a}
\\
*+\txt{}
	&*++[o][F=]{3} \ar[ur]_{\tt a} \ar@/^/[u]^{\tt b}
}
\]
in einen "aquivalenten regul"aren Ausdruck umwandeln.

Im ersten Schritt f"ugen wir die neuen Start- und Akzeptierzust"ande 
hinzu, die Zust"ande $2$ und $3$ sind damit nicht mehr Akzeptierzust"ande:
\[
\entrymodifiers={++[o][F]}
\xymatrix{
*+\txt{}\ar[r]
	&S\ar[r]^{\varepsilon}
		&{1}\ar@/^/[r]^{\tt a} \ar@/^/[d]^{\tt b}
			&{2} \ar@(ul,ur)^{\tt b} \ar@/^/[l]^{\tt a}\ar[r]^{\varepsilon}
				&*++[o][F=]{A}
\\
*+\txt{}
	&*+\txt{}
		&{3} \ar[ur]_{\tt a} \ar@/^/[u]^{\tt b} \ar[urr]^{\varepsilon}
}
\]
Jetzt entfernen wir nacheinander alle ``inneren'' Zust"ande. Wir beginnen mit
dem Zustand $1$, dabei sind die Pfade
$S$-$1$-$2$,
$S$-$1$-$3$,
$2$-$1$-$3$,
$3$-$1$-$2$,
$2$-$1$-$2$
und $3$-$1$-$2$
zu ber"ucksichtigen
\[
\entrymodifiers={++[o][F]}
\xymatrix{
*+\txt{}\ar[r]
	&S\ar[rr]^{\tt a} \ar[dr]_{\tt b}
		&*+\txt{}
			&{2} \ar@(ul,ur)^{\tt aa|b} \ar@/_/[dl]_{\tt ab} \ar[r]^{\varepsilon}
				&*++[o][F=]{A}
\\
*+\txt{}
	&*+\txt{}
		&{3} \ar@/_/[ur]_{\tt a|ba} \ar@/_20pt/[urr]_{\varepsilon} \ar@(dl,dr)_{\tt bb}
}
\]
Jetzt wird der Zustand $2$ entfernt, dabei sind die Pfade 
$S$-$2$-$A$,
$S$-$2$-$3$,
$3$-$2$-$A$,
und
$3$-$2$-$3$
zu ber"ucksichtigen:
\[
\entrymodifiers={++[o][F]}
\xymatrix{
*+\txt{}\ar[r]
	&S\ar[rrr]^{\tt a(aa|b)^*} \ar[dr]_{\tt b|a(aa|b)^*ab}
		&*+\txt{}
			&*+\txt{}
				&*++[o][F=]{A}
\\
*+\txt{}
	&*+\txt{}
		&{3} \ar[urr]_{{\tt (a|ba)(aa|b)^*}|\varepsilon} \ar@(dl,dr)_{{\tt bb|(a|ba)(aa|b)^*ab}}
}
\]
Jetzt muss nur noch der Zustand $3$ entfernt werden:
\[
\entrymodifiers={++[o][F]}
\xymatrix{
*+\txt{}\ar[r]
	&S\ar[rrrrrrrr]^{\tt (a(aa|b)^*)|(b|a(aa|b)^*ab)(bb|(a|ba)(aa|b)^*ab)^*(((a|ba)(aa|b)^*)|\varepsilon)}
		&*+\txt{}
			&*+\txt{}
			&*+\txt{}
			&*+\txt{}
			&*+\txt{}
			&*+\txt{}
			&*+\txt{}
				&*++[o][F=]{A}
}
\]
Damit ist der regul"are Ausdruck gefunden, der die gleiche Sprache
akzeptiert wie der urspr"ungliche DEA.

\section{Anhang: Regul"are Ausdr"ucke in der Praxis\label{regulaer:praxis}}
\rhead{Regex in der Praxis}
\subsection{Flex}
\index{Flex}
\index{Scanner}
\index{Scannergenerator}
Flex ist ein Scannergenerator. Er verarbeitet eine Spezifikation einer
Sprache als Menge von regul"aren Ausdr"ucken in einen deterministischen 
endlichen Automaten. Als Beispiel soll verfolgt werden, wie die
Flex-Spezifikation
\verbatiminput{lex/example4.l}
in einen endlichen Automaten umgesetzt wird. Mit der Option {\tt -T}
kann man sich die erzeugten Tabellen des NEA anzeigen lassen:
\verbatiminput{lex/dumpnfa}
Die Tabelle ist wie folgt zu lesen.
In der mittleren Spalte steht das zu verarbeitende Zeichen, also
die Beschriftung eines Pfeils. F"ur jedes Zeichen sind zwei Zielzust"ande
m"oglich, die in den letzten beiden Spalten eingetragen werden. Einen
Zustand {\tt 0} gibt es nicht, "Uberg"ange nach ${\tt 0}$ m"ussen als
nicht verwendete Pfeile interpretiert werden. $\varepsilon$-"Uberg"ange
werden durch mit dem Symbol {\tt 257} bezeichnet. Sind von einem Punkt
aus mehr also zwei Alternativen m"oglich, wie im vorliegenden Fall
bei der Alternative {\tt h|s|r}, dann muss dies mit Hilfe von zus"atzlichen
$\varepsilon$-"Ubergangen aus Zweier-Alternativen zusammengesetzt werden.
Zeichnet man diese NEA, ergibt sich das folgende Zustandsdiagramm:
\[
\entrymodifiers={++[o][F]}
\xymatrix{
*+\txt{}\ar[r]
	&{15} \ar[r]^{\varepsilon} \ar[d]^{\varepsilon}
		&{13}\ar[r]^{{\tt EOF}}
			&*++[o][F=]{14}
\\
*+\txt{}
	&{12}\ar[d]^{\varepsilon} \ar[r]^{\varepsilon}
		&{1}
\\
*+\txt{}
	&{8}\ar[d]^{\varepsilon} \ar[r]^{\varepsilon}
		&{7}\ar[dr]^{\tt r}
\\
*+\txt{}
	&{5}\ar[dr]^{\varepsilon} \ar[r]^{\varepsilon}
		&{3}\ar[r]^{\tt h}
			&{6}\ar[r]^{\varepsilon}
				&{9}\ar[r]^{\tt 1}
					&{10}\ar[r]^{\tt 0}
						&*++[o][F=]{11}
\\
*+\txt{}
	&*+\txt{}
		&{4}\ar[ur]^{\tt s}
\\
*+\txt{}
	&{2}
}
\]
Man kan im unteren Teil sehr sch"on erkennen, wie der Automat, der
die Dreier-Altenrative {\tt h|s|r} akzeptiert, "uber einen
$\varepsilon$-"Ubergang mit dem Automaten verkettet wird, der 
den String {\tt 10} akzeptiert. Zustand {\tt 14} wurde von Flex
hinzugef"ugt, er erlaubt das Fileende zu erkennen, so dass das
erzeugte Programm am Fileende auf jeden Fall terminieren kann.

Der NEA kann nat"urlich auch in einen DEA umgewandelt werden.
Auch dies protokolliert Flex: 
\verbatiminput{lex/dumpdfa}
Zu jedem Zustand wird jetzt aufgef"uhrt, welche Zeichen zu welchen
"Uberg"angen f"uhren. Flex hat die Zeichen in Klassen zusammengefasst,
die Ziffern bedeuten:
\begin{center}
\begin{tabular}{|c|c|}
\hline
Zeichenklasse&Inhalt\\
\hline
{\tt 1}&andere\\
{\tt 2}&{\tt 0}\\
{\tt 3}&{\tt 1}\\
{\tt 4}&{\tt h}\\
{\tt 5}&{\tt s}\\
{\tt 6}&{\tt r}\\
\hline
\end{tabular}
\end{center}
Auch den DEA kann man in ein Zustandsdiagramm zur"uck"ubersetzen:
\[
\entrymodifiers={++[o][F]}
\xymatrix{
*+\txt{}
	&*+\txt{}\ar[d]
\\
*+\txt{}
	&{1}\ar[dr]^{{\tt h},{\tt s},{\tt r}} \ar[dl]_{{\tt 0}, {\tt 1}, \text{andere}}
\\
*++[o][F=]{4}
	&*+\txt{}
		&*++[o][F=]{5}\ar[r]^{\tt 1}
			&{6}\ar[r]^{\tt 0}
				&*++[o][F=]{7}
\\
*+\txt{}
	&{2}\ar[ur]_{{\tt h},{\tt s},{\tt r}} \ar[ul]^{{\tt 0}, {\tt 1}, \text{andere}}
\\
*+\txt{}
	&*++[o][F=]{3}
}
\]
Man kann sehen, dass Flex nicht alles wirklich aufr"aumt, es bleiben
unerreichbare Zust"ande {\tt 2} und {\tt 3} stehen, einer davon sogar ein
Akzeptierzustand. Die Zust"ande 4 und 5 sind als
Akzeptierzust"ande markiert, aber nur 7 ist als Akzeptierzustand zu
verstehen, der ein auf den regul"aren Ausdruck passendes Wort
akzeptiert.

\subsection{Performance}
\index{Laufzeit!eines DEA}
\index{Laufzeit!eines NEA}
Ein DEA hat immer Laufzeit $O(n)$, d.\,h.~ein Regex-Matcher auf der
Basis eines DEA wird innerhalb einer Zeit proportional zur
L"ange des Inputstrings erkennen, ob ein Inputstring passt.
Die einfachste Implementation eines NEA hingegen wird alle M"oglichkeiten
von nicht eindeutigen "Uberg"angen durchprobieren m"ussen, mit
m"oglicherweise exponentieller Laufzeit. Normalerweise sind in der
Praxis eingesetzte regul"are Ausdr"ucke klein, und die untersuchten
Strings sind ebenfalls nicht allzu gross. Man kann allerdings auch
Ausdr"ucke konstruieren, die den Unterschied zwischen DEA und simplistischer
NEA-Implementation offensichtlich werden lassen. Ein solcher Ausdruck
ist
\begin{center}
\tt a?a?a?a?a?aaaaa
\end{center}
Also $n$ fakultative {\tt a} gefolgt von $n$ obligatorischen {\tt a}.
Um ein Wort bestehend aus $n$ {\tt a} zu akzeptieren, gibt es genau eine
M"oglichkeit, die man erh"alt, wenn man in der Alternativen
${\tt a?} = \varepsilon|{\tt a}$
jeweils das leere Wort w"ahlt. Falls das immer die zweite ausprobierte 
M"oglichkeit ist, ist die Laufzeit dieses Algorithmus proportional zu
$2^n$, wird also bei langen Strings schnell sehr gross.

Es ist einfach, einen geeigneten DEA f"ur den regul"aren
Ausdruck {\tt a?a?a?aaa} hinzuschreiben:
\[
\entrymodifiers={++[o][F]}
\xymatrix{
*+\txt{}\ar[r]
	&\ar[r]^{\tt a}
		&\ar[r]^{\tt a}
			&\ar[r]^{\tt a}
				&*++[o][F=]{}\ar[r]^{\tt a}
					&*++[o][F=]{}\ar[r]^{\tt a}
						&*++[o][F=]{}\ar[r]^{\tt a}
							&*++[o][F=]{}
}
\]
Und auch einen NEA findet man aus der Konstruktion eines Automaten
aus einem regul"aren Ausdruck ohne Probleme:
\[
\entrymodifiers={++[o][F]}
\xymatrix{
*+\txt{}\ar[r]
	&\ar@/^/[r]^{\tt a} \ar@/_/[r]_{\varepsilon}
		&\ar@/^/[r]^{\tt a} \ar@/_/[r]_{\varepsilon}
			&\ar@/^/[r]^{\tt a} \ar@/_/[r]_{\varepsilon}
				&\ar[r]^{\tt a}
					&\ar[r]^{\tt a}
						&\ar[r]^{\tt a}
							&*++[o][F=]{}
}
\]
Auch die Vereinfachung der $\varepsilon$-"Uberg"ange ist noch sehr
"ubersichtlich:
\[
\entrymodifiers={++[o][F]}
\xymatrix{
*+\txt{}\ar[r]
	&\ar[r]^{\tt a}
	\ar@/^20pt/[rr]^{\tt a}
	\ar@/^20pt/[rrr]^{\tt a}
	\ar@/^20pt/[rrrr]^{\tt a}
		&\ar[r]^{\tt a}
		\ar@/_20pt/[rr]_{\tt a}
		\ar@/_20pt/[rrr]_{\tt a}
			&\ar[r]^{\tt a} \ar@/_20pt/[rr]_{\tt a}
				&\ar[r]^{\tt a}
					&\ar[r]^{\tt a}
						&\ar[r]^{\tt a}
							&*++[o][F=]{}
}
\]
Auch in diesem Beispiel kann man sich von Flex helfen lassen,
einen NEA zu erstellen.  Unverkennbar sind auch hier wieder die
Spuren der Konstruktionsschritte, mit denen man aus regul"aren
Ausdr"ucken Teilautomaten baut und diese dann zum ganzen Automaten
zusammenbaut.
\[
\entrymodifiers={++[o][F]}
\xymatrix{
*+\txt{}\ar[r]
	&{19}\ar[d]^{\varepsilon}\ar[r]^{\varepsilon}
		&{17}\ar[r]^{\tt EOF}
			&*++[o][F=]{18}
\\
*+\txt{}
	&{16}\ar[d]^{\varepsilon} \ar[r]^{\varepsilon}
		&{1}
\\
*+\txt{}
	&{5} \ar[r]^{\varepsilon}
	      \ar[dr]^{\varepsilon}
		&{3} \ar[d]^{\tt a}
			&{8} \ar[r]^{\varepsilon}
			     \ar[dr]^{\varepsilon}
				&{6} \ar[d]^{\tt a}
					&{11} \ar[r]^{\varepsilon}
					     \ar[dr]^{\varepsilon}
						&{9} \ar[d]^{\tt a}
\\
*+\txt{}
	&*+\txt{}
		&{4} \ar[ur]^{\varepsilon}
			&*+\txt{}
				&{7} \ar[ur]^{\varepsilon}
					&*+\txt{}
						&{10} \ar@/_10pt/[dlll]_{\varepsilon}
\\
*+\txt{}
	&*+\txt{}
		&*+\txt{}
			&{12} \ar[r]^{\tt a}
				&{13} \ar[r]^{\tt a}
					&{14} \ar[r]^{\tt a}
						&*++[o][F=]{15}
}
\]
Daraus macht Flex dann den folgenden DEA:
\[
\entrymodifiers={++[o][F]}
\xymatrix{
*+\txt{}\ar[d]
\\
{1}\ar[r]^{\tt a} \ar[d]^{\text{andere}} \
	&{5}\ar[r]^{\tt a}
		&{6}\ar[r]^{\tt a}
			&*++[o][F=]{7} \ar[r]^{\tt a}
				&*++[o][F=]{8} \ar[r]^{\tt a}
					&*++[o][F=]{9} \ar[r]^{\tt a}
						&*++[o][F=]{10}
\\
{4}
\\
{3}
}
\]
$3$ und $4$ sind wieder Zust"ande die ben"otigt werden, um andere Inputs
wie zum Beispiel EOF zu erkennen. Der resultierend DEA entspricht genau
dem oben vorgeschlagenen DEA.

\begin{sloppypar} % To prevent "java.lang.string" from overflowing
Nicht alle Implementationen verwenden jedoch diese Theorie. Die
Regex-Funktionen in der C-Library verwenden dies, und sind entsprechend
schnell. Alle Unix-Tools, die auf dieser Bibliothek basieren, sind
entsprechend schnell. Perl hat seine eigene erweiterte Regex-Library,
verwendet keinen DEA, und ist langsam. Ebenfalls langsam ist die
die Implementation der {\tt matches} Methode in {\tt java.lang.String}.
Man w"urde hoffen, dass wenigstens das Package {\tt java.util.regex} einen
besseren Regex-Matcher beinhalten w"urde. Leider wurde da nur am
Interface herumgebastelt, nicht an der Substanz. Vielleicht sollten
die Implementatoren von {\tt java.util.regex} mal die Vorlesung AutoSpr
besuchen.
\end{sloppypar}

\section{Zusammenfassung: Das Wichtigste in K"urze}
\begin{enumerate}
\item Zu jedem DEA gibt es einen
minimalen Automaten, der zum Beispiel dazu verwendet werden kann,
DEA zu vergleichen
(\ref{regulaer:minimalautomat}).
\item Eine Sprache ist regul"ar, wenn sie von einem endlichen Automaten
akzeptiert wird (\ref{regulaer:definition:regulaere-sprache}).
\item Selbst wenn man von einer Sprache nur weiss, dass sie regul"ar
ist, kann man einen DEA finden, der die Sprache akzeptiert
(Satz \ref{satz_dea_aus_sprache}).
\item Jeder NEA kann in einen DEA umgewandelt werden (\ref{regulaer:nea-dea}).
\item Regul"are Sprachen erf"ullen das Pumping Lemma
(\ref{regulaer:pumpinglemma}).
Eine Sprache ist nicht regul"ar, wenn sie ein Wort enth"alt, welches
nicht aufgepumpt werden kann: wie auch immer man das Wort
auf eine vom Pumping Lemma zerlegt, mindestens eine aufgepumptes
oder abgepumptes Wort ist nicht mehr in der Sprache drin.
\item Die regul"aren Operationen Vereinigung, Verkettung und $*$-Operation
erzeugen aus regul"aren Sprachen neue regul"are Sprachen
(\ref{regulaer:regulaere-operationen}).
\item Zu jedem endlichen Automaten kann man einen regul"aren Ausdruck
finden, der die gleiche Sprache akzeptiert wie der Automat.
(\ref{regulaer:dea-re}).
\end{enumerate}

%
% einleitung.tex
%
% (c) 2009 Prof Dr Andreas Mueller, Hochschule Rapperswil
%
\lhead{Einleitung}
\rhead{}
\chapter*{Worum geht es?\label{chapter-intro}}
Urspr"unglich wurden Computer dazu entwickelt, aufwendige
Berechnungen zu automatisieren. Inzwischen wird nur noch ein
ganz kleiner Teil der weltweit installierten Rechenleistung f"ur
diese Art von Problemen genutzt. 
Die heutige Informatik befasst sich vorwiegend mit der maschinellen
Verarbeitung von Zeichenfolgen. 

W"ahrend fr"uher noch die Frage nach der effizientesten L"osung von
Berechnungsaufgaben mit dem Computer, also die Numerik, im Vordergrund stand,  
ist heute die L"osung komplizierterer und oft nicht numerischer Probleme 
gefragt. Zum Beispiel ist der k"urzeste Weg zu finden, den ein Spediteur
zur Auslieferung aller ihm anvertrauten Sendungen verwenden kann, eine
Aufgabe, in der die Numerik offensichtlich eine untergeordnete Rolle spielt.

Trotzdem bleiben eine Reihe von Fragen, die grunds"atzlich mathematischer
Art sind:
\begin{enumerate}
\item Gibt es Probleme, die mit dem Computer grunds"atzlich nicht
l"osbar sind?
\item K"onnen wir dar"uber eine Aussage machen, wie schnell ein Computer
ein bestimmtes Problem l"osen kann? Gibt es eine maximale Geschwindigkeit,
die nicht zu "ubertreffen ist?
\item Welchen Einfluss hat die Architektur eines Rechners auf die Klasse
von Problemen, die damit l"osbar sind?
\item Gibt es Probleme, die sich besonders schnell l"osen lassen?
\end{enumerate}
Die im vorliegenden Skript behandelte mathematische Theorie befasst sich
genau mit den logischen Grundlagen dieses Prozesses.
Sie definiert, welche Mengen von Zeichenfolgen "uberhaupt untersucht
werden sollen, und nennt sie {\em Sprachen}.
Sie studiert, welche abstrakten Typen von
Rechnern welche Arten von Zeichenfolgen erkennen oder hervorbringen
k"onnen. 

Es geht weniger darum, Methoden oder Algorithmen zur Verarbeitung
von Zeichenfolgen zu behandeln.
Vielmehr sollen mathematische S"atze aufgestellt werden, welche
Arten von Sprachen mit welcher Art von Rechnermodell bearbeitet
werden k"onnen.
Zum Beispiel werden wir Resultaten wie die folgenden finden:
\begin{enumerate}
\item Ein Zustandsautomat kann die korrekte Schachtelung von
Klammern nicht erkennen.
\item Mit regul"aren Ausdr"ucken kann man korrekte arithmetische
Ausdr"ucke nicht erkennen.
\item Um korrekte arithmetische Ausdr"ucke zu erkennen, braucht man als
Speicher mindestens einen Stack.
\item Es gibt kein Programm, das ein beliebiges anderes Programm
analysieren kann und mit Sicherheit herausfinden kann, ob es je anhalten
wird.
\item Es gibt wahrscheinlich keinen Algorithmus mit polynomialer Laufzeit,
der entscheidet, ob es einen fehlerfreien Stundenplan gibt (d.\,h.~alle 
Studenten k"onnen alle Veranstaltungen besuchen, f"ur die sie sich
angemeldet haben, es gibt keine "Uberschneidungen).
\item Fast alle reellen Zahlen k"onnen gar nicht berechnet werden.
\end{enumerate}
Die zu erwartenden Resultate sind also nicht von der Art von Kochrezepten:
``Um $X$ zu erhalten, mache man erst $Y$, dann $Z$ und zum Schluss $V$''.
Die Fragestellungen drehen sich alle um die fundamentale Frage:
\begin{quote}
Welche grunds"atzlichen M"oglichkeiten und Einschr"ankungen haben
Computer?
\end{quote}
Die Aussagen sind also absolute Wahrheiten, die unabh"angig vom Stand
der Com\-pu\-ter-Technik oder der Programmier-Kunst gelten.
Solche Aussagen sind nur dann glaub\-w"urdig, wenn man sich von deren
Richtigkeit mit Hilfe eines mathematischen Beweises jederzeit "uberzeugen
kann. In dieser Vorlesung geht es also vor allem auch um Beweise, 
warum eine Sprache eine Eigenschaft hat, oder ob eine Eigenschaft
unter bestimmten Voraussetzungen "uberhaupt vorhanden sein kann.
Daher werden im ersten Kapitel kurz einige Beweistechniken rekapituliert.

Problem 5 in obiger Aufz"ahlung k"ummert sich nicht darum, ob eine
Aufgabe l"osbar ist, sondern darum, wie effizient sie l"osbar ist.
Dabei geht es nicht darum, wer den schnellsten Algorithmus findet,
sondern um die grunds"atzliche Frage, ob es einen Algorithmus mit
einer gewissen Laufzeitkomplexit"at "uberhaupt geben kann.
Wiederum geht es um eine absolute mathematische Aussage, welche
einen Beweis erfordert. 

In dieser Vorlesung wird nichts berechnet.
In anderen
Mathematikvorlesungen werden zum Beispiel Techniken zur Berechnung
von L"osungen von
Gleichungen, zur Fl"achenberechnung oder zur Modellierung eines 
physikalischen Prozesses behandelt werden.
Dazu wird ein Kalk"ul mit einer eigenen Formelsprache entwickelt,
mit dem diese Berechnungen durchgef"uhrt werden k"onnen.
In dieser Vorlesung werden kaum Formeln vorkommen. 
Und wenn Formeln vorkommen, dann werden es logische Formeln sein,
nicht algebraische.

Nachdem jetzt gekl"art ist, was in dieser Vorlesung nicht behandelt
wird, hier einige Hinweise, was in den kommenden Kapiteln untersucht
wird:
\begin{description}
\item[Kapitel \ref{chapter-regular}:] Deterministische endliche Automaten (DEA). Von einem
DEA akzeptierte Sprache. Rekonstruktion des DEA aus der Sprache.
Regul"are Sprachen und DEAs. Regul"are Ausdr"ucke. Nicht deterministische
endliche Automaten (NEA). Wie beweist man, dass eine Sprache nicht regul"ar
ist?

In diesem Kapitel geht es nicht darum, ein Virtuose in der
Anwendung regul"arer Ausdr"ucke zu werden. Viel wichtiger ist, Klarheit dar"uber
zu bekommen, in welchen F"allen regul"are Ausdr"ucke anwendbar sind,
und wo ihre Grenzen sind.

\item[Kapitel \ref{chapter-cfl}:] Stack-Automaten und von einem Stack-Automaten
erzeugte Sprache. Kontextfreie Grammatiken und davon erzeugte Sprachen. 
Normalform einer Grammatik. Komplexit"at des Parse-Problems.
Wiederum geht es nicht darum, besonders gewandt im Umgang mit Grammatiken
zu werden.

Kontextfreie Grammatiken werden zur Beschreibung von Programmiersprachen verwendet.
Trotzdem ist das Ziel dieses Abschnittes nicht Gewandtheit in der Spezifikation
von Programmiersprachen, sondern das Verst"andnis, welche Eigenschaften eine
Sprache haben muss, damit sie mit Hilfe einer Grammatik beschrieben werden kann.
In einem Beispiel wird auch gezeigt, wie man aus einer Grammatik mit geeigneten
Software-Werkzeugen direkt einen Parser f"ur die von der Grammatik beschriebene
Sprache generieren kann. Wieder ist nicht die Beherrschung solcher Tools das
Ziel, dies wird in der Vorlesung Compilerbau geschehen.

\item[Kapitel \ref{chapter-turing}:] Turing Maschinen und davon erkannte Sprachen. 
Rekursiv aufz"ahlbare Sprachen und Aufz"ahler.

Eine Turing Maschinen ist ein stark vereinfachtes Modell eines Computers.
Alle modernen Computer sind im Prinzip nur besonders raffinierte Turing Maschinen.
Turing Maschinen sind aber einfach genug, um sich mit ihrer Hilfe dar"uber Gedanken
zu machen, ob und mit welcher Geschwindigkeit ein Problem mit einem
Computer gel"ost werden kann. In diesem Zusammenhang m"ussen wir uns
dar"uber Gedanken machen, was es heissen soll, dass ein Problem mit
einem Computer l"osbar sein soll. Dabei werden wir zeigen, dass es Probleme
gibt, die nicht mit dem Computer gel"ost werden k"onnen. Allerdings wird
es uns schwerfallen, solche Probleme explizit zu beschreiben. 
Gelingen wird uns dies erst im Kapitel 6.

\item[Kapitel \ref{chapter-entscheidbarkeit}:] Entscheidbare Sprachen. Unl"osbare Probleme.

Ziel dieses Kapitels ist zu lernen, wie man ein beliebiges Problem als
Sprache formulieren kann, und damit die bisher gelernte Theorie darauf
anwenden kann. Damit wird es m"oglich, konkrete und f"ur die Praxis eines
Informatikers wichtige Probleme zu beschreiben, die
nicht mit Computern gel"ost werden k"onnen.
Es ist zum Beispiel unm"oglich, Programme maschinell daraufhin
zu pr"ufen, ob sie je anhalten werden.
Ausserdem lernen wir eine Technik kennen, mit der wir Probleme 
nach ``Schwierigkeitsgrad'' miteinander vergleichen k"onnen. 

\item[Kapitel \ref{chapter-komplexitaet}:] Komplexit"atstheorie. In diesem Kapitel sollen 
die prinzipiellen Grenzen der Geschwindigkeit ermittelt werden,
mit der ein Problem mit dem Computer gel"ost werden kann. Es wird sich
herausstellen, dass es zwar viele Probleme gibt, die mit aktuellen Computern in
vern"unftiger Zeit l"osbar sind, dass aber auch eine Klasse von Aufgaben
existiert, f"ur die es nach aktuellem Wissen keinen Algorithmus geben kann,
der die Aufgabe bei etwas gr"osseren Problemen l"osen kann. Diese Probleme sind
zwar im Prinzip l"osbar, aber alle Rechenleistung der Welt kann die Probleme nicht l"osen.

\item[Kapitel \ref{chapter-vollstaendigkeit}:]
Turing-Maschinen haben eine universelle Eigenschaft: eine Turing-Maschine
kann jede andere Turing-Maschine simulieren. Doch in der Praxis
programmiert niemand direkt eine Turing-Maschine, dazu werden
spezialisierte Programmiersprachen verwendet.
Welche Eigenschaften muss eine Sprache haben, damit
man all das machen kann, was man mit einer Turing-Maschine
machen kann?
\end{description}



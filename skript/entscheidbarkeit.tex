%mm
% 
%
\chapter{Entscheidbarkeit\label{chapter-entscheidbarkeit}}
\lhead{Entscheidbarkeit}
\rhead{}
Die Menge aller Turing-entscheidbaren Sprachen ist abz"ahlbar,
da die Turingmaschinen abz"ahlbar sind.
Die Menge $\Sigma^*$ aller W"orter "uber dem Alphabet $\Sigma$ ist abz"ahlbar.
Andererseits ist die Menge aller Sprachen "uber einem Alphabet $\Sigma$
als Potenzmenge einer abz"ahlbaren Menge "uberabz"ahlbar. Es muss
also auch Sprachen geben, die nicht entscheidbar sind. Es ist also
ein durchaus interessantes Unterfangen, nicht entscheidbare Sprachen
zu finden.

In diesem Kapitel zeigen wir, wie Probleme als Sprachprobleme
formuliert werden k"onnen und untersucht werden kann, ob sie entscheidbar sind,
also ob es ein Computerprogramm geben kann, mit dem sie gel"ost werden
k"onnen.

\section{Entscheidbare Sprachen}
\rhead{Entscheidbare Sprachen}
\index{Sprache!entscheidbare}
Wenn ein Problem von einem Computer gel"ost werden soll, dann
muss der Computer die Problembeschreibung und eine m"ogliche L"osung
lesen k"onnen, und entscheiden k"onnen, ob die beiden zusammen passen.
Er muss also korrekte Problem- und L"osungs-Beschreibungen von
inkorrekten unterscheiden k"onnen. Die Menge der korrekten
Beschreibungen bilden eine Sprache, das Problem ist also darauf
zur"uckgef"uhrt, eine Sprache zu entscheiden.

\subsection{Entscheidbare Sprachen}
Eine Sprache heisst entscheidbar, wenn es einen Entscheider gibt, der
dazu verwendet werden kann, herauszufinden, ob ein Wort $w$ in der Sprache
liegt. 

\begin{beispiel}
Sei $A$ ein DEA. Dann kann man folgenden Algorithmus $M_A$ mit Input $w$ bauen:
\begin{compactenum}
\item Simuliere den DEA $A$ auf dem Input $A$
\item Falls $A$ in einem Akzeptierzustand steht: $q_{\text{accept}}$
\item Andernfalls: $q_{\text{reject}}$
\end{compactenum}
Dieser Algorithmus ist sicher ein Entscheider, er terminiert immer.
Ob er ein Wort $w$ akzeptiert, h"angt davon ab, ob $A$ das Wort akzeptiert:
\begin{align*}
  A&\text{ akzeptiert $w$}&
   &\Leftrightarrow&
M_A&\text{ erkennt $w$}
\\
  A&\text{ akzeptiert $w$ nicht}&
   &\Leftrightarrow&
M_A&\text{ erkennt $w$ nicht}
\end{align*}
Das bedeutet aber, dass $L(A)=L(M_A)$, oder $M_A$ ist ein Entscheider f"ur
die Sprache $L(A)$.
\end{beispiel}

Zu jeder regul"aren Sprache mit DEA $A$ gibt es also einen
Algorithmus $M_A$, der die gleiche Sprache erkennt: $L(A)=L(M_A)$.
K"urzer:

\begin{satz}
\label{regulaer-entscheidbar}
Regul"are Sprachen sind entscheidbar.
\end{satz}

\begin{beispiel}
Sei $G$ eine kontextfreie Grammatik. Dann kann man den folgenden Algorithmus
$M_G$ mit Input $w$ bauen:
\begin{compactenum}
\item Wandle $G$ in Chomsky-Normalform $G'$ um
\item Wende den CYK-Algorithmus (Satz \ref{cyk-algorithm}) auf $G'$ und das
Wort $w$ an
\item Wenn der CYK-Algorithmus das Wort $w$ akzeptiert: $q_{\text{accept}}$
\item Andernfalls: $q_{\text{reject}}$.
\end{compactenum}
Weil der CYK-Algorithmus (Satz \ref{cyk-algorithm}) deterministisch 
feststellen kann, ob ein Wort von einer Grammatik erzeugt wird, ist
$M_G$ ein Entscheider. $M_G$ erkennt ein Wort genau dann, wenn des von
der Grammatik produziert wird, also $L(M_G)=L(G)$.
\end{beispiel}

Zu jeder kontextfreien Sprache mit Grammatik $G$ gibt es also einen 
Algorithmus $M_G$, der die gleiche Sprache erkennt: $L(G)=L(M_G)$.
K"urzer:

\begin{satz}
Kontextfreie Sprachen sind entscheidbar.
\end{satz}
Dieser Satz zeigt, dass die entscheidbaren Sprachen eine 
Obermenge der kontextfreien Sprachen sind, wie in folgender
Graphik dargestellt.
\begin{center}
%\includegraphics[width=0.7\hsize]{images/lang-2}
\includegraphics{images/lang-2}
\end{center}

\subsection{Entscheidbare Probleme f"ur regul"are Sprachen}
Wir wollen jetzt eine Ebene h"oher geben. Unser Fokus sind jetzt nicht
mehr einzelne Sprachen, sondern eine ganze Menge von Sprachen, und eventuell
weitere Daten. Wir suchen also nicht mehr Algorithmen, die eine Frage
f"ur eine einzelne Sprache beantworten k"onnen, sondern Algorithmen, die
die Frage gleich f"ur mehrere Sprachen beantworten.

In Satz~\ref{regulaer-entscheidbar} ging es darum, einen Algorithmus
zu finden, der entscheidet, ob ein Wort $w$ von {\em einem} ganz bestimmten,
vorgegebenen endlichen Automaten $A$ akzeptiert wird.
Jetzt fragen wir, ob es einen Algorithmus gibt, der f"ur {\em jede beliebige}
Kombination von DEA und Wort $w$ diese Entscheidung treffen kann.
Dies ist offenbar ein viel allgemeineres Problem, ob $A$ das Wort
akzeptiert ist ein Spezialfall. Dieses Problem heisst das
Akzeptanzproblem f"ur endliche Automaten.

\subsubsection{Akzeptanzproblem}
\index{Akzeptanzproblem!f\"ur regul\"are Sprachen}
Wir betrachten das folgende Problem: Gegeben sei ein endlicher Automat $B$
und ein Wort $w$,
kann man entscheiden, ob der Automat das Wort akzeptiert?
Um dies als Sprachproblem zu formulieren, braucht  man eine Codierung
$\langle B,w\rangle$ von Automat und Wort. Die Sprache, die zu
entscheiden ist, lautet dann:
\[
A_{\text{DEA}} =\{
\langle B,w\rangle\;|\;\text{$B$ ist ein DEA und $B$ akzeptiert $w$}.
\}
\]
\begin{satz}
\label{adea_decidable}
$A_{\text{DEA}}$ ist entscheidbar.
\end{satz}
\index{Akzeptanzproblem!f\"ur DEAs}

\begin{proof}[Beweis]
Der folgende Algorithmus entscheidet
$A_{\text{DEA}}$:
\medskip
\begin{compactenum}
\item Simuliere $B$ auf einer TM mit Input $w$.
\item Falls $B$ akzeptiert, gehe in den Zustand $q_{\text{accept}}$.
\item Falls $B$ nicht akzeptiert, gehe in den Zustand $q_{\text{reject}}$.
\end{compactenum}
\medskip
\end{proof}

Das gleiche Problem kann man auch f"ur nicht deterministische Automaten
formulieren:

\begin{satz}
\index{Akzeptanzproblem!f\"ur NEAs}
Das Akzeptanzproblem f"ur nicht deterministische endlich Automaten
\[
A_{\text{NEA}} =\{
\langle B,w\rangle\;|\;\text{$B$ ist ein NEA und $B$ akzeptiert $w$}.
\}
\]
ist entscheidbar.
\end{satz}

\begin{proof}[Beweis]
Der folgende Algorithmus entscheidet 
$A_{\text{NEA}}$:
\medskip
\begin{compactenum}
\item Wandle $B$ in einen DEA $B'$ um.
\item Verwende die TM von Satz \ref{adea_decidable}, um zu
entscheiden, ob $B'$ das Wort $w$ akzeptiert.
\end{compactenum}
\medskip
\end{proof}

\begin{satz} Das Akzeptanzproblem f"ur regul"are Ausdr"ucke
\index{Akzeptanzproblem!f\"ur regul\"are Ausdr\"ucke}
\[
A_{\text{REX}}=\{
\langle R,w\rangle\;|\;\text{$R$ ist ein regul"arer Ausdruck und $R$ akzeptiert $w$}
\}
\]
ist entscheidbar.
\end{satz}

\subsubsection{Leerheitsproblem}
\index{Leerheitsproblem!f"ur DEAs}
Wir wollen einem endlichen Automaten ansehen k"onnen, ob er "uberhaupt
irgend ein Wort akzeptieren wird, und untersuchen daher
\[
E_{\text{DEA}}
=\{
\langle A\rangle \;|\;\text{$A$  ist ein DEA und $L(A)=\emptyset$}
\}.
\]
\begin{satz}
$E_{\text{DEA}}$
ist entscheidbar.
\end{satz}

\begin{proof}[Beweis]
Man muss herausfinden, ob "uberhaupt je ein Akzeptierzustand erreicht
werden kann.
Der folgende Algorithmus schafft dies:
\medskip
\begin{compactenum}
\item Markiere den Startzustand
\item Solange sich noch neue Zust"ande markieren liessen:
 Markiere alle Zust"ande, die sich von den bereits markierten aus
erreichen lassen.
\item Falls ein Akzeptierzustand markiert wurde: $q_{\text{reject}}$,
andernfalls
$q_{\text{accept}}$
\end{compactenum}
\medskip
\end{proof}

\subsubsection{Gleichheitsproblem}
\index{Gleichheitsproblem!f"ur DEAs}
Akzeptieren zwei DEAs die gleiche Sprache? Im Kapitel \ref{chapter-regular}
haben wir einen Algorithmus skizziert, mit dem dies entschieden werden kann.

\begin{satz}
\label{satz:eqdea}
Die Sprache
\[
\text{\it EQ}_{\text{DEA}}=\{
\langle A,B\rangle\;|\;\text{$A$ und $B$ sind DEAs und $L(A)=L(B)$}
\}
\]
ist entscheidbar.
\end{satz}

\begin{proof}[Beweis 1]
In Anlehnung an Kapitel \ref{chapter-regular} k"onnen wir den folgenden
Algorithmus verwenden:
\medskip
\begin{compactenum}
\item Erzeuge den minimalen Automaten $A'$ zu $A$.
\item Erzeuge den minimalen Automaten $B'$ zu $B$.
\item Falls $A'=B'$: $q_{\text{accept}}$, andernfalls $q_{\text{reject}}$.
\end{compactenum}
\medskip
Dieser Algorithmus entscheidet offenbar 
$\text{\it EQ}_{\text{DEA}}$.
\end{proof}

\begin{proof}[Beweis 2]
Noch etwas eleganter ist der folgende Beweis. Ob $L(A)=L(B)$ ist
offenbar gleichbedeutend damit, dass die symmetrische Differenz
(Abbildung~\ref{symdiff})
\[
L(A){\;\Delta\;} L(B)=
(L(A)\setminus L(B)) \cup (L(B)\setminus L(A))
\]
leer ist.
Da sich die Mengenoperationen aber durch Operationen mit
endlichen Automaten abbilden lassen, gibt es einen Algorithmus,
der  zu jedem Paar $(A,B)$
von endlichen Automaten  einen endlichen Automaten $C$ konstruiert mit
$L(C)=L(A){\;\Delta\;} L(B)$. Dann muss nur noch mit Hilfe eines Entscheiders
f"ur $E_{\text{DEA}}$ entschieden werden, ob $\langle C\rangle\in
E_{\text{DEA}}$.
\end{proof}

Das Prinzip dieses zweiten Beweises wird uns sp"ater weiter besch"aftigen.
Entscheidend war, dass es einen Algorithmus gibt, mit dem das
Entscheidungsproblem 
$\text{\it EQ}_{\text{DEA}}$
auf das Leerheitsproblem
$E_{\text{DEA}}$ zur"uckgef"uhrt werden konnte. Die Entscheidung
erfolgte dann mit dem Entscheider f"ur
$E_{\text{DEA}}$.

\subsection{Entscheibarkeitsproblem f"ur kontextfreie Sprachen}
\subsubsection{Akzeptanzproblem}
\index{Akzeptanzproblem!f"ur kontextfreie Grammatiken}
\begin{satz}
\label{satz:acfg-entscheidbar}
Die Sprache
\[
A_{\text{CFG}}=\{
\langle G,w\rangle\;|\; \text{die kontextfreie Grammatik $G$ erzeugt $w$}
\}
\]
ist entscheidbar.
\end{satz}

\begin{proof}[Beweis 1]
Der CYK-Algorithmus von Satz \ref{cyk-algorithm} entscheidet
$A_{\text{CFG}}$.
\end{proof}

\begin{proof}[Beweis 2]
Der folgende Algorithmus entscheidet
$A_{\text{CFG}}$:
\medskip
\begin{compactenum}
\item Wandle $G$ in Chomsky Normalform um.
\item Erzeuge alle W"orter, die sich durch Anwendung
von $|w|-1$ Regeln der Form $A\to BC$ und $|w|$ Regeln
der Form $A\to a$ ableiten lassen. 
\item Falls $w$ darunter vorkommt: $q_{\text{accept}}$, 
andernfalls $q_{\text{reject}}$.
\end{compactenum}
\medskip
\end{proof}

\subsubsection{Leerheitsproblem}
\index{Leerheitsproblem!f\"ur kontextfreie Grammatiken}
\begin{satz}
Die Sprache
\[
E_{\text{CFG}}=\{
\langle G\rangle\;|\; \text{$G$ ist eine kontextfreie Grammatik und $L(G)=\emptyset$}
\}
\]
ist entscheidbar.
\end{satz}

\begin{proof}[Beweis]
Eine Grammatik erzeugt W"orter, wenn alle Variablen, die im Laufe
der Ableitung auftreten, am Ende mit Hilfe von Regeln der Form $A\to a$
in Terminalsymbole umgewandelt werden k"onnen. Man k"onnte also einen
Markierungsalgorithmus entwickeln, der alle Terminalsymbole markiert,
und alle Symbole, die mit geeigneten Regeln ebenfalls markierte Symbole
erzeugen. Falls die Startvariable nicht markiert wird, ist die Sprache
leer. Der Algorithmus 
\medskip
\begin{compactenum}
\item Wandle $G$ in Chomsky Normalform um.
\item Markiere alle Terminalsymbole.
\item Solange sich neue Variablen markieren lassen: markiere alle
Variablen $A$, zu denen es eine Regel gibt, so dass alle Symbole auf
der rechten Seite der Regel bereits markiert sind.
\item Falls $S$ markiert wurde: $q_{\text{reject}}$, andernfalls
$q_{\text{accept}}$.
\end{compactenum}
\medskip
entscheidet also
$E_{\text{CFG}}$.
\end{proof}

\subsubsection{Gleichheit}
\index{Gleichheitsproblem!f\"ur kontextfreie Grammatiken}
Es ist verlockend zu vermuten, dass das Gleichheitsproblem
\[
\text{\it EQ}_{\text{CFG}}=\{
\langle G,H\rangle\;|\; \text{$G$  und $H$ sind kontextfreie Grammatiken und $L(G)=L(H)$}
\}
\]
mit der gleichen Methode mit Hilfe der symmetrischen Differenz
wie bei regul"aren Sprachen entschieden werden k"onnte. Dem ist
allerdings nicht so, denn das Komplement einer kontextfreien Sprache
ist im Allgemeinen nicht kontextfrei, also auch die symmetrische
Differenz nicht. In der Tat ist 
$\text{\it EQ}_{\text{CFG}}$ nicht entscheidbar, die Methoden zum Beweis
werden allerdings erst sp"ater bereitgestellt.

\section{Das Akzeptanzproblem f"ur Turingmaschinen}
\index{Akzeptanzproblem!f\"ur Turingmaschinen}
\rhead{Akzeptanzproblem f"ur Turingmaschinen}
Das Akzeptanzproblem f"ur Turing-erkennbare Sprachen fragt, ob 
eine Turingmaschine $M$ und ein Inputwort $w$ erkennen wird.
Das Problem ist entscheidbar, wenn man aus der Beschreibung
der Maschine $M$ und dem Inputwort mit einem Algorithmus, der
f"ur alle Turingmaschinen und alle W"orter immer anh"alt, ableiten
kann, ob $M$ das Wort $w$ akzeptieren wird. Man kann also immer
entscheiden, ob $M$ im Zustand $q_{\text{accept}}$ anhalten wird.

Dabei k"onnen wir nicht einfach $M$ laufen lassen, weil $M$
ja auch in eine Endlosschleife fallen k"onnte, womit sich die
Frage nicht mehr entscheiden liesse.

Als Sprachproblem formuliert, suchen wir jetzt also einen
Entscheider f"ur die Sprache
\[
A_{\text{TM}}=\{
\langle M,w\rangle\;|\; \text{$M$ ist eine TM und $M$ erkennt $w$}
\}
\]

\begin{satz}
\label{ATM}
$A_{\text{TM}}$ ist nicht entscheidbar.
\end{satz}

Auf den ersten Blick ist dieses Resultat sehr "uberraschend, warum
soll f"ur Turing-Maschinen alles grunds"atzlich anders sein, als
f"ur endliche Automaten und Stackautomaten? Der wesentliche Unterschied
verbirgt sich in dem Wort ``entscheidbar'' selbst. Ein Entscheider ist eine
TM, welche in diesem Fall Aussagen "uber andere TMs machen muss, also
zum Beispiel auch "uber sich selbst. Bei DEAs und CFGs war die Sache
insofern einfacher, als eine TM Aussagen "uber einen DEA oder eine
CFG gemacht hat.

Ein solcher Selbstbezug kann einen in Teufels K"uche bringen. 
Nehmen wir an, in einem Dorf lebt ein Barbier, der genau diejenigen
Leute rasiert, die sich selbst nicht rasieren. Soll er sich rasieren?
Wenn er sich rasiert, geh"ort er zu den Leuten, die sich selbst rasieren,
die rasiert er nicht, also darf er sich nicht rasieren. Wenn er sich
aber nicht rasiert, geh"ort er zu den Leuten, die sich nicht selbst
rasieren, daher muss er sich rasieren. Der R"uckbezug ist, was den
Barbier in die Zwickm"uhle bringt. Und genau so einen ``Barbier''
konstruiert der Beweis des Satzes~\ref{ATM}.

\begin{proof}[Beweis]
Wir f"uhren den Beweis mit Hilfe eines Widerspruchs. Dazu nehmen wir
an, es g"abe eine Turingmaschine $H$, welche das Akzeptanzproblem entscheidet.
Sie verh"alt sich wie folgt:
\[
H(\langle M,w\rangle)\begin{cases}
\text{erkennt}&\qquad\text{falls $M$ auf $w$ im Zustand $q_{\text{accept}}$ anh"alt}
\\
\text{verwirft}&\qquad\text{falls $M$ auf $w$ nicht oder im Zustand $q_{\text{reject}}$ anh"alt}
\end{cases}
\]
Daraus konstruieren wir jetzt eine neue TM $D$ mit Input $\langle M\rangle$:
\medskip
\begin{compactenum}
\item Lasse $H$ auf dem Input $\langle M,\langle M\rangle\rangle$ laufen
\item Falls $H$ erkennt: $q_{\text{reject}}$
\item Falls $H$ verwirft: $q_{\text{accept}}$
\end{compactenum}
\medskip
Die Maschine $D$ verh"alt sich wie folgt:
\[
D(\langle M\rangle)\begin{cases}
\text{erkennt}&\qquad\Leftrightarrow\qquad \text{$M$ verwirft $\langle M\rangle$}
\\
\text{verwirft}&\qquad\Leftrightarrow\qquad \text{$M$ erkennt $\langle M\rangle$}
\end{cases}
\]
($D$ entspricht dem Barbier, der genau die rasiert, die sich selbst
nicht rasieren.)

Lassen wir jetzt $D$ auf seiner eigenen Beschreibung operieren
(wir fragen uns, ob der Barbier sich selbst rasieren soll), muss
sich folgendes Resultat ergeben
\[
D(\langle D\rangle)\begin{cases}
\text{erkennt}&\qquad\Leftrightarrow\qquad \text{$D$ verwirft $\langle D\rangle$}
\\
\text{verwirft}&\qquad\Leftrightarrow\qquad \text{$D$ erkennt $\langle D\rangle$}
\end{cases}
\]
$D$ muss also genau das Gegenteil dessen tun, was $D$ tut.
$D$ ist der Barbier, der sich nicht entscheiden kann, ob er sich selbst
rasieren will.
Dieser
Widerspruch zeigt, dass die Annahme, es g"abe einen Entscheider $H$
nicht haltbar ist. Also ist das Problem nicht entscheidbar.
\end{proof}

\index{Goedel@G\"odel, Kurt}
Die Entdeckung, dass gewisse Probleme nicht entscheidbar sind,
geht auf Kurt G"odel zur"uck. G"odel hat sie jedoch in leicht
anderem Zusammenhang gefunden. Er untersuchte die Frage, ob ein
Axiomensystem eine Aussage beweisen kann. Dabei zeigte sich,
dass es Aussagen geben muss, die nicht bewiesen werden k"onnen,
obwohl sie wahr sind.

Die Formulierung f"ur Turing-Maschinen geht auf eine Arbeit von
Alan Turing 1936 zur"uck.
\index{Turing, Alan}
Aus der L"osung des Halteproblems kann
man eine konkrete solche Aussage ableiten.

Am Input $w$ allein liegt es nicht, dass das Problem nicht
entschieden werden kann.
Nicht einmal f"ur den speziellen Input $\varepsilon$ ist es
entscheidbar, auch die Sprache
\[
\textit{HALT}\varepsilon_{\text{TM}}
=\{
\langle M\rangle \;|\;
\text{$M$ ist eine TM und h"alt auf Input $\varepsilon$}
\}
\]
ist nicht entscheidbar. Nehmen wir an, es g"abe einen
Entscheider $H$ f"ur $\textit{HALT}\varepsilon_{\text{TM}}$, dann k"onnen wir daraus
auch einen Entscheider f"ur $A_{\text{TM}}$ konstruieren.
Dazu gehen wir wie folgt vor. Auf dem Input $\langle M,w\rangle$
bauen wir folgende Maschine $M'$:
\medskip
\begin{compactenum}
\item Schreibe $w$ auf das Band
\item Lasse $M$ laufen
\item Falls $M$ erkennt: $q_{\text{accept}}$.
\item Falls $M$ verwirft: gehe in eine Endlosschleife.
\end{compactenum}
\medskip
Die Maschine $M'$ h"alt auf leerem Input genau dann, wenn $M$
das Wort $w$ erkennt. Der Entscheider $H$, angewendet auf $M'$
entscheidet also, ob $w\in L(M)$, l"ost also das Akzeptanzproblem
$A_{\text{TM}}$. 
Da $A_{\text{TM}}$ nicht entscheidbar ist, darf es keinen solchen
Entscheider geben. Dieser Widerspruch zeigt, dass die Annahme, es
g"abe einen Entscheider f"ur $\textit{HALT}\varepsilon_{\text{TM}}$ gibt,
nicht haltbar ist. $\textit{HALT}\varepsilon_{\text{TM}}$ heisst
auch das {\em spezielle Halteproblem}.
\index{Halteproblem!spezielles}

Das Halteproblem beweist, dass es Sprachen gibt, die nicht
entscheidbar sind, obwohl sie von einer Turing-Maschine
erkennbar sind. Die Turing-erkennbaren Sprachen bilden also eine
echte Obermenge der entscheidbaren Sprachen.
\begin{center}
%\includegraphics[width=0.8\hsize]{images/lang-3}
\includegraphics{images/lang-3}
\end{center}

\section{Reduktion}
\index{Reduktion}
\rhead{Reduktion}
Im letzten Abschnitt wurde die Entscheidbarkeit der Sprache $L$
auf das Akzeptanzproblem in dem Sinne zur"uckgef"uhrt, dass
aus einem Entscheider f"ur $L$ ein Entscheider f"ur das Akzeptanzproblem
konstruiert wurde. Die Nichtentscheidbarkeit des Akzeptanzproblems
hat dann gezeigt, dass auch $L$ nicht entscheidbar sein kann.

Dieses Vorgehen funktioniert in sehr vielen F"allen, wir wollen es
daher etwas abstrakter formulieren und damit die Anwendung in weiteren
Beispielen vereinfachen. Gleichzeitig erm"oglicht es uns, die
Sprachen mindestens teilweise nach ihrer Schwierigkeit, sie zu entscheiden,
anzuordnen. 

\subsection{Reduktionsabbildung}
\index{Reduktionsabbildung}
\begin{figure}
\begin{center}
%\includegraphics[width=\hsize]{images/red-1}
\includegraphics{images/red-1}
\end{center}
\caption{Reduktionsabbildung $f\colon A\to B$.\label{reduktionsabbildung}}
\end{figure}
Wir m"ochten zwei verschiedene Sprachen 
$A$ und $B$ 
vergleichen. Dazu bilden wir W"orter von der einen Sprache in
die andere Sprache ab, wir brauchen also eine Abbildung, die
$f\colon\Sigma^*\to \Sigma^*$, die $A$ in $B$ "uberf"uhrt,
d.\,h.
\[
w\in A\Leftrightarrow f(w)\in B.
\]
Diese Abbildung hilft uns allerdings nur dann, Entscheidbarkeitsfragen
von $B$ nach $A$ zu transportieren, wenn sie mit einer Turingmaschine
berechnet werden kann. Wir definieren daher

\begin{definition}
\index{berechenbar}
Eine Abbildung $f\colon \Sigma^*\to \Sigma^*$ heisst berechenbar wenn es eine
Turingmaschine gibt, die auf jedem Input $w\in \Sigma^*$ anh"alt
und ausschliesslich das Wort $f(w)$ auf dem Band zur"uckl"asst.
\end{definition}

\begin{definition}
\index{Reduktion}
Eine Sprache $A$ ist reduzierbar auf die Sprache $B$, in Zeichen
\[
A\le B
\]
wenn es eine
berechenbare Abbildung $f\colon \Sigma^*\to \Sigma^*$ gibt, mit
\[
w\in A\Leftrightarrow f(w)\in B.
\]
Wir schreiben daf"ur auch
\[
f\colon A\le B.
\]
\end{definition}
Die Schreibweise $A\le B$ soll ausdr"ucken, dass die Sprache $A$ ``leichter''
zu entscheiden ist als die Sprache $B$. Die nachfolgenden zwei S"atze
zeigen, dass diese Leseart berechtigt ist.

\begin{satz}
Seien $A\le B$ Sprachen. Ist $B$ entscheidbar, dann auch $A$.
\end{satz}

\begin{proof}[Beweis]
Ist $B$ entscheidbar, dann gibt es einen Entscheider f"ur $B$, also
eine Turingmaschine $M$, die auf jedem Inputwort anh"alt, und
entscheidet, ob es zu $B$ geh"ort. Daraus k"onnen wir jetzt einen
Entscheider f"ur $A$ konstruieren. Um zu entscheiden, ob $w\in A$
ist, verwenden wir den folgenden Algorithmus $M'$ mit Input
$w$:
\medskip
\begin{compactenum}
\item Berechne $f(w)$, daf"ur gibt es wegen $A\le B$ eine Turingmaschine.
\item Verwende den Entscheider $M$ um zu entscheiden, ob $f(w)\in B$.
\item Falls $M$ akzeptiert ist $f(w)\in B\Rightarrow w\in A$
\item Falls $M$ nicht akzeptiert ist $f(w)\not\in B\Rightarrow w\not\in A$.
\end{compactenum}
\medskip
$M'$ ist offenbar ein Entscheider f"ur $A$, also ist $A$ entscheidbar.
\end{proof}

Die Kontraposition dieser Aussage ist
\begin{satz}
Ist $A\le B$ und $A$ nicht entscheidbar, dann ist auch $B$ nicht
entscheidbar.
\end{satz}
Damit haben wir eine Methode, um Sprachen als nicht entscheidbar 
zu beweisen. Wir m"ussen nur eine Abbildung $f\colon \Sigma^*\to \Sigma^*$
konstruieren, wobei $A$ ein bekanntermassen nicht
entscheidbares Problem ist, und $f$ berechenbare Abbildung ist,
die die Reduktion $A\le B$ vermittelt.

\subsection{Weitere nicht entscheidbare Sprachprobleme}
\index{Halteproblem}
\begin{satz} Das Halteproblem
\[
\text{\it HALT}_{\text{TM}}=\{\langle M,w\rangle\;|\;
\text{$M$ ist eine TM und $M$ h"alt auf Input $w$}\}
\]
ist nicht entscheidbar.
\end{satz}

\begin{proof}[Beweis]
Wir konstruieren eine Reduktion
$
A_{\text{TM}}
\le
\text{\it HALT}_{\text{TM}}
$.
Die Reduktion muss aus einer TM $M$ und einem Inputwort $w$ eine
neue TM $S$ und ein Inputwort $w$ machen, so dass $S$ auf
Input $w$ genau dann h"alt, wenn $M$ das Wort $w$ akzeptiert. Die Abbildung
$f$ ist also
\[
f\colon 
A_{\text{TM}}
\le
\text{\it HALT}_{\text{TM}}
\colon
\langle M,w\rangle\mapsto \langle S,w\rangle.
\]
$S$ konstruieren wir wie folgt:
\medskip
\begin{compactenum}
\item lasse $M$ laufen auf Input $w$
\item falls $M$ das Wort $w$ erkennt: $q_{\text{accept}}$
\item falls $M$ verwirft: beginne eine Endlosschleife
\end{compactenum}
\medskip
Offenbar ist
$\langle S,w\rangle\in \text{\it HALT}_{\text{TM}}$
genau dann, wenn $\langle M,w\rangle\in A_{\text{TM}}$.
Wir haben also
\[
\langle M,w\rangle\in A_{\text{TM}}
\qquad
\Leftrightarrow
\qquad
\langle S,w\rangle=f(\langle M,w\rangle)
\in \text{\it HALT}_{\text{TM}}.
\]
Somit ist 
$A_{\text{TM}}\le\text{\it HALT}_{\text{TM}}$, und da
$A_{\text{TM}}$ nicht entscheidbar ist, ist auch
$\text{\it HALT}_{\text{TM}}$ nicht entscheidbar.
\end{proof}
Statt des Anhaltens k"onnte man auch viele andere spezielle
Betriebszust"ande einer Turingmaschine als ``Haltekriterien''
heranziehen. Dann bedeutet der Satz, dass es keinen allgemeinen
Algorithmus geben kann, der ein St"uck Code inspizieren kann und
daraus ableiten kann, ob dieser Code auf einem System Schaden
anrichtet. Echte Schadcode-Erkennung gibt es also grunds"atzlich
nicht (auch wenn das Marketing-Material gewisser Produkte sich
da zu ganz erstaunlichen Aussagen versteigt). Das beste, was man
tun kann, ist Mustererkennung (endliche Sprachen sind alle regul"ar).

\begin{satz}
\index{Leerheitsproblem!f\"ur Turingmaschinen}
Das Leerheitsproblem f"ur Turingmaschinen
\[
E_{\text{TM}}
=
\{ \langle M\rangle\;|\; \text{$M$ ist eine TM und $L(M)=\emptyset$}\}
\]
ist nicht entscheidbar.
\end{satz}

\begin{proof}[Beweis]
Wir m"ochten eine Reduktion $A_{\text{TM}}\le E_{\text{TM}}$ konstruieren.
Dazu m"ussen wir aus $\langle M,w\rangle$ eine Maschine $S$ konstruieren,
die genau dann kein einziges Wort akzeptiert, wenn $M$ das Wort $w$
akzeptiert. Es ist allerdings einfacher, die Negation zu behandeln,
also zu zeigen, dass 
\[
\bar E_{\text{TM}}=
\{ \langle M\rangle\;|\; \text{$M$ ist eine TM und $L(M)\ne\emptyset$}\}
\]
nicht entscheidbar ist. Wir brauchen dann eine Reduktion
$A_{\text{TM}}\le \bar E_{\text{TM}}$, d.\,h.~eine Maschine $S$,
die genau dann mindestens ein Wort erkennt, wenn $M$ das Wort $w$ 
erkennt.

Die Maschinen $S$ muss auf beliebigen W"ortern $u$ als Input ausgef"uhrt
werden k"onnen. Wir k"onnten folgenden Algorithmus verwenden:
\medskip
\begin{compactenum}
\item Falls $u\ne w$: $q_{\text{reject}}$.
\item Lasse $M$ auf $w$ laufen.
\item Falls $M$ erkennt: $q_{\text{accept}}$.
\end{compactenum}
\medskip
Wenn $M$ das Wort $w$ erkennt, dann erkennt $S$ die Sprache 
$L(S)=\{w\}$. Wenn $M$ das Wort $w$ nicht erkennt, ist $L(S)=\emptyset$.
Oder
\begin{align*}
M&\operatorname{erkennt}w              &&\Rightarrow&L(S)&=\{w\}\ne \emptyset
\\
M&\operatorname{erkennt}w\text{ nicht} &&\Rightarrow&L(S)&=\emptyset
\end{align*}
oder
\[
\langle M,w\rangle \in A_{\text{TM}}
\qquad
\Leftrightarrow
\qquad
\langle S\rangle = f(\langle M,w\rangle)\in
\bar E_{\text{TM}},
\]
also $A_{\text{TM}}\le \bar E_{\text{TM}}$.
\end{proof}

\begin{satz}
Es ist nicht entscheidbar, ob eine Turingmaschine eine regul"are
Sprache erkennt.
\end{satz}

\begin{proof}[Beweis]
Als Sprache formuliert m"ochten wir eine Reduktion 
\[
A_{\text{TM}}\le\text{\it REGULAR}_{\text{TM}}=\{
\langle M\rangle\;|\;\text{$M$ ist eine TM und $L(M)$ ist regul"ar}
\}
\]
konstruieren. Auch in diesem Fall ist es einfacher, eine
Reduktion
\[
A_{\text{TM}}\le\overline{\text{\it REGULAR}}_{\text{TM}}
\]
zu konstruieren.

Wir m"ussen also eine TM $S$ konstruieren, die genau
dann eine nicht regul"are Sprache erkennt, wenn $w$ von $M$ erkannt wird.
Der folgende Algorithmus tut dies. Auf Input $u$ geht er wie folgt
vor:
\medskip
\begin{compactenum}
\item Falls $u$ nicht von der Form $0^n1^n$ ist: $q_{\text{reject}}$
\item Lasse $M$ auf $w$ laufen.
\item Falls $M$ $w$ erkennt: $q_{\text{accept}}$
\item Falls $M$ $w$ verwirft: $q_{\text{reject}}$
\end{compactenum}
\medskip
Wegen Schritt~1 kann $S$ h"ochstens die W"orter der Form
$\texttt{0}^n\texttt{1}^n$ erkennen, alle anderen werden bereits in
Schritt $1$ verworfen.
Ob die W"orter 
$\texttt{0}^n\texttt{1}^n$ erkannt werden, h"angt davon ab, ob $M$ das Wort
$w$ erkennt:
\begin{align*}
   M&\text{ erkennt $w$}      &
    &\Leftrightarrow&
L(S)&=\{\texttt{0}^n\texttt{1}^n\,|\, n\ge 0\}&
    &\Leftrightarrow&
    &\text{$L(S)$ nicht regul"ar}\\
   M&\text{ erkennt $w$ nicht}&
    &\Leftrightarrow&
L(S)&=\{\quad\}=\emptyset&
    &\Leftrightarrow&
    &\text{$L(S)$ regul"ar}
\end{align*}
$S$ erkennt die nicht regul"are Sprache $\{0^n1^n\;|\;n\in\mathbb N\}$
genau dann, wenn $M$ das Wort $w$ erkennt.
Die Reduktion $\langle M,w\rangle\to \langle S\rangle$ "ubersetzt
``erkennen'' in ``nicht regul"are Sprache erkennen'', also also $A_{\text{TM}}$
in $\overline{\textsl{REGULAR}}_{\text{TM}}$.
\end{proof}

\index{Gleichheitsproblem!f\"ur Turingmaschinen}
\begin{satz}
Man kann nicht entscheiden, ob zwei Turingmaschinen die gleiche Sprache
erkennen:
\[
\text{\it EQ}_{\text{TM}}=\{
\langle M_1,M_2\rangle\;|\;\text{$M_i$ sind Turingmaschinen und $L(M_1)=L(M_2)$}
\}
\]
ist nicht entscheidbar.
\end{satz}

\begin{proof}[Beweis]
Wir konstruieren eine Reduktion
$
E_{\text{TM}}\le \text{\it EQ}_{\text{TM}}
$
Dazu sei $M_0$ eine Turingmaschine, die jeden Input verwirft,
also $L(M_0)=\emptyset$.
F"ur die Abbildung $f$ verwendenw ir
\[
f
\colon
E_{\text{TM}}\le \text{\it EQ}_{\text{TM}}
\colon
\langle M\rangle\mapsto \langle M,M_0\rangle
\]
Es ist $\langle M,M_0\rangle\in \text{\it EQ}_{\text{TM}}$
genau dann, wenn
$L(M)=L(M_0)=\emptyset$, also genau dann, wenn
$\langle M\rangle\in E_{\text{TM}}$.
\end{proof}

Dieser Satz hat zum Beispiel zur Folge, dass es keine zuverl"assige
maschinelle
M"oglichkeit gibt, herauszufinden, ob zwei Programme das gleiche tun.
Damit ist die Entscheidung, ob ein beliebiges Programm ein beliebiges
Softwarepatent verletzt, nur durch den subjektiven Prozess in Form
eines Gerichtes m"oglich, kann also grunds"atzlich niemals objektiv sein.

\section{Nicht entscheidbare Probleme f"ur CFGs}
\rhead{Nicht entscheidbare Probleme f"ur CFGs}
Der letzte Abschnitt k"onnte den Eindruck hinterlassen haben,
dass nicht entscheidbare Probleme erst bei Turingmaschinen
auftreten, und dass alle naheliegenden Probleme bei regul"aren
und kontextfreien Sprachen entscheidbar sind. Dem ist jedoch
nicht so. Es ist zum Beispiel nicht entscheidbar, ob zwei
Grammatiken die gleiche Sprache erzeugen.

\begin{satz} Es ist nicht entscheidbar, ob eine kontextfreie
Grammatik alle W"orter erzeugt:
\[
\text{\it ALL}_{\text{CFG}}=\{\langle G\rangle\;|\; \text{$G$
ist eine kontextfreie Grammatik und $L(G)=\Sigma^*$}\}
\]
ist nicht entscheidbar.
\end{satz}

\begin{proof}[Beweis]
Wir konstruieren eine Reduktion 
$A_{\text{TM}}\le \text{\it ALL}_{\text{CFG}}$.
Wir m"ussen also aus einem Paar $\langle M,w\rangle$ einen
Grammatik erzeugen, die genau dann alle W"orter erzeugt, wenn
$M$ das Wort $w$ erkennt. Dabei m"ussen wir diese Strings in
Zusammenhang bringen mit einer Berechnung der Turingmaschine, die
auf $w$ im Zustand $q_{\text{accept}}$ endet.

Dazu k"onnen wir
die Berechnungsgeschichte verwenden. Jede Konfiguration der
Turingmaschine ist ja ein String der Form $C=w_1qw_2$. Die gesamte
Berechnung besteht aus einer Folge $C_i$ von solchen Konfigurationen.
Wir k"onnen sie protokollieren, indem wir alle $C_i$
mit Hilfe eines zus"atzlichen Zeichens verketten,
welches nicht in $\Gamma$ vorkommt. Wir bezeichnen diese Zeichen
mit {\tt\#}.

Die Grammatik soll jetzt also alle Strings ausser der Berechnungsgeschichte
erzeugen, die zum Erkennen des Wortes $w$ f"uhren. Diese Grammatik
erzeugt also genau dann alle Strings, wenn $M$ das Wort $w$ erkennt.

Um alle Strings ausser der erkennden Berechnung zu bekommen,
"uberlegen wir uns, wie ein String
$h={\tt\#}C_1{\tt\#}C_2{\tt\#}\dots{\tt\#}C_n{\tt\#}$
keine korrekte Berechungsgeschichte sein k"onnte:
\medskip
\begin{compactenum}
\item $C_1$ k"onnte nicht die Startkonfiguration sein, also
$C_1\ne q_0w$.
\item $C_n$ k"onnte nicht eine akzeptierende Konfiguration 
sein, also etwas von der Form $xq_{\text{accept}}y$.
\item Ein Zwischenschritt k"onnte nicht den Regeln der Turingmaschine
$M$ entsprechen.
\end{compactenum}
\medskip
Die gesuchte Grammatik muss all Strings erzeugen, die einen dieser ``Fehler''
machen. Falls es eine erkennende Berechnung f"ur $w$ gibt, wird diese
die einzige sein, die nicht erzeugt wird.

Statt die Grammatik zu konstruieren, konstruieren wir einen Stackautomaten,
den wir sp"ater in eine Grammatik umwandeln k"onnen. Der Stackautomat
muss alle Strings akzeptieren, die einen der Defekte 1-3 haben, wir
k"onnen also damit beginnen, nichtdeterministisch zwischen diesen drei
F"allen zu unterscheiden:
\[
\entrymodifiers={++[o][F]}
\xymatrix @+3mm{
*+\txt{}
	&q_1
\\
q_0	\ar[ur]^{\varepsilon,\varepsilon\to\varepsilon}
	\ar[r]^{\varepsilon,\varepsilon\to\varepsilon}
	\ar[dr]_{\varepsilon,\varepsilon\to\varepsilon}
	&q_2
\\
*+\txt{}
	&q_3
}
\]
Im Zustand $q_1$ m"ussen wir einen Stackautomaten anh"angen, der
"uberpr"uft, ob im Input der String $q_0w$ steht. Dies kann man 
bereits mit einem DEA machen, man braucht den Stack daf"ur gar nicht.
Ist $w=a_1a_2\dots a_k$, dann ist der Automat daf"ur geeignet
\[
\xymatrix @+3mm{
*++[o][F]{q_1}\ar[r]^{q_0,\varepsilon\to\varepsilon}
	&*++[o][F]{}\ar[r]^{a_1,\varepsilon\to\varepsilon}
		&*++[o][F]{}\ar[r]^{a_2,\varepsilon\to\varepsilon}
			&\dots\ar[r]^{a_{k-1},\varepsilon\to\varepsilon}
				&*++[o][F]{}\ar[r]^{a_k,\varepsilon\to\varepsilon}
					&*++[o][F]{}
}
\]
Akzeptieren darf er aber nur, wenn diese Regeln nicht eingehalten
werden, man muss also noch einen Fehlerzustand hinzuf"ugen, der
alles akzeptiert, was in diesem Automaten nicht funktioniert.

Im Zustand $q_2$ m"ussen wir einen Stackautomaten anh"angen, welcher
eine akzeptierende Konfiguration erkennt. Er muss also zun"achst 
alle $C_i$ mit $i<n$ "uberspringen und dann nichtdeterministisch
die Zeichen von $C_i$ lesen und kontrollieren, ob $q_{\text{accept}}$
darin vorkommt. Ein Automat daf"ur ist 
\[
\entrymodifiers={++[o][F]}
\xymatrix @+3mm{
q_2\ar@(ur,ul)
	\ar[r]^{{\tt\#},\varepsilon\to\varepsilon}
	&{}\ar@(ur,ul)
		\ar[rr]^{q_{\text{accept}},\varepsilon\to\varepsilon}
		&*+\txt{}
			&{}\ar@(ur,ul)
				\ar[r]^{{\tt\#},\varepsilon\to\varepsilon}
				&*++[o][F]{}
}
\]
Darin enhalten die Schleifen beim zweiten und dritten Zustand keine
"Uberg"ange mit dem Zeichen {\tt\#}. Dazu braucht es zus"atzliche
"Ubergange, damit der Automat alles ausser diesen Strings akzeptiert.

Im Zustand $q_3$ muss ein Automat konstruiert werden, der erkennen kann,
ob die Folge $C_i{\tt\#}C_{i+1}$ einem korrekten "Ubergang zwischen
zwei Konfigurationen der Turingmaschen $M$ entspricht. Dazu schreibt
der Stackautomat zun"achst alle Zeichen von $C_i$ auf den
Stack, und vergleicht dann dann den Stackinhalt mit den Zeichen von
$C_{i+1}$. Bis auf die Zeichen um die Kopfposition d"urfen sich
die Strings nicht unterscheiden. Dieser Automat enth"alt also
einerseits alle "Uberg"ange, welche das Zeichen auf dem Stack
mit dem n"achsten Inputzeichen vergleichen:
\[
\entrymodifiers={++[o][F]}
\xymatrix{
q\ar@(ur,dr)^{a,a\to\varepsilon}
}
\]
Ausserdem "Uberg"ange, die zu g"ultigen
Turingmaschinen-Konfigurations"uberg"angen geh"oren. Ist zum Beispiel
$px\to yq$ ein solcher "Ubergang, dann m"ussen folgende
Stackautomaten-"Uberg"ange hinzugef"ugt werden:
\[
\entrymodifiers={++[o][F]}
\xymatrix{
{}\ar@/^/[r]^{y,p\to\varepsilon}
	&\ar@/^/[l]^{q,x\to\varepsilon}
}
\]
Leider funktioniert dies noch nicht ganz: die Zeichen von $C_i$
befinden sich auf dem Stack in umgekehrter Reihenfolge, passen
also nicht zu $C_{i+1}$. Diesen Mangel kann man mit folgendem
Trick beheben: man schreibt in der Berechnungsgeschichte jede
zweite Konfiguration r"uckw"arts, also
\[
{\tt\#}C_1{\tt\#}C_2^t{\tt\#}C_3{\tt\#}C_4^t{\tt\#}\dots
\]
Die Regeln f"ur die korrekten Turingmaschinen-"Ubergang m"ussen
nat"urlich an diese umgekehrte Ordnung angepasst werden, dabei
muss beachtet werden, dass mit jedem gelesenen {\tt\#}-Zeichen
die ``Leserichtung'' "andert. Und wiederum muss der Automat alles
akzeptieren, was nicht den Turingmaschinen-Regeln entspricht.

Auf diese Weise kann ein Stackautomat konstruiert werden, der
alle Strings akzeptiert ausser der korrekten Berechnungsgeschichte
(mit alternierenden Schreibrichtungen geschrieben). Die daraus
erzeugte Grammatik erzeugt alle Strings ausser der
korrekten Berechnungsgeschichte. Damit ist die Reduktionsabbildung
konstruiert.
\end{proof}

\begin{satz}
\index{Gleichheitsproblem!f\"ur kontextfreie Grammatiken}
Das Gleichheitsproblem f"ur kontextfreie Grammatiken
ist nicht entscheidbar:
\[
\text{\it EQ}_{\text{CFG}}=\{
\langle G_1,G_2\rangle\;|\;\text{$G_i$ sind kontextfreie Grammatiken
und $L(G_1)=L(G_2)$}
\}
\]
ist nicht entscheidbar.
\end{satz}

\begin{proof}[Beweis]
Wir konstruieren eine Reduktion
\[
\text{\it ALL}_{\text{CFG}}
\le
\text{\it EQ}_{\text{CFG}}.
\]
Da $\text{\it ALL}_{\text{CFG}}$ nicht entscheidbar ist, folgt
auch, dass $\text{\it EQ}_{\text{CFG}}$ nicht entscheidbar
ist.

Sie $G_0$ die Grammatik, die $\Sigma^*$ erzeugt. Offenbar ist
$\langle G\rangle$ genau dann in $\text{ALL}_{\text{CFG}}$,
wenn $G$ und $G_0$ die gleiche Sprache erzeugen, also
$\langle G,G_0\rangle\in\text{\it EQ}_{\text{CFG}}$.
Die Reduktionsabbildung ist also
\[
\langle G\rangle \mapsto \langle G,G_0\rangle,
\]
dies ist sicher berechenbar.
\end{proof}


\section{Satz von Rice}
\rhead{Satz von Rice}
Eigenschaften von Turing-erkennbaren Sprachen sind im allgmeinen
nicht entscheidbar, wie die vorangegangenen Beispiel zeigen.
Allerdings m"ussen sie gen"u\-gend kompliziert sein. Dann l"asst
sich aber sogar der Beweis vereinheitlichen. Zum Beispiel wurde
beim Beweis von $\text{\it REGULAR}_{\text{TM}}$ eigentlich
nur verwendet, dass wir eine regul"are Sprache ($\emptyset$) und
eine nicht regul"are ($\{0^n1^n\;|\;n\in\mathbb N\}$) kennen.

\begin{definition}
\index{Eigenschaft!nicht-triviale}
Sei $P$ eine Eigenschaft einer Sprache $L$, $P(L)$ ist also wahr oder
falsch. $P$ heisst nicht trivial, falls es eine Sprache $L_1$ gibt,
die die Eigenschaft $P$ hat, und eine Sprache $L_2$, die die
Eigenschaft nicht hat.
\end{definition}

\begin{satz}[Rice]
\index{Rice!Satz von}
\label{rice-theorem}
Sei $P$ eine nicht triviale Eigenschaft von Turing-erkennbaren Sprachen,
dann ist $P$ nicht entscheidbar. Als Sprachproblem formuliert:
\[
P_{\text{TM}}=\{ \langle M\rangle\;|\;
\text{$M$ ist eine TM und $L(M)$ hat die Eigenschaft $P$}
\}
\]
ist nicht entscheidbar.
\end{satz}

\begin{proof}[Beweis]
Wir konstruieren eine Reduktion $A_\text{TM}\le P_{\text{TM}}$.
Zu einem Paar
$\langle M,w\rangle$
m"ussen wir also auf
berechenbare Weise eine Turingmaschine
$M'$ konstruieren,
\[
f\colon \langle M,w\rangle\mapsto M',
\]
deren Sprache genau dann die Eigenschaft $P$
hat, wenn $M$ das Wort $w$ akzeptiert.

Dazu brauchen wir die beiden Turing-erkennbaren Sprachen $L_1$ und $L_2$,
wobei $L_1$ die Eigenschaft $P$ hat, $L_2$ aber nicht.
Wir k"onnen dabei annehmen, dass $L_2=\emptyset$ ist, dass also
$\emptyset$ die Eigenschaft $P$ nicht hat. W"are dies nicht so, k"onnten
wir stattdessen $L_1=\emptyset$  w"ahlen, und die Eigenschaft $\neg P$
untersuchen. Aussserdem gibt es eine Turingmaschine $M_1$ mit $L(M_1)=L_1$.

Die Turingmaschine $M'$ operiert auf dem Input $u$ wie folgt:
\medskip
\begin{compactenum}
\item Lasse $M$ auf Input $w$ laufen
\item Falls $M$ das Wort $w$ akzeptiert, lasse $M_1$ auf $u$ laufen.
Falls $M_1$ im Zustand $q_{\text{accept}}$ anh"alt, halte ebenfalls
im Zustand $q_{\text{accept}}$ an.
\item In allen anderen F"allen: $q_{\text{reject}}$
\end{compactenum}
\medskip

Falls $w\in L(M)$ akzeptiert die Turingmaschine $M'$ genau die W"orter
$u\in L_1$. Falls $w\not\in L(M)$ akzeptiert $M'$ "uberhaupt keine
W"orter, in diesem Fall ist also $L(M')=\emptyset$:
\begin{align*}
w\in L(M)&\qquad \Rightarrow\qquad L(M')=L_1\\
w\not\in L(M)&\qquad \Rightarrow\qquad L(M')=\emptyset
\end{align*}
Da $f$ berechnenbar ist, haben wir die verlangte Reduktion
gefunden, und es folgt, dass $P_{\text{TM}}$ nicht
entscheidbar ist.
\end{proof}

\subsubsection{Anwendungen}

\begin{beispiel}[\bf $\text{\textsl{ALL}}_{\text{TM}}$ ist nicht entscheidbar]
$\text{\textsl{ALL}}_{\text{TM}}$ ist die Sprache
\[
\text{\textsl{ALL}}_{\text{TM}}=\{
\langle M\rangle\;|\; \text{$M$ ist eine TM und $L(M)=\Sigma^*$}
\}
\]
Der Satz von Rice verlangt, dass wir das Problem als eine Eigenschaft
der von einer TM erkannten Sprache formulieren.
\[
P=\text{\textsl{ALL}}:\text{Die erkannte Sprache ist $\Sigma^*$}
\]
Offensichtlich ist dies eine Eigenschaft der Sprache, und nicht zum
Beispiel der konkreten Implementation der TM.

Ausserdem verlangt der Satz von Rice, dass die Eigenschaft nicht trivial ist,
dass es also Sprachen gibt, die die Eigenschaft haben, und andere, die
sie nicht haben. 
\begin{enumerate}
\item Die Sprache $\Sigma^*$ ist Turing-erkennbar und hat die Eigenschaft
$P=\text{\textsl{ALL}}$.
\item Die Sprache $\emptyset$ ist auch Turing-erkennbar, hat aber die
Eigenschaft
$P=\text{\textsl{ALL}}$ nicht.
\end{enumerate}
Damit ist klar, dass die Eigenschaft nicht trivial ist.

Jetzt kann der Satz von Rice angewandt werden, er besagt, dass die
Eigenschaft
$P=\text{\textsl{ALL}}$ nicht entscheidbar ist.
\end{beispiel}

\begin{beispiel}[\bf Primzahlpr"ufer] Es ist entscheidbar, ob eine Zahl $n$
prim ist oder nicht, man testet einfach jeden m"oglichen Teiler $<n$,
wenn immer ein Rest bleibt, ist die Zahl prim. Dies ist nat"urlich
kaum der effizienteste Algorithmus, tats"achlich gibt es viele Alternativen,
die wir in eine Menge zusammenfassen k"onnen:
\[
\text{\textsl{PRIMALITY-TESTER}}=
\left\{\langle M\rangle\;\left|\;
\begin{minipage}{2.15truein}
\raggedright
$M$ ist eine Turingmaschine und\\
ein korrekter Primzahltester
\end{minipage}
\right.\right\}.
\]
\textsl{PRIMALITY-TESTER} enth"alt also genau diejenigen Programme, welche
als Primzahlpr"ufer korrekt funktionieren.

Gibt es ein Programm, welches beliebige Primzahlpr"ufer auf Korrektheit
testen kann? Wenn ja, dann ist \textsl{PRIMALITY-TESTER} entscheidbar.
Leider kann es kein solches Programm geben,
\textsl{PRIMALITY-TESTER} ist nicht entscheidbar, wie man mit dem Satz
von Rice einsehen kann.

Zun"achst muss man wieder die Eigenschaft formulieren
\[
P: \text{Die akzeptierte Sprache ist die Menge der Primzahlen}
\]
Weiter muss es zwei Sprachen geben, die eine muss die Eigenschaft
$P$ haben, die anderen nicht:
\begin{enumerate}
\item
Der oben beschriebene primitive Primzahlpr"ufer, der alle m"oglichen Teiler
durchprobiert, akzeptiert genau die Primzahlen, die akzeptierte Sprache
hat also die Eigenschaft $P$.
\item 
Die leere Sprache $\emptyset$ ist Turing-erkennbar, hat aber ganz bestimmt
die Eigenschaft $P$ nicht.
\end{enumerate}
Die Eigenschaft $P$ ist also nicht trivial, und nach dem Satz von Rice
kann sie daher auch nicht entscheidbar sein.
\end{beispiel}

Die Beispiel illustrieren, dass es im Allgmeinen unm"oglich ist, Programme
automatisch darauf zu testen, ob sie in allen F"allen korrekt arbeiten
werden. Dies ist nur dann m"oglich, wenn die akzeptierte Sprache einfach,
zum Beispiel regul"ar, ist, oder nur f"ur eine Teilmenge aller m"oglichen
Programmen.

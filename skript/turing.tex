%
% Turing Maschinen und rekursiv aufzaehlbare Sprachen
%
% (c) 2009
%
\lhead{Algorithmen}
\chapter{Turing Maschinen\label{chapter-turing}}
Die bisher entwickelten Modelle von Rechenmaschinen haben zwar
interessante Sprachklassen hervorgebracht, sind aber offensichlich nicht
geeignet, die M"oglichkeiten eines Computers abzubilden. Ein
Computer verf"ugt "uber einen Speicher, auf den er, im Gegensatz zum
Stack eines Stackautomen, beliebigen Zugriff hat. Das Ziel dieses
Abschnittes ist daher, einen endlichen Zustandsautomaten so zu
erweitern, dass er mit einem solchen Speicher arbeiten kann.
Weiterhin soll das Modell so formuliert sein, dass sich damit
Sprachen akzeptieren lassen, und wir hoffen nat"urlich, dass dadurch
eine nochmals gr"ossere Klasse von Sprachen entsteht, die diesmal
aber alles umfasst, was wir mit modernen Computern abbilden k"onnen.

\section{Turing Maschinen}
\rhead{Turingmaschinen}
\subsection{Zwei Stacks: Random Access}
Ein Stack hat die M"oglichkeiten eines endlichen Automaten dramatisch
erweitert, ein Stackautomat kann die bedeutend gr"ossere Klasse der
kontextfreien Sprachen analysieren. Wir haben aber auch schon gesehen,
dass es nicht kontextfreie Sprachen gibt. Mit welcher m"oglichst kleinen
Erweiterung k"onnen wir eine Sprache wie $\{a^nb^nc^n\,|\, n\ge 0\}$
analysieren?

Eine erste Idee k"onnte sein, einen zweiten Stack hinzuzuf"ugen.
Damit k"onnte man mit Sicherheit die Sprache $\{a^nb^nc^n\,|\,n\ge 0\}$
behandeln. Doch man f"ugt damit wesentlich mehr hinzu. Ein Buch ist
sicher ein Random-Access Speichermedium. Durch bl"attern im Buch
kann man jede beliebige Seite erreichen. Wenn man aber ein Buch
aufschl"agt, besteht es eigentlich nur aus zwei Stapeln von Bl"attern.
Bl"attern bedeutet nichts anderes, dass man ein Blatt des Buches vom
einen Stapel wegnimmt und auf den anderen Stapel legt. Zwei Stacks
bilden also bereits ein Random-Access Medium.

Antike Schriftrollen sind auch B"ucher, die aber nicht als Stapel
von Seiten organisiert sind, sondern als langes Band. Es ist aber klar,
dass man durch zerschneiden und binden jede Schriftrolle in ein
Buch verwandeln kann, und ein Buch durch zusammenkleben der einzelnen
Seiten zu einem Band in eine Schriftrolle.
Die heutigen B"ucher (Kodices) haben sich gegen"uber den Schriftrollen
etwa im ersten Jahrhundert n.~Chr. durchgesetzt.
F"ur unsere Anwendung ist die Vorstellung eines Random-Access Mediums
als endloses Band von Speicherzellen praktischer. Sie passt besser zu
unserem Bild vom Hauptspeicher eines Computers oder von Massenspeichermedien.
\subsection{Definition}
\index{Band}
\index{Bandalphabet}
\index{Schreib-/Lese-Kopf}
Wir wollen einem endlichen Automaten statt eines Stacks einen unendlich
grossen Speicher zur Verf"ugung stellen. Wir stellen uns diesen
Speicher als ein in beide Richtungen unendlich langes Band vor,
welches in einzelne Speicherzellen aufgeteilt ist, in die jeweils
genau ein Zeichen geschrieben werden kann (Abbildung~\ref{turingfig}).
Wie beim Stackautomaten gehen
wir davon aus, dass dieser Speicher auch Dinge speichern kann, die nicht
im Input vorkommen, wir verwenden daher ein Bandalphabet $\Gamma$, welches
das Input-Alphabet umfasst: $\Gamma\supset \Sigma$.

\begin{figure}
\begin{center}
%\includegraphics[width=0.8\hsize]{images/turing-1}
\includegraphics{images/turing-1}
\end{center}
\caption{Schematische Darstellung einer Turing-Maschine\label{turingfig}}
\end{figure}

Der Zugriff auf den Speicher erfolgt jeweils zellenweise, wir stellen
uns einen Schreib-/Lese-Kopf vor, der auf dem Band positioniert werden
kann, und genau ein Zeichen lesen kann, oder das auf dem Band vorhandene
Zeichen "uberschreiben kann. Um zu Beginn einen wohldefinierten Zustand
zu haben, brauchen wir ein zus"atzliches ``Blank''-Zeichen $\blank$,
welches in $\Gamma$ aber nicht in $\Sigma$ enthalten ist\footnote{W"are
das ``Blank''-Zeichen $\blank$ auch in $\Sigma$, k"onnte man das
leere Band nicht vom Input unterscheiden.}.

Da wir jetzt diesen unendlich grossen Speicher haben, gibt es keinen
Grund mehr, sich vorzustellen, dass der Input Zeichen f"ur Zeichen
in den endlichen Automaten gef"uttert wird. Stattdessen gehen wir
davon aus, dass der Input auf das Band geschrieben wird, der
Schreib-/Lesekopf auf das erste Feld des Input positioniert wird,
und man dann die Maschine arbeiten l"asst.

Die Maschine muss jetzt Zeichen vom Band lesen, und zusammen mit
ihrem aktuellen Zustand entscheiden, in welchem neuen Zustand
sie nach der Verarbeitung dieses Zeichens sein soll, welches Zeichen
an dieser Stelle auf dem Band stehen soll, und ob der Schreib-/Lesekopf
bewegt werden soll. Wir erlauben nur zwei Bewegungen: L und R, insbesondere
ist es nicht zul"assig, dass der Kopf nach der Verarbeitung stehen
bleibt.

Im Gegensatz zu DEA oder Stackautomat ist die Berechnung in diesem
Modell nach dem Ende des Input-Strings noch nicht zu Ende. Die Maschine
kann mehrmals "uber die auf dem Band gespeicherten Daten fahren und
sie immer wieder modifizieren. Sie k"onnte also im Prinzip auch
gar nie aufh"oren zu arbeiten. Daher brauchen wir Zust"ande, die
die Maschine anhalten. Wenn die Maschine angehalten hat, m"ussen wir
zudem wissen, ob das auf das Band geschriebene Input-Wort f"ur die
Sprache akzeptabel war oder nicht.

All dies f"uhrt uns auf folgende formale Definition einer Turing-Maschine
\begin{definition}
\index{Turing-Maschine}
Eine Turing-Maschine ist ein $7$-tupel
$M=(Q,\Sigma,\Gamma,\delta,q_0,q_{\text{accept}},q_{\text{reject}})$,
wobei $Q$, $\Sigma$ und $\Gamma$ endliche Mengen sind mit folgenden
zus"atzlichen Eigenschaften
\begin{compactenum}
\index{Zustand}
\item $Q$ heisst die Menge der Zust"ande
\index{Inputalphabet}
\index{Eingabe-Alphabet}
\index{blank@Blank}
\index{$\blank$|see{Blank}}
\item $\Sigma$ heisst das Inputalphabet, es enth"alt das spezielle
``Blank''-Zeichen $\blank$ nicht.
\index{Bandalphabet}
\item $\Gamma$ ist das Bandalphabet, es gilt $\blank\in\Gamma$ und
$\Sigma\subset\Gamma$.
\index{Uebergangsfunktion@\"Ubergangsfunktion}
\item $\delta\colon Q\times \Gamma\to Q\times\Gamma\times\{\text{L},\text{R}\}$
ist die "Ubergangsfunktion
\index{Startzustand}
\item $q_0\in Q$ ist der Startzustand
\index{Akzeptierzustand}
\item $q_{\text{accept}}\in Q$ ist der Akzeptierzustand
\index{Ablehnungszustand}
\item $q_{\text{reject}}\in Q$ ist der Ablehnungszustand
\end{compactenum}
\end{definition}
Die "Ubergangsfunktion liefert also zu jedem Zustand und zum
Inhalt des Feldes unter dem Schreib-/Lesekopf einen neuen
Zustand, einen neuen Bandinhalt und die Information, ob
der Kopf nach links (L) oder rechts (R) bewegt werden muss.

\subsection{Sprache}
Sei eine Turingmaschine
$M=(Q,\Sigma,\Gamma,\delta,q_0,q_{\text{accept}},q_{\text{reject}})$
gegeben.
Zu $M$ kann man auf folgende Weise eine Sprache konstruieren:
Um zu entscheiden, ob ein Wort $w\in\Sigma^*$
zu der Sprache geh"ort,
schreibt man es auf das Band, platziert den Schreib-/Lesekopf auf
das erste Zeichen und bringt die Maschine in den Zustand $q_0$.
\index{Konfiguration}
Man beschreibt diese Konfiguration der Maschine mit der Zeichenkette
$q_0w$, die Maschine ist im Zustand $q_0$ und der Schreib-/Lesekopf
steht "uber dem ersten Zeichen von $w$.

Mit Hilfe der "Ubergangsfunktion wird dann der n"achste Zustand,
der neue Inhalt des Feldes unter dem Schreib-/Lesekopf und die
Kopfbewegung ermittelt. Auch diesen neuen Zustand kann man wieder
als Konfiguration hinschreiben. Der Schreib-/Lesekopf steht
jetzt wom"oglich mitten im Wort drin, und die Maschine befindet
sich in irgend einem Zustand $q$: $w_1qw_2$.

Nehmen wir an, die Maschine befindet sich zu Beginn eines Zyklus im
Zustand $q_1$ und hat den Kopf auf das Zeichen $a_k$ positioniert.
Die Funktion $\delta$ berechnet daraus ein Tripel $(q_1,b_k,R)$
oder $(q_2,b_k,L)$. Die Konfiguration "andert sich dabei wie
folgt:
\begin{align*}
&\text{Ausgangskonfiguration:}&a_1\dots a_{k-1}&\;q_1\;a_k\dots a_n\\
&\text{"Ubergang mit } \delta(q_1,a_k)=(q_2,b_k,L)&a_1\dots a_{k-2}&\;q_2\;a_{k-1}b_k\dots a_n\\
&\text{"Ubergang mit } \delta(q_1,a_k)=(q_2,b_k,R)&a_1\dots a_{k-1}b_k&\;q_2\;a_{k+1}\dots a_n
\end{align*}

Die Maschine arbeitet so immer weiter, bis einer der Zust"ande $q_{\text{accept}}$
oder $q_{\text{reject}}$ erreicht wird, dann h"alt sie an.
Das Inputwort gilt als akzeptiert, wenn der Zustand $q_{\text{accept}}$
erreicht wurde, es gilt als verworfen, wenn $q_{\text{reject}}$ erreicht
wurde.

\index{Berechnungsgeschichte}
Die Folge von Konfigurationen, die die Turing-Maschine
w"ahrend der Berechnung durchl"auft, nennt man auch
{\em Berechnungsgeschichte}.

\begin{definition}
\index{Sprache!von einer Turing-Maschine erkannte}
Ist $M$ eine Turingmaschine, dann heisst
\[
L(M)=\{w\in\Sigma^*\;|\;\text{$M$ akzeptiert $w$}\}.
\]
die von $M$ erkannte Sprache.
\end{definition}

\begin{definition}
\index{Turing-erkennbar}
Eine Sprache $L$ heisst Turing-erkennbar, wenn es eine Turing-Maschine
$M$ gibt mit $L=L(M)$.
\end{definition}

Die Definition der erkannten Sprache erlaubt, dass die zur Erkennung
verwendete Turing-Maschine auf einigen W"ortern $w\in\Sigma^*$ nicht
anh"alt. Diese W"orter werden nie akzeptiert, und geh"oren
damit nicht zur Sprache. Testet man sie mit $M$, wird die Berechnung
aber nie anhalten, man weiss also eigentlich nie, ob das Wort
nicht zur Sprache geh"ort, oder ob man einfach noch etwas mehr
Geduld braucht.

Gewissheit hat man erst, wenn man sicher sein kann, dass $M$ auf
jedem Input anh"alt.

\begin{definition}
\index{Entscheider}
Ein Entscheider ist eine Turingmaschine, die auf jedem Input $w\in\Sigma^*$
anh"alt. Eine Sprache heisst entscheidbar, wenn ein Entscheider sie
erkennt.
\index{entscheidbar}
\end{definition}
Es ist klar, dass eine Turing-entscheidbare Sprache auch Turing-erkennbar
ist.
Die Definition der Eigenschaft ``Turing-entscheidbar'' unterscheidet sich
von der Definition der Eigenschaft ``Turing-erkennbar'' nur dadurch,
dass bei einer entscheidbaren Sprache die Turing-Maschine auf jedem
beliebigen Input anhalten muss, w"ahrend bei einer nur erkennbaren
Sprache einzelne Input-W"orter auch dazu f"uhren k"onnen, dass die
Turing-Maschine endlos weiterrechnet.

\begin{beispiel}
Die kleinsten Mersenne-Primzahlen sind $3=2^2-1$, $7=2^3-1$, $31=2^5-1$,
andererseits ist $2^{11}-1=2047=23\cdot 89$ keine Primzahl.
Das finden von Mersenneschen Primzahlen ist mit GIMPS (Great Internet
Mersenne Prime Search \cite{skript:gimps}) zu einem Volkssport geworden.
Es ist nicht bekannt, ob es unendlich viele Mersennesche Primzahlen gibt.

Wir bilden jetzt die folgende Sprache "uber dem Alphabet
$\Sigma=\{\texttt{0},\texttt{1}\}$:
\[
L=\{w\in\Sigma^*\,|\,\text{die Anzahl Mersennescher Primzahlen ist gr"osser als $w$}\}.
\]
Eine Turing-Maschinen-Programm, welches pr"ufen muss, ob $w\in L$ ist,
muss herausfinden, ob es mindestens $w$ (interpretiert als Zahl)
Mersennsche Primzahlen gibt.
Dazu m"ussen der Reihe nach alle Zweierpotenzen $2^k$ daraufhin getestet werden,
ob $2^k-1$ eine Primzahl ist.
Wenn mindestens $n$ Mersennsche Primzahlen gefunden wurden, wird $n$
akzeptiert, sonst nicht. Falls $n\in L$ ist, wird der Algorithmus
die Zahl $n$ akzeptieren, die Sprache $L$ ist also Turing-erkennbar.

Falls es nur endlich viele Mersennesche Primzahlen gibt, zum Beispiel
nur $N$, dann wird dieser Algorithmus f"ur jede Zahl $n>N$ nicht
aufh"oren zu laufen.
Der Algorithmus kann ja nicht absch"atzen, ob noch eine Mersennesche
Primzahl kommen wird, ob es also einfach noch etwas mehr Geduld
braucht.
Diese Sprache $L$ ist also nicht Turing-entscheidbar.
\end{beispiel}

\subsection{Zustandsdiagramm}
\index{Graph!gerichteter!beschrifteter!einer Turing-Maschine}
Auch eine Turingmaschine kann man als gerichteten beschrifteten Graphen
darstellen. "Uberg"ange zwischen zwei Zust"anden sind immer begleitet von
einer "Anderung des Inhaltes des Feldes unter dem Schreib-/Lesekopf
und von einer Kopfbewegung. Beide m"ussen mit der Beschriftung
der Pfeile codiert werden. In Anlehnung an die bei den Stackoutomaten
verwendete Notation schreiben wir f"ur den "Ubergang
$\delta(q_1,a)=(q_2,b,\text{R})$
die Notation
\[
\entrymodifiers={++[o][F]}
\xymatrix{
{q_1}\ar[r]^{a\to b,\text{R}}
	&{q_2}
}
\]
und f"ur den "Ubergang
$\delta(q_1,a)=(q_2,b,\text{L})$
\[
\entrymodifiers={++[o][F]}
\xymatrix{
{q_1}\ar[r]^{a\to b,\text{L}}
	&{q_2}
}
\]

\subsection{Beispiel}
Wir konstruieren eine Turingmaschine, welche die bereits als weder regul"ar
noch kontextfrei erkannte Sprache $L=\{0^{2^n}\;|\; n\in\mathbb N\}$
erkennt.

Zweierpotenzen kann man daran erkennen, dass man die Zahl ohne
Rest durch zwei teilen kann, bis nur noch 1 "ubrig bleibt. Genau
diesen Prozess kann man mit einer Turingmaschine auf dem Band
nachbilden. Die Maschine muss also folgendes tun:

Auf dem Input $w$:
\begin{compactenum}
\item Fahre von links nach rechts "uber das Band und streiche jedes zweite $0$.
\item Falls im Schritt 1 das Band genau eine $0$ enthielt, {\it akzeptiere}.
\item Falls im Schritt 1 das Band eine ungerade Zahl und mehr als ein $0$
enthielt, {\it verwerfe}.
\item Fahre mit dem Kopf zur"uck zum linken Ende des Bandes
\item Weiter bei Schritt 1
\end{compactenum}
In jedem Durchgang wird die Anzahl der Nullen halbiert, es sei denn,
sie war nicht gerade. In diesem Fall bleibt nach dem Durchgang eine
ungerade Anzahl $0$ auf dem Band stehen, was beim darauffolgenden Durchgang
in Schritt~3 erkannt wird. Geht die Division immer auf, bleibt am Schluss
nur genau eine $0$ stehen, was im Schritt~2 erkannt wird. Daher funktioniert
dieser Algorithmus.

Um den Algorithmus mit einer Turing-Maschine zu realisieren, brauchen wir
ein neues Zeichen {\tt x}, mit dem wir gestrichene Zeichen markieren
k"onnen. Es ist also $\Gamma = \{0,{\tt x},\blank\}$.
Ausserdem brauchen wir
Zust"ande, mit denen man die Parit"at der Anzahl $0$ bestimmen kann, und
mit denen man die Aktivit"aten in Schritt 4 von den anderen unterscheiden
kann. Schritt $4$ wird zum Beispiel implementiert durch das
Zustandsdiagramm
\[
\entrymodifiers={++[o][F]}
\xymatrix{
{}\ar@(ul,ur)^{\genfrac{}{}{0pt}{1}{0\to 0,\text{L}}{{\tt x}\to{\tt x},\text{L}}}
\ar[r]^{\blank\to\blank,\text{R}}
	&
}
\]
Jede zweite Null streichen wird implementiert durch
\[
\entrymodifiers={++[o][F]}
\xymatrix{
*+\txt{}\ar[d]
\\
\ar@/^/[r]^{0\to{\tt x},\text{R}}
\ar@(ul,dl)_{\tt x\to\tt x,\text{R}}
	&\ar@/^/[l]^{0\to 0,\text{R}}
         \ar@(ur,dr)^{\tt x\to\tt x,\text{R}}
	 \ar[d]^{\blank\to\blank,?}
\\
*+\txt{}
	&{}
}
\]
Dass akzeptiert werden soll, wenn nach dem Ersetzen einer $0$ durch
ein {\tt x} keine weiteren $0$ mehr gefunden werden k"onnen, kann durch
\[
\entrymodifiers={++[o][F]}
\xymatrix{
\ar[r]^{0\to{\tt x},\text{R}}
	&{} \ar@(dl,dr)_{{\tt x}\to{\tt x},\text{R}}
	    \ar[r]^{\blank\to\blank,\text{R}}
		&*++[o][F=]{q_a}
}
\]
ausgedr"uckt werden. Die Schleife beim mittleren Zustand
bildet ab, dass {\tt x}-Zeichen "ubersprungen werden sollen.
Mit solchen Elementen k"onnen wir die Turingmaschine jetzt schrittweise
aufbauen.

Zun"achst stellen wir sicher, dass der Input tats"achlich aus Nullen
besteht:
\[
\entrymodifiers={++[o][F]}
\xymatrix{
*+\txt{}\ar[d]
\\
{q_1}\ar[d]_{\genfrac{}{}{0pt}{1}{\blank\to\blank,\text{R}}{{\tt x}\to{\tt x},\text{R}}}
\\
*++[o][F=]{q_r}
}
\]
Jetzt h"angen wir das Segment an, welches pr"uft, ob es noch genau ein $0$
auf dem Band hat:
\[
\entrymodifiers={++[o][F]}
\xymatrix{
*+\txt{}\ar[d]
\\
{q_1}\ar[d]_{\genfrac{}{}{0pt}{1}{\blank\to\blank,\text{R}}{{\tt x}\to{\tt x},\text{R}}}
	\ar[r]^{0\to{\tt x},\text{R}}
	&{q_2}\ar[d]^{\blank\to\blank,\text{R}}
              \ar@(u,ul)_{{\tt x}\to {\tt x},\text{R}}
\\
*++[o][F=]{q_r}
	&*++[o][F=]{q_a}
}
\]
Falls dies nicht zutrifft, m"ussen weitere Nullen weggestrichen werden,
wobei "uber die Parit"at der noch vorhandenen Nullen Buch gef"uhrt werden
muss. Daf"ur braucht es die beiden Zust"ande $q_3$ und $q_4$:
\[
\entrymodifiers={++[o][F]}
\xymatrix{
*+\txt{}\ar[d]
\\
{q_1}\ar[d]_{\genfrac{}{}{0pt}{1}{\blank\to\blank,\text{R}}{{\tt x}\to{\tt x},\text{R}}}
	\ar[r]^{0\to\blank,\text{R}}
	&{q_2}\ar[d]_{\blank\to\blank,\text{R}}
	      \ar[r]^{0\to{\tt x},\text{R}}
              \ar@(u,ul)_{{\tt x}\to {\tt x},\text{R}}
		&{q_3}\ar@(u,r)^{{\tt x}\to{\tt x},\text{R}}
		      \ar@/^/[d]^{0\to 0,\text{R}}
\\
*++[o][F=]{q_r}
	&*++[o][F=]{q_a}
		&{q_4}\ar@/^/[u]^{0\to {\tt x},\text{R}}
		      \ar@(d,r)_{{\tt x}\to{\tt x},\text{R}}
}
\]
Wenn die Maschine nach dem Abarbeiten aller 0 und {\tt x} auf einen
neuen Blank \textvisiblespace\ trifft, ist sie entweder im Zustand
$q_3$ oder in $q_4$. Im letzten Fall hat sie ein ungerade Anzahl
von 0 gefunden, gem"ass Schritt 3 des Algorithmus muss also verworfen
werden:
\[
\entrymodifiers={++[o][F]}
\xymatrix{
*+\txt{}\ar[d]
\\
{q_1}\ar[d]_{\genfrac{}{}{0pt}{1}{\blank\to\blank,\text{R}}{{\tt x}\to{\tt x},\text{R}}}
	\ar[r]^{0\to\blank,\text{R}}
	&{q_2}\ar[d]_{\blank\to\blank,\text{R}}
	      \ar[r]^{0\to{\tt x},\text{R}}
              \ar@(u,ul)_{{\tt x}\to {\tt x},\text{R}}
		&{q_3}\ar@(u,r)^{{\tt x}\to{\tt x},\text{R}}
		      \ar@/^/[d]^{0\to 0,\text{R}}
\\
*++[o][F=]{q_r}
	&*++[o][F=]{q_a}
		&{q_4}\ar@/^/[u]^{0\to {\tt x},\text{R}}
		      \ar@(d,r)_{{\tt x}\to{\tt x},\text{R}}
		      \ar@/^20pt/[ll]^{\blank\to\blank,\text{R}}
}
\]
Wenn sie dagegen im Zustand $q_3$ ist, muss sie nach links fahren,
bis zum ersten Blank, und dann im Zustand $q_2$ ankommen. Dies erreicht
man mit Hilfe eines weitere Zustands $q_5$.
\[
\entrymodifiers={++[o][F]}
\xymatrix{
*+\txt{}\ar[d]
	&*+\txt{}
		&{q_5}\ar@(ul,ur)^{\genfrac{}{}{0pt}{1}{0\to 0,\text{L}}{{\tt x}\to {\tt x},\text{L}}}
		      \ar[dl]_{\blank\to\blank,\text{R}}
\\
{q_1}\ar[d]_{\genfrac{}{}{0pt}{1}{\blank\to\blank,\text{R}}{{\tt x}\to{\tt x},\text{R}}}
	\ar[r]^{0\to\blank,\text{R}}
	&{q_2}\ar[d]_{\blank\to\blank,\text{R}}
	      \ar[rr]^{0\to{\tt x},\text{R}}
              \ar@(u,ul)_{{\tt x}\to {\tt x},\text{R}}
		&*+\txt{}
			&{q_3}\ar@(u,r)^{{\tt x}\to{\tt x},\text{R}}
			      \ar@/^/[d]^{0\to 0,\text{R}}
			      \ar[ul]_{\blank\to\blank,\text{L}}
\\
*++[o][F=]{q_r}
	&*++[o][F=]{q_a}
		&*+\txt{}
			&{q_4}\ar@/^/[u]^{0\to {\tt x},\text{R}}
			      \ar@(d,r)_{{\tt x}\to{\tt x},\text{R}}
			      \ar@/^20pt/[lll]^{\blank\to\blank,\text{R}}
}
\]
Mit diesem Beispiel haben wir gezeigt, dass die Menge der
Sprachen, die von einer Turingmaschine erkennbar sind, Sprachen
enth"alt, die mit den bisherigen Mitteln nicht erkennbar waren.

\begin{definition}
Die von einer Turingmaschine $M$ erkannten W"orter bilden die
Sprache $L(M)$.
\end{definition}

\section{Varianten von Turing Maschinen}
\rhead{Varianten von Turingmaschinen}
Turingmaschinen sollen als universelles Modell f"ur Computer gelten,
dies ist jedoch nur m"oglich, wenn sich die Menge der erkannten Sprachen
nicht "andert, wenn an dem Modell kleine Ver"anderungen vorgenommen
werden.

Das Band einer Turingmaschine k"onnen wir als den RAM-Speicher
eines modernen Computers interpretieren. Moderne Computer verwenden
jedoch verschiedene Wortl"ange bei der Arbeit mit ihrem Speicher,
was auch das Bandalphabet $\Gamma$ ver"andert. Ein Computer mit
Speicherwortl"ange $l$ hat als Bandalphabet die Menge $\Gamma_l=[2^l]$.

\index{von Neumann-Architektur}
\index{Harvard-Architektur}
Computer mit von Neumann-Architektur verwenden nur einen einzigen RAM-Speicher,
der Programm und Daten enth"alt. Computer mit Harvard-Architektur
verwenden dagegen zwei verschiedene Speicher: einen Programm-Speicher
und einen Datenspeicher. Die beiden Speicher k"onnen sogar verschiedene
Wortl"ange haben. AVR-Microcontroller verwenden zum Beispiel
den Flashspeicher, den sie als 16 Bit breiten Speicher adressieren,
w"ahrend sie das RAM als 8 Bit breiten Speicher adressieren.
ARM-Microcontroller dagegen betrachten RAM und Flash einfach als
verschiedene Bereich in einem einzigen grossen Adressraum.
AVR-Microcontroller haben also ``zwei B"ander'' mit unterschiedlichen
Bandalphabeten, ARM-Microcontroller verwenden dagegen ein einziges Band
mit immer dem gleichen Alphabet.

\subsection{Mehrspurige Turingmaschine}
\index{Turing-Maschine!mehrspurige}
\begin{satz}\label{mehrspurigeturingmaschine}
Jede Sprache, die von einer mehrspurigen Turingmaschine
erkannt werden kann, kann auch von einer einspurigen Turingmaschine
erkannten werden.
\end{satz}

\begin{proof}[Beweis]
Sei $L$ eine Sprache, die von der Turingmaschine $M$ mit $n$ Spuren
erkannten wird. Das Bandalphabet ist $\Gamma$. Wir konstruieren aus
$M$ eine neue Turingmaschine $M'$, welches nur noch eine Spur hat,
jedoch als Bandalphabet die Menge $\Gamma^n$. Dadurch wird die ``Wortbreite''
des Bandes erh"oht, es wird nur noch eine breite Spur verwendet, um die
gleiche Information unterzubringen.
\end{proof}

\subsection{Turingmaschine mit mehreren B"andern}
\index{Turing-Maschine!mit mehreren B\"andern}
\begin{figure}
\begin{center}
%\includegraphics[width=0.9\hsize]{images/turing-2}
\includegraphics{images/turing-2}
\end{center}
\caption{Turingmaschine mit mehreren B"andern\label{multitapetm}}
\end{figure}
\begin{satz}
\label{mehrbandturingmaschine}
Jede Sprache, die von einer Turingmaschine mit mehreren B"andern
(Abbildung~\ref{multitapetm})
erkannt
wird, kann auch von einer Turing-Maschine mit nur einem Band erkannt
werden.
\end{satz}

\begin{proof}[Beweis]
Die Konfiguration einer Turingmaschine $M$ mit $n$ B"andern beinhaltet
ausser dem Inhalt der zus"atzlichen B"ander auch noch die Position
des Schreib-/Lesekopfes jedes einzelnen Bandes. Um diese Information
zu codieren, konstruieren wir eine neue Turingmaschine $M'$ mit
$2n$ Spuren. Spur $2i$ enth"alt dabei die Daten von Band $i$.
Auf Spur $2i-1$ verwenden wir eine spezielle Marke $\uparrow$, um die Position
des Schreib-/Lesekopfes auf Band $i$ zu markieren. Somit l"asst sich
eine Turingmaschine mit $n$ B"andern auf einer Turingmaschine mit
$2n$ Spuren und Bandalphabet $\Gamma\cup\{\uparrow\}$ codieren.
Und nach Satz \ref{mehrspurigeturingmaschine} l"asst sich eine solche wiederum
auf einer Standardturingmaschine simulieren.
\end{proof}
Die geraden Spuren werden also dazu verwendet, die Position des
Schreib-/Lese-Kopfes zu speichern, w"ahrend die ungeraden Spuren
die Daten enthalten:
\begin{center}
\begin{tabular}{c|c|c|c|c|c|c|c|c}
\hline
&a&b&c&d&e&f&c&\\
\hline
& & &$\uparrow$& & & & &\\
\hline
&1&2&3&4&5&6&7&\\
\hline
& & & & & &$\uparrow$& &\\
\hline
\end{tabular}
\end{center}

\subsection{Bandalphabet}
Auch das Bandalphabet hat keinen Einfluss auf die M"oglichkeiten einer
Turingmaschine. Die Festplatte eines modernen Computers ist eigentlich
ein ``Band'' mit Bandalphabet $\{0,1,\blank\}$, erst zus"atzliche
Logik im Controller macht daraus einzelne Bytes, also ein gr"ossers
Bandalphabet.

\begin{satz}
Jede Turingmaschine mit Bandalphabet $\Gamma$ kann auf einer Turingmaschine
mit Bandalphabet $\{0,1,\blank\}$ simuliert werden.
\end{satz}

\begin{proof}[Beweis]
Sei also $M$ eine Turingmaschine mit Bandalphabet $\Gamma$ gegeben. Wir
wollen daraus eine Turingmaschine $M'$ mit dem Alphabet
$\Gamma_0= \{0,1,\blank\}$ konstruieren, die dieselbe
Sprache erkennt. Dazu m"ussen wir die Zeichen aus $\Gamma$ als
Bitfolgen codieren, eine beliebige injektive Abbildung
\[
e\colon\Gamma\to \Gamma_0^l
\]
l"ost dieses Problem, eine solche existiert, wenn $2^l>|\Gamma|$.
Zus"atzlich k"onnen
wir verlangen, dass $e(x)\in \{0,1\}^l$, die Zeichen $\ne\blank$
also ausschliesslich mit $0$ und $1$ codiert werden, und
$e(\blank)=(\blank,\dots,\blank)$.

Mit der Codierung $e$ wird der Bandinhalt jetzt umcodiert. Der Inhalt
$x$
eines Feldes des Bandes von $M$ wird auf $l$ aufeinanderfolgende Felder
des Bandes von $M'$ verteilt, die mit den einzelnen Komponenten
von $e(x)$ gef"ullt werden.

In den folgenden Darstellungen verwenden wir der gr"osseren
"Ubersichtlichkeit halber $\Gamma=\{0,1,2,3\}$, wobei eines der
Zeichen die Rolle von $\blank$ "ubernimmt. Ausserdem ist $l=2$.

Jetzt m"ussen auch die "Uberg"ange in $M$
durch entsprechend erweiterte Konstruktionen in $M'$ ersetzt werden.
In einem Zyklus m"ussen zun"achst die n"achsten $l$ Zeichen vom Band
von $M'$ gelesen werden. Welches Zeichen dieser Bitfolge entspricht,
muss in Zust"anden von $M'$ festgehalten werden. Ausgehend vom Zustand
$q$ von $M$ kann zur Decodierung
des Bandes folgender Automat verwendet werden, der $l$ Bits liest
und den Kopf wieder auf den Anfang des Codewortes zur"uckf"ahrt:
\[
\entrymodifiers={++[o][F]}
\xymatrix{
*+\txt{}
	&*+\txt{}
		&*+\txt{}
			&q\ar[dll]_{0\to 0,\text{R}} \ar[drr]^{1\to 1,\text{R}}
\\
*+\txt{}
	&\ar[dl]_{0\to 0,\text{R}} \ar[dr]^{1\to 1,\text{R}}
		&*+\txt{}
			&*+\txt{}
				&*+\txt{}
					&\ar[dl]_{0\to 0,\text{R}} \ar[dr]^{1\to 1,\text{R}}
						&*+\txt{}
\\
\ar[d]^{.\to .,\text{L}}
	&*+\txt{}
		&\ar[d]^{.\to .,\text{L}}
			&*+\txt{}
				&\ar[d]^{.\to .,\text{L}}
					&*+\txt{}
						&\ar[d]^{.\to .,\text{L}}
\\
\ar[d]^{.\to .,\text{L}}
	&*+\txt{}
		&\ar[d]^{.\to .,\text{L}}
			&*+\txt{}
				&\ar[d]^{.\to .,\text{L}}
					&*+\txt{}
						&\ar[d]^{.\to .,\text{L}}
\\
q_0
	&*+\txt{}
		&q_1
			&*+\txt{}
				&q_2
					&*+\txt{}
						&q_3
}
\]
Die Zust"ande $q_i$ codieren jetzt zus"atzlich, welches
Zeichen $i\in\Gamma$ sich unter dem Schreib-/Lesekopf befindet.

Ein "Ubergang
\[
\entrymodifiers={++[o][F]}
\xymatrix{
q\ar[r]^{a\to b,\text{R}}
	&p
}
\]
muss jetzt "ubersetzt werden in einen "Ubergang von $q_a$ in einen
Zwischenzustand, von dem aus das Codewort von $b$ geschrieben.
wird. Wird $b$ als $b_1\dots b_l$ codiert, wird daraus also
\[
\entrymodifiers={++[o][F]}
\xymatrix{
q_a\ar[r]^{.\to b_1,\text{R}}
	&\ar[r]^{.\to b_2,\text{R}}
		&p
}
\]
Falls dem "Ubergang jedoch eine Kopfbewegung nach links folgt, wie in
\[
\entrymodifiers={++[o][F]}
\xymatrix{
q\ar[r]^{a\to b,\text{L}}
	&p
}
\]
dann muss diese ebenfalls noch angeh"angt werden:
\[
\entrymodifiers={++[o][F]}
\xymatrix{
q_a\ar[r]^{.\to b_1,\text{R}}
	&\ar[r]^{.\to b_2,\text{R}}
		&\ar[r]^{.\to .,\text{L}}
			&\ar[r]^{.\to .,\text{L}}
				&\ar[r]^{.\to .,\text{L}}
					&\ar[r]^{.\to .,\text{L}}
						&p
}
\]
Diese Konstruktion zeigt, dass sich die Turingmaschine $M$ auf einer
Turingmaschine $M'$ mit Bandalphabet $\{0,1,\blank\}$ simulieren
l"asst.
\end{proof}

\subsection{Aufz"ahler}
\begin{figure}
\begin{center}
%\includegraphics[width=\hsize]{images/turing-3}
\includegraphics{images/turing-3}
\end{center}
\caption{Schematische Darstellung eines Aufz"ahlers.\label{turing-aufzaehler}}
\end{figure}
Unser bisheriges Konzept einer Turingmaschine hat keine M"oglichkeit,
Output zu produzieren. Solange die Turingmaschine arbeitet, wissen wir
nicht, welcher Teil des Bandinhaltes m"oglicherweise schon ``fertig''
berechnet ist. Erst wenn sie angehalten hat, weil sie $q_{\text{accept}}$
oder $q_{\text{reject}}$ erreicht hat, ist die Berechnung fertig.

F"ur die Praxis w"unschen wir uns jedoch auch eine M"oglichkeit,
die Maschine ohne Ende weiterlaufen zu lassen, wobei sie immer neue
Resultate produziert. Zum Beispiel k"onnte so eine Maschine der Reihe
nach alle W"orter einer Sprache aufz"ahlen wollen. Zu diesem Zweck
f"ugen wir der Turingmaschine einen Drucker hinzu
(Abbildung~\ref{turing-aufzaehler}). Die Turingmaschine
kann jederzeit ein Wort auf den Drucker schreiben und dann
weiterarbeiten.

\begin{definition}
\index{Aufzahler@Aufz\"ahler}
Eine Turingmaschine mit einem Drucker, die auf einem leeren Band beginnt,
heisst ein Aufz"ahler. Die von einem Aufz"ahler auf dem Drucker ausgegebene
W"orter bilden eine Sprache, die vom Aufz"ahler aufgez"ahlte Sprache.
\end{definition}

Wieder haben wir das Berechnungsmodell erweitert, und es stellt sich
die Frage, ob sich dadurch die Menge der Sprachen erweitert.

\begin{satz}
\index{Turing-erkennbar}
Eine Sprache ist genau dann Turing-erkennbar, wenn sie
von einem Aufz"ahler aufgez"ahlt wird.\end{satz}

\begin{proof}[Beweis]
Da die "Aquivalenz der beiden Modelle zu zeigen ist, sind zwei Implikationen
zu beweisen. Einerseits muss gezeigt werden, dass die von einem
Aufz"ahler aufgez"ahlte Sprache auch von einer Turingmaschine
erkannt werden kann, andererseits muss zu einer Turingmaschine, die
die Sprache erkennt, ein Aufz"ahler konstruiert werden, der die Sprache
aufz"ahlt.

Sei als $A$ ein Aufz"ahler. Wir konstruieren eine Turingmaschine $M$, die
zum Inputwort $w$ folgenden Algorithmus implementiert.
\begin{compactenum}
\item Lasse $A$ laufen. Jedesmal, wenn $A$ ein Wort auf den Drucker schreibt,
vergleiche das Wort mit $w$.
\item Falls das Wort mit $w$ "ubereinstimmt, gehe in den Zustand
$q_{\text{accept}}$.
\end{compactenum}
Falls das Wort zur Sprache geh"ort, wird der Aufz"ahler es fr"uher
oder sp"ater aufz"ahlen, und der Algorithmus wird es akzeptieren.

F"ur die umgekehrte Richtung m"ussen wir zur Turingmaschine $M$
einen Aufz"ahler produzieren.
Ein erster Versuch besteht darin, der Reihe nach alle W"orter aus
$\Sigma^*$ zu produzieren, und jedes mit $M$ zu testen.
Ein Algorithmus, der die W"orter produziert, ist einfach herzustellen,
man produziert zuerst alle W"orter der L"ange 1, dann alle der L"ange 2,
immer lexikographisch geordnet.
Sei also $s_1,s_2,s_3,\dots$ eine Liste aller W"orter von $\Sigma^*$.

Beim Testen der W"orter mit $M$
haben wir die Schwierigkeit, dass $M$ auf einem Inputwort
m"oglicherweise nicht anh"alt. Wir m"ussen also mit einem Trick
simulieren, dass unsere Algorithmus nicht in einer ``Endlosschleife''
in $M$ stecken bleibt. Wir lassen $M$ daher nur jeweils f"ur einige
Schritte laufen.
Genauer:
\begin{compactenum}
\item F"ur $n=1,2,3,\dots$ f"uhre die folgenden zwei Schritte aus:
\item Lasse $M$ auf jedem Wort $s_i, i \le n$ w"ahrend $n$ Schritten
laufen.
\item Falls $M$ das Wort $s_i$ akzeptiert, schreibe es auf den Drucker.
\end{compactenum}
Dieser Algorithmus verhindert, dass $M$ in eine ``Endlosschleife''
ger"at, und druckt alle W"orter aus, von der Turingmaschine erkannt
werden, sogar unendlich oft.
\end{proof}

\subsection{Nicht deterministische Turingmaschinen}
\index{Turing-Maschine!nicht deterministische}
Sowohl bei endlichen Automaten wie auch bei Stackautomaten war
Nichtdeterminismus ein Konzept, welches die Formulierung eines
Automaten wesentlich vereinfachen konnte, ohne die M"oglichkeiten
zu ver"andern. Die einzige "Anderung in der Definition ist die
Definition von $\delta$, welche nicht mehr Werte in
$Q\times \Gamma\times\{\text{L},\text{R}\}$ annimmt, sondern
in der Potenzmenge:
\[
\delta\colon Q\times\Gamma\to
P(Q\times \Gamma\times\{\text{L},\text{R}\}).
\]
Die Berechnung muss in jedem Schritt eine der M"oglichkeiten
aus $\delta(q,a)$ ausw"ahlen.
Eine deterministische Turingmaschine ist offenbar auch eine
nichtdeterministische Turingmaschine, die den nichtdeterminismus
gar nicht ausn"utzt, also $|\delta(q,a)|=1$.

\begin{satz}
\label{nichtdeterministischeturingmaschine}
Jede nichtdeterministische Turingmaschine ist "aquivalent zu einer
deterministischen Turingmaschine.
\end{satz}

\begin{proof}[Beweis]
Die Behauptung ist bewiesen, wenn wir zu einer nicht deterministischen
Turingmaschine $M$ eine deterministische Turingmaschine konstruiert haben,
die die gleichen W"orter erkennt.

Die nicht deterministische Maschine kann viele verschiedene Berechnungswege
verwenden, um ein Wort zu akzeptieren.
Es ist aber auch m"oglich, dass ein solcher Weg in eine Endlosschleife f"uhrt.
Daher k"onnen wir nicht einfach die nichtdeterministische Turingmaschine
auf einem Berechnungsweg laufen lassen, wir w"urden nie mit einem
anderen Weg beginnen, der m"oglicherweise das Wort akzeptiert.
Wir d"urfen also die Maschine immer nur einige Schritte laufen lassen,
und m"ussen dann die anderen Wege durchprobieren.

Wir konstruieren jetzt eine Turingmaschine mit drei B"andern, die die
nichtdeterministische Turingmaschine simuliert. Das erste Band
enth"alt das Input-Wort $w$.
Das zweite Band dient als Arbeitsband f"ur die Maschine $M$. Das
dritte Band hat Bandalphabet $S=Q\times \Gamma\times\{\text{L},\text{M}\}$
und speichert die nicht deterministischen Auswahlen aus $\delta(q,a)$,
die  in jedem Schritt n"otig sein k"onnen. Eine Folge von Auswahlen
ist ein String aus $S^*$. Wir wissen bereits, dass die Strings aus $S^*$
aufgez"ahlt werden k"onnen.

Auf dieser Maschine f"uhren
wir jetzt folgenden Algorithmus aus:

\begin{compactenum}
\item f"ur $n=1,2,3,\dots$ f"uhre die Schritte 2 bis 5 aus:
\item Schreibe den String $s_i$  auf Band 3,
und f"uhre damit die Schritte 3 bis 5 aus.
\item Kopiere $w$ von Band 1 auf Band 2
\item Lasse $M$ f"ur $n$ Schritte auf Band 2 laufen, verwende f"ur die
nichtdeterministischen Entscheidungen in jedem Schritt das entsprechende
Feld auf Band 3.
\item Falls $M$ akzeptiert, akzeptiere.
\end{compactenum}
Dieser Algorithmus probiert nacheinander alle m"oglichen Berechnungswege
durch, l"asst $M$ aber immer nur eine beschr"ankte Anzahl Schritte lang
rechnen. Falls $w$ akzeptiert werden kann, wir der dazu n"otige Berechnungsweg
irgendwann auf Band 3 auftauchen, und in entsprechend vielen Schritten wird
das Wort akzeptiert werden.
\end{proof}

\begin{satz}
Eine Sprache ist Turing-erkennbar, wenn sie von einer nicht deterministischen
Turingmaschine erkannt werden kann.
\end{satz}

Auch eine nicht deterministische Turingmaschine kann ein Entscheider sein.
Da ein Entscheider aber niemals in eine ``Endlosschleife'' geraten darf,
m"ussen wir verlangen, dass alle m"oglichen Berechnungsabl"aufe
irgenwann terminieren.

\begin{definition}
\index{Entscheider}
\index{Berechnungsgeschichte}
Eine nicht deterministische Turingmaschine ist ein Entscheider, wenn
jede m"ogliche Berechnungsgeschichte terminiert.
\end{definition}

\begin{satz}
Eine Sprache ist entscheidbar, wenn sie von einer nicht deterministischen
Turingmaschine entschieden wird.
\end{satz}

Nichtdeterminismus "andert also die erkennbaren oder entscheidbaren Sprachen
nicht. Doch der Algorithmus l"asst bereits vermuten, dass die Simulation
einer nichtdeterministischen Turingmaschine auf einer deterministischen
Maschine sehr zeitaufwendig ist. Wenn man also eine Performance-Vorgabe
macht, kann die Menge der von einer deterministischen Turingmaschine
erkennbaren Sprachen deutlich kleiner sein. Dies wird uns im Kapitel
\ref{chapter-komplexitaet} besch"aftigen.

\section{Berechenbarkeit}
\rhead{Berechenbarkeit}
In diesem Abschnitt wollen wir kl"aren, welche Dinge "uberhaupt
berechnet werden k"onnen. Dabei sehen wir die Tatsache, dass f"ur
die Darstellung einer reellen Zahl eine unendliche Folge von Ziffern
n"otig sein kann, nicht als Hindernis an. Wir nennen eine Zahl
berechenbar, wenn wir ein Programm schreiben k"onnen, welches
m"oglicherweise unendlich lange l"auft und dabei eine Stelle der
Zahl nach der anderen liefern kann. Die Turingmaschine
funktioniert also als Aufz"ahler, der die Stellen der zu berechnenden
Zahl aufz"ahlen soll.  In diesem Sinne ist die
Zahl $1/3=0.3333\dots$ berechenbar, aber auch die Zahlen $\sqrt{2}$, $\pi$,
$e$ und weitere. Wir werden allerdings sehen, dass die meisten
Zahlen gar nicht berechenbar sind.

\subsection{Abz"ahlbar und "uberabz"ahlbar}
Unendlich ist nicht gleich unendlich, die Mengen $\mathbb N$ und
$\mathbb R$ haben zwar beide unendlich viele Elemente, dennoch
gibt es keine bijektive Abbildung $\mathbb N\to \mathbb R$.

\begin{definition}
\index{Machtigkeit@M\"achtigkeit}
Zwei Mengen $A$ und $B$ heissen gleich m"achtig, wenn es eine bijektive
Abbildung $A\to B$ gibt.
\end{definition}

Es ist bekannt, dass $\mathbb Q$ und $\mathbb N$ gleich m"achtig sind,
aber die Menge $\mathbb R$ scheint wesentlich gr"osser
zu sein als $\mathbb N$.
Diese Unterschiede zwischen verschiedenen
unendlichen Mengen wird illustriert von der nachfolgenden Geschichte,
die David Hilbert als Illustration zu erz"ahlen pflegte.

\subsubsection{Hotel ``Unendlich''}
\index{Hotel ``Unendlich''}
\index{Hilbert!-Hotel}
In einem fernen Land gibt es ein Hotel, welches Hotel ``Unendlich''
genannt wird. Unsere Geschichte beginnt kurz nachdem der Nachtportier
die Nachtschicht "ubernommen hat. Ein versp"ateter Gast meldet sich
am Eingang, w"ahrend der Nachtportier damit besch"aftigt ist,
die herumlungernden Besoffenen
aus den naheliegenden Pubs zu vertreiben, die auch gerade geschlossen haben.
Der Gast m"ochte im Hotel "ubernachten, doch der Nachtportier zeigt sich
unnachgiebig: ``Du kommst hier nicht rein!''

Zum Gl"uck hat der Hotelmanager den Wortwechsel mitbekommen und greift
ein. Selbstverst"andlich haben wir noch ein Zimmer f"ur jeden noch so
sp"aten Gast. Der Nachportier protestiert, es seien doch alle Zimmer
belegt. ``Kein Problem'' sagt der Manager, und greift zum Mikrofon
der Sprechanlage. Er bittet alle G"aste, aus ihrem aktuellen Zimmer
ins n"achste Zimmer umzuziehen. Dieses Ansinnen zu sp"ater Stunden
verursacht zwar bei einigen G"asten etwas Unmut, aber jeder ist damit
einverstanden, denn h"atten sie sich selbst versp"atet, w"aren sie ja
auch auf das Entgegenkommen der anderen G"aste angewiesen. Auf diese
Weise wird das Zimmer mit der Nummer $0$ frei f"ur den neu angekommenen
Gast.

Kurze Zeit sp"ater kommt eine versp"atete Reisegruppe aus dreissig
Touristen an, die ebenfalls
Einlass verlangt. Der Nachtportier will nicht nochmals einen schlechten
Eindruck hinterlassen und ruft den Manager. Dieser greift
wieder zum Mikrofon, und bittet wieder alle G"aste, diesmal aus
ihrem Zimmer mit der Nummer $n$ ins Zimmer mit der Nummer $n+30$
umzuziehen. So werden die Zimmer mit den Nummern $0$ bis $29$ frei,
Platz genug f"ur die neu angekommene Reisegruppe.

Diese Nacht ist ziemlich viel los im Hotel ``Unendlich'', denn
kaum waren die dreissig G"aste untergebracht, f"ahrt ein Hilbert-Bus
vor: ein Bus mit unendlich vielen Sitzpl"atzen, numeriert mit den
nat"urlichen Zahlen. Der Nachtportier ist schockiert: Unendlich
viele G"aste, ohne Reservation. Dass man f"ur endlich viele G"aste
immer noch Platz schaffen kann, hat er inzwischen verstanden, aber
f"ur unendlich viele G"aste sei das unm"oglich, stellt er fest, und
will die Gesellschaft wegschicken. Doch der Hotelmanager m"ochte
sich das Gesch"aft nicht entgehen lassen. Er greift erneut zum
Mikrofon und bittet die G"aste, vom Zimmer $n$ ins Zimmer $2n$ umzuziehen.
So werden alle Zimmer mit ungeraden Nummern frei, Platz genug f"ur alle
Passagiere des Hilbert-Buses. Nur die Stimmung unter den Stammg"asten
hat sich bereits deutlich verschlechtert.

Schliesslich fahren gleich dreissig Hilbertbusse vor. Ein Hilbert-Jumbo
hatte sich versp"atet, und so dass die Ferieng"aste erst sehr sp"at
mit ihren Hilbertbussen vom Flughafen zum Hotel abfahren konnten.
Der Nachtportier hat schon einen Verdacht, dass der Manager auch
hierf"ur eine L"osung bereit hat. Tats"achlich: per Mikrofondurchsage
gibt er wieder alle ungerade Zimmer frei. Dann bringt er die Neuank"ommlinge
darin unter, und zwar sch"on nacheinander immer je einen aus jedem Bus,
also in die ersten dreissig freien Zimmer die jeweils ersten G"aste
aus jedem Bus, dann die zweiten aus jedem Bus in die zweiten dreissig
freien Zimmer und so weiter.

Man ahnt es schon, in dieser Nacht gibt es f"ur die G"aste im Hotel
``Unendlich'' keine Ruhe. Als n"achstes findet sich ein Hilbert-Konvoi
vor dem Hotel ein. Also ein Folge von Hilbert-Bussen, die mit nat"urlichen
Zahlen numeriert waren. Der Nachtportier hat gar nicht erst gewagt,
den Neuank"ommlingen abschl"agigen Bescheid zu geben, sondern gleich
den Manager gerufen. Der muss zwar auch einen Moment nachdenken, findet
dann aber eine L"osung. Zun"achst wendet er nochmals den Trick an,
mit dem er jetzt schon mehrmals die ungeraden Zimmer frei bekommen hat.
Dabei kommt es auch zu w"usten Szenen mit G"asten, die jetzt schon vier mal
geweckt worden sind, und sich ihre Nachtruhe nicht von der Geldgier des
Managers ruinieren lassen wollen.

Die G"aste aus dem Hilbert-Konvoi m"ussen sich in Einerkolonnen auf
dem Vorplatz aufstellen, der Manager rief sie daraufhin in der
eingezeichneten Reihenfolge ab. Auch dies tr"agt nicht  unbedingt
zur Zufriedenheit der G"aste bei, hinter vorgehaltener Hand wird
"uber den ``Kasernenton'' und ``Zust"ande wie in einem Konzentrationslager''
geschimpft.
Aber jeder kommt fr"uher oder sp"ater dran,
am Ende hat jeder Gast ein Zimmer, was die Gem"uter wieder
etwas beruhigt.

\[
\xymatrix{
\ar[r]
\cdot	     &\cdot\ar[dl] &\cdot\ar[r] &\cdot\ar[dl]  &\cdot\ar[r] &\cdot\ar[dl] &\cdot
\\
\cdot\ar[d]  &\cdot\ar[ur] &\cdot\ar[dl] &\cdot\ar[ur] &\cdot\ar[dl]&\cdot        &\cdot
\\
\cdot\ar[ur] &\cdot\ar[dl] &\cdot\ar[ur] &\cdot\ar[dl] &\cdot       &\cdot        &\cdot
\\
\cdot\ar[d]  &\cdot\ar[ur] &\cdot\ar[dl] &\cdot        &\cdot       &\cdot        &\cdot
\\
\cdot\ar[ur] &\cdot        &\cdot        &\cdot        &\cdot       &\cdot        &\cdot
}
\]

Ein Ereignis steht allerdings noch bevor, und hier sollte sich die
Geldgier des Managers r"achen. Gegen morgen n"amlich f"ahrt
ein voller Cantor-Bus vor. Dieses hochmoderne Verkehrsmittel hat Sitze,
\index{Cantor!-Bus}
die mit den reellen Zahlen angeschrieben sind. Nat"urlich wollen auch
diese G"aste im Hotel ``Unendlich'' untergebracht werden. Aber so sehr
sich der Manager auch anstrengt, in seinem Hotel kann er diese
G"aste nicht unterbringen. Daraufhin wird er zwar vom Verwaltungsrat
entlassen, der noch weniger Mathematik versteht, was ihm dank "uppiger
Abgangsentsch"adigung jedoch nicht weiter Sorgen macht. Unbest"atigten
Ger"uchten zufolge soll er jetzt bei einer Hotelkette ``Cantor Hotels''
arbeiten, die mit mit dem Spruch ``Present in uncountable locations
throughout the Universe'' f"ur sich wirbt.

\subsubsection{Lehren aus der Geschichte}
Eine unendliche Menge ist offenbar so gross, dass man darin immer noch
Platz genug f"ur eine Kopie der ganzen Menge finden kann. Oder anders
herum: endliche Mengen sind solche, in denen man niemals Platz finden
k"onnte:

\begin{satz}
Eine Menge ist endlich, wenn jede injektive Abbildung auch surjektiv ist.
\end{satz}

In der Geschichte kam eine ganze Reihe von unendlichen Mengen auf, die
alle in $\mathbb N$ untergebracht werden konnten, die also nicht
gr"osser waren als $\mathbb N$ selbst, und das obwohl sie aus nat"urlichen
Zahlen konstruiert worden waren. Mit $\mathbb N$ vergleichbare Mengen
bilden also eine robuste Klasse von Mengen.

\begin{definition}
\index{abzahlbar unendlich@abz\"ahlbar unendlich}
\index{uberabzahlbar unendlich@\"uberabz\"ahlbar unendlich}
Eine unendliche Menge $A$ heisst abz"ahlbar unendlich, wenn sie
gleichm"achtig ist wie die nat"urlichen Zahlen. $A$ heisst
"uberabz"ahlbar unendlich, wenn es keine Bijektion zwischen
$\mathbb N$ und $A$ gibt.
\end{definition}
Bildlich gesprochen ist eine abz"ahlbare Menge eine solche,
bei der sich die Elemente in eine mit den nat"urlichen Zahlen numerierte
Liste einordnen lassen.

Auch die Menge der Paare $(k,l)\in \mathbb N^2$ ist abz"ahlbar.
Ein Gast aus dem Hilbert-Konvoi wird durch seine Busnummer $k$ und
seine Platz-Nummer $l$ identifiziert, der Konvoi hat also gleich
viele G"aste wie $\mathbb N^2$ Paare enth"alt. Und alle diese Paare
passen in die Menge $\mathbb N$, also die Zimmer des Hotels ``Unendlich''
hinein. Somit sind $\mathbb N^2$ und $\mathbb N$ gleich m"achtig.

Die Geschichte hat exemplarisch den folgenden Satz gezeigt:

\begin{satz}Die Vereinigung von endlich vielen abz"ahlbaren
Mengen ist abz"ahlbar. Das kartesische Produkt zweier abz"ahlbarer
Mengen ist abz"ahlbar.
\end{satz}

\begin{proof}[Beweis]
Sei $A_1,\dots,A_n$ eine endliche Familie abz"ahlbarer Mengen,
je mit einer Funktion $f_i\colon \mathbb N\to A_i$, dann
k"onnen wir eine Abz"ahlfunktion f"ur die Vereinigung konstruieren:
\[
f\colon \mathbb N\to\bigcup_{i=1}^n A_i\colon nk+l \mapsto f_l(k)
\]
wobei wir verwenden, dass wir jede Zahl eindeutig als $nk+l$
schreiben k"onnen mit geeigneten Zahlen $k$ und $l$.

F"ur das kartesische Produkt m"ussen wir zeigen, dass sich die
Menge der Paare $\mathbb N^2$ aufz"ahlen l"asst, was mit dem
Diagonaltrick des Hotelmanagers geschehen kann.
\end{proof}

Daraus folgen jetzt weitere S"atze "uber die Kardinalit"at
der bekannten Zahlmengen.

\begin{satz}
Die Mengen $\mathbb Z$ und $\mathbb Q$ sind abz"ahlbar unendlich.
\end{satz}

\begin{proof}[Beweis]
Die Menge $\mathbb Z$ ist abz"ahlbar, da sie die Vereinigung
von zwei Mengen ist, die offensichtlich abz"ahlbar sind:
\[
\mathbb Z = \mathbb N\cup \{-n\;|\;n\in\mathbb N\}.
\]
Die rationalen Zahlen k"onnen durch Br"uche dargestellt werden,
also durch Paare $(p,q)$ von ganzen Zahlen:
\[
\mathbb Q=\left\{\left.\frac{p}{q}\;\right|\;p\in \mathbb Z,q\in\mathbb N\setminus\{0\}\right\}
\]
Da sich einige Br"uche noch k"urzen lassen, ist $\mathbb Q$ noch
kleiner als $\mathbb N^2$, aber insbesondere abz"ahlbar.
\end{proof}

\begin{satz}
Die Menge $\mathbb R$ ist "uberabz"ahlbar unendlich.
\end{satz}

\begin{proof}[Beweis]
Wir f"uhren den Beweis mit Hilfe eines Widerspruchs. Nehmen wir
an, $\mathbb R$ w"are abz"ahlbar unendlich. Dann m"ussten auch
die reellen Zahlen zwischen $0$ und $1$ abz"ahlbar sein,  es g"abe also eine
Liste all dieser Zahlen, wir schreiben die Zahlen
dieser Liste in Dezimaldarstellung
\begin{align*}
0.&r_{11}r_{12}r_{13}r_{14}\dots\\
0.&r_{21}r_{22}r_{23}r_{24}\dots\\
0.&r_{31}r_{32}r_{33}r_{34}\dots\\
0.&r_{41}r_{42}r_{43}r_{44}\dots\\
&\vdots
\end{align*}
Jetzt konstruieren wir eine Zahl $x$, die nicht in dieser Liste vorkommen
kann. Um die $k$-te Stelle $x_k$ von $x$ zu konstruieren, betrachten
wir die $k$-te Stelle der $k$-ten Zahl $r_{kk}$, und setzen
\[
x_k=\begin{cases}
r_{kk}-1&\qquad r_{kk}>0\\
5&\qquad r_{kk}=0
\end{cases}
\]
$x_k$ ist also verschieden von $r_{kk}$, und zwar f"ur jedes $k$.
Die Ziffer $9$ kommt in $x$ nicht vor, es kann also keine Zahl
mit lauter $9$ am Ende sein, die gleichbedeutend mit einer anderen
Zahl ist, die m"oglicherweise bereits in der Liste ist.
Also unterscheidet sich $x$ von jeder der Zahlen aus der Liste, $x$
kommt also in der Liste nicht vor. Wir hatten angenommen, dass die
Liste alle reellen Zahlen zwischen $0$ und $1$ umfasst, dieser
Widerspruch zeigt, dass es keine solche Liste geben kann, die
reellen Zahlen sind also "uberabz"ahlbar.
\end{proof}

\index{Cantor!Georg}
Georg Cantor (1845-1918) leistete wichtige Beitr"age zur Mengelehre,
unter anderem entdeckte er den Unterschied der M"achtigkeit von
nat"urlichen und reellen Zahlen. Daher haben wir in der Geschichte
den Bus, dessen Sitzpl"atze mit reellen Zahlen angeschrieben waren,
als Cantor-Bus bezeichnet.

Verwandt damit und mit einem "ahnlichen Beweis kann man auch einsehen,
dass die Potenzmenge einer abz"ahlbaren Menge "uberabz"ahlbar ist.

\begin{satz}\label{powersetuncountable}
Ist $A$ eine abz"ahlbar unendliche Menge, dann ist $P(A)$
"uberabz"ahlbar.
\end{satz}

\begin{proof}[Beweis]
Da $A$ abz"ahlbar ist gibt es eine bijektive Abbildung
$i\mapsto a_i\in A$. Nehmen wir an, es g"abe eine bijektive
Abbildung von $\mathbb N$ in $P(M)$, also $i\mapsto A_i\subset A$.
Dann kann man wie folgt eine Menge bilden, die unter den Mengen
$A_i$ nicht vorkommt.
\[
a_i\in B\quad\Leftrightarrow\quad a_i\not\in A_i
\]
Die Menge $B$ unterscheidet sich von jeder Menge $A_i$. Falls
$A_i$ das Elemente $a_i$ enth"alt, enth"alt $B$ es nicht und
umgekehrt. Somit ist $B\ne A_i\forall i$, im Widerspruch zur Annahme
dass $i\mapsto A_i$ eine Bijektion ist.
\end{proof}

Jetzt k"onnen wir auch einige Mengen aus der bisher betrachteten
Theorie der Sprachen auf ihre M"achtigkeit untersuchen.

\begin{satz}
$\Sigma^*$ ist abz"ahlbar unendlich. Die Menge aller Sprachen "uber dem
Alphabet $\Sigma$ ist "uberabz"ahlbar unendlich.
\end{satz}

\begin{proof}[Beweis]
Eine Aufz"ahlung von $\Sigma^*$ kann man konstruieren, indem man
die W"orter von $\Sigma^*$ der L"ange nach sortiert, und innerhalb
der W"orter gleicher L"ange die lexikographische Ordnung verwendet.
Dazu braucht man nat"urlich eine Anordnung der Zeichen des Alphabets $\Sigma$,
da dieses aber endlich ist, kann man immer eine solche Anordnung finden.

Die Menge aller Sprache ist die Potenzmenge der Menge $\Sigma^*$, welche
abz"ahlbar unendlich ist. Nach Satz \ref{powersetuncountable} ist
$P(\Sigma^*)$ "uberabz"ahlbar unendlich.
\end{proof}

\begin{satz}\label{countablefinite}
Eine abz"ahlbare Vereinigung $\bigcup_{i\in\mathbb N}A_i$ von endlichen
Mengen $(|A_i|<\infty)$ ist abz"ahlbar.
\end{satz}

\begin{proof}[Beweis]
Endliche Mengen kann man immer abz"ahlen, also kann man eine Abz"ahlung
von $\bigcup_{i\in\mathbb N}A_i$ einfach dadurch konstruieren, dass
man zuerst die Elemente von $A_0$ durchz"ahlt, dann die von $A_1$, und
so weiter. Weil keine der Mengen $A_i$ unendlich ist, kommt jede
Menge irgend wann dran.
\end{proof}

\begin{satz} Sei $\Sigma$ ein festes Alphabet. Dann sind die folgenden
Mengen alle abz"ahlbar unendlich:
\begin{enumerate}
\item Die Menge aller deterministischen endlichen Automaten.
\item Die Menge aller nichtdeterministischen endlichen Automaten.
\item Die Menge der regul"aren Sprachen.
\item Die Menge aller kontextfreien Grammatiken.
\item Die Menge aller kontextfreien Sprachen.
\item Die Menge aller Stackautomaten.
\item Die Menge aller Turingmaschinen.
\end{enumerate}
\end{satz}

\begin{proof}[Beweis]
\begin{enumerate}
\item Es gen"ugt zu zeigen, dass die Menge der deterministischen
endlichen Automaten mit $k$ Zust"anden endlich ist, dann ist nach
Satz \ref{countablefinite} auch die Vereinigung abz"ahlbar. Ein
deterministischer endlicher Automat ist aber durch die Tabellendarstellung
gegeben. In der Tabelle sind $|Q|\cdot|\Sigma|$ Felder mit Zust"anden
zu besetzen, daf"ur gibt es $|Q|^{|Q|\cdot|\Sigma|}$ M"oglichkeiten.
F"ur jede solche M"oglichkeit ist ausserdem festzulegen, welche Zust"ande
Akzeptierzust"ande sind, das sind $|P(Q)|=2^{|Q|}$ M"oglichkeiten. Es gibt
also
\[
|Q|^{|Q|\cdot|\Sigma|}\cdot 2^{|Q|}
\]
DEAs mit $|Q|$ Zust"anden.
\item Obwohl in der Definition der nichtdeterministischen endlichen
Automaten auch die Potenzmenge vorkommt, wird dadurch die Menge
noch nicht "uber\-abz"ahlbar. Der Funktionswert der "Ubergangsfunktion
$\delta$ ist ja immer eine Teilmenge einer endlichen Menge, die Potenzmenge
einer endlichen Menge ist aber auch endlich. Es "andert sich am Argument
nur, dass in die Tabelle eines Automaten mit $|Q|$ Zust"anden und
$|\Sigma|$ Zeichen im Alphabet an $|Q|\cdot|\Sigma|$ Stellen
nicht $|Q|$ verschieden Objekte einf"ullen lassen, sondern $2^{|Q|}$.
An der Abz"ahlbarkeit "andert dies nichts.
\item Die regul"aren Sprachen werden von deterministischen endlichen
Automaten akzeptiert, es gibt also eine surjektive Abbildung von
den deterministischen endlichen Automaten auf die regul"aren Sprachen,
eine Aufz"ahlung der deterministischen endlichen Automaten liefert also
auch automatisch eine Aufz"ahlung der regul"aren Sprachen.
\item Es gibt nur endliche viele Grammatiken mit $n$ verschiedenen
Variablen und h"ochstens $n$ Zeichen langen rechten Seiten der Regeln,
einfach weil es nur endlich viele rechte Seiten gibt. Also ist die
Menge aller kontextfreien Grammatiken abz"ahlbar.
\item Kontextfreie Sprachen werden von einer kontextfreie Grammatik
erzeugt. Die Abbildung $G\mapsto L(G)$ macht aus einer Aufz"ahlung
der kontextfreien Grammatiken eine Aufz"ahlung der kontextfreien Sprachen.
\item Stackautomaten sind nichtdeterministische endliche Automaten mit
zus"atzlicher Beschriftung der "Ubergangspfeile. Trotzdem bleibt die
Anzahl der Stackautomaten mit $k$ Zust"anden endlich, also ist die
Menge der Stackautomaten abz"ahlbar.
\item Turingmaschinen sind im wesentlichen deterministsiche endliche Automaten
mit zus"atzlichen Beschriftungen der Pfeile. Wie in 1.~gibt es nur endlich
viele Turingmaschinen mit $k$ Zust"anden, also sind die Turingmaschinen
abz"ahlbar unendlich.
\end{enumerate}
\end{proof}

\subsection{Nicht berechenbare Zahlen}

\begin{satz}
Die rationalen Zahlen sind berechenbar.
\end{satz}

\begin{proof}[Beweis]
Der Algorithmus der schriftlichen Division erlaubt alle Stellen
zu finden, man kann sich gut vorstellen, dass er sich auf einer
Turingmaschine als Aufz"ahler implementieren l"asst. Z"ahler und
Nenner werden zu Beginn auf das Band geschrieben, der Algorithmus
berechnet dann alle Stellen des Quotienten. Also sind die rationalen
Zahlen berechenbar.
\end{proof}

\begin{satz}Die algebraischen Zahlen, also die Nullstellen von
Polynomen mit rationalen Koeffizienten sind berechenbar.
\end{satz}

\begin{proof}[Beweis]
Zun"achst k"onnen wir das Polynom mit dem gemeinsamen Nenner der
Koeffizienten multiplizieren und erhalten ein Polynom mit ganzzahligen
Koeffizienten.
Die Nullstellen eines Polynoms k"onnen mit dem Newton-Algorithmus
mit beliebiger Genauigkeit bestimmt werden. Dazu muss nur eine
Sch"atzung $\hat x_0$ f"ur die Nullstelle bekannt sein, dann k"onnen
mit der Iteration
\[
x_{n+1}=x_n-\frac{f(x_n)}{f'(x_n)},\quad x_0 = \hat x_0
\]
immer genauere Approximationen berechnet werden. Alle Operationen
in der Iterationsformel sind Operationen in rationalen Zahlen, sind
also berechenbar.
\end{proof}

\begin{satz}
Die Menge der nicht berechenbaren Zahlen ist "uberabz"ahlbar.
\end{satz}

\begin{proof}[Beweis]
Die berechenbaren Zahlen sind abz"ahlbar. Wir k"onnen n"amlich
eine Liste aller Turingmaschinen erstellen, zun"achst schreiben
wir die Turingmaschinen mit nur einem Zustand hin, dann all jene
mit genau zwei Zust"anden u.\,s.\,w. Da jede Turingmaschine genau eine
Zahl berechnen kann, ist die Menge der berechnenbaren Zahlen abz"ahlbar,
die Menge der reellen Zahlen ist also "uberabz"ahlbar, daher m"ussen die
nicht berechenbaren Zahlen auch "uberabz"ahlbar sein.
\end{proof}

\subsection{Das 10.~Hilbertsche Problem}
\index{Hilbert!David}
\index{Hilbertsche Probleme}
Im Jahre 1900 hielt der deutsche Mathematiker David Hilbert am
\index{Internationaler Kongress}
internationalen Mathematikerkongress in Paris einen ber"uhmten Vortrag,
in dem er eine Reihe von Problemen zusammenstellte, deren L"osung
nach seiner Meinung die Mathematik im zwanzigsten Jahrhundert
entscheidend voranbringen w"urden. Einige dieser Probleme wurden
inzwischen gel"ost, andere, darunter die Riemannsche Vermutung,
sind immer noch offen.

Von besonderem Interesse f"ur unser Thema war das zehnte Problem:
Gibt es ein Verfahren, das f"ur eine beliebige diophantische Gleichung
entscheidet, ob sie l"osbar ist?

Schon die Problemstellung wirft einige Fragen auf.
Zun"achst zu den Begriffen:
\index{Gleichung!diophantische}
eine diophantische Gleichung ist eine
Polynomgleichung mit mehreren Variablen aber ausschliesslich
ganzzahligen Koeffizienten. Die Gleichung
\[
x^2+y^2-z^2=0
\]
ist eine diophantische Gleichung, und es ist auch bekannt, dass sie
L"osungen hat, zum Beispiel $x=3$, $y=4$ und $z=5$. Seit wenigen
Jahren ist auch bekannt, dass
\[
x^n+y^n-z^n=0,
\]
ebenfalls eine diophantische Gleichung, nur ganz wenige L"osungen hat.
Dies ist die ber"uhmte Fermatsche Vermutung, die Andrew Wiles 1995
\index{Fermatsche Vermutung}
\index{Wiles, Andrew}
vollst"andig bewiesen hat.

Die schwierigerere Frage aber ist: Was f"ur eine Art von Verfahren
ist gemeint? Hilbert hat keine Definition gegeben. Heute w"urden
wir wohl fragen, ob ein Algorithmus angegeben werden kann, mit dem
die Frage entschieden werden kann. Doch das f"uhrt uns nur wieder
auf die Frage nach einer  mathematisch strengen Definition was
ein Algorithmus ist.

\index{Turing, Alan}
Im Jahre 1936 gab Alan Turing eine Antwort: ein Algorithmus ist eine
Rechenvorschrift, die sich mit einer Turingmaschine implementieren
l"asst. Probleme, f"ur die es einen L"osungsalgorithmus gibt, sind
also solche, deren Inputdaten man als Wort auf das Band einer
Turingmaschine schreiben kann, die Turingmaschine verarbeitet die
Problembeschreibung, schreibt die L"osung auf das Band und h"alt im
Zustand $q_{\text{accept}}$ an. Falls das Problem keine L"osung hat,
h"alt die Maschine im Zustand $q_{\text{reject}}$ an. Solche
Problem definieren also auch eine Sprache: $L(M)$ ist die
Menge der Inputw"orter, f"ur die eine L"osung existiert.

Ein Beispiel soll dies illustrieren. Wir m"ochten die Frage
von einem Algorithmus beantworten lassen, ob es in einem Graphen $G$
einen Weg gibt, der alle Knoten trifft.  Dazu m"ussen wir eine
Beschreibung des Graphen erzeugen, wir nennen sie $\langle G\rangle$,
diese auf das Band einer geeignet programmierten Turingmaschine $M$ schreiben,
diese laufen lassen, und warten, bis sie die Frage beantwortet. $L(M)$
besteht aus Beschreibungen $\langle G\rangle$ von Graphen $G$, die
einen solchen Weg haben.

Auf dieser Basis gelang es 1970 Yuri Matijasevi\v c "ubrigens,
\index{Matijasevi\v c, Yuri}
das zehnte Hilbertsche Problem etwas "uberraschend zu l"osen: es gibt
keinen Algorithmus, mit dem entschieden werden kann, ob eine
diophantische Gleichung eine L"osung hat. Es ist also durchaus
nicht selbstverst"andlich, wenn ein Problem eine algorithmische
L"osung hat. Wir werden im n"achsten Kapitel sehen, dass es viele
Probleme gibt, die keine algorithmische L"osung haben. Im "ubern"achsten
Kapitel werden wir dann Probleme kennenlernen, die zwar von einem Computer
gel"ost werden k"onnen, deren L"osung aber l"anger dauert als man
jeden realistischen Computer laufen lassen kann.
Die im folgenden entwickelte Theorie zeigt also, wo die Grenzen
der Berechenbarkeit mit Computern liegen.

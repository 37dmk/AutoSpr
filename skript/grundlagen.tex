%
% grundlagen.tex
%
% (c) 2009 Prof Dr Andreas Mueller
%
\chapter{Grundlagen}
\lhead{Grundlagen}
\rhead{Notation}
\section{Notation}
In diesem Abschnitt stellen wir eine Reihe bereits bekannter Notationen
zusammen.
\subsection{Logik}
\index{Logik}
\index{Pr\"adikatenlogik}
\subsubsection{Pr"adikate}
Wir verwenden die Pr"adikatenlogik.
\index{Pr\"adikat}
{\em Pr"adikate} sind formale Aussagen "uber mathematische Objekte, die auch
variable Teile enthalten k"onnen. Pr"adikate sind also ``Funktionen'' mit
Wahrheitswerten als R"uckgabewerten.
\begin{align*}
P(x,y)&=x < y&&\text{wahr falls $x$ kleiner als $y$}\\
P(n)&=n \equiv 0\mod 2&&\text{wahr falls $n$ gerade}
\end{align*}
Pr"adikate k"onnen durch logische Verkn"upfungen zu neuen Aussagen kombiniert
werden:
\begin{center}
\begin{tabular}{|l|c|c|l|}
\hline
Name&Verkn"upfung&Zeichen&Bedeutung\\
\hline
\index{Konjunktion}
Konjunktion&UND&$P\wedge Q$&wahr falls sowohl $P$ als auch $Q$ wahr sind\\
\index{Disjunktion}
Disjunktion&ODER&$P\vee Q$&wahr falls $P$ oder $Q$ (oder beide) wahr sind\\
\index{Negation}
Negation&NICHT&$\neg P$&wahr falls $P$ nicht wahr ist\\
\hline
\end{tabular}
\end{center}
Die Kombination $\neg P\vee Q$ ist also wahr, wenn $P$ nicht
wahr ist, oder falls $P$ wahr ist, auf jeden Fall auch $Q$
wahr ist. Die Aussage, dass $\neg P\vee Q$ wahr sei, heisst also
nichts anderes, als dass $Q$ folgt, wenn $P$ wahr ist. Daher schreibt
man
\[
\neg P\vee Q\qquad =\qquad P\Rightarrow Q.
\]

Hat man mit einer grossen Zahl von Pr"adikaten $P_1,P_2,\dots$ zu tun,
dann kann man deren Konjunktion oder Disjunktion "ahnlich wie beim Summenzeichen
mit einem ``grossen'' Verkn"upfungszeichen schreiben:
\begin{align*}
\bigwedge_{i=1}^n P_i&=P_1\wedge P_2\wedge P_3\wedge\dots\wedge P_n\\
\bigvee_{i=1}^n P_i&=P_1\vee P_2\vee P_3\vee\dots\vee P_n
\end{align*}
Die Indizes m"ussen nicht eine fortlaufende Folge bilden, sie k"onnen
auch aus einer beliebigen Indexmenge $I$ stammen, wof"ur man dann
schreibt
\[
\bigwedge_{i\in I}P_i
\qquad
\text{bzw.}
\qquad
\bigvee_{i\in I}P_i
\]

Die beiden Operationen $\wedge$ und $\vee$ sind miteinander vertr"aglich,
es gelten die {\em Distributiv-Gesetze}
\begin{align*}
P\wedge(Q\vee R)&=(P\wedge Q)\vee (P\wedge R)\\
P\vee(Q\wedge R)&=(P\vee Q)\wedge (P\vee R).
\end{align*}
\index{Distributivgesetz}

\subsubsection{Normalformen}
\index{Normalform}
\index{Normalform!konjunktive}
Durch wiederholte Anwendung der Distributivgesetze kann man 
eine Formel in eine von zwei Normalformen bringen, je nachdem
ob man die ``$\wedge$'' oder die ``$\vee$'' nach aussen bringt.
Sind die ``"ausseren'' Verkn"upfungen die ``$\wedge$'', spricht
man von {\em konjunktiver Normalform}, eine solche Formel sieht so aus:
\[
(x_1\vee x_3)\wedge(x_2\vee \bar x_4\vee x_5)\wedge\dots
\]
Die Klammerausdr"ucke heissen {\em Klauseln}.
\index{Klausel}

Bringt man stattdessen die ``$\vee$'' nach aussen, spricht man von
{\em disjunktiver Normalform}, also eine Formel der Form
\index{Normalform!diskjunktive}
\[
(x_1\wedge x_3)\vee(x_2\wedge \bar x_4\wedge x_5)\vee\dots
\]

Die Umwandlung zwischen den Normalformen kann sehr verschwenderisch sein.
Die Formel
\[
(x_1\wedge y_1)\vee(x_2\wedge y_2)\vee\dots\vee (x_n\wedge y_n)
\]
in disjunktiver Normalform wird bei dieser Art von Umwandlung zu
\begin{equation}
(x_1\vee x_2\vee\dots\vee x_{n-1}\vee x_n)
\wedge
(x_1\vee x_2\vee\dots\vee x_{n-1}\vee y_n)
\wedge
\dots 
=\bigwedge (z_1\vee z_2\vee\dots \vee z_{n-1}\vee z_n),
\label{bigcnf}
\end{equation}
wobei $(z_1,\dots,z_n)$ eine beliebige Kombination $z_i=x_i$ oder $z_i=y_i$
ist. Die Formel (\ref{bigcnf}) hat also $2^n$ Klauseln.

\subsubsection{de Morgansche Regeln}
\index{de Morgansche Regeln}
Die de Morganschen Regeln erlauben, ODER in UND zu verwandeln, indem
man die einzelnen Terme negiert:
\begin{align}
P\wedge Q&= \neg(\neg P\vee\neg Q)\notag\\
P\vee Q&= \neg(\neg P\wedge\neg Q)\notag\\
P\Rightarrow Q&=\neg P\vee Q=\neg\neg(\neg P\vee Q)=\neg(P\wedge\neg Q)=\neg\neg Q\vee \neg P\notag\\
&=\neg Q\Rightarrow \neg P\label{kontraposition}
\end{align}
\index{Kontraposition}
Die letzte Formel heisst auch die Kontraposition, sie ist der Kern
eines Beweises mit Widerspruch (siehe Abschnitt \ref{widerspruchsbeweis})


\subsubsection{Quantoren}
\index{Quantor}
Pr"adikate mit mindestens einer freien Variablen k"onnen mit Hilfe
von Quantoren zu neuen Pr"akdikaten zusammengebaut werden. 
Die Aussage $n^2\ge 0$ ist zum Beispiel f"ur alle nat"urlichen
Zahlen richtig, man schreibt daf"ur
\[
\forall (n\in\mathbb N\Rightarrow n^2>0)
\]
\index{All-Quantor}
Das Zeichen $\forall$ wird ``f"ur alle'' gelesen, es heisst
{\em All-Quantor}.
Oft wird daf"ur abk"urzend auch geschrieben
\[
\forall n\in\mathbb N(n^2>0)
\]
\index{Existenz-Quantor}
Jede positive reelle Zahl $x$ hat eine Wurzel, es gibt also eine
Zahl $w$ mit der Eigenschaft $x=w^2$. Formal dr"uckt man dies mit
dem {\em Existenz-Quantor} $\exists$ aus:
\[
\exists w\in\mathbb R(x =w^2)
\]
Dies liest man ``es gibt eine reelle Zahl $w$, deren Quadrat $x$ ist.
Will man obigen Satz, dass jede positive relle Zahl eine Wurzel hat,
formal ausdr"ucken, schreibt man
\[
\forall x>0\exists w\in\mathbb R(x=w^2)
\]

Zwei Aussagen $P$ und $Q$ sind {\em "aquivalent}, wenn sowohl
$P\Rightarrow Q$ als auch $Q\Rightarrow P$, kurz $P\Leftrightarrow Q$.

\subsection{Mengenlehre}
\index{Mengenlehre}
Wir gehen von einem intuitiven Mengenverst"andnis aus, und bezeichnen
Mengen meistens mit grossen Buchstaben $A$, $B$, $C$ usw.
Aus einer Menge $A$ kann eine neue Menge gebildet werden, indem
nur die Elemente ausgew"ahlt werden, die eine bestimmte Eigenschaft
$E$ haben:
\[
\{a\in A\;|\;E(a)\}
\]
Die Menge der $Q$ der Quadratzahlen besteht zum Beispiel aus den
Zahlen $n$, f"ur
die es eine andere Zahl $w$ gibt (die Wurzel), mit der Eigenschaft
$n=w^2$.  In Formeln:
\[
Q=\{n\; |\; \exists m\in\mathbb N(n=m^2)\}
\]

Sind $A$ und $B$ zwei Mengen, dann k"onnen wir daraus neue Mengen
bilden:
\begin{center}
\begin{tabular}{|l|c|l|}
\hline
Operation&Zeichen&Bedeutung\\
\hline
\index{Durchschnitt}
Durchschnitt&$A\cap B$&Enth"alt die Elemente, die in $A$ {\bf und} $B$ sind\\
\index{Vereinigung}
Vereinigung&$A\cup B$&Enth"alt die Elemente, die in $A$ {\bf oder} $B$ sind\\
\index{Komplement}
Komplement&$\bar A$&Enth"alt die Elemente, die {\bf nicht} in $A$ sind\\
\index{Differenz}
Differenz&$A\setminus B$&Enth"alt die Elemente aus A, die nicht in $B$ sind\\
\hline
\end{tabular}
\end{center}
\index{symmetrische Differenz}
F"ur die Anwendungen wichtig ist ausserdem die {\em symmetrische Differenz}
zweier Mengen: 
\[
A{\;\Delta\;}B = (A\setminus B)\cup (B\setminus A).
\]
Die symmetrische Differenz verschwindet genau dann, wenn die beiden
Mengen gleich sind:
\[
A{\;\Delta\;}B = \emptyset
\quad\Leftrightarrow\quad
A=B.
\]
\begin{figure}
\begin{center}
\includegraphics[width=0.6\hsize]{images/turing-4}
\end{center}
\caption{Symmetrische Differenz der Mengen $A$ und $B$ (schraffiert)
\label{symdiff}}
\end{figure}

Die folgenden Zahlmengen sind wohlbekannt und wir werden sie ebenfalls
oft verwenden:
\begin{center}
\begin{tabular}{|c|l|}
\hline
Symbol&Beschreibung\\
\hline
\index{nat\"urliche Zahlen}
$\mathbb N$&Menge der nat"urlichen Zahlen $\{0,1,2,\dots\}$\\
$\mathbb N^*$&Menge der positiven nat"urlichen Zahlen $\{1,2,3,\dots\}$\\
\index{ganze Zahlen}
$\mathbb Z$&Menge der ganzen Zahlen $\{\dots,-2,-1,0,1,2,\dots\}$\\
\index{rationale Zahlen}
$\mathbb Q$&Menge der rationalen Zahlen $\{\frac{p}{q}|p\in\mathbb Z,q\in\mathbb N^*\}$\\
\index{reelle Zahlen}
$\mathbb R$&Menge der reellen Zahlen\\
\index{leere Menge}
$\emptyset=\{\}$&leere Menge\\
\hline
\end{tabular}
\end{center}
Daraus lassen sich weitere Mengen konstruieren:
\begin{align*}
\mathbb Z^*&=\mathbb Z\setminus \{0\}\\
\mathbb Q^*&=\mathbb Q\setminus \{0\}\\
\mathbb R^*&=\mathbb R\setminus \{0\}\\
\mathbb R_{> 0}&=\mathbb R^+=\{x\in\mathbb R\,|\,x>0\}\\
[n]&=\{x\in \mathbb N^*\,|\,x\le n\}
\end{align*}
\index{Machtigkeit@M\"achtigkeit}
Enth"alt eine Menge $A$ nur endlich viele Elemente, schreiben wir die Anzahl
ihrer Elemente $|A|$, sie heisst auch die {\em M"achtigkeit} von $A$.
\[
|\{n\in\mathbb N\,|\,\text{$n$ prim}\wedge n<10\}|=|\{2,3,5,7\}|=4.
\]
Die Mengen $[n]$ enthalten genau $n$ Elemente, $|[n]|=n$.

Enth"alt eine Menge $B$ alle Elemente von $A$, sagt man, $A$ sei in $B$
enthalten, schreibt daf"ur $A\subset B$ und sagt, $A$ sei eine
{\em Teilmenge} von $B$. Ist $A\subset B$ und auch
$B\subset A$, dann enthalten $A$ und $B$ die gleichen Elemente, also
$A=B$.

\index{Potenzmenge}
Die Menge aller Teilmengen von $A$ heisst die {\em Potenzmenge} $P(A)$ von $A$:
\[
P(A)=\{ T\,|\,T\subset A\}
\]
Die Potenzmenge einer endlichen Elemente $A$ kann man wie folgt bilden.
Um eine Teilmenge von $A$ zu bilden, muss man f"ur jedes Elemente
von $A$ entscheiden, ob es in die Teilmenge kommen soll oder nicht.
Man hat also f"ur jedes der $|A|$ Elemente eine Entscheidung mit zwei
m"oglichen Ausg"angen zu f"allen, was auf
\[
\underbrace{2\cdot\dots\cdot 2}_{\text{$|A|$ Faktoren}}=2^{|A|}
\]
m"oglich ist, also $|P(A)|=2^{|A|}$.

\subsection{Paare und Tupel}
\index{Paar}
\index{Tupel@$n$-Tupel}
\index{kartesisches Produkt}
Die Beschreibung eines Punktes in der Ebene erfordert die Angabe zweier
Koordinaten, die man "ublicherweise als Paar $(x,y)$ schreibt. Allgemein
kann man aus den Elementen $a\in A$ und $b\in B$ zweier Mengen $A$ und $B$
die Menge der Paare (2-Tupel) bilden, diese Menge heisst das
{\em kartesische Produkt} der beiden Mengen:
\[
A\times B = \{(a,b)\,|\,a\in A\wedge b\in B\}.
\]
\index{Tripel}
Analog kann man aus drei Mengen die Menge aller Tripel
\[
A\times B\times C=\{(a,b,c)\,|\,a\in A\wedge b\in B\wedge c\in C\}
\]
bilden, oder f"ur $n$ Mengen die Menge der $n$-Tupel
\[
A_1\times A_2\times\dots\times A_n
=\{(a_1,a_2,\dots,a_n)\,|a_1\in A_1\wedge a_2\in A_2\wedge\dots\wedge a_n\in A_n\}.
\]
Falls die Faktormengen alle identisch sind, also $A_1=A_2=\dots=A_n=A$
schreiben wir auch
\[
A^n= \underbrace{A\times \dots \times A}_{\text{$n$ Faktoren}}
\]
f"ur das kartesische Produkt von $n$ Faktoren $A$.

\subsection{Relationen}
\index{Relation}
Eine zweistellige {\em Relation} $R$ ist ein Spezialfall eines Pr"adikates.
Realisieren k"onnte man eine Relation dadurch, dass man die Menge
aller Paare bildet, f"ur die die Relation erf"ullt ist. Um zu entscheiden,
ob $xRy$ gilt, muss man also nur noch in der Menge nachschauen,
ob $(x,y)$ dort drin ist. Man kann also die Relation mit der
Menge
\[
\{(x,y)\in A\times B\,| xRy\}
\]
identifizieren.
\index{Graph!einer Relation}
Diese Menge heisst auch der {\em Graph} der Relation.

Besonders wichtig sind {\em "Aquivalenzrelationen}, die sich durch folgende
Eigenschaften auszeichnen.
\begin{compactenum}
\item $R$ ist {\em reflexiv}, falls f"ur jedes $x$ gilt $xRx$: $\forall x(xRx)$
\item $R$ heisst {\em symmetrisch}, falls mit $xRy$ auch $yRx$ gilt: $\forall x\forall y(xRy\Rightarrow yRx)$
\item $R$ heisst {\em transitiv}, falls mit $xRy$ und $yRz$ auch $xRz$ gilt.
\end{compactenum}
Beispiele von "Aquivalenzrelationen sind:
\begin{itemize}
\item Gleichheit
\item $xRy$ f"ur $x,y\in\mathbb N$ falls $x$ und $y$ den gleichen
Rest bei Teilung durch eine feste Primzahl $p$ haben.
\item $xRy$ f"ur Dreiecke $x$ und $y$, falls $x$ und $y$ kongruent sind.
\end{itemize}

\subsection{Funktionen, Abbildungen}
\index{Funktion}
\index{Abbildung}
Eine {\em Funktion} oder {\em Abbildung} $f\colon A\to B$ ist
eine Zuordnung, die
jedem Element von $A$ genau ein Element von $B$ zuordnet.
\index{Bild}
\index{Urbild}
Das Element
$b=f(a)$ heisst das {\em Bild} von $a$, $a$ heisst ein {\em Urbild} von $b$. Ein
Element in $b$ kann mehrere Urbilder haben.

Eine Funktion ist eine spezielle Relation: 
$a$  und $b$ stehen in der Relation zueinander, wenn $f(a)=b$. 
\index{Graph!einer Abbildung}
Die zugeh"orige Menge von Paaren ist
\[
G(f)=\{(a,b)\,|\,b=f(a)\},
\]
und heisst der {\em Graph}. Ist $A=\mathbb R$, $B=\mathbb R$ und
$f\colon\mathbb R\to\mathbb R$, dann ist $G(f)$ die Menge
der Punkte $(x,y)$ in der Ebene, die $y=f(x)$ erf"ullen, also
genau der Graph im "ublichen Sinne.

\index{Menge aller Abbildungen}
Die Menge aller Abbildungen von $A$ nach $B$ schreibt man auch
$B^A$. Die Notation wird verst"andlich, wenn man f"ur endliche
Mengen $A$ und $B$ z"ahlt, wieviel
Abbildungen zwischen $A$ und $B$ es gibt. Um eine Abbildung von
$A$ nach $B$ zu konstruieren, muss man f"ur jedes der $|A|$ Elemente von $A$
eines der $|B|$ Elemente von $B$ ausw"ahlen. Man muss also $|A|$ mal
eine Auswahl mit $|B|$ M"oglichkeiten treffen, man hat also
\[
\underbrace{|B|\cdot\dots\cdot|B|}_{\text{$|A|$ Faktoren}}=|B|^{|A|}
\]
Abbildungen.

Die Elemente der Potenzmenge von $A$ entstehen dadurch, dass man f"ur jedes
Element von $A$ einen Wert $0$ oder $1$ ausw"ahlen muss, $0$ gibt an,
dass das Element nicht in die Teilmenge kommt, $1$ gibt an, dass es dazugeh"ort.
Eine Teilmenge von $A$ entspricht also genau einer Abbildung $A\to\{0,1\}$,
man kann die Potenzmenge mit der Menge der Abbildungen $A\to\{0,1\}$
identifizieren:
\[
P(A) = \{0,1\}^{A}.
\]

Ein $n$-Tupel von Elementen aus $A$ ordnet jedem der Pl"atze
$1,\dots,n$ genau ein Element aus $A$ zu, ein Tupel ist also eigentlich
eine Abbildung $[n]=\{1,\dots,n\}\to A$, die Menge der Tupel ist somit
\[
A^n=A^{\{1,\dots,n\}}=A^{[n]}.
\]

\subsection{Graphen}
\index{Kante}
\index{Ecke}
\index{Vertex}
Ein Graph besteht aus Ecken (Vertizes) und Kanten,
die die Ecken verbinden.
Dabei
soll es aber anders als zum Beispiel bei einem Verkehrsnetz zwischen zwei
Ecken immer nur eine Verbindung geben k"onnen. Die Kante ist also durch
die beiden Endpunkte vollst"andig bestimmt, ausserdem spielt deren
Reihenfolge keine Rolle. Um eine Kante zu beschreiben, braucht man
also nur die Menge $\{a,b\}$ der Endpunkte zu kennen.
Wir fassen das in folgende Definition zusammen.

\begin{definition}
\index{Graph}
\label{def_graph}
Ein Graph ist ein Paar $(V,E)$ bestehend aus einer Menge $V$ von Ecken
(Vertices),
und einer Menge $E$ von zweielementigen Teilmengen von $V$, die Kanten
genannt werden.
\end{definition}

\index{Grad}
Der {\em Grad} einer Ecke in einem Graphen ist die Anzahl der Kanten,
die von dieser Ecke ausgehen.

\index{Pfad}
Ein {\em Pfad} in einem Graphen ist eine Folge von Ecken, die durch Kanten
verbunden sind.
\index{Pfad!einfacher}
Ein {\em einfacher Pfad} ist ein Pfad, der keine Ecke mehr
als einmal trifft.
\index{Zyklus}
Ein {\em Zyklus} ist ein Pfad, der mit der gleichen Ecke
endet, mit der er begonnen hat.
\index{Zyklus!einfacher}
Ein einfacher Zyklus ist ein Zyklus,
der keine Ecke zweimal besucht.

\index{Baum}
Ein {\em Baum} ist ein Graph ohne Zyklen. Ein Baum kann einen ausgezeichneten
Punkt, die {\em Wurzel} des Baumes, enthalten. Alle anderen Ecken vom Grad 1
\index{Blatt}
heissen Bl"atter des Baumes.

M"ochte man die Richtung der Kanten ber"ucksichtigen,
hat man mit einem gerichteten
Graphen zu tun. Da es jetzt auf die Reihenfolge der Endpunkte an kommt,
sich insbesondere die Kante von $a$ nach $b$ von der Kante von $b$ nach
$a$ unterscheidet, k"onnen wir die Kanten nicht mehr durch die ungeordneten
Mengen beschreiben, sondern m"ussen geordnete Tupel verwenden.

\begin{definition}
\index{Graph!gerichteter}
\label{def_gerichteter_graph}
Ein gerichteter Graph ist ein Paar $(V,E)$ bestehend aus einer
Mengen $V$ von Ecken (Vertizes) und einer Menge von Paaren $E\subset V\times V$,
den Kanten.
\end{definition}

Ein Verkehrsnetz wie auch die sp"ater zu definierenden endlichen
Automaten ist aber noch komplizierter: auf den Verbindungen zwischen
den Netzwerken sind auch verschiedene Bahnlinien im Einsatz, so gibt
es zwischen Rapperswil und Pf"affikon zum Beispiel je eine Kante, die
mit ``S5'' bzw.~mit ``Voralpenexpress'' angeschrieben ist. Dies
wird von folgender Definition eingefangen:

\begin{definition}
\label{def_gerichteter_beschrifteter_graph}
\index{Graph!gerichteter!beschrifteter}
Ein gerichteter beschrifteter Graph ist ein Tripel $(V,E,L)$ bestehend
aus einer Menge $V$ von Ecken, einer Menge $L$ von Beschriftungen (Labels) und 
einer Menge $E$ von Tripeln $(a,e,l)\in V\times V\times L$. Das
Tripel $(a,e,l)$ heisst Kante von $a$ nach $e$ mit Beschriftung $l$.
\end{definition}

\rhead{Beweistechniken}
\section{Beweistechniken}
Da in dieser Vorlesung Beweise zentral sind hier drei grundlegende 
Techniken zur Erinnerung.

\subsection{Konstruktion}
\index{Beweis!konstruktiver}
Ein konstruktiver Beweis gibt einfach eine Konstruktion an, die
das behauptete Objekt liefert oder f"ur die die behauptete Eigenschaft
offensichtlich ist. 

\begin{satz}
\index{Gleichung!quadratische}
Falls $b^2-4ac>0$ hat die quadratische Gleichung
\[
ax^2+bx+c=0
\]
zwei verschiedene L"osungen.
\end{satz}

\begin{proof}[Beweis]
Man kann die quadratische Gleichung mit vollst"andigem Erg"anzen wie
folgt umformen:
\begin{align*}
ax^2+bx+c&=0\\
x^2+2\cdot \frac{b}{2a} x +\frac{c}a&=0\\
x^2+2\cdot \frac{b}{2a} x 
+\left(\frac{b}{2a}\right)^2
+\frac{c}a&=0
+\left(\frac{b}{2a}\right)^2\\
\left(x+\frac{b}{2a}\right)^2 &= \left(\frac{b}{2a}\right)^2 -\frac{c}a \\
\left(x+\frac{b}{2a}\right)^2 &=
\frac{b^2-4ac}{4a^2}\\
x+\frac{b}{2a}&=\frac{\pm\sqrt{b^2-4ac}}{2a}\tag{*}\\
x_{\pm}&=\frac{-b\pm\sqrt{b^2-4ac}}{2a}
\end{align*}
Da $b^2-4ac>0$ kann im Schritt (*) tats"achlich die Wurzel gezogen werden,
und die beiden Wurzeln sind auch verschieden. Man hat also zwei verschiedene
L"osungen konstruiert, womit die Behauptung bewiesen ist.
\end{proof}

\subsection{Widerspruch\label{widerspruchsbeweis}}
\index{Beweis!mit Widerspruch}
\index{Widerspruch}
In der Mathematik darf es keine Widerspr"uche geben, denn aus einem
Widerspruch liesse sich jede beliebige Aussage ableiten. W"are zum
Beispiel $P$ eine Aussage, die sowohl wahr wie auch falsch ist,
dann h"atte die Menge
\[
\{x\in\{P\}\,|\, \text{$x$ ist wahr}\}
\]
je nachdem ob $P$ nun wahr oder falsch ist $1$ oder $0$ Elemente.
Da das aber nat"urlich immer die gleiche Menge ist, m"usste man
folgern $0=1$. Daraus l"asst sich dann ableiten, dass alle nat"urlichen
Zahlen gleich sind (zum Beispiel mit vollst"andiger Induktion, siehe
unten), dass alle Mengen gleich viele Elemente haben, n"amlich gar
keine, dass es die ganze Mathematik also gar nicht gibt.

Wenn man also eine Aussage $P$ beweisen will, kann man annehmen,
dass $P$ falsch ist, oder $\neg P$ wahr ist. Wenn daraus jetzt
eine Aussage $Q$ folgt, von der wir bereits wissen, dass sie falsch
ist, dann bedeutet das, dass die Annahme $\neg P$ nicht haltbar
war.

Und noch etwas formaler: wenn $\neg P\Rightarrow \neg Q$, dann k"onnen
wir das umformen zu $\neg Q\Rightarrow P$ (nach Kontrapositionsformel
(\ref{kontraposition})), wir wissen bereits, dass
$Q$ falsch ist, die linke Seite also wahr ist, also muss auch $P$
wahr sein.

Das klassische Beispiel f"ur einen Beweis mit Widerspruch ist das 
folgende:
\begin{satz}$\sqrt{2}$ ist irrational: $\sqrt{2}\not\in\mathbb Q$.
\index{irrational}
\end{satz}
\begin{proof}[Beweis]
Wir nehmen $\sqrt{2}\in\mathbb Q$ und f"uhren dies zu einem
Widerspruch. Wenn $\sqrt{2}\in\mathbb Q$ ist, kann man $\sqrt{2}$
als gek"urzten Bruch $\frac{p}{q}$ schreiben. Dann gilt aber
auch
\[
\sqrt{2}^2=2=\frac{p^2}{q^2}\quad\Rightarrow\quad p^2=2q^2.
\]
Insbesondere muss $p$ eine gerade Zahl sein, also $p^2$ sogar
durch $4$ teilbar sein. In $2q^2$ muss daher auch in $q$ nochmals
ein Primfaktor $2$ vorkommen, auch $q$ muss also gerade sein.
Also k"onnen wir sowohl $p$ als auch $q$ durch $2$ teilen,
den Bruch $\frac{p}q$ also mit $2$ k"urzen.
Das steht aber im Widerspruch zu der Annahme, dass $\frac{p}q$ bereits
ein gek"urzter Bruch war. Dieser Widerspruch zeigt, dass die
Annahme $\frac{p}q\in\mathbb Q$ nicht zu halten ist.
\end{proof}

\subsection{Induktion}
\index{Beweis!mit Induktion}
\index{Induktion}
M"ochte man eine Aussage mit einem nat"urlichen Parameter $n$ f"ur jeden
m"oglichen Wert von $n$ beweisen, kann man dazu vollst"andige Induktion
verwenden. Dazu geht man wie folgt vor:
\begin{description}
\index{Verankerung}
\item[Verankerung:] Beweise die Aussage f"ur den kleinsten Wert,
f"ur den sie bewiesen werden soll.
\index{Induktionsannahme}
\item[Induktionsannahme:] Nehme an, f"ur einen bestimmten Wert
$n$ sei die Aussage bereits bewiesen (nach der Verankerung ist sie 
f"ur den kleinsten m"oglichen Wert ja tats"achlich bereits bewiesen).
\index{Induktionsschritt}
\item[Induktionsschritt:] Beweise jetzt die Aussage f"ur den Wert $n+1$.
\end{description}

Als Beispiel soll der folgene Satz dienen:

\begin{satz} F"ur alle nat"urlichen Zahlen $n$ gilt
\[
\sum_{k=1}^nk=\frac{n(n+1)}2
\]
\end{satz}
\begin{proof}[Beweis]
\begin{description}
\item[Verankerung:] f"ur $n=0$ ist die Summe leer, ergibt also $0$.
Auf der rechten Seite steht $0(0+1)/2=0$, die Formel trifft also zu.
\item[Induktionsannahme:]Wir nehmen also an, dass
\[
\sum_{k=1}^nk=\frac{n(n+1)}2
\]
\item[Induktionsschritt:] Wir m"ussen die Behauptung f"ur $n+1$ zeigen. Dazu
berechnen wir
\begin{align}
\sum_{k=1}^{n+1}k&=\biggl(\sum_{k=1}^nk\biggr) + (n+1)\notag\\
&=\frac{n(n+1)}2+(n+1)\label{verwendung_annahme}\\
&=\frac{n(n+1)+2(n+1)}2\notag\\
&=\frac{(n+1)(n+2)}2=\frac{(n+1)((n+1)+1)}2\notag
\end{align}
In Schritt (\ref{verwendung_annahme}) haben wir die Induktionsannahme
verwendet. In der letzten Zeile auf der rechten Seite steht tats"achlich
die Formel des Satzes, in der $n$ durch $n+1$ ersetzt worden ist.
Damit ist die Formel f"ur $n+1$ bewiesen.
\end{description}
\end{proof}

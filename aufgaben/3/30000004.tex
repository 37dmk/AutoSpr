Betrachten Sie den folgenden, in Tabellenform gegebenen
Automaten "uber dem Alphabet $\Sigma=\{0,1\}$:
\begin{center}
\begin{tabular}{|c|cc|}
\hline
&{\tt 0}&{\tt 1}\\
\hline
$z$&$e$&$z_1$\\
$z_0/E$&$z_0$&$z_1$\\
$z_1$&$z_2$&$z_0$\\
$z_2$&$z_1$&$z_3$\\
$z_3$&$z_1$&$z_2$\\
$e$&$e$&$e$\\
\hline
\end{tabular}
\end{center}
Zeichnen Sie das Zustandsdiagramm des Automaten. Konstruieren Sie den
zugeh"origen minimalen Automaten.

\begin{loesung}
Das folgende Zustandsdiagramm beschreibt den DEA:
\[
\UseTips
\entrymodifiers={++[o][F]}
\xymatrix @-1mm {
*+\txt{}
        &*+\txt{}
                &*+\txt{}
                        &z_2 \ar@/_/[d]^{\tt 1} \ar@/^/[dl]_{\tt 0}
\\
*+\txt{}\ar[r]
        &z \ar[r]^{1} \ar[d]_{\tt 0}
                &z_1 \ar@/_/[d]_{\tt 1}  \ar@/^/[ur]^{\tt 0}
                        &z_3 \ar[l]^{\tt 0} \ar@/_/[u]_{\tt 1}
\\
*+\txt{}
        &e \ar@(dr,dl)^{{\tt 0},{\tt 1}}
                &*++[o][F=]{z_0} \ar@(dr,dl)^{\tt 0} \ar@/_/[u]_{\tt 1}
}
\]
F"ur die Bestimmung der "aquivalenten Zust"ande bauen wir die Tabellen
der nicht "aquivalenten Zust"ande auf. Im ersten Schritt werden die
Paare aus einem Endzustand und einem Nicht-Endzustand markiert:
\begin{center}
\begin{tabular}{|c|cccccc|}
\hline
&$z$&$z_0/E$&$z_1$&$z_2$&$z_3$&$e$\\
\hline
$z$&$\equiv$&$\times$&&&&\\
$z_0/E$&$\times$&$\equiv$&$\times$&$\times$&$\times$&$\times$\\
$z_1$&&$\times$&$\equiv$&&&\\
$z_2$&&$\times$&&$\equiv$&&\\
$z_3$&&$\times$&&&$\equiv$&\\
$e$&&$\times$&&&&$\equiv$\\
\hline
\end{tabular}
\end{center}
In den folgenden Schritten pr"uft man jedes noch nicht markierte Paar
von Zust"anden, ob sie sich durch Anwendung der "Ubergangsfunktion
in ein Paar "uberf"uhren lassen, welches bereits markiert ist. Ist dies
der Fall, wird das entsprechende Feld ebenfalls markiert. In den folgenden
Tabellen sind die jeweils bereits bekannten Markierungen mit dem Zeichen
$\otimes$ ausgef"uhrt.
\begin{center}
\begin{tabular}{|c|cccccc|}
\hline
&$z$&$z_0/E$&$z_1$&$z_2$&$z_3$&$e$\\
\hline
$z$&$\equiv$&$\otimes$&$\times$&&&\\
$z_0/E$&$\otimes$&$\equiv$&$\otimes$&$\otimes$&$\otimes$&$\otimes$\\
$z_1$&$\times$&$\otimes$&$\equiv$&$\times$&$\times$&$\times$\\
$z_2$&&$\otimes$&$\times$&$\equiv$&&$\times$\\
$z_3$&&$\otimes$&$\times$&&$\equiv$&$\times$\\
$e$&&$\otimes$&$\times$&$\times$&$\times$&$\equiv$\\
\hline
\end{tabular}
\end{center}
In der zweiten Iteration k"onnen noch sechs Paare markiert werden:
\begin{center}
\begin{tabular}{|c|cccccc|}
\hline
&$z$&$z_0/E$&$z_1$&$z_2$&$z_3$&$e$\\
\hline
$z$&$\equiv$&$\otimes$&$\otimes$&$\times$&$\times$&$\times$\\
$z_0/E$&$\otimes$&$\equiv$&$\otimes$&$\otimes$&$\otimes$&$\otimes$\\
$z_1$&$\otimes$&$\otimes$&$\equiv$&$\otimes$&$\otimes$&$\otimes$\\
$z_2$&$\times$&$\otimes$&$\otimes$&$\equiv$&&$\otimes$\\
$z_3$&$\times$&$\otimes$&$\otimes$&&$\equiv$&$\otimes$\\
$e$&$\times$&$\otimes$&$\otimes$&$\otimes$&$\otimes$&$\equiv$\\
\hline
\end{tabular}
\end{center}
Aber in der letzten Iteration bleiben die Felder $(z_2,z_3)$ und
$(z_3,z_2)$ unmarkiert, die beiden Zust"ande sind also "aquivalent:
\begin{center}
\begin{tabular}{|c|cccccc|}
\hline
&$z$&$z_0/E$&$z_1$&$z_2$&$z_3$&$e$\\
\hline
$z$&$\equiv$&$\otimes$&$\otimes$&$\otimes$&$\otimes$&$\otimes$\\
$z_0/E$&$\otimes$&$\equiv$&$\otimes$&$\otimes$&$\otimes$&$\otimes$\\
$z_1$&$\otimes$&$\otimes$&$\equiv$&$\otimes$&$\otimes$&$\otimes$\\
$z_2$&$\otimes$&$\otimes$&$\otimes$&$\equiv$&&$\otimes$\\
$z_3$&$\otimes$&$\otimes$&$\otimes$&&$\equiv$&$\otimes$\\
$e$&$\otimes$&$\otimes$&$\otimes$&$\otimes$&$\otimes$&$\equiv$\\
\hline
\end{tabular}
\end{center}
Durch Zusammenlegen im Zustandsdiagramm erh"alt man jetzt als
kleinsm"oglichen Automaten:
\[
\UseTips
\entrymodifiers={++[o][F]}
\xymatrix @-1mm {
*+\txt{}\ar[r]
        &z \ar[r]^{1} \ar[d]_{\tt 0}
                &z_1 \ar@/_/[d]_{\tt 1}  \ar@/^/[r]^{\tt 0}
                        &z_2,z_3 \ar@/^/[l]^{\tt 0} \ar@(dr,dl)^{\tt 1}
\\
*+\txt{}
        &e \ar@(dr,dl)^{{\tt 0},{\tt 1}}
                &*++[o][F=]{z_0} \ar@(dr,dl)_{\tt 0} \ar@/_/[u]_{\tt 1}
}
\]
\end{loesung}


Die ``International Bank Account Number'' vereinfacht den
grenz"uberschreitenden Zahlungsverkehr durch ein vereinheitlichtes
Format. Die IBAN beginnt immer mit der zweistelligen ISO-L"anderkennzeichnung
({\tt CH} f"ur die Schweiz, {\tt GR} f"ur Griechenland,$\dots$). Danach
folgen zwei bis drei Stellen f"ur eine Pr"ufzahl (Italien verwendet
drei Stellen, der Rest Europas zwei).
Danach folgen weitere Ziffern, die die Bank und das Konto bezeichnen.
Der kontospezifische Teil kann weitere bankspezifische Pr"ufziffernteile
enthalten, die jedoch nicht standardisiert sind, und uns hier nicht
interessieren.
Insgesamt ist die IBAN h"ochstens 34 Stellen lang.
Beispielsweise ist
\begin{center}
{\tt DE88200800000970375700}
\end{center}
eine g"ultige IBAN Nummer.

Die Berechnung der Pr"ufziffer funktioniert wie folgt.
Zun"achst
werden die beiden Buchstaben des L"andercodes durch Zahlenwerte
beginnend mit $\text{\tt A}=10$ ersetzt, und f"ur die zweistellige
Pr"ufzahl werden {\tt 0} eingesetzt. Dann werden diese sechs Stellen
vorne entfernt und hinten an die IBAN angeh"angt. F"ur obiges Beispiel
wird also {\tt D} durch $13$ und {\tt E} durch $14$ ersetzt, so dass dass die 
IBAN zun"achst zu
\begin{center}
{\tt 131488200800000970375700}
\end{center}
Dann werden die ersten 6 Stellen nach hinten verschoben:
\begin{center}
{\tt 200800000970375700131400}
\end{center}
Dann wird von dieser Zahl der Rest bei Division durch 97 ermittelt:
in diesem Fall 10. Die Pr"ufzahl ist die Differenz zu 98, in diesem
Fall also $98-10=88$, die korrekte IBAN ist also
\begin{center}
{\tt DE88200800000970375700}
\end{center}

Die Modulo-Berechnung f"ur grosse Zahlen ist etwas unhandlich.
Gibt es eventuell einen endlichen Automaten, mit dem man die IBANs
auch auf Korrektheit pr"ufen kann?

\begin{hinweis}
Es wird nicht verlangt, dass Sie einen solchen Zustandsautomaten angeben,
sondern h"ochstens dass sie darlegen, wie man gegebenenfalls zu einem solchen
Zustandsautomaten kommen k"onnte, sollte es einen geben.
\end{hinweis}

\begin{loesung}
Ja, da IBANs eine endliche, wenn auch sehr grosse ($25^2\cdot 10^{32}$)
Sprache bilden, ist die Menge der IBANs regul"ar.
Insbesondere gibt es einen regul"aren Ausdruck, der genau die 
korrekten IBANs akzeptiert.

Einen solchen regul"aren Ausdruck zu finden, ist jedoch nicht einfach.
Man kann aber wie folgt vorgehen: In der Vorlesung wurde gezeigt,
dass man den Rest bei einer Division durch 97 mit einem endlichen
Automaten bestimmen kann. Die obige Beschreibung der Pr"ufzahl
l"auft darauf hinaus, dass man "uberpr"ufen muss, ob die ersten
vier Stellen den zum zweiten Teil ``passenden'' Rest bei Teilung
durch 97 liefert.

Man formuliert also zun"achst einen Automaten $A_1$,
welcher genau die Reste von der ersten vier Stellen ermittelt, seine
Zust"ande sind die m"oglichen Reste $0$ bis $96$.
Dann konstruiert man einen Automaten $A_2$, welcher den Rest 
des zweiten Teils ermittelt, wenn diesem noch 6 Stellen mit $0$
folgen w"urden.

Dieser Rest muss jetzt zusammen mit dem von $A_1$
bestimmten Rest einen bestimmten Wert ergeben. Aus $A_1$ und $A_2$ bildet
man den Produktautomaten und markiert alle Paare von Zust"anden die
in diesem Sinne ``zusammenpassen'' also Akzeptiertzust"ande.
Um einen regul"aren Ausdruck zu finden, wandelt man diesen endlichen
Automaten mit dem Standardalgorithmus in einen regul"aren Automaten
um.
\end{loesung}

Das Programm {\em ImageMagick} erm"oglicht Bildverarbeitung mit
Kommandozeilenbefehlen. Viele Optionen dieses Programms m"ussen die
Geometrie eines Bildes oder Bildausschnittes spezifizieren k"onnen.
Dazu wurde eine Notation entwickelt, die in der Dokumentation etwa
wie folgt zusammengefassst ist:
\begin{center}
\begin{tabular}{>{\tt}lp{5.75in}}
\rm geometry&General description\\
\hline
50\%&Height and width both scaled by specified (50) percentage.\\
47\%x42\%&Height (47\%) and width (42\%) individually scaled by specified
percentages. (Only one \% symbol needed.)\\
47&Width 47 given, height automagically selected to preserve aspect ratio.\\
x42&Height 42 given, width automagically selected to preserve aspect ratio.\\
47x42&Maximum values of height (42) and width (47) given, aspect ratio
preserved.\\
47x42\^&Minimum values of width (47) and height (42) given, aspect ratio
preserved.\\
47x42!&Width (47) and height (42) emphatically given, original
aspect ratio ignored.\\
47x42>&Shrinks an image with dimension(s) larger than the
corresponding width (47) and/or height (42) argument(s).\\
47x42<&Enlarges an image with dimension(s) smaller than the
corresponding width (47) and/or height (42) argument(s).\\
1974@&Resize image to have specified area (1974) in pixels. Aspect ratio is
preserved.\\
\hline
\end{tabular}
\end{center}
Ausserdem kann einer solchen Gr"ossenspezifikation auch noch ein Offset
angeh"angt werden, der immer von der Form
\texttt{+18-48}
sein muss. Die erste Zahl gibt den Versatz in $x$-Richtung an, die zweite
den Versatz in $y$-Richtung. In beiden Richtungen sind beide Vorzeichen 
\texttt{+} und \texttt{-} m"oglich.

Finden Sie einen regul"aren Ausdruck, der genau alle m"oglichen
Geometrie-Spezifikationen akzeptiert.

\begin{loesung}
Ein solcher regul"arer Ausdruck kann zum Beispiel in einem Programm dazu
verwendet werden, um mit Hilfe eines Pattern-Matchings die verschiedenen
F"alle zu unterscheiden, die Parameter herauszulesen, und die entsprechend
Aktion einzuleiten.
Dazu ist notwendig, dass der regul"are Ausdruck so aufgebaut wird, dass
er die m"oglichen F"alle zu unterscheiden erlaubt.
Die folgende L"osung strebt genau dies an.

Es gibt offenbar vier Arten von Geometriespezifikationen: Skalierungen 
(erste zwei Zeilen),
partielle Spezifikationen (Zeilen 3 und 4), vollst"andige Spezifikationen
und Fl"achenspezifikationen (letzte Zeile).
F"ur jede kann man einen regul"aren Ausdruck angeben:
\begin{align*}
\text{Skalierung:}\qquad&r_1 = \texttt{\color{red}[0-9]+\%|[0-9]+\%?x[0-9]+\%|[0-9]\%+x[0-9]+\%?}
\\
\text{partiell:}\qquad&r_2 = \texttt{\color{green}[0-9]+|x[0-9]+}
\\
\text{vollst"andig:}\qquad&r_3 = \texttt{\color{blue}[0-9]+x[0-9]+[!><\^{}]?}
\\
\text{Fl"ache:}\qquad&r_4 = \texttt{\color{yellow}[0-9]+@}
\end{align*}
Man beachte, dass in der Liste der angeh"angten Sonderzeichen bei den
vollst"andigen Geometriespezifikation das \texttt{\^{}} ganz am Ende
steht, weil ein \texttt{\^{}} am Anfang einer Zeichenklasse die
Negation bedeutet.


Jedem dieser Ausdr"ucke kann ausserdem eine Offset-Spezifikation angeh"angt
werden, die auf den regul"aren Ausdruck
\begin{center}
\texttt{[-+][0-9]+[-+][0-9]+}
\end{center}
passen muss. Zusammen ergibt das den regul"aren Ausdruck
\[
\texttt{(}
{\color{red} r_1}
\texttt{|}
{\color{green} r_2}
\texttt{|}
{\color{blue} r_3}
\texttt{|}
{\color{yellow} r_4}
\texttt{)([-+][0-9]+[-+][0-9]+)?}
\]
oder ausgeschrieben
\begin{center}
\texttt{(({\color{red}[0-9]+\%|[0-9]+\%?x[0-9]+\%|[0-9]\%+x[0-9]+\%?})|}%
\texttt{({\color{green}[0-9]+|x[0-9]+})|}\\
\texttt{({\color{blue}[0-9]+x[0-9]+[{}!><\^{}]?}))}%
\texttt{({\color{yellow}[0-9]+@})?}%
\texttt{)([-+][0-9]+[-+][0-9]+)?}
\end{center}
(der regul"are Ausdruck ist aus Platzgr"unden auf zwei Zeilen aufgeteilt,
beide Teile sind als ein regul"arer Ausdruck ohne Zeilenumbruch zu lesen).

Man kann einen regul"aren Ausdruck auch noch viel direkter bekommen, ohne
die F"alle in der Tabelle auf Gemeinsamkeiten zu untersuchen.
Man schreibt einfach f"ur jeden Ausdruck in der Tabelle einen regul"aren
Ausdruck hin, und verkettet diese alle, also
\begin{center}
\texttt{([0-9]+\%)|}
\texttt{([0-9]+\%x[0-9]+)|}
\texttt{([0-9]+x[0-9]+\%)|}
\texttt{([0-9]+\%x[0-9]+\%)|}
\\
\texttt{([0-9]+)|}
\\
\texttt{(x[0-9]+)|}
\\
\texttt{([0-9]+x[0-9]+)|}
\\
\texttt{([0-9]+x[0-9]+\^{})|}
\\
\texttt{([0-9]+x[0-9]+!)|}
\\
\texttt{([0-9]+x[0-9]+>)|}
\\
\texttt{([0-9]+x[0-9]+<)|}
\\
\texttt{([0-9]+@)}
\end{center}
(eine Zeile f"ur jede Spezifikation in der Tabelle).
Daran h"angt man dann wieder wie oben die Offset\-spezifikation an.
\end{loesung}

\begin{diskussion}
Die urspr"ungliche Formulierung konnte so missverstanden werden, dass auch
der Ausdruck \texttt{.*} eine korrekte Antwort gewesen w"are.
Es ist aus dem Zusammenhang der Aufgabe klar, dass dies nicht gemeint war.
\end{diskussion}

\begin{diskussion}
Der Ausdruck $r_2$ k"onnte auch als \texttt{x?[0-9]+} geschrieben werden,
doch ist die angegebene Form besser.
Typische Pattern-Matcher k"onnen nach dem Match auch einzelne markierte
Teile einer Zeichenkette angeben.
In der Standard-Regex-Library kann man zum Beispiel mit Klammern solche
Teile markieren.
Schreibt man $r_2$ dann in der Form \texttt{([0-9]+)|x([0-9]+)}
dann welche kann man am Resultat des Matchings ablesen, welcher der
Teile vorliegt, und kann mit nur einem Match zwischen einer Breitenangabe
und einer H"ohenangabe unterscheiden. Das folgende Programm illustriert dies:
{\small
\verbatimainput{part.c.expanded}
}
\end{diskussion}

\begin{bewertung}
Skalierung wiedergegeben ({\bf S}) 1 Punkt,
Partielle Gr"ossenangabe ({\bf P}) 1 Punkt,
vollst"andige Gr"ossenangabe ({\bf V}) 1 Punkt,
Fl"acheangabe ({\bf F}) 1 Punkt,
Offset ({\bf O}) 1 Punkt,
Komposition ({\bf K}) 1 Punkt.
\end{bewertung}

Das Programm {\em Imagemagick} erm"oglicht Bildverarbeitung mit
Kommandozeilenbefehlen. Viele Optionen dieses Programms m"ussen die
Geometrie eines Bildes oder Bildausschnittes spezifizieren k"onnen.
Dazu wurde eine Notation entwickelt, die in der Dokumentation etwa
wie folgt zusammengefassst ist:
\begin{center}
\begin{tabular}{>{\tt}lp{5.75in}}
\rm geometry&General description\\
\hline
50\%&Height and width both scaled by specified (50) percentage.\\
47\%x42\%&Height (47\%) and width (42\%) individually scaled by specified
percentages. (Only one \% symbol needed.)\\
47&Width 47 given, height automagically selected to preserve aspect ratio.\\
x42&Height 42 given, width automagically selected to preserve aspect ratio.\\
47x42&Maximum values of height (42) and width (47) given, aspect ratio
preserved.\\
47x42\^&Minimum values of width (47) and height (42) given, aspect ratio
preserved.\\
47x42!&Width (47) and height (42) emphatically given, original
aspect ratio ignored.\\
47x42>&Shrinks an image with dimension(s) larger than the
corresponding width (47) and/or height (42) argument(s).\\
47x42<&Enlarges an image with dimension(s) smaller than the
corresponding width (47) and/or height (42) argument(s).\\
1974@&Resize image to have specified area (1974) in pixels. Aspect ratio is
preserved.\\
\hline
\end{tabular}
\end{center}
Ausserdem kann einer solchen Gr"ossenspezifikation auch noch ein Offset
angeh"angt werden, der immer von der Form
\texttt{+18-48}
sein muss. Die erste Zahl gibt den Versatz in $x$-Richtung an, die zweite
den Versatz in $y$-Richtung. In beiden Richtungen sind beide Vorzeichen 
\texttt{+} und \texttt{-} m"oglich.

Finden Sie einen regul"aren Ausdruck, der alle m"oglichen
Geometrie-Spezifikationen akzeptiert.


\begin{loesung}
Es gibt offenbar vier Arten von Geometriespezifikationen: Skalierungen 
(erste zwei Zeilen),
partielle Spezifikationen (Zeilen 3 und 4), vollst"andige Spezifikationen
und Fl"achenspezifikationen (letzte Zeile).
F"ur jede kann man einen regul"aren Ausdruck angeben:
\begin{align*}
\text{Skalierung:}\qquad&r_1 = \texttt{\color{red}[0-9]+\%|[0-9]+\%?x[0-9]+\%|[0-9]\%+x[0-9]+\%?}
\\
\text{partiell:}\qquad&r_2 = \texttt{\color{green}[0-9]+|x[0-9]+}
\\
\text{vollst"andig:}\qquad&r_3 = \texttt{\color{blue}[0-9]+x[0-9]+[\^{}!><]?}
\\
\text{Fl"ache:}\qquad&r_4 = \texttt{\color{yellow}[0-9]*@}
\end{align*}
Jedem dieser Ausdr"ucke kann ausserdem eine Offset-Spezifikation angeh"angt
werden, die auf den regul"aren Ausdruck
\begin{center}
[-+][0-9]+[-+][0-9]+
\end{center}
passen muss. Zusammen ergbiet das den regul"aren Ausdruck
\[
\texttt{(}
{\color{red} r_1}
\texttt{|}
{\color{green} r_2}
\texttt{|}
{\color{blue} r_3}
\texttt{|}
{\color{yellow} r_4}
\texttt{)([-+][0-9]+[-+][0-9]+)?}
\]
oder ausgeschrieben
\begin{center}
\texttt{(({\color{red}[0-9]+\%|[0-9]+\%?x[0-9]+\%|[0-9]\%+x[0-9]+\%?})|}%
\texttt{({\color{green}[0-9]+|x[0-9]+})|}\\
\texttt{({\color{blue}[0-9]+x[0-9]+[\^{}!><]?}))}%
\texttt{({\color{yellow}[0-9]*@})?}%
\texttt{)([-+][0-9]+[-+][0-9]+)?}
\end{center}
(der regul"are Ausdruck ist aus Platzgr"unden auf zwei Zeilen aufgeteilt,
beide Teile sind als ein regul"arer Ausdruck ohne Zeilenumbruch zu lesen).
\end{loesung}


Sei $\Sigma=\{a,b\}$. Konstruieren Sie einen deterministischen
endlichen Automaten, der die Sprache
\[
L=\{w|\;\text{$|w|_a$ ist ungerade und $w$ endet mit $ab$}\}
\]
akezptiert.

\begin{loesung}
Wir betrachten die Sprache $L$ als Durchschnitt zweier Sprachen
\begin{align*}
L_1&=\{w|\;\text{$|w|_a$ ist ungerade}\}\\
L_2&=\{w|\;\text{$w$ endet mit $ab$}\}
\end{align*}
und konstruieren f"ur die $L_i$ je einen Automaten $A_i$.

F"ur $L_1$ verwenden wir den Automaten $A_1$:
\[
\entrymodifiers={++[o][F]}
\xymatrix @-1mm {
*+\txt{} \ar[r]
&x_0 \ar@/^/[r]^{a} \ar@(dl,dr)_{b}
&*++[o][F=]{x_1} \ar@/^/[l]^{a} \ar@(dl,dr)_{b}
}
\]

F"ur $L_2$ funktioniert der Automat $A_2$:
\[
\entrymodifiers={++[o][F]}
\xymatrix @-1mm {
*+\txt{} \ar[r]
&y_0 \ar@/^/[r]^{a} \ar@(dl,dr)_{b}
&y_1 \ar@/^/[r]^{b} \ar@(dl,dr)_{a}
&*++[o][F=]{y_2} \ar@/^/[l]^{a} \ar `u^l[ll]+/u0.8cm/ `^d[ll]_{b} [ll]
}
\]
Die Zust"ande haben dabei eine einfache Bedeutung:
\begin{itemize}
\item[$y_0$:] ``Noch nichts gelesen, oder eine Folge von mehr als einem $b$
gelesen. Um zu akzeptieren brauchen wir die Folge $ab$.''
\item[$y_1$:] ``$a$ gelesen, um zu akzeptieren brauchen wir ein Zeichen $b$.''
\item[$y_2$:] ``Soeben $ab$ gelesen, Wort w"are akzeptabel, wenn es hier zu
Ende w"are.''
\end{itemize}

Die Standardkonstruktion f"ur $L_1\cap L_2$ kann jetzt auf die Automaten
$A_1$ und $A_2$ angewendet werden, sie liefert den gesuchten Automaten.
Dazu werden als Zust"ande die Paare $(x_i,j_i)$ gebildet, es ergibt sich
das folgende Zustandsdiagram:
\[
\entrymodifiers={++[o][F]}
\xymatrix @+3mm {
*+\txt{} \ar[r]
&x_0,y_0 \ar[dr]^{a} \ar@(ul,u)^{b}
&x_0,y_1 \ar@/^/[d]^{a} \ar[r]^{b}
&x_0,y_2 \ar `u^l[ll]+/u0.8cm/ `^d[ll]_{b} [ll] \ar[dl]_{a}
\\
*+\txt{}
&x_1,y_0 \ar[ur]^{a} \ar@(d,dl)^{b}
&x_1,y_1 \ar@/^/[u]^{a} \ar[r]_{b}
&*++[o][F=]{x_1,y_2} \ar[ul]^{a}
\ar `d[ll]+/d0.8cm/ `[ll]^{b} 
}
\]
Die Zust"ande in der ersten Zeile sind nicht akzeptabel, weil sie eine
Gerade Anzahl Zeichen $a$ im Wort anzeigen. Man kann leicht verifizieren,
dass man nur mit einer Zeichenfolge $ab$ beim Zustand $(x_1,y_2)$
ankommen kann.
\end{loesung}

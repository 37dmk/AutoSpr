Ein etwas primitives Daten"ubermittlungsprotokoll verwendet
das Zeichen \texttt{/} als Delimiter, dazwischen k"onnen Messages beliebiger
L"ange stehen, die aber das Zeichen \texttt{/} nicht verwenden d"urfen.
Die Erfahrung hat gezeigt, dass es sehr selten vorkommt, dass einzelne
Zeichen nicht "ubertragen werden, es fehlen immer gleich mindestens drei
Zeichen. Daher kam jemand auf die Idee die "Ubertragung dadurch zu "uberpr"ufen,
dass man jeder Message einen String von \texttt{x}-Zeichen anh"angt, der
ein Drittel so lange ist wie die Message:
\begin{align*}
&\texttt{/}\underbrace{\texttt{wasjetzt}}_8\texttt{/}\underbrace{\texttt{xx}}_2\texttt{/}
&\texttt{/}\underbrace{\texttt{noch eine etwas laengere message}}_{32}\texttt{/}\underbrace{\texttt{xxxxxxxxxx}}_{10}\texttt{/}
\end{align*}
Bei der Bestimmung der L"ange der Folge von \texttt{x}-Zeichen wird ganzzahlige
Division verwendet, also $\lfloor 8 / 3\rfloor = 2$ und $\lfloor 32 / 3 \rfloor = 10$.
Ihre Aufgabe ist, einen regul"aren Ausdruck zu formulieren, der korrekte
Messages von fehlerhaften unterscheidet.

\begin{loesung}
So einen regul"aren Ausdruck kann es nicht geben, denn die Sprache 
\[
\{ \texttt{/}w\texttt{/}s\texttt{/}\, |\, w\in \Sigma^*, s\in\{\texttt{x}\}^*,
|s| = \lfloor |w|/3\rfloor
\}
\]
ist nicht regul"ar, wie wir mit dem Pumping-Lemma beweisen k"onnen.

Nehmen wir an, die Sprache $L$ sei regul"ar. Nach dem Pumping-Lemma gibt
es dann die Pumping Length $N$. Wir konstruieren jetzt ein Wort in $L$:
\[
w=
\texttt{/a}^{3N}\texttt{/x}^N\texttt{/}
\]
Das Pumping-Lemma sagt ausserdem, dass $w$ zerlegt werden k"onnen muss in
drei Teile $w=xyz$ so, dass $|xy|\le N$, $|y|>0$ und alle aufgepumpten
W"orter $xy^kz$ sollten wieder in $L$ liegen. Dies ist nur m"oglich,
wenn $x$ und $y$ vollst"andig im Teile $\texttt{a}^{3N}$ enthalten ist.
Beim Aufpumpen wird $xy^kz$ also zus"atzlick $k|y|$ Zeichen \texttt{a}
enthalten, bei dreimaligem Aufpumpen "andert sich also die Zahl der 
\texttt{a}s dermassen, dass auch die Zahl der \texttt{x} vergr"ossert
werden m"usste, um wieder ein Wort in der Sprache zu bekommen. Die Zahl
der \texttt{x} "andert aber beim Pumpen nie, also sind nicht mehr alle
aufgepumpten W"orter in $L$. Dieser Widerspruch zeigt, dass $L$ nicht
regul"ar sein kann.
\end{loesung}

\begin{bewertung}
Pumping Lemma (\textbf{L}) 1 Punkt, Pumping Length (\textbf{N}) 1 Punkt,
geeignetes Wort (\textbf{W}) 1 Punkt, Zerlegung (\textbf{Z}) 1 Punkt,
Widerspruch beim Pumpen (\textbf{P}) 1 Punkt,
Schlussfolgerung nicht regul"ar (\textbf{R}) 1 Punkt.
\end{bewertung}

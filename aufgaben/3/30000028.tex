Zeigen Sie, dass die Sprache
\[
L=\{
\texttt{0}^m\texttt{1}^n|m\ne n
\}
\]
"uber dem Alphabet $\Sigma=\{\texttt{0},\texttt{1}\}$ nicht regul"ar ist.

\begin{loesung}
Der einfachste Beweis verwendet die Mengenoperationen:
Die Sprache
\[
L'=\{\texttt{0}^n\texttt{1}^m\,|\,n,m\ge 0\}
\]
ist regul"ar, denn sie wird von dem Automaten
\[
\entrymodifiers={++[o][F]}
\xymatrix{
*+\txt{}\ar[r]
        &*++[o][F=]{z_0}\ar[r]^{\texttt{1}} \ar@(ur,ul)_{\texttt{0}}
                &*++[o][F=]{z_1} \ar[r]^{\texttt{0}} \ar@(ur,ul)_{\texttt{1}}
			& {z_2}\ar@(ur,dr)^{\texttt{0},\texttt{1}}
}
\]
akzeptiert.
W"are $L$ regul"ar, m"usste auch $L'\setminus L$ regul"ar sein.
Doch $L'\setminus L$ ist die Sprache
\[
L'\setminus L
=
\{\texttt{0}^n\texttt{1}^n\,|\,n\ge 0\},
\]
von der in der Vorlesung gezeigt wurde, dass sie nicht regul"ar
ist. Dieser Widerspruch zeigt, dass auch $L$ nicht regul"ar sein kann.

Man kann die Behauptung aber auch direkt mit dem Pumping-Lemma beweisen.
\ding{182}
Wir nehmen also an, die Sprache sei regul"ar.
\ding{183}
Sei $N$ die Pumping Length.
Jedes gen"ugend lange Wort $x$ der Sprache kann dann geschrieben werden als $uvw$,
wobei $|uv|\le N$, und alle W"orter $uv^nw\in L$, f"ur alle $n\ge 1$.
Dem Teilwort $v$ entspricht eine Schleife im endlichen Automaten, der
$L$ akzeptiert.

\ding{184}
Wir w"ahlen jetzt ein Wort $x=\texttt{0}^N\texttt{1}^{N+b}$.
Es hat L"ange $2N+b$ und ist f"ur $b>0$ in $L$, es erf"ullt also die
Voraussetzungen des Pumping Lemma.
\ding{185}
Also kann es als $x=uvw$ geschrieben werden, wobei dem Teilwort $v$ eine
Schleife im Graphen des Automaten entspricht.
Das Wort $v$ enth"alt nur Nullen, da $|uv|\le N$ sein muss.
Vergr"ossert man $b$, ist der erste Teil des Wortes davon unber"uhrt,
es gilt also immer noch die gleiche Aufteilung in $u$ und $v$,
$w$ wird um einige Einsen l"anger.

\ding{186}
Wir versuchen jetzt das Wort genau einmal aufzupumpen, also $uv^ky$ mit
$k=2$ zu bilden m"ochten erreichen, dass dieses Wort nicht mehr in $L$
ist.
Da durch das Aufpumpten genau so viele \texttt{0} hinzukommen wie das
Teilwort $v$ lang ist, das Wort $uv^2w$ beginnt also mit $N+|v|$ Nullen.
Damit $uv^2w$ nicht mehr in der Sprache ist, m"ussen gleich viele Nullen
wie Einsen vorhanden sein, es muss also $N+|v|=N+b$ gelten.
\begin{center}
\includeagraphics[]{pl-1.pdf}
\end{center}
Wenn man also $b=|v|$ w"ahlt, ist das einmal aufgepumpte Wort nicht
mehr in $L$, im Widerspruch zur Behauptung des Pumping Lemma.

\ding{187}
Somit kann $L$ nicht regul"ar sein.
\end{loesung}

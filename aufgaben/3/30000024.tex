F"ur eine Anwendung, die als Antwort auf "aussere Reize (Events)
nach programmierbaren Zeitintervallen (Delays) gewisse Aktionen (Actions)
ausl"ost, wurde folgende einfache Programmiersprache entworfen. Ein
Programm ist eine Liste von ``Befehlsw"ortern'', die Events, Delays und Actions
codieren. Letztere werden
jeweils durch die Buchstaben {\tt E}, {\tt D} und {\tt A} codiert.
Parameter werden daran angeh"angt. Events oder Actions haben als Parameter
einen Kleinbuchstaben gefolgt von einer Ziffernfolge. Delays haben als
Parameter eine Ziffernfolge ohne f"uhrende Nullen und maximal f"unf
Zeichen. Die einzelnen Befehlsw"orter d"urfen zur besseren Lesbarkeit mit Zeilenumbr"uchen
(Newline, Zeichen \verb+\n+) getrennt werden. Auf jeden Event muss mindestens
ein Delay folgen, auf jedes Delay eine Action. Auf ein Event darf keine Action folgen.
\begin{teilaufgaben}
\item Ist diese Sprache regul"ar?
\item Falls ja: Beschreiben Sie einen endlichen Automaten, der die
Sprache akzeptiert.
\end{teilaufgaben}

\begin{loesung}
\begin{teilaufgaben}
\item
Die Sprache ist regul"ar, denn man kann einen regul"aren Ausdruck
angeben, der genau die W"orter dieser Sprache akzeptiert. Um diesen
regul"aren Ausdruck aufzubauen kann man zun"achst Ausdr"ucke f"ur
die einzelnen Events, Delays und  Actions konstruieren:
\begin{center}
\begin{tabular}{|c|c|}
\hline
Event&{\tt E[a-z][0-9]*($|\backslash$n)}\\
Delay&{\tt D[1-9][0-9]*($|\backslash$n)}\\
Action&{\tt A[a-z][0-9]*($|\backslash$n)}\\
\hline
\end{tabular}
\end{center}
Ein Event ist immer gefolgt von mindestens einer Gruppe aus
einem Delay gefolgt von mindestens einer Action:
\begin{center}
Event
        Delay Action Action{\tt *}
        {\tt (} Delay Action Action{\tt *} {\tt )*}
\end{center}
Solche Events kann es jetzt beliebig viele geben, also
\begin{center}
{\tt (}Event
        Delay Action Action{\tt *}
        {\tt (} Delay Action Action{\tt *} {\tt )*}{\tt )*}
\end{center}
Setzt man jetzt die Ausdr"ucke aus der Tabelle in, erh"alt man
den endg"ultigen regul"aren Ausdruck (ist alles als ein einziger
Ausdruck zu lesen):
\begin{center}
{\tt (}
{\tt E[a-z][0-9]*($|\backslash$n)}
{\tt (D[1-9][0-9]*($|\backslash$n))}
{\tt (A[a-z][0-9]*($|\backslash$n))}
{\tt (A[a-z][0-9]*($|\backslash$n))}
{\tt *}
{\tt (}
{\tt (D[1-9][0-9]*($|\backslash$n))}
{\tt (A[a-z][0-9]*($|\backslash$n))}
{\tt (A[a-z][0-9]*($|\backslash$n))}
{\tt *}
{\tt )*}{\tt )*}
\end{center}
\item
Einen endlichen Automat kann dadurch erhalten, dass man obigen
regul"aren Ausdruck vom Standardalgorithmus in einen endlichen
Automaten umwandeln l"asst. Man kann den endlichen Automaten
aber auch explizit angeben. Zun"achst kann man einzelne
Teilautomaten $E$, $D$ und $A$ konstruieren, die jeweils
ein Event-Wort, eine Delay-Wort oder ein Action-Wort akzeptieren:
\[
\entrymodifiers={++[o][F]}
\xymatrix @-1mm {
*+\txt{} \ar[r]
        &z_{10}\ar[r]^{\tt E}
                &z_{11}\ar[r]^{\tt [a-z]}
                        &z_{12}\ar[r]^{\tt [0-9]}
                                &*++[o][F=]{z_{13}} \ar@(ur,dr)^{\tt [0-9]}
\\
*+\txt{} \ar[r]
        &z_{20}\ar[r]^{\tt D}
                &z_{21}\ar[r]^{\tt [1-9]}
                        &*++[o][F=]{z_{22}} \ar@(ur,dr)^{\tt [0-9]}
\\
*+\txt{} \ar[r]
        &z_{30}\ar[r]^{\tt E}
                &z_{31}\ar[r]^{\tt [a-z]}
                        &z_{32}\ar[r]^{\tt [0-9]}
                                &*++[o][F=]{z_{33}} \ar@(ur,dr)^{\tt [0-9]}
}
\]
Diese Teilautomaten k"onnen wir jetzt zu einem neuen Automaten
zusammensetzen, der "Ubersichtlichkeit halber schreiben wir jeweils
nur $E$, $D$ und $E$ f"ur die Teilautomaten.
\[
\entrymodifiers={++[o][F]}
\xymatrix @-1mm {
*+\txt{} \ar[r]
        &E \ar[r]^{\tt \backslash n} \ar[dr]^{\tt D}
                &z_1 \ar[d]^{\tt D}
                        &*+\txt{}
\\
*+\txt{}
        &*+\txt{}
                &D \ar[r]^{\tt \backslash n} \ar@/^/[dr]^{\tt A}
                        &z_2 \ar[d]^{\tt A}
\\
*+\txt{}
        &z_3 \ar[uu]^{\tt E} \ar[ur]^{\tt D}
                &*+\txt{}
                        &A      \ar@(ur,dr)^{\tt A}
                                \ar@/^/[lu]^/1em/{\tt D}
                                \ar@/^1.7pc/[lluu]^/2em/{\tt E}
                                \ar[ll]^{\tt \backslash n}
}
\]
Dabei verbinden die Pfeile jeweils die Akzeptierzust"ande der Teilautomaten
mit den Startzus"anden.
\end{teilaufgaben}
\end{loesung}



Finden Sie einen regul"aren Ausdruck f"ur die Sprache $L$ "uber dem
Alphabet $\Sigma=\{{\tt 0},{\tt 1}\}$ bestehend aus den W"ortern,
die eine ungerade Anzahl {\tt 1} enthalten.

\begin{loesung}
Diese Sprache ist regul"ar, man kann sofort einen endlichen Automaten
daf"ur angeben:
\[
\entrymodifiers={++[o][F]}
\xymatrix @-1mm {
*+\txt{}\ar[r]
	&{q_0}\ar@/^/[r]^{{\tt 1}}
	      \ar@(ul,ur)^{{\tt 0}}
		&*++[o][F=]{q_1} \ar@/^/[l]^{{\tt 1}}
	               \ar@(ul,ur)^{{\tt 0}}
}
\]
Auf diesen entlichen Automaten kann man jetzt den Standardalgorithmus
anwenden, mit dem man einen "aquivalenten regul"aren Ausdruck
gewinnen kann.

Im ersten Schritt muss man neue Start- und Akzeptierzust"ande 
konstruieren:
\[
\entrymodifiers={++[o][F]}
\xymatrix @-1mm {
*+\txt{}\ar[r]
	&{S}\ar[r]^{\varepsilon}
		&{q_0}\ar@/^/[r]^{{\tt 1}}
		      \ar@(ul,ur)^{{\tt 0}}
			&{q_1} \ar@/^/[l]^{{\tt 1}}
			       \ar@(ul,ur)^{{\tt 0}}
			       \ar[r]^{\varepsilon}
				&*++[o][F=]{A}
}
\]
Im zweiten Schritt m"ussen die Zwischenzust"ande entfernt werden,
bis nur noch $S$ und $A$ "ubrig bleiben. Wir beginnen mit $q_0$:
\[
\entrymodifiers={++[o][F]}
\xymatrix @-1mm {
*+\txt{}\ar[r]
	&{S}\ar[rr]^{{\tt 0}^*{\tt 1}}
		&*+\txt{}
			&{q_1} \ar@(ul,ur)^{{\tt 0}|{\tt 1}{\tt 0}^*{\tt 1}}
			       \ar[r]^{\varepsilon}
				&*++[o][F=]{A}
}
\]
Jetzt kann auch noch $q_1$ entfernt werden:
\[
\entrymodifiers={++[o][F]}
\xymatrix @-1mm {
*+\txt{}\ar[r]
	&{S}\ar[rrr]^{{\tt 0}^*{\tt 1} ({\tt 0}|{\tt 1}{\tt 0}^*{\tt 1})^*}
		&*+\txt{}
			&*\txt{}
				&*++[o][F=]{A}
}
\]
Der gesuchte regul"are Ausdruck ist also
\[
r={\tt 0}^*{\tt 1} ({\tt 0}|{\tt 1}{\tt 0}^*{\tt 1})^*
\]
\end{loesung}

\begin{bewertung}
Methode "uber VNEA ({\bf V}) 1 Punkt,
DEA f"ur die Sprache ({\bf D}) 1 Punkt,
Erweiterung um neuen Start-/Akzeptierzustand ({\bf S}) 1 Punkt,
Entfernung der Zwischenzust"ande in zwei Schritten ({\bf E}) 2 Punkte,
Regul"arer Ausdruck ({\bf R}) 1 Punkt.
\end{bewertung}

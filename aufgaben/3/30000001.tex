In einem endlichen Automaten
\[
A=(\{q_1,q_2,q_3,q_4,q_5\}, \{\text{\tt u},\text{\tt d}\}, \delta,
q_3, \{q_3\})
\]
ist die Funktion $\delta$ durch die folgende Tabelle definiert:
\begin{center}
\begin{tabular}{>{$}c<{$}|>{$}c<{$}>{$}c<{$}}
&\text{\tt u}&\text{\tt d}\\
\hline
q_1&q_1&q_2\\
q_2&q_1&q_3\\
q_3&q_2&q_4\\
q_4&q_3&q_5\\
q_5&q_4&q_5\\
\end{tabular}
\end{center}
\begin{teilaufgaben}
\item Zeichnen Sie das Zustandsdiagramm.
\item Akzeptiert der Automat des leere Wort $\varepsilon$?
\item Akzeptiert der Automat ein Wort der L"ange $3$?
\item Bestimmen Sie alle W"orter der L"ange $\le 4$, die
der Automat akzeptiert.
\item Wieviele verschiedene W"orter akzeptiert der Automat?
\end{teilaufgaben}

\begin{loesung}
\begin{teilaufgaben}
\item
\[
\entrymodifiers={++[o][F]}
\xymatrix @-1mm {
*+\txt{}&{q_1}\ar@/^/[d]^{\tt d} \ar@(ul,ur)^{\tt u}
\\
*+\txt{}&{q_2}\ar@/^/[u]^{\tt u}\ar@/^/[d]^{\tt d}
\\
*+\txt{}\ar[r]&*++[o][F=]{q_3}\ar@/^/[u]^{\tt u}\ar@/^/[d]^{\tt d}
\\
*+\txt{}&{q_4}\ar@/^/[u]^{\tt u}\ar@/^/[d]^{\tt d}
\\
*+\txt{}&{q_5}\ar@/^/[u]^{\tt u} \ar@(dr,dl)^{\tt d}
}
\]
\item
Das leere Wort "uberf"uhrt den Startzustand $q_3$ in den Zustand
$q_3\in \{q_3\}$, also einen Akzeptierzustand, somit ist
$\varepsilon\in L(A)$.
\item
Ein Wort der L"ange $3$ wird von diesem Automaten nicht akzeptiert.
Man kann das zum Beispiel dadurch einsehen, dass man eine Liste
aller W"orter der L"ange drei macht (es gibt nur 8 solche W"orter),
und die dann einzeln pr"uft.

Oder man sagt sich, dass ein Wort mit der L"ange $3$ niemals die kleine
Schleife an den Zust"anden $q_1$ oder $q_5$ nutzen kann, weil das Wort
zu kurz ist, um wieder $q_3$ zu erreichen. Die verbleibenden Pfeile,
die das Wort nutzen kann, "andern also mit jedem Zeichen den Rest
der Zustandsnummer bei Teilung durch zwei. Man kann das symbolisieren,
in dem man die $q_1$, $q_3$ und $q_5$ schwarz einf"arbt, die anderen
weiss. Ein Zeichen l"asst dann immer die Farbe wechseln. Mit drei Zeichen
landet man auf weiss, aber der einzige Akzeptierzustand ist schwarz,
also ist es nicht m"oglich, ein Wort der L"ange drei zu akzeptieren.

Noch ein weiteres Argument verl"auft wie folgt:
Zun"achst halten wir fest, dass der Automat auch keine W"orter der
L"ange $1$ akzeptiert, es gibt keine Pfeile, welche von $q_3$ nach
$q_3$ zur"uck f"uhren. Ein Wort der L"ange drei muss also mindestens
die Zust"ande $q_2$ oder $q_4$ erreichen. Eine Zustands"anderung von
dort zur"uck zu $q_3$ ergibt zwar ein akzeptiertes Wort der L"ange
$2$, es ist aber nicht m"oglich, dieses zu einem akzeptierten Wort
der L"ange $3$ zu erg"anzen.
Bleibt also nur ein Wort, welches
sogar $q_1$ oder $q_5$ erreicht. Da es keine Pfeile von $q_1$
oder $q_5$ direkt zum einzigen Akzeptierzustand $q_3$ gibt,
braucht es also mindestens zwei zus"atzliche Zeichen, um den
Akzeptierzustand $q_3$ zu erreichen. Somit muss ein solches
Wort mindestens L"ange $4$ haben.
\item Wir m"ussen alle W"orter finden, die gleich viele {\tt u} wie {\tt d}
enhalten. Die Anzahl solcher W"orter kann man wie folgt bestimmmen.
Man muss auf die vier Stellen eines solchen Wortes zwei Buchstaben
{\tt u} verteilen, die anderen zwei Pl"atze werden dann mit {\tt d}
gef"ullt. Das erste {\tt u} k"onnen wir auf Platz 1, 2 oder 3 setzen,
dann bleiben jeweils 3, 2 bzw.~1 M"oglichkeiten f"ur das zweite {\tt u},
insgesamt also $3+2+1=6$ M"oglichkeiten. Die sechs W"orter sind:
{\tt udud}, {\tt dudu}, {\tt uddu}, {\tt duud}, {\tt uudd}, {\tt dduu}.
Dazu kommen jetzt noch die zwei W"orter der L"ange $2$ und das leere Wort,
welches L"ange $0$ hat. Insgesamt sind dies $9$ verschiedene W"orter
\[
\{
{\tt udud}, {\tt dudu}, {\tt uddu}, {\tt duud}, {\tt uudd}, {\tt dduu},
{\tt ud},{\tt du}, \varepsilon
\}
\]
\item Der Automat akzeptiert alle W"orter der Form
\[
\text{\tt uu}
\text{\tt u}^n
\text{\tt dd},
\]
f"ur jedes $n\in \mathbb N$,
dies sind unendlich viele verschiedene W"orter. Alternativ kann
man auch die W"orter der Form $({\tt ud})^n$, also die Sprache
$\{{\tt ud}\}^*$ anf"uhren, welche ebenfalls unendlich viele W"orter
umfasst.
\qedhere
\end{teilaufgaben}
\end{loesung}


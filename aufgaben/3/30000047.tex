Sei $\Sigma=\{\texttt{a},\texttt{b},\dots,\texttt{z}\}$ das Alphabet bestehend
aus allen Kleinbuchstaben.
Die Sprache
\[
L=\{w\in\Sigma^*\;|\;\text{es gibt zwei verschiedene Buchstaben
$a,b\in\Sigma$ mit $|w|_a=|w|_b > 1$}\}
\]
besteht aus W"ortern, die mindestens zwei Buchstaben mehr als einmal
und in gleicher Anzahl enthalten.
Die W"orter
\[
\texttt{essen},\qquad
\texttt{rapperswil},\qquad
\texttt{seenachtfest}
\]
sind in $L$, nicht aber
\[
\texttt{trinken},\qquad
\texttt{pfaeffikon},\qquad
\texttt{montag}.
\]
Ist die Sprache $L$ regul"ar?

\begin{loesung}
Nein, wie man mit dem Pumping-Lemma f"ur regul"are Sprachen bewesein kann.
Wir nehmen dazu an, dass $L$ regul"ar sei, das Pumping Lemma f"ur regul"are
Sprachen besagt dann, dass es eine Zahl $N$, die Pumping Length, gibt,
so dass W"orter gr"osserer L"ange die Pumpeigenschaft haben.

Wir konstruieren daher ein Wort in $L$, welches diese Eigenschaft hat:
\[
w=\texttt{a}^N\texttt{b}^N
\]
Das Wort ist in $L$, wenn $N>1$ ist, was wir im Folgenden annehmen.
Das Pumping-Lemma besagt jetzt, dass es eine Zerlegung in drei 
Teile $w=xyz$ gibt, wobei $|xy|\le N$ und $|y|>1$ ist.
Insbesondere bestehen $x$ und $y$ ausschliesslich aus Buchstaben \texttt{a}.
Ein aufgepumptes Wort hat die Form
\[
w_k=xy^kz=\texttt{a}^{N+|y|(k-1)}\texttt{b}^N,
\]
d.~h.~f"ur $k>1$ ist die Zahl der einzigen beiden vorkommenden Buchstaben
\texttt{a} und \texttt{b} nicht mehr gleich, also $w_k\not\in L$,
im Widerspruch zur Behauptung des Pumping-Lemmas.
Dieser Widerspruch zeigt, dass $L$ nicht regul"ar sein kann.
\end{loesung}


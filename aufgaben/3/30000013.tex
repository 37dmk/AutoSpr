Ein CSV-File (Comma Separated Values) besteht aus Zeilen, die aus
mit Komma getrennten Feldern bestehen. Jedes Feld ist begrenzt mit
Anf"uhrungszeichen.
Die Felder selbst k"onnen auch Kommas und Anf"uhrungszeichen enthalten,
aber um die Anf"uhrungszeichen in einem Feld vom feldbegrenzenden
Anf"uhrungszeichen zu unterscheiden, m"ussen zwei Anf"uhrungszeichen
eingegeben werden. Enth"alt ein Feld weder Komma noch Anf"uhrungszeichen,
k"onnen die feldbegrenzenden Anf"uhrungszeichen weggelassen werden.
Hier ein Beispiel eines syntaktisch korrekten CSV-Files:
\verbatimainput{beispiel.csv}

Die syntaktisch korrekten Zeilen von CSV-Files bilden eine Sprache.
Ist diese Sprache regul"ar?

\ifthenelse{\boolean{loesungen}}{
\begin{loesung}
Ja, sie ist regul"ar, denn man kann einen endlichen Automaten daf"ur
angeben.
Das folgende Zustandsdiagramm beschreibt einen VNEA, der korrekte
CSV-Zeilen akzeptiert.
\[
\entrymodifiers={++[o][F]}
\xymatrix{
*+\txt{}\ar[r]
        &*++[o][F=]{z_0}\ar@/_/[d]_{\text{\tt[\char94,\char34]}} \ar[r]^{\text{\tt \char34}}
            \ar@(ul,ur)^{\text{\tt,}}
                &{z_1}\ar@(ul,ur)^{\text{\tt [\char94{}\char34]}}
                    \ar@/_/[d]_{\text{\tt\char34}}
\\
*+\txt{}
        &*++[o][F=]{z_2}\ar@/_/[u]_{\text{\tt,}}
            \ar@(dl,dr)_{\text{\tt [\char94,\char34]}}
                &*++[o][F=]{z_3}\ar@/_/[u]_{\text{\tt\char34}}
                    \ar[ul]^{\text{\tt,}}
}
\]
Wenn sich der Automat im Zustand $z_0$ befindet, erwartet er als
n"achstes ein Feld. Dieses kann leer sein, was durch in Komma ({\tt ,})
angezeigt wird. Es kann mit einem Feldbegrenzer {\tt\char34} beginnen,
was den Automaten in den Zustand $z_1$ bringt. In diesem Zustand
sind alle Zeichen akzeptabel, aber ein Anf"uhrungszeichen kann zwei
verschiedene Bedeutungen haben. Es kann gefolgt sein von noch einem
Anf"uhrungszeichen, in diesem Fall ist es ein eingebetetes Anf"uhrungszeichen.
Es kann aber auch gefolgt sein von einem Komma, in diesem Fall war
es ein Feldbegrenzer. Zu dieser Unterscheidung dient der Zustand $z_3$.

Ein Feld kann im Zustand $z_0$ auch mit einem beliebigen Zeichen
beginnen, das verschieden von Komma und Anf"uhrungszeichen ist.
Dazu dient der Zustand $z_2$. In diesem Zustand sind Anf"uhrungszeichen
nicht erlaubt, und ein Komma beendet das Feld, f"uhrt den Automaten
also wieder in den Zustand $z_0$.

Die Zust"ande $z_2$ und $z_3$ sind Akzeptierzust"ande, weil eine Zeile
nur dann noch nicht zu ende sein darf, wenn ein ge"offnetes Anf"uhrungszeichen
noch nicht geschlossen worden ist. Dies ist genau im Zustand $z_1$ der
Fall, alle Zust"ande ausser $z_1$ k"onnen daher akzeptieren.

Nat"urlich ist es auch m"oglich, die Regularit"at dadurch nachzuweisen,
dass man einen regul"aren Ausdruck f"ur die Zeilen angibt. Wir bauen
den regul"aren Ausdruck Schrittweise auf. Wenn $r_f$ ein regul"arer Ausdruck
ist, auf den einzelne Felder einer CSV-Zeile passen, dann ist
\[
r_z=r_f\text{\tt (,}r_f\text{\tt )*}
\]
ein regul"arer Ausdruck, auf den ganze CSV-Zeilen passen. Ein Feld kann
Feldbegrenzer haben, oder eben nicht, es ist also $r_f=r_b|r_u$, wobei
auf $r_b$ begrenzte Felder passen, und auf $r_u$ nur unbegrenzte.
Ungebegrenzte Felder d"urfen keine Kommata oder Anf"uhrungszeichen
enthalten, also $r_u=\text{\tt [\char94,\char34]*}$. Begrenzte Felder
d"urfen Kommata enthalten. Wenn sie Anf"uhrungszeichen enthalten,
dann jeweils in Zweiergruppen:
\[
r_b=\text{\tt
\char34([\char94\char34]|(\char34\char34))*\char34
}
\]
Damit kann man jetzt den regul"aren Ausdruck zusammensetzen:
\begin{align*}
r_u&=\text{\tt
[\char94,\char34]*
}\\
r_b&=\text{\tt
\char34([\char94\char34]|(\char34\char34))*\char34
}\\
r_f&=\text{\tt
[\char94,\char34]*|\char34([\char94\char34]|(\char34\char34))*\char34
}\\
r_z&=\text{\tt
[\char94,\char34]*|\char34([\char94\char34]|(\char34\char34))*\char34
(,%
[\char94,\char34]*|\char34([\char94\char34]|(\char34\char34))*\char34
)*
}
\end{align*}
\end{loesung}
}{}


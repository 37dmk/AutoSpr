Sei $n>0$ eine natürliche Zahl und sei $L_n$ die Sprache bestehend
aus W"ortern "uber dem Alphabet
$\Sigma=\{{\tt 0},{\tt 1 }\}$ deren Anzahl {\tt 0} und {\tt 1}
den gleichen Rest bei Teilung durch $n$ haben. In Formeln:
\[
L_n=\{ w\in\Sigma^*| \; |w|_0\equiv |w|_1\mod n\}.
\]
Zur Sprache $L_3$ geh"ort also beispielsweise dazu:
\[
\text{\tt 1010000},\;
\text{\tt 00101110},\;
\text{\tt 010101},\;
\text{\tt 101010},\;
\]
es geh"ort aber nicht dazu:
\[
\text{\tt 1110},\;
\text{\tt 0001},\;
\text{\tt 1110001},\;
\text{\tt 10101},\;
\]
Zeigen Sie, dass jede der Sprachen $L_n$ regul"ar ist.

\begin{hinweis}
Zeichnen Sie einen DEA f"ur $L_8$ und "uberlegen sie
sich dann, wie ein DEA f"ur $L_n$ mit beliebigem $n$ aussehen m"usste.
\end{hinweis}

\begin{loesung}
Wir wissen bereits, dass es endliche Automaten gibt, die den Rest
einer Zahl bei Teilung durch $n$ bestimmen k"onnen. Als Zust"ande
wird der Rest verwendet. Da es nur $n$ verschiedene Reste gibt,
kann ein endlicher Automat alle m"oglichen Reste darstellen.
Daher ist plausibel, dass auch die Gleichheit zweier Reste
mit Hilfe eines endlichen Automaten erkannt werden kann.

Wir zeigen, dass es einen endlichen Automaten gibt, der die W"orter
von $L_n$ akzeptiert. Wenn $r_0(w)\equiv |w|_{\tt 0}\mod n$
der Rest der Anzahl {\tt 0} bei
Teilung durch $n$ und $r_1(w)\equiv |w|_{\tt 1}\mod n$ der Rest der Anzahl
{\tt 1} bei Teilung
durch $n$ ist, dann interessieren uns nur die W"orter, f"ur die die
beiden Zahlen gleich sind, also
\[
L_n=\{w\in\Sigma^*\;|\;r_0(w)-r_1(w)=0\}
\]
Die Differenz kann aber mit folgendem endlichen Automaten bestimmt
werden. Als Zust"ande dienen die m"oglichen Differenzen
$\{0,1,2,\dots,n-1\}$.
Startzustand ist $0$.
Jede {\tt 0} in $w$ erh"oht den Zustand um
eins, jede {\tt 1} erniedrigt den Zustand um $1$. Akzeptierzustand
ist $0$. Das Zustandsdiagramm f"ur den Fall $n=8$ ist
\[
\entrymodifiers={++[o][F]}
\xymatrix @-1mm {
*+\txt{}\ar[dr]
\\
*+\txt{}
        &*++[o][F=]{0}\ar@/^/[r]^{\tt 0} \ar@/^/[dl]^{\tt 1}
                &{1}\ar@/^/[dr]^{\tt 0} \ar@/^/[l]^{\tt 1}
                        &*+\txt{}
\\
{7}\ar@/^/[ur]^{\tt 0} \ar@/^/[d]^{\tt 1}
        &*+\txt{}
                &*+\txt{}
                        &{2}\ar@/^/[d]^{\tt 0} \ar@/^/[ul]^{\tt 1}
\\
{6}\ar@/^/[u]^{\tt 0} \ar@/^/[dr]^{\tt 1}
        &*+\txt{}
                &*+\txt{}
                        &{3}\ar@/^/[dl]^{\tt 0} \ar@/^/[u]^{\tt 1}
\\
*+\txt{}
        &{5}\ar@/^/[ul]^{\tt 0} \ar@/^/[r]^{\tt 1}
                &{4}\ar@/^/[l]^{\tt 0} \ar@/^/[ur]^{\tt 1}
                        &*+\txt{}
}
\]
Somit ist klar, dass $L_n$ regul"ar ist.

Alternativ k"onnte man auch zwei Automaten bilden, deren Zust"ande
die Reste von $|w|_0$ und $|w|_1$ bei Teilung durch $n$ sind:
\[
\xymatrix @-1mm {
        &
                &
                        &
                                &
                                        &
                                                &
\\
*+\txt{}\ar[r]
        &*++[o][F]{0}\ar[r]^{\tt 0}
                &*++[o][F]{1}\ar[r]^{\tt 0}
                        &*++[o][F]{2}\ar[r]^{\tt 0}
                                &*++[o][F]{3}\ar[r]^{\tt 0}
                                        &{\dots}\ar[r]^{\tt 0}
                                                &*++[o][F]{n-1}\ar `u[l] `[lllll]_{\tt 0} [lllll]
\\
        &
                &
                        &
                                &
                                        &
                                                &
\\
*+\txt{}\ar[r]
        &*++[o][F]{0}\ar[r]^{\tt 1}
                &*++[o][F]{1}\ar[r]^{\tt 1}
                        &*++[o][F]{2}\ar[r]^{\tt 1}
                                &*++[o][F]{3}\ar[r]^{\tt 1}
                                        &{\dots}\ar[r]^{\tt 1}
                                                &*++[o][F]{n-1}\ar `u[l] `[lllll]_{\tt 1} [lllll]
}
\]
Bildet man jetzt den kartesischen Produkt-Automaten, sind genau
die Zust"ande auf den Diagonalen diejenigen, die W"orter akzeptieren,
die module $n$ die gleiche Zahl von {\tt 0} und {\tt 1} haben:
\[
\xymatrix @-1mm {
*+\txt{}\ar[dr]
        &{\vdots}\ar[d]^{\tt 1}
                &{\vdots}\ar[d]^{\tt 1}
                        &{\vdots}\ar[d]^{\tt 1}
                                &
                                        &{\vdots}\ar[d]^{\tt 1}
\\
{\dots}\ar[r]^{\tt 0}
        &*++[o][F=]{ }\ar[r]^{\tt 0}\ar[d]^{\tt 1}
                &*++[o][F]{ }\ar[r]^{\tt 0}\ar[d]^{\tt 1}
                        &*++[o][F]{ }\ar[r]^{\tt 0}\ar[d]^{\tt 1}
                                &{\dots}\ar[r]^{\tt 0}
                                        &*++[o][F]{ }\ar[r]^{\tt 0}\ar[d]^{\tt 1}
                                                &{\dots}
\\
{\dots}\ar[r]^{\tt 0}
        &*++[o][F]{ }\ar[r]^{\tt 0}\ar[d]^{\tt 1}
                &*++[o][F=]{ }\ar[r]^{\tt 0}\ar[d]^{\tt 1}
                        &*++[o][F]{ }\ar[r]^{\tt 0}\ar[d]^{\tt 1}
                                &{\dots}\ar[r]^{\tt 0}
                                        &*++[o][F]{ }\ar[r]^{\tt 0}\ar[d]^{\tt 1}
                                                &{\dots}
\\
{\dots}\ar[r]^{\tt 0}
        &*++[o][F]{ }\ar[r]^{\tt 0}\ar[d]^{\tt 1}
                &*++[o][F]{ }\ar[r]^{\tt 0}\ar[d]^{\tt 1}
                        &*++[o][F=]{ }\ar[r]^{\tt 0}\ar[d]^{\tt 1}
                                &{\dots}\ar[r]^{\tt 0}
                                        &*++[o][F]{ }\ar[r]^{\tt 0}\ar[d]^{\tt 1}
                                                &{\dots}
\\
        &\vdots\ar[d]^{\tt 1}
                &\vdots\ar[d]^{\tt 1}
                        &\vdots\ar[d]^{\tt 1}
                                &
                                        &\vdots\ar[d]^{\tt 1}
\\
{\dots}\ar[r]^{\tt 0}
        &*++[o][F]{ }\ar[r]^{\tt 0}\ar[d]^{\tt 1}
                &*++[o][F]{ }\ar[r]^{\tt 0}\ar[d]^{\tt 1}
                        &*++[o][F]{ }\ar[r]^{\tt 0}\ar[d]^{\tt 1}
                                &{\dots}\ar[r]^{\tt 0}
                                        &*++[o][F=]{ }\ar[r]^{\tt 0}\ar[d]^{\tt 1}
                                                &{\dots}
\\
        &{\vdots}
                &{\vdots}
                        &{\vdots}
                                &
                                        &{\vdots}
                                                &
}
\]
Darin f"uhren die Pfeile, die das Diagramm am rechten oder unteren Rand
verlassen wieder zu den Zust"anden in der gleichen Zeile bzw.~Spalte
am linken bzw.~oberen Rand. Nat"urlich ist dieser Automat nicht minimal,
denn alle Zust"ande auf der Diagonalen oder Zust"ande auf einer Geraden
parallel zur Diagonalen sind "aquivalent.

Man kann den Nachweis aber auch f"uhren, ohne einen Automaten zu zeichnen.
Dazu "uberlegt man sich, dass die Sprachen
\[
L_{ik}=\{w\in\Sigma^*|\; |w|_i\cong k\mod n\}= L( i^k(i^n)^* )
\]
regul"ar sind, weil sie sich durch den angegeben regul"aren Ausdruck
beschreiben lassen. Die Sprache
$L_{{\tt 0}k}\cap L_{{\tt 1}k}$
besteht
aus den W"ortern, in denen die Anzahl der {\tt 0} und {\tt 1} den
Rest $k$ bei Teilung durch $n$ haben. Als Durchschnitt regul"arer
Sprachen ist sie nat"urlich auch wieder regul"ar. In der Sprache $L_n$
sind jetzt aber alle solchen W"orter mit Resten $k=0,\dots,n-1$, also
\[
L_n=\bigcup_{k=0}^{n-1}
L_{{\tt 0}k}\cap L_{{\tt 1}k},
\]
was als Vereinigung regul"arer Sprachen nat"urlich auch wieder regul"ar ist.
\end{loesung}

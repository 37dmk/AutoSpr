Mitglieder sogenannter Netzgruppen (netgroups) in Unix sind Tripel
bestehend aus aus Hostname, Username und Domain. Es wird die "ubliche
mathematische Notation f"ur Tripel verwendet, zum Beispiel sind
\begin{center}
\begin{tabular}{l}
{\tt ( asterix, hans, )}\\
{\tt ( obelix , heiri, )}\\
{\tt ( asterix,, hsr.ch )}
\end{tabular}
\end{center}
solche Tripel, sie werden auch Netgrouptriples genannt.
Die Namen k"onnen Buchstaben, Ziffern, Punkt und Unterstrich
enthalten, die Komponenten k"onnen auch leer sein.
\begin{teilaufgaben}
\item
Ist die Sprache $L_1$ der Netgrouptriples regul"ar?
\item
Wenn zus"atzlich verlangt wird, dass mindestens eine der drei
Komponenten nicht leer ist--wir nennen diese neue Sprache $L_2$-- ist $L_2$
dann auch regul"ar?
\end{teilaufgaben}

\begin{loesung}
\begin{teilaufgaben}
\item
Es gibt einen regul"aren Ausdruck, der die W"orter der Sprache $L_1$
akzeptiert:
\verbatimainput{aufg1a.lsg}
also ist die Sprache $L_1$ regul"ar.
\item
Die Sprache $L_\emptyset$ der leeren Netgrouptriples ist regul"ar, denn sie ist die
Sprache der W"orter, die auf den regul"aren Ausdruck
\verbatimainput{aufg1b.lsg}
passen, $L_\emptyset = L(\text{\tt ( *, *, *)})$. Dann ist
aber auch $L_2=L_1\setminus L_\emptyset$ als Differenz regul"arer
Sprachen regul"ar.
\end{teilaufgaben}
\end{loesung}

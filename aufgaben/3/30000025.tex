In dieser Aufgabe geht es um die Frage, ob man bin"are Ganzzahlen mit
Hilfe eines endlichen Automaten auf Gleichheit testen kann.
Sei $\Sigma=\{\texttt{0},\texttt{1},\texttt{=}\}$. Ist die Sprache
\[
L=\{ w\texttt{=}w|w\in \{\texttt{0},\texttt{1}\}^*\}
\]
regul"ar?

\begin{hinweis}
Bitte beachten Sie die beiden unterschiedlichen Bedeutungen von $=$
bzw.~\texttt{=}.
Das eine ist das mathematische Gleichheitszeichen, das andere ein Symbol
im Alphabet $\Sigma$.
Die beiden Zeichen sind zwar in zwei verschiedenen Schriftarten gesetzt,
die normalerweise Zeichen von Formelsymbolen zu unterscheiden gestatten,
aber im Falle des Gleichheitszeichens sind die Unterschiede verschwindend
klein.
\end{hinweis}

\begin{loesung}
Die Sprache ist nicht regul"ar, wie man mit Hilfe des Pumping
Lemmas beweisen kann. Dazu nehmen wir an, die Sprache sei regul"ar.
Das Pumping Lemma garantiert, dass es eine Zahl $N$ gibt, so dass
sich W"orter l"anger als diese Zahl in einer speziellen Art
zerlegen und ``aufpumpen'' lassen. Wir w"ahlen das Wort
\[
w=\texttt{10}^N\texttt{=10}^N,
\]
welches offenbar zur Sprache geh"ort und auch l"anger ist als $N$.
Das Pumping Lemma garantiert, dass $w=xyz$  geschrieben werden kann
mit $|xy|\le N$ und $|y|>0$:
\begin{center}
\includeagraphics[]{pl-1.pdf}
\end{center}
Aus der Konstruktion sieht man, dass $y$ eine der beiden Formen
\[
\texttt{0}^k
\quad\text{oder}\quad
\texttt{10}^k
\]
haben muss (die zweite nur falls $|x|=0$).
Welche Form auch immer $y$ hat, nach dem Aufpumpen wird
auf der linken Seite des Gleichheitszeichens eine Bin"arzahl mit
einer f"uhrenden \texttt{1} und mit mehr Stellen stehen, die Gleichheit
ist also nicht mehr erf"ullt, das aufgepumpte Wort ist nicht in der
Sprache, $xy^lz\not\in L$. Dieser Widerspruch zur Aussage des Pumping
Lemmas zeigt, dass die Annahme, $L$ sei regul"ar, falsch gewesen
sein muss.
\end{loesung}



Verwandeln sie die folgenden regul"aren Ausdr"ucke in NEAs "uber
dem Alphabet $\Sigma =\{{\tt a},{\tt b}\}$:
\begin{teilaufgaben}
\item ${\tt a}({\tt abb})^*| {\tt b}$
\item ${\tt a}^+|({\tt ab})^+$
\end{teilaufgaben}

\begin{hinweis}
Das Zeichen {\tt +} bei regul"aren Ausdr"ucken bedeutet
``mindestens ein mal''.
\end{hinweis}

\ifthenelse{\boolean{loesungen}}{
\begin{loesung}
Wir wenden die Standardkonstruktionen f"ur die Umwandlung eines regul"aren
Ausdrucks in einen NEA an. Die regul"aren Ausdr"ucke werden dabei
``von innen nach aussen'' "ubersetzt.
\begin{teilaufgaben}
\item
Der Teilausdruck {\tt abb} ist eine Verkettung von drei NEAs, die
genau ein ein Zeichen langes Wort akzeptieren:
\[
\entrymodifiers={++[o][F]}
\xymatrix{
*+\txt{}\ar[r]
        &{}\ar[r]^{\tt a}
                &\ar[r]^{\varepsilon}
                        &{}\ar[r]^{\tt b}
                                &\ar[r]^{\varepsilon}
                                        &{}\ar[r]^{\tt b}
                                                &*++[o][F=]{}
}
\]
F"ur die $*$-Operation braucht man einen zus"atzlichen Akzeptierzustand und
``R"uckw"arts''-Pfeile:
\[
\entrymodifiers={++[o][F]}
\xymatrix{
*+\txt{}\ar[r]
        &*++[o][F=]{} \ar[r]^{\varepsilon}
                &\ar[r]^{\tt a}
                        &\ar[r]^{\varepsilon}
                                &\ar[r]^{\tt b}
                                        &\ar[r]^{\varepsilon}
                                                &{}\ar[r]^{\tt b}
                                                        &{}\ar@/^10pt/[llllll]^{\varepsilon}
}
\]
Dem $({\tt abb})^*$ muss jetzt noch ein {\tt a} vorangehen:
\[
\entrymodifiers={++[o][F]}
\xymatrix{
*+\txt{}\ar[r]
        &\ar[r]^{\tt a}
        &{} \ar[r]^{\varepsilon}
        &*++[o][F=]{} \ar[r]^{\varepsilon}
                &\ar[r]^{\tt a}
                        &\ar[r]^{\varepsilon}
                                &\ar[r]^{\tt b}
                                        &\ar[r]^{\varepsilon}
                                                &{}\ar[r]^{\tt b}
                                                        &{}\ar@/^10pt/[llllll]^{\varepsilon}
}
\]
Zum Schluss m"ussen wir eine Alternative zwischen diesem Ausdruck und dem
Ausdruck {\tt b} bilden:
\[
\entrymodifiers={++[o][F]}
\xymatrix{
*+\txt{}&*+\txt{}
        &\ar[r]^{\tt a}
        &{} \ar[r]^{\varepsilon}
        &*++[o][F=]{} \ar[r]^{\varepsilon}
                &\ar[r]^{\tt a}
                        &\ar[r]^{\varepsilon}
                                &\ar[r]^{\tt b}
                                        &\ar[r]^{\varepsilon}
                                                &{}\ar[r]^{\tt b}
                                                        &{}\ar@/^10pt/[llllll]^{\varepsilon}
\\
*+\txt{} \ar[r]
        &\ar[ur]^{\varepsilon} \ar[dr]^{\varepsilon}
        &*+\txt{}
        &*+\txt{}
        &*+\txt{}
        &*+\txt{}
        &*+\txt{}
        &*+\txt{}
        &*+\txt{}
\\
*+\txt{}&*+\txt{}
                &\ar[r]^{\tt b}
                        &*++[o][F=]{}
                                &*+\txt{}
}
\]
\item
Es stellt sich die Frage, wie man $r^+$ implementieren will. Man k"onnte
einerseits $r^+$ als $r^*$ implementieren, welches aber das leere Wort
doch nicht akzeptiert. Dazu w"urde man den neuen Start-/Akzeptierzustand,
der in der Konstruktion von $r^*$ vorkommt, nur als Startzustand, nicht
aber als Akzeptierzustand hinzuf"ugen. Dann sehe der Automat wie folgt
aus:
\[
\entrymodifiers={++[o][F]}
\xymatrix{
*+\txt{}
        &*+\txt{}
%               &*++[o][F=]{} \ar[r]^{\varepsilon}
                &{} \ar[r]^{\varepsilon}
                        &{} \ar[r]^{\tt a}
                                &*++[o][F=]{} \ar@/^10pt/[ll]^{\varepsilon}
\\
*+\txt{}\ar[r]
        &\ar[ur]^{\varepsilon} \ar[dr]^{\varepsilon}
\\
*+\txt{}
        &*+\txt{}
%               &*++[o][F=]{} \ar[r]^{\varepsilon}
                &{} \ar[r]^{\varepsilon}
                        &\ar[r]^{\tt a}
                                &\ar[r]^{\varepsilon}
                                        &\ar[r]^{\tt b}
                                                &*++[o][F=]{}\ar@/^10pt/[llll]^{\varepsilon}
}
\]
Man k"onnte aber auch $r^+=rr^*$ implementieren, also als Verkettung
eines Automaten, der $r$ akzeptiert mit einer $*$-Konstruktion von $r$.
Dann s"ahe der Automat wie folgt aus:
\[
\entrymodifiers={++[o][F]}
\xymatrix{
*+\txt{}
        &*+\txt{}
%               &*++[o][F=]{} \ar[r]^{\varepsilon}
                &{}\ar[r]^{\tt a}
                        &{}\ar[r]^{\varepsilon}
                &*++[o][F=]{} \ar[r]^{\varepsilon}
                        &{} \ar[r]^{\tt a}
                                &{} \ar@/^10pt/[ll]^{\varepsilon}
\\
*+\txt{}\ar[r]
        &\ar[ur]^{\varepsilon} \ar[dr]^{\varepsilon}
\\
*+\txt{}
        &*+\txt{}
%               &*++[o][F=]{} \ar[r]^{\varepsilon}
                        &\ar[r]^{\tt a}
                                &\ar[r]^{\varepsilon}
                                        &\ar[r]^{\tt b}
                &{} \ar[r]^{\varepsilon}
                        &*++[o][F=]{} \ar[r]^{\varepsilon}
                        &\ar[r]^{\tt a}
                                &\ar[r]^{\varepsilon}
                                        &\ar[r]^{\tt b}
                                                &{}\ar@/^10pt/[llll]^{\varepsilon}
}
\]
\end{teilaufgaben}
\end{loesung}
}{}


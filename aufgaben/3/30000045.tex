Sei $\Sigma=\{\texttt{a},\texttt{b},\dots,\texttt{z}\}$ das Alphabet
bestehend aus allen Kleinbuchstaben. Die Sprache
\[
L=\{ w\in\Sigma^*
\,
|\,
\text{
f"ur alle $a,b\in\Sigma$ gilt
$|w|_a < |w|_b$, wenn $b$ im Alphabet auf $a$ folgt%
}
\}
\]
besteht aus W"ortern, in denen Buchstaben h"aufiger sind, die sp"ater im
Alphabet stehen (man beachte, dass $a$ und $b$ hier Variablen sind,
nicht die Buchstaben \texttt{a} und \texttt{b}).
Die W"orter
\[
\texttt{abbccc},\;
\texttt{cbabcc},\;
\texttt{sos},\;
\texttt{auauu}
\]
geh"oren zu dieser Sprache, weil die alphabetisch sp"ateren Buchstaben
in den W"ortern immer h"aufiger sind als die alphabetisch
fr"uheren Buchstaben. Hingegen sind die W"orter 
\[
\texttt{aua},\;
\texttt{politik},\;
\texttt{mimikri},\;
\texttt{ente}
\]
nicht in $L$ (\texttt{e} ist alphabetisch vor \texttt{n}, aber h"aufiger
in \texttt{ente}).

Ist die Sprache $L$ regul"ar?

\begin{loesung}
Nein, $L$ ist nicht regl"ar, wie man mit dem Pumping-Lemma beweisen kann.
Wir nehmen dazu an, dass $L$ regul"ar sei, das Pumping Lemma f"ur
regul"are Sprachen besagt dann, dass es die Pumping Length $N$ f"ur die
Sprachen $L$ gibt.
Wir konstruieren jetzt ein Wort, welches die Bedingungen der Sprache
erf"ullt:
\[
w=\texttt{a}^{N+1}\texttt{b}^{N+2}
\]
Das Pumping Lemma besagt dann, dass $w$ geschrieben werden kann als
$w=xyz$, wobei $|xy|\le N$ und $|y|\ge 1$. 
Aufgrund der Konstruktion von $w$ bestehen sowohl $x$ als auch $y$
ausschliesslich aus Buchstaben \texttt{a}.
Ein aufgepumptes Wort hat daher die Form
\[
w_k=xy^kz = \texttt{a}^{N+1+(k-1)|v|}\texttt{b}^{N+2}.
\]
Es gilt f"ur $k > 1$:
\[
N+1+(k-1)|y|\ge N+1+(k-1)=N+k\ge N+2,
\]
das Wort $w_k=xy^kz$ erf"ullt nicht mehr die Spezifikation der Sprache,
da diese $N+1+(k-1)|y| < N+2$ verlangt.
Diese Widerspruch zeigt, dass $L$ nicht regul"ar sein kann.
\end{loesung}

\begin{diskussion}
Dieser Beweis mit dem Pumping-Lemma kann nat"urlich mit jedem beliebigen
anderen Wort der Form $\texttt{a}^N\texttt{b}^{m}$ durchgef"uhrt werden,
solange $m > N$ ist. Zum Beispiel ist $m=2N$ genauso gut.

Der Beweis ist sogar m"oglich mit dem Wort $w=\texttt{a}^{N-1}\texttt{b}^N$,
aber man muss einen zus"atzlichen Spezialfall untersuchen.
Man kann n"amlich $w$ wie folgt unterteilen:
$x=\texttt{a}^{N-1}$, $y=\texttt{b}$ und $z=\texttt{b}^{N-1}$.
Beim Aufpumpen dieser Zerlegung gilt
\[
w_k=xy^kz=\texttt{a}^{N-1}\texttt{b}^k\texttt{b}^{N-1}
=
\texttt{a}^{N-1}\texttt{b}^{k+N-1},
\]
die Bedingung der Sprache ist also f"ur $k>0$ immer erf"ullt, beim Aufpumpen
verl"asst man die Sprache nicht.
F"ur $k=0$, also beim Abpumpen, entsteht aber das Wort
$\texttt{a}^{N-1}\texttt{b}^{N-1}\not\in L$, Abpumpen verl"asst also die
Sprache, man kann den gew"unschten Widerspruch also auch f"ur diese
Unterteilung des Wortes erhalten.
\end{diskussion}

\begin{bewertung}
Annahme regul"ar ({\bf A}) 1 Punkt,
Pumping length ({\bf N}) 1 Punkt,
Konstruktion eines Wortes in der Sprache, welches $N$ verwendet ({\bf W})
1 Punkt,
Unterteilung (allgemein, keine willk"urliche/spezielle Wahl) ({\bf U}) 1 Punkt,
Verletzung der Spracheigenschaft beim Pumpen ({\bf P}) 1 Punkt,
Schluss\-folgerung (nicht regul"ar) ({\bf S}) 1 Punkt.
\end{bewertung}
 

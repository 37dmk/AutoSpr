1905 beschlossen die Kantone, dass alle Autos in der Schweiz mit einem
Kennzeichen versehen werden m"ussen, und teilten jedem Kanton einen
Nummernblock zu. Nach dem Motto ``Mehr wird niemand brauchen'', welches
ja auch schon in der Informatik spektakul"ar gescheitert ist, erhielt
der Kanton Z"urich nur gerade 1000 Nummern, der
Kanton Appenzell Innerrhoden sogar nur 100.
Daher wurde 1933 das heute noch g"ultige System der Autokennzeichen
eingef"uhrt. Die Kennzeichen bestehen aus dem Kantonsk"urzel
oder den Buchstaben ``A'' (Administration), ``P'' (Post) oder
``M'' f"ur Milit"arfahrzeuge gefolgt von einer maximal sechsstelligen Zahl
(keine f"uhrende Nullen).
In Sonderf"allen kann ein weiterer Buchstabe angeh"angt sein, welcher
Sondernutzungen anzeigt: ``U'' f"ur ``Garagennummern'',
``V'' f"ur Mietfahrzeuge und ``Z'' f"ur Zollschilder (befristet).
Formulieren Sie einen regul"aren Ausdruck f"ur schweizerische Autokennzeichen,
der alle eben beschriebenen Kriterien abbildet.

\begin{loesung}
Wir brauchen zun"achst einen regul"aren Ausdruck f"ur die Kantonsk"urzel
und die anderen drei Pr"afixe:
\[
{\tt AG}|
{\tt AR}|
{\tt AI}|
{\tt BL}|
{\tt BS}|
{\tt BE}|
{\tt FR}|
{\tt GE}|
{\tt GL}|
{\tt GR}|
{\tt JU}|
{\tt LU}|
{\tt NE}|
{\tt NW}|
{\tt OW}|
{\tt SH}|
{\tt SZ}|
{\tt SO}|
{\tt SG}|
{\tt TI}|
{\tt TG}|
{\tt UR}|
{\tt VD}|
{\tt VS}|
{\tt ZG}|
{\tt ZH}|
{\tt A}|
{\tt P}|
{\tt M}
\]
Diesem folgt eine maximal sechsstellige Zahl, die aber nicht mit einer
f"uhrenden Null beginnen darf. Wir verwenden die Notationen
\begin{align*}
\text{\tt [1-9]}&=\text{Ziffern von {\tt 1} bis {\tt 9}}
&&=
({\tt 1}|
{\tt 2}|
{\tt 3}|
{\tt 4}|
{\tt 5}|
{\tt 6}|
{\tt 7}|
{\tt 8}|
{\tt 9})
\\
\text{\tt [0-9]}&=\text{Ziffern von {\tt 0} bis {\tt 9}}
&&=
({\tt 0}|
{\tt 1}|
{\tt 2}|
{\tt 3}|
{\tt 4}|
{\tt 5}|
{\tt 6}|
{\tt 7}|
{\tt 8}|
{\tt 9})
\\
r{\text{\tt \{0,5\}}}&=\text{Ziffern 0 bis 5 Kopien von r}
&&=(|r|rr|rrr|rrrr|rrrrr)
\end{align*}
Damit kann man eine maximal sechsstellige Zahl ohne f"uhrende Nullen als
\[
\text{\tt [1-9][0-9]\{0,5\}}
\]
schreiben.
Dann wird noch maximal einer der Zusatzbuchstaben angeh"angt:
\[
(\varepsilon|{\tt U}|{\tt V}|{\tt Z})
\]
Alles zusammen begkommen wir den regul"aren Ausdruck
\[
\small
(
{\tt AG}|
{\tt AR}|
{\tt AI}|
{\tt BL}|
{\tt BS}|
{\tt BE}|
{\tt FR}|
{\tt GE}|
{\tt GL}|
{\tt GR}|
{\tt JU}|
{\tt LU}|
{\tt NE}|
{\tt NW}|
{\tt OW}|
{\tt SH}|
{\tt SZ}|
{\tt SO}|
{\tt SG}|
{\tt TI}|
{\tt TG}|
{\tt UR}|
{\tt VD}|
{\tt VS}|
{\tt ZG}|
{\tt ZH}|
{\tt A}|
{\tt P}|
{\tt M})
\text{\tt [1-9][0-9]\{0,5\}}
(\varepsilon|{\tt U}|{\tt V}|{\tt Z})
\]
Man k"onnte einwenden, dass es nach diesem regul"aren Ausdruck auch
Milit"arfahrzeuge gibt, die man mieten kann, doch in der Aufgabenstellung
wird dies nicht verboten.
\end{loesung}

\begin{bewertung}
Bewertet wird, ob der regul"are Ausdruck alle genannten Kriterien 
erf"ullt, als da sind:
zweistellige Kantonsk"urzel ({\bf K}) 1 Punkt,
alternative K"urzel A/P/M ({\bf A}) 1 Punkt,
Zahl ein- bis sechsstellig ({\bf L}) 1 Punkt,
keine f"uhrende Nullen ({\bf N}) 1 Punkt,
Sondernutzung nur U/V/Z ({\bf S}) 1 Punkt,
Sondernutzugn optional ({\bf O}) 1 Punkt.
\end{bewertung}


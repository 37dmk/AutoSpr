In der Astronomie werden Punkte am Himmel mit Hilfe der Winkel
der geographischen L"ange und Breite angegeben, die im astronomischen
Zusammenhang auch
Rektaszension und Deklination genannt werden.
Allerdings sind viele verschiedene Formate f"ur die
Winkelangabe gebr"auchlich, manchmal werden sogar im gleichen Sternkatalog
verschiedene Format verwendet.
Die Astronomen haben eine gute Intuition f"ur einen Winkel in Minuten,
da der Vollmonddurchmesser ziemlich genau 30 Winkelminuten betr"agt.
"Ubliche Formate sind
\begin{center}
\begin{tabular}{ll}
Format&Beispiel\\
\hline
dezimale Grade&\texttt{65.4321}\\
Grad, dezimale Minuten&\texttt{65 25.926}\\
Grad, Minuten, dezimale Sekunden&\texttt{65 25 55.56}\\
\hline
\end{tabular}
\end{center}
Nat"urlich kann der Nachkommateil auch fehlen, und es ist auch m"oglich, dass
ein Winkel negativ ist.
Stellen Sie einen regul"aren Ausdruck auf, der genau diese Formate
akzeptiert. 

\begin{loesung}
Eine Winkelangabe endet immer mit einem optionalen Nachkommateil
\begin{center}
\tt($\backslash$.[0-9]*)?
\end{center}
Davor stehen maximal drei Gruppen von Ziffern, die jeweils durch
Leerzeichen getrennt sind.
\begin{center}
\tt[0-9]+( *[0-9]+)\{0,2\}
\end{center}
Zusammengesetzt, und mit einem optionalen Vorzeichen versehen:
\begin{center}
\tt (|-|$\backslash$+)[0-9]+( *[0-9]+)\{0,2\}($\backslash$.[0-9]*)?
\end{center}

F"ur die Praxis w"are aber eine andere Form n"utzlicher, n"amlich ein
regul"rare Ausdruck, der nicht nur die G"ultigkeit des Formates
erkennt, sondern auch die einzelnen Teile identifiziert, damit man
diese Teilstrings gleich zur Berechnung des Winkelwertes heranziehen
kann. Dieser Ausdruck beginnt immer mit einer Gruppe von Ziffern
\begin{center}
\tt [0-9]+
\end{center}
Diese Gruppe kann jetzt entweder von einem Nachkommateil gefolgt sein
oder von einer weiteren Gruppe
\begin{center}
\tt [0-9]+($\backslash$.[0-9]*| {\rm weitere Gruppen } )?
\end{center}
Die weiteren Gruppen haben aber genau das gleiche Format wie das,
was der bisherige Ausdruck darstellt, man kann also den Ausdruck einfach
rekursiv nochmals einpacken
\begin{center}
\tt
(|-|$\backslash$+)[0-9]+($\backslash$.[0-9]*| [0-9]+($\backslash$.[0-9]*| [0-9]+($\backslash$.[0-9]*)?)?)?
\end{center}
Bei der letzten Gruppe kann nur ein Kommateil folgen, etwas kleineres
als Sekunden gibt es ja nicht.
\end{loesung}

\begin{bewertung}
Regul"arer Ausdruck akzeptiert die drei Beispiele ({\bf B}) je 1 Punkt,
regul"arer Ausdruck azkeptiert Vorzeichen +, - ({\bf V}) 1 Punkt,
regul"arer Ausdruck akzeptiert Zahl von Nachkommateil ({\bf N}) 1 Punkt,
regul"arer Ausdruck akzeptiert keine anderen Zahlen ({\bf X}) 1 Punkt.
\end{bewertung}


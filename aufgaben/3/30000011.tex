Welche der folgenden Aussagen ist wahr~bzw.~falsch:
\begin{teilaufgaben}
\item Jede Teilmenge einer regul"aren Sprache ist regul"ar.
\item Jede Obermenge einer regul"aren Sprache ist regul"ar.
\item Die Vereinigung zweier nicht regul"arer Sprachen ist nicht regul"ar.
\item Die Schnittmenge zweier nicht regul"arer Sprachen ist nicht
regul"ar.
\end{teilaufgaben}

\begin{loesung}
\begin{teilaufgaben}
\item $\Sigma^*$ ist regul"ar, w"are a) war, m"usste jede Sprache regul"ar
sein, aber wir kennen nicht regul"are Sprachen, zum Beispiel
$\{0^n1^n\,|\,n\mathbb N\}$.
\item $\emptyset$ ist regul"ar, da aber $\emptyset$ in jeder anderen
Teilmenge von $\Sigma^*$ enthalten ist, m"ussten alle Sprachen regul"ar
sein, im Widerspruch zur Tatsache, dass wir nicht regul"are Sprachen
kennen.
\item Wenn $L$ nicht regul"ar ist, dann ist auch $\bar L$ nicht regul"ar.
W"are n"amlich $\bar L$ regul"ar, m"usste auch $\bar{\bar L}=L$ regul"ar
sein. Es ist aber auch $L\cup \bar L=\Sigma^*$ die Vereinigung zweier
nicht regul"arer Sprachen, trotzdem ist $\Sigma^*$ nat"urlich regul"ar.
\item Das Komplement einer nicht regul"aren Sprache ist ebenfalls nicht
regul"ar. W"are die Vereinigung zweier nicht regul"arer Sprachen $L_1$
und $L_2$ nicht regul"ar m"usste auch $\bar L_1\cup \bar L_2$ nicht
regul"ar sein, und damit auch $\overline{\bar L_1\cup \bar L_2}=L_1\cap L_2$.
Dass die Schnittmenge nicht regul"arer Sprachen aber trotzdem regul"ar
sein kann haben wir bereits in b) eingesehen.
\end{teilaufgaben}
\end{loesung}

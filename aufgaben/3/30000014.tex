EAN-Code und ISBN-Nummer verwenden zum Schutz vor "Ubertragungsfehlern
eine Pr"ufziffer. Zur Berechnung der Pr"ufziffer werden alle
Stellen abwechselnd mit 3 oder 1 multipliziert, diese Produkte werden
addiert. Der Zehner-Rest der Summe ist dann die Pr"ufziffer.
Beispiel: der Code {\tt 1291} bekommt die Pr"ufziffer
\[
1\cdot 3+2\cdot1+9\cdot3+1\cdot 1=35\equiv 5\mod10.
\]
Der mit Pr"ufziffer gesicherte Code {\tt 1291} ist also {\tt 12915}.
Gibt es einen regul"aren Ausdruck, auf den nur die Codes mit korrekter
Pr"ufziffer passen?

\ifthenelse{\boolean{loesungen}}{
\begin{loesung}
Ja. Die Sprache der Codes mit korrekter Pr"ufziffer ist regul"ar,
denn es gibt einen endlichen Automaten daf"ur, den wir unten konstruieren
werden. Folglich gibt es auch einen regul"aren Ausdruck daf"ur.

Auf den ersten Blick ist vielleicht etwas "uberraschend, dass
ein endlicher Automat die Pr"ufziffer ausrechnen k"onnen soll,
schliesslich hat man ja gelernt, dass ein endlicher Automat
nicht rechnen k"onne und sich das Zwischenresultat nicht merken
k"onnen. Allerdings gibt es nicht viel, was man sich merken
muss. Von der Summe muss man sich nur an die letzte Ziffer
erinnern, da ja nur der Zehnerrest interessiert. Ausderdem
muss man sicht daran erinnern, ob die n"achste Ziffer mit
3 oder mit 1 multipliziert werden muss. An eine solch beschr"ankte
Informationsmenge kann sich eine DEA sehr wohl erinnern, das sind
seine endlich vielen Zust"ande, die er hat. Auch muss er nicht
wirklich rechnen k"onnen, es reicht, wenn er Reste berechnen
kann, und dass das m"oglich ist, wurde in der Vorlesung gezeigt.
Auch muss man sich nicht an bereits analysierte Ziffern erinnern,
man braucht nur die bis zu diesem Zeitpunkt berechnete Pr"ufziffer.

Nehmen wir, wir h"atten ein Automaten, der Pr"ufziffern
berechnen kann. Dieser Automat befindet sich nach dem Lesen eines
Wortes in einem Zustand $z$, und zu jedem Zustand geh"ort eine
einzige korrekte Pr"ufziffer $p(z)$. Wenn wir daraus einen
Automaten bauen wollen, der Codes mit korrekter Pr"ufziffer
akzeptiert, dann m"ussen wir den bestehenden Automaten f"ur
jeden Zustand um das Fragment
\[
\entrymodifiers={++[o][F]}
\xymatrix{
{z}\ar[r]^{p(z)}
        &*++[o][F=]{a}
}
\]
erweitern.
Diese bedeutet, dass das Wort akzeptiert wird, wenn in diesem
Zustand die zugeh"orige Pr"ufziffer $p(z)$ folgt. So entsteht
ein NEA.

Wir m"ussen jetzt also einen Automaten konstruieren, der die
Pr"ufziffer berechnen kann.
Ohne die Faktoren $3$ in der Pr"ufzifferformel
w"are dies einfach. Wir nehmen als Zust"ande die Pr"ufziffer,
die folgen m"usste, wenn der Code an dieser Stelle fertig w"are.
Der Automat hat dann die Tabellendarstellung
\begin{center}
\begin{tabular}{|c|cccccccccc|}
\hline
&{\tt 0} &{\tt 1} &{\tt 2} &{\tt 3} &{\tt 4} &{\tt 5} &{\tt 6} &{\tt 7} &{\tt 8} &{\tt 9}\\
\hline
0&0&1&2&3&4&5&6&7&8&9\\
1&1&2&3&4&5&6&7&8&9&0\\
2&2&3&4&5&6&7&8&9&0&1\\
3&4&5&6&7&8&9&0&1&2&3\\
4&5&6&7&8&9&0&1&2&3&4\\
5&6&7&8&9&0&1&2&3&4&5\\
6&7&8&9&0&1&2&3&4&5&6\\
7&8&9&0&1&2&3&4&5&6&7\\
8&9&0&1&2&3&4&5&6&7&8\\
9&0&1&2&3&4&5&6&7&8&9\\
\hline
\end{tabular}
\end{center}
W"urde stattdessen immer der Faktor $3$ ber"ucksichtigt, w"are
die "Ubergangsfunktion
\begin{center}
\begin{tabular}{|c|cccccccccc|}
\hline
&{\tt 0} &{\tt 1} &{\tt 2} &{\tt 3} &{\tt 4} &{\tt 5} &{\tt 6} &{\tt 7} &{\tt 8} &{\tt 9}\\
\hline
0&0&3&6&9&2&5&8&1&4&7\\
1&1&4&7&0&3&6&9&2&5&8\\
2&2&5&8&1&4&7&0&3&6&9\\
3&3&6&9&2&5&8&1&4&7&0\\
4&4&7&0&3&6&9&2&5&8&1\\
5&5&8&1&4&7&0&3&6&9&2\\
6&6&9&2&5&8&1&4&7&0&3\\
7&7&0&3&6&9&2&5&8&1&4\\
8&8&1&4&7&0&3&6&9&2&5\\
9&9&2&5&8&1&4&7&0&3&6\\
\hline
\end{tabular}
\end{center}
Nun m"ussen aber beide "Uberg"ange gemischt verwendet werden.
Bei den Ziffern mit gerader Nummer m"ussen wir die erste Tabelle
verwenden, bei der Ziffer mit ungerader Nummer die zweite Tabelle.
Wir m"ussen also unterscheiden k"onnen, ob wir gerade an einer Ziffer
mit gerader oder ungerader Nummer arbeiten. Wir tun dies, indem wir
die Zustande verdoppeln. Zu jedem Zustand $i$ gibt es jetzt noch
einen Zustand $i'$. Das Zeichen {\tt 0} f"uhrt dann den Zustand $i$
nicht mehr in den Zustand $i$, sondern $i'$. Und das Zeichen {\tt 1}
f"uhrt $i'$ nicht in $(i+3)'$, sondern in $i+3$ "uber. Der vollst"andige
Automat zur Berechnung der Pr"ufziffer ist also
\begin{center}
\begin{tabular}{|>{$}c<{$}|>{$}c<{$}>{$}c<{$}>{$}c<{$}>{$}c<{$}>{$}c<{$}>{$}c<{$}>{$}c<{$}>{$}c<{$}>{$}c<{$}>{$}c<{$}|}
\hline
&{\tt 0} &{\tt 1} &{\tt 2} &{\tt 3} &{\tt 4} &{\tt 5} &{\tt 6} &{\tt 7} &{\tt 8} &{\tt 9}\\
\hline
0&0'&1'&2'&3'&4'&5'&6'&7'&8'&9'\\
1&1'&2'&3'&4'&5'&6'&7'&8'&9'&0'\\
2&2'&3'&4'&5'&6'&7'&8'&9'&0'&1'\\
3&4'&5'&6'&7'&8'&9'&0'&1'&2'&3'\\
4&5'&6'&7'&8'&9'&0'&1'&2'&3'&4'\\
5&6'&7'&8'&9'&0'&1'&2'&3'&4'&5'\\
6&7'&8'&9'&0'&1'&2'&3'&4'&5'&6'\\
7&8'&9'&0'&1'&2'&3'&4'&5'&6'&7'\\
8&9'&0'&1'&2'&3'&4'&5'&6'&7'&8'\\
9&0'&1'&2'&3'&4'&5'&6'&7'&8'&9'\\
0'&0&1&2&3&4&5&6&7&8&9\\
1'&1&2&3&4&5&6&7&8&9&0\\
2'&2&3&4&5&6&7&8&9&0&1\\
3'&4&5&6&7&8&9&0&1&2&3\\
4'&5&6&7&8&9&0&1&2&3&4\\
5'&6&7&8&9&0&1&2&3&4&5\\
6'&7&8&9&0&1&2&3&4&5&6\\
7'&8&9&0&1&2&3&4&5&6&7\\
8'&9&0&1&2&3&4&5&6&7&8\\
9'&0&1&2&3&4&5&6&7&8&9\\
\hline
\end{tabular}
\end{center}
Es gibt also einen DEA, der die Pr"ufziffer berechnet. Damit k"onnen wir
wie zu Beginn festgestellt auch einen NEA zur "Uberpr"ufung der Pr"ufziffer
bauen, indem wir die "Uberg"ange
\[
\entrymodifiers={++[o][F]}
\xymatrix{
{i}\ar[r]^{i}
        &*++[o][F=]{a}
                &*+\txt{und}
                        &{i'}\ar[r]^{i}
                                &*++[o][F=]{a}
}
\]
hinzuf"ugen.

Vielfach wurde argumentiert, diese Sprache k"onne nie und nimmer
regul"ar sein, denn bei Pr"ufziffer w"urde ja beim Aufpumpen
kaputt gehen. Das muss nicht sein, aus Teilst"uck $y$ darf
einfach keinen Einfluss auf die Pr"ufziffer haben. Solche
Teilst"ucke h"atten f"ur sich allein genommen die Pr"ufziffer 0.
Dazu geh"oren zum Beispiel
{\tt 00},
{\tt 13},
{\tt 26},
{\tt 39},
{\tt 42},
{\tt 55},
{\tt 68},
{\tt 71},
{\tt 84}
und
{\tt 97}, wenn sie an einer geraden Stelle beginnen, oder
die gleichen Zeichenfolgen in umgekehrter Reihenfolge, wenn sie
an einer ungeraden Stelle beginnen.
Dazu gibt es noch viele Bl"ocke mit 4 Zeichen, zum Beispiel
{\tt 1114} oder {\tt 1230}. Es gibt also eine grosse Auswahl von
m"oglichen Strings, die man aufpumpen kann, und wenn ein Wort
lang genug ist (Pumping-Length), wird man sicher eines finden.

Dass dem so ist kann man sich auch so "uberlegen. Es gibt 10
verschiedene m"ogliche Pr"ufziffern. Nach jeder Ziffer des Codes m"usste,
wenn der Code dort zu Ende w"are, eine dieser 10 m"oglichen Pr"ufziffern
folgen. Da es nur 10 m"ogliche Pr"ufziffern gibt, muss sie sich
wiederholen.
Das Wortst"uck zwischen der ersten Wiederholung f"uhrt hat f"ur sich
genommen Pr"ufziffer 0 haben. Damit ist es aber nat"urlich noch
nicht aufpumpbar, es muss auch noch gerade L"ange haben. Wenn man
aber ein Wort von mindestens 22 Zeichen L"ange hat, dann gibt es
darin nach genau diesem Argument ein Teilst"uck gerader L"ange,
welches f"ur sich genommen die Pr"ufziffer 0 hat, also ein Teilst"uck,
welches aufgepumpt werden kann. Dieses Argument ist nat"urlich
sehr "ahnlich wie der urspr"ungliche Beweis des Pumping Lemmas.

Als Beispiel betrachten wir das Wort
\[
w=\text{\tt 10101010101010101010}
\]
Die Pr"ufziffern von Teilw"ortern sind in folgender Tabelle
zusammengstellt:
\begin{center}
\begin{tabular}{|l|r|}
\hline
Teilwort&Pr"ufziffer\\
\hline
$\varepsilon$             &0\\
{\tt 1}                   &1\\
{\tt 10}                  &1\\
{\tt 101}                 &2\\
{\tt 1010}                &2\\
{\tt 10101}               &3\\
{\tt 101010}              &3\\
{\tt 1010101}             &4\\
{\tt 10101010}            &4\\
{\tt 101010101}           &5\\
{\tt 1010101010}          &5\\
{\tt 10101010101}         &6\\
{\tt 101010101010}        &6\\
{\tt 1010101010101}       &7\\
{\tt 10101010101010}      &7\\
{\tt 101010101010101}     &8\\
{\tt 1010101010101010}    &8\\
{\tt 10101010101010101}   &9\\
{\tt 101010101010101010}  &9\\
{\tt 1010101010101010101} &0\\
{\tt 10101010101010101010}&0\\
\hline
\end{tabular}
\end{center}
Daraus kann man ablesen, dass die ersten 20 Zeichen ein Teilst"uck
bilden, welches aufgepumpt werden kann. Jedes Wort, welches mit
$w$ beginnt und im "ubrigen eine korrekte Pr"ufziffer hat, kann
aufgeumpt werden, indem diese 20 Zeichen aufgepumpt werden.
Ein Beispiel eines solchen Wortes w"are
\[
\text{\tt 1010101010101010101012915}.
\]
\end{loesung}
}{ }


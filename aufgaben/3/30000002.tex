Konstruieren Sie zu jeder der folgenden Sprachen einen
endlichen Automaten, der genau diese Sprache akzeptiert.
Die Sprachen verwenden die Alphabete
$\Sigma_1=\{{\tt a},{\tt b}\}$ und
$\Sigma_2=\{{\tt 0},{\tt 1}\}$.
\begin{teilaufgaben}
\item
$L=\{w\in \Sigma_1^*|\text{
$w$ beginnt mit $a$ und enth"alt h"ochstens ein $b$
}\}$
\item
$L=\{w\in \Sigma_1^*|\text{
$w$ enth"alt eine gerade Zahl {\tt a}s und mindestens
zwei {\tt b}s
}\}$
\item
$L=\{w\in \Sigma_1^*|\text{
$w$ hat gerade L"ange und eine ungerade Anzahl {\tt b}s
}\}$
\item
$L=\{w\in \Sigma_1^*|\text{
$w$ enth"alt nicht genau zwei {\tt a}s
}\}$
\item
$L=\{w\in \Sigma_2^*|\text{
an jeder ungeraden Position von $w$ steht ein {\tt 1}
}\}$
\item
$L=\{w\in \Sigma_2^*|\;|w|\le 5\}$
\item
$L=\emptyset\subset \Sigma_2^*$
\end{teilaufgaben}

\begin{loesung}
\begin{teilaufgaben}
\item
Die Zust"ande haben folgende Bedeutung:
\begin{center}
\begin{tabular}{c|l}
Zustand&Bedeutung\\
\hline
$z_0$&noch kein {\tt a}
\\
$z_1$&beginnt mit {\tt a}, noch kein {\tt b}
\\
$z_2$&beginnt mit {\tt a}, genau ein {\tt b}
\\
$z_3$&beginnt mit {\tt b} oder mehr als ein {\tt b}
\\
\end{tabular}
\end{center}
\[
\entrymodifiers={++[o][F]}
\xymatrix @-1mm {
*+\txt{} \ar[r]
        &{z_0} \ar[r]^{\tt a} \ar[d]^{\tt b}
                &*++[o][F=]{z_1} \ar[d]^{\tt b} \ar@(dr,ur)_{\tt a}
\\
*+\txt{}
        &{z_3} \ar@(dr,dl)^{{\tt a},{\tt b}}
                &*++[o][F=]{z_2} \ar@(dr,dl)^{\tt a} \ar[l]^{\tt b}
}
\]

Selbstverst"andlich kann man diese Aufgabe auch mit Hilfe des Satzes~3.1
aus der Vorlesung l"osen. Dazu m"ussen die Mengen $L(w)$ bestimmt
werden:
\begin{center}
\begin{tabular}{|c|l|c|}
\hline
$w$&$L(w)$&$z_{?}$\\
\hline
$\varepsilon$&$L$&$z_0$\\
\tt a&$\{w\in\Sigma^*\,|\,\text{$w$ enth"alt h"ochstens 1 {\tt b}}\}$&$z_1$\\
\tt b&$\emptyset$&$z_3$\\
\tt aa&$\{w\in\Sigma^*\,|\,\text{$w$ enth"alt h"ochstens 1 {\tt b}}\}$&$z_1$\\
\tt ab&$\{w\in\Sigma^*\,|\,|w|_{\tt b} = 0\}$&$z_2$\\
\tt ba&$\emptyset$&$z_3$\\
\tt bb&$\emptyset$&$z_3$\\
$\vdots$&$\vdots$&\\
\hline
\end{tabular}
\end{center}
Offenbar gibt es vier verschiedene Mengen, also kommte der DEA von $L$
mit 4 Zust"anden aus. $L({\tt a})$ und $L({\tt ab})$ enth"alten das
leere Wort, sind als Akzeptierzust"ande. Die "Uberg"ange k"onnen ebenfalls
aus der Tabelle abgelesen werden, es ergibt sich der gleiche Automat.

\item Die Zeilen z"ahlen die Anzahl {\tt b}, die bereits gelesen
wurden, die Spalten f"uhren Buch "uber den Zweierrest der Anzahl
{\tt a}.
\[
\entrymodifiers={++[o][F]}
\xymatrix @-1mm {
*+\txt{} \ar[r]
        &{z_0}\ar@/^/[r]^{\tt a} \ar[d]^{\tt b}
                &{z_1}\ar@/^/[l]^{\tt a} \ar[d]^{\tt b}
\\
*+\txt{}
        &{z_2}\ar@/^/[r]^{\tt a} \ar[d]^{\tt b}
                &{z_3}\ar@/^/[l]^{\tt a} \ar[d]^{\tt b}
\\
*+\txt{}
        &*++[o][F=]{z_4}\ar@/^/[r]^{\tt a} \ar@(dr,dl)^{\tt b}
                &{z_5}\ar@/^/[l]^{\tt a} \ar@(dr,dl)^{\tt b}
}
\]
L"osung mit Myhill-Nerorde: Startzustand ist $L$, weitere Mengen
der Form $L(w)$ sind
\begin{center}
\begin{tabular}{c|ll|c}
$w$&$L(w)$&$z_i$&$\varepsilon\in L(w)$\\
\hline
$\varepsilon$&$L$&$=L_0$&nein\\
  {\tt a}&$\{w\in\Sigma^*\;|\;\text{$|w|_{\text{\tt a}}$ ungerade, $|w|_{\text{\tt b}}\ge 2$}\}$&$=L_1$&nein\\
  {\tt b}&$\{w\in\Sigma^*\;|\;\text{$|w|_{\text{\tt a}}$ gerade,   $|w|_{\text{\tt b}}\ge 1$}\}$&$=L_2$&nein\\
 {\tt aa}&$\{w\in\Sigma^*\;|\;\text{$|w|_{\text{\tt a}}$ gerade,   $|w|_{\text{\tt b}}\ge 2$}\}$&$=L_0$&nein\\
 {\tt ab}&$\{w\in\Sigma^*\;|\;\text{$|w|_{\text{\tt a}}$ ungerade, $|w|_{\text{\tt b}}\ge 1$}\}$&$=L_3$&nein\\
 {\tt bb}&$\{w\in\Sigma^*\;|\;\text{$|w|_{\text{\tt a}}$ gerade,   $|w|_{\text{\tt b}}\ge 0$}\}$&$=L_4$&nein\\
{\tt aaa}&$\{w\in\Sigma^*\;|\;\text{$|w|_{\text{\tt a}}$ ungerade, $|w|_{\text{\tt b}}\ge 2$}\}$&$=L_1$&nein\\
{\tt aab}&$\{w\in\Sigma^*\;|\;\text{$|w|_{\text{\tt a}}$ gerade,   $|w|_{\text{\tt b}}\ge 1$}\}$&$=L_2$&nein\\
{\tt abb}&$\{w\in\Sigma^*\;|\;\text{$|w|_{\text{\tt a}}$ ungerade, $|w|_{\text{\tt b}}\ge 0$}\}$&$=L_5$&ja\\
{\tt bbb}&$\{w\in\Sigma^*\;|\;\text{$|w|_{\text{\tt a}}$ gerade,   $|w|_{\text{\tt b}}\ge 0$}\}$&$=L_4$&nein\\
\hline
\end{tabular}
\end{center}
Die Zust"ande $L_i$ entsprechen den $z_i$ im Zustandsdiagramm.
\item Die Zeilen codieren den Zweierrest der Anzahl {\tt b}s,
die Spalten den Zweierrest der Anzahl {\tt a}s.
\[
\entrymodifiers={++[o][F]}
\xymatrix @-1mm {
*+\txt{} \ar[r]
        &{z_0} \ar@/^/[r]^{\tt a} \ar@/^/[dr]_{\tt b}
                &{z_1} \ar@/^/[l]_{\tt a}\ar@/^/[dl]%^{\tt b}
\\
*+\txt{}
        &*++[o][F=]{z_2}\ar@/^/[r]_{\tt a}\ar@/^/[ur]
                &{z_3} \ar@/^/[l]^{\tt a}\ar@/^/[ul]
}
\]
L"osung mit Myhill-Nerode: Als Zust"ande findet man die folgenden
Mengen:
\begin{center}
\begin{tabular}{c|ll|c}
$w$&$L(w)$&$z_i$&$\varepsilon\in L(w)$\\
\hline
$\varepsilon$&$L$&$=L_0$&nein\\
  {\tt a}&$\{w\in\Sigma^*\;|\;\text{$|w|$ ungerade, $|w|_{\text{\tt b}}$ ungerade}\}$&$=L_1$&nein\\
  {\tt b}&$\{w\in\Sigma^*\;|\;\text{$|w|$ ungerade, $|w|_{\text{\tt b}}$   gerade}\}$&$=L_3$&nein\\
 {\tt aa}&$\{w\in\Sigma^*\;|\;\text{$|w|$ gerade,   $|w|_{\text{\tt b}}$ ungerade}\}$&$=L_0$&nein\\
 {\tt ab}&$\{w\in\Sigma^*\;|\;\text{$|w|$ gerade,   $|w|_{\text{\tt b}}$   gerade}\}$&$=L_2$&ja\\
 {\tt bb}&$\{w\in\Sigma^*\;|\;\text{$|w|$ gerade,   $|w|_{\text{\tt b}}$ ungerade}\}$&$=L_0$&nein\\
{\tt aaa}&$\{w\in\Sigma^*\;|\;\text{$|w|$ ungerade, $|w|_{\text{\tt b}}$ ungerade}\}$&$=L_1$&nein\\
{\tt aab}&$\{w\in\Sigma^*\;|\;\text{$|w|$ ungerade, $|w|_{\text{\tt b}}$   gerade}\}$&$=L_3$&nein\\
{\tt abb}&$\{w\in\Sigma^*\;|\;\text{$|w|$ ungerade, $|w|_{\text{\tt b}}$ ungerade}\}$&$=L_1$&nein\\
{\tt bbb}&$\{w\in\Sigma^*\;|\;\text{$|w|$ ungerade, $|w|_{\text{\tt b}}$   gerade}\}$&$=L_3$&nein\\
\hline
\end{tabular}
\end{center}
Die Zustande $L_i$ entsprechen wieder den $z_i$ von vorhin.
\item
In den Zust"anden $z_i,i <3$ hat das Worte genau $i$ {\tt a}s,
im Zustand $z_3$ hat das Wort 3 oder mehr {\tt a}s.
\[
\entrymodifiers={++[o][F]}
\xymatrix @-1mm {
*+\txt{} \ar[r]
        &*++[o][F=]{z_0}\ar[r]^{\tt a} \ar@(dr,dl)^{\tt b}
                &*++[o][F=]{z_1}\ar[r]^{\tt a} \ar@(dr,dl)^{\tt b}
                        &{z_2} \ar@(dr,dl)^{\tt b} \ar[r]^{\tt a}
                                &*++[o][F=]{z_3} \ar@(dr,dl)^{{\tt a},{\tt b}}
}
\]
L"osung mit Myhill-Nerode:
\begin{center}
\begin{tabular}{c|ll|c}
$w$&$L(w)$&$z_i$&$\varepsilon\in L(w)$\\
\hline
$\varepsilon$&$L$&$=L_0$&nein\\
  {\tt a}&$\{w\in\Sigma^*\;|\;\text{$|w|_{\tt a}\ne 1$}\}$&$=L_1$&ja\\
  {\tt b}&$\{w\in\Sigma^*\;|\;\text{$|w|_{\tt a}\ne 2$}\}$&$=L_0$&ja\\
 {\tt aa}&$\{w\in\Sigma^*\;|\;\text{$|w|_{\tt a}\ne 0$}\}$&$=L_2$&nein\\
 {\tt ab}&$\{w\in\Sigma^*\;|\;\text{$|w|_{\tt a}\ne 1$}\}$&$=L_1$&ja\\
 {\tt ba}&$\{w\in\Sigma^*\;|\;\text{$|w|_{\tt a}\ne 1$}\}$&$=L_1$&ja\\
 {\tt bb}&$\{w\in\Sigma^*\;|\;\text{$|w|_{\tt a}\ne 2$}\}$&$=L_0$&ja\\
{\tt aaa}&$\Sigma^*$                                      &$=L_3$&ja\\
{\tt aab}&$\{w\in\Sigma^*\;|\;\text{$|w|_{\tt a}\ne 0$}\}$&$=L_2$&nein\\
{\tt aba}&$\{w\in\Sigma^*\;|\;\text{$|w|_{\tt a}\ne 0$}\}$&$=L_2$&nein\\
{\tt abb}&$\{w\in\Sigma^*\;|\;\text{$|w|_{\tt a}\ne 1$}\}$&$=L_1$&ja\\
{\tt baa}&$\{w\in\Sigma^*\;|\;\text{$|w|_{\tt a}\ne 0$}\}$&$=L_2$&nein\\
{\tt bab}&$\{w\in\Sigma^*\;|\;\text{$|w|_{\tt a}\ne 1$}\}$&$=L_1$&ja\\
{\tt bba}&$\{w\in\Sigma^*\;|\;\text{$|w|_{\tt a}\ne 1$}\}$&$=L_1$&ja\\
{\tt bbb}&$\{w\in\Sigma^*\;|\;\text{$|w|_{\tt a}\ne 2$}\}$&$=L_0$&ja\\
\hline
\end{tabular}
\end{center}
\item Bedeutung der Zust"ande:
\begin{center}
\begin{tabular}{c|l}
Zustand&Beschreibung\\
\hline
$z_0$&gerades Zeichen, Bedingung an ungeraden Stellen immer erf"ullt\\
$z_1$&ungerades Zeichen war eine {\tt 1}\\
$z_2$&{\tt 0} an gerader Stelle\\
\end{tabular}
\end{center}
\[
\entrymodifiers={++[o][F]}
\xymatrix @-1mm {
*+\txt{} \ar[r]
        &*++[o][F=]{z_0}\ar@/^/[r]^{{\tt 1}} \ar[d]^{\tt 0}
                &*++[o][F=]{z_1} \ar@/^/[l]^{{\tt 0},{\tt 1}}
\\
*+\txt{}
        &{z_2}\ar@(ur,dr)^{{\tt 0},{\tt 1}}
                &*+\txt{}
}
\]
L"osung mit Myhill-Nerode: Die Zust"ande sind
\begin{center}
\begin{tabular}{c|ll|c}
$w$&$L(w)$&$z_i$&$\varepsilon\in L(w)$\\
\hline
$\varepsilon$&$L$&$=L_0$&ja\\
  {\tt 0}&$\emptyset$&$=L_0$&ja\\
  {\tt 1}&$\{ w\in\Sigma^*\;|\; \text{an jeder geraden Stelle eine {\tt 1}}\}$&$=L_1$&nein\\
 {\tt 00}&$\emptyset$&$=L_2$&nein\\
 {\tt 01}&$\emptyset$&$=L_2$&nein\\
 {\tt 10}&$L$&$=L_0$&ja\\
 {\tt 11}&$L$&$=L_0$&ja\\
{\tt 000}&$\emptyset$&$=L_2$&nein\\
{\tt 001}&$\emptyset$&$=L_2$&nein\\
{\tt 010}&$\emptyset$&$=L_2$&nein\\
{\tt 011}&$\emptyset$&$=L_2$&nein\\
{\tt 100}&$\emptyset$&$=L_2$&nein\\
{\tt 101}&$L$&$=L_0$&ja\\
{\tt 110}&$\emptyset$&$=L_2$&nein\\
{\tt 111}&$\{ w\in\Sigma^*\;|\; \text{an jeder geraden Stelle eine {\tt 1}}\}$&$=L_1$&nein\\
\hline
\end{tabular}
\end{center}
\item Zustand $z_i, i <6$ bedeutet, dass das Wort genau $i$
Zeichen lang ist. Zustand $z_6$ bedeutet, dass das Wort mindestens
$6$ Zeichen lang ist.
\[
\entrymodifiers={++[o][F]}
\xymatrix @-1mm {
*+\txt{} \ar[r]
        &*++[o][F=]{z_0}\ar[r]^{{\tt 0},{\tt 1}}
        &*++[o][F=]{z_1}\ar[r]^{{\tt 0},{\tt 1}}
        &*++[o][F=]{z_2}\ar[r]^{{\tt 0},{\tt 1}}
        &*++[o][F=]{z_3}\ar[r]^{{\tt 0},{\tt 1}}
        &*++[o][F=]{z_4}\ar[r]^{{\tt 0},{\tt 1}}
        &*++[o][F=]{z_5}\ar[r]^{{\tt 0},{\tt 1}}
        &{z_6}\ar@(dr,ur)_{{\tt 0},{\tt 1}}
}
\]
\item Kein Wort darf akzeptiert werden, d.~h.~die Menge der Endzust"ande
ist leer. Der einfachste Automat, der dies erf"ullt, hat genau einen
Zustand und eine triviale "Ubergangsfunktion:
\[
\entrymodifiers={++[o][F]}
\xymatrix @-1mm {
*+\txt{} \ar[r]
        &{z_0}\ar@(dr,ur)_{{\tt 0},{\tt 1}}
}
\]
L"osung mit Myhill-Nerode: Startzustand ist die Menge $L=\emptyset$
Da $\varepsilon\not\in\emptyset$ ist dieser Zustand kein Startzustand.
Ausserdem sind die $L(w)$ f"ur jedes $w\in\Sigma^*$ weitere Zust"ande.
Nach Definition ist $L(w)=\{w_e\in\Sigma^*\,|\,ww_e\in \emptyset\}=\emptyset$,
es gibt also nur einen einzigen Zustand, damit ist der Automat bereits
vollst"andig definiert.
\end{teilaufgaben}
\end{loesung}


"Ubersetzen sie die folgenden Aussagen soweit m"oglich in
logische Formeln (sie m"ussen die Aussagen selbst nicht beweisen).
\begin{teilaufgaben}
\item Eine Aussage ist entweder wahr oder falsch.
\item Eine reelle Zahl ist entweder positiv, negativ oder $0$.
\item Negative (reelle) Zahlen haben keine reellen Wurzeln.
\end{teilaufgaben}

\begin{loesung}
\begin{teilaufgaben}
\item Wenn die Aussage $P$ heisst: $P\vee\neg P$.
Man k"onnte argumentieren, dass damit noch nicht ausgedr"uckt ist,
dass mit ``entweder\dots oder'' ein ausschliessendes Oder formuliert
worden ist. Dazu m"usse man mit einer Und-Verkn"ufpung den Termen
$\neg(P\wedge \neg P)$
hinzuf"ugen. Aber
$\neg(P\wedge \neg P)=\neg P\vee \neg\neg P=\neg P\vee P=P\vee\neg P$,
dieser Zusatzterm ist also genau dann war, wenn der urspr"ungliche
auch wahr war, bereits $P\vee \neg P$ beinhaltet also ein ausschliessendes
Oder.
\item $\forall x\in\mathbb R(x > 0\vee x<0\vee x=0)$
\item $x\in\mathbb R\wedge x < 0\Rightarrow
\neg\exists w\in\mathbb R(x=w^2)$
oder mit einem All-Quantor: $\forall x\in\mathbb R(x<0\Rightarrow
\neg\exists w\in\mathbb R(x=w^2))$
\end{teilaufgaben}
\end{loesung}


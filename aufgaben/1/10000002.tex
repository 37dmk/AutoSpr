"Ubersetzen sie die folgenden logischen Formeln in deutsche S"atze:
\begin{teilaufgaben}
\item $\exists m\in\mathbb N (n=2m)$
\item $\exists m\in\mathbb N (n=2m)\Rightarrow (-1)^n > 0$
\item $\neg x\in\mathbb R_{>0}\vee \exists w\in\mathbb R(x=w^2)$
\end{teilaufgaben}

\ifthenelse{\boolean{loesungen}}{
\begin{loesung}
\begin{teilaufgaben}
\item $n$ ist eine gerade Zahl.
\item Eine gerade Potenz von $-1$ ist positiv.
\item $\neg x\in\mathbb R_{>0}\vee \exists w\in\mathbb R(x=w^2)$ wird zun"achst
"ubersetzt in
$x\in\mathbb R_{>0}\Rightarrow \exists w\in\mathbb R(x=w^2)$, was man lesen
kann als: wenn $x$ eine positive reelle Zahl ist, dann hat $x$ eine
Quadratwurzel.
\end{teilaufgaben}
\end{loesung}
}{}


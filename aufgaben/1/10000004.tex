Was ist falsch an folgendem Beweis der Behauptung alle Pferde h"atten die
gleiche Farbe?

Beweis mit vollst"andiger Induktion. Die Induktionsverankerung
erfolgt mit $n=1$. F"ur eine Menge mit genau einem Pferd ist
die Behauptung sicher richtig. Wir nehmen jetzt also an, die
Behauptung sei f"ur eine Menge mit $n$ Pferden bereits bewiesen.
Sei also $H$ eine Menge von $n+1$ Pferden. Entfernen wir aus $H$
ein Pferd, erhalten wir eine Menge $H_1$ von $n$ Pferden, nach
Induktionsannahme haben alle diese Pferde die gleiche Farbe.
Tauschen wir das vorher entfernte Pferd gegen ein anderes, erhalten
wir die Menge $H_2$ mit ebenfalls genau $n$ Pferden. Nach Induktionsannahme
haben auch alle Pferde in dieser Menge die gleiche Farbe. Somit
haben die beiden entfernten Pferde die gleiche Farbe wie der Rest,
somit haben alle Pferde die gleiche Farbe. Damit ist der Induktionsschritt
vollzogen, und es ist gezeigt, dass alle Pferde in $H$ die gleiche Farbe
haben.

\begin{loesung}
Das Problem ist der Schritt von $n=1$ zu $n+1=2$.
Aus der Menge mit $n+1$ Pferden soll man ein Pferd entfernen.
Dieses Pferd tauscht man sp"ater aus, um zu zeigen dass die
beiden gew"ahlten Pferde die gleiche Farbe haben. Der zentrale
Schritt dabe ist, dass diese Pferde, wenn sie mit den anderen nicht
gew"ahlten Pferden zusammen in der kleineren Menge drin sind,
die gleiche Farbe haben m"ussen. Man vergleicht also das ausgew"ahlte
Pferd mit den anderen Pferden, die immer in der Menge drin bleiben.
Im Falle $n=1$ gibt es aber solche
anderen Pferde nicht, man kann also auch nicht schliessen, dass
sie die gleiche Farbe h"atten.

Noch konkreter: in der Menge $H$ seien ein weisses und ein schwarzes Pferd.
Entfernen wir das schwarze Pferd, bleibt in $H_1$ das weisse Pferde,
in $H_1$ hat es also tats"achlich nur Pferde einer Farbe. Entfernen
wir das weisse Pferd aus $H$, bleiben in $H_2$ nur schwarze
Pferde. Die Aussagen sind also alle richtig, und trotzdem sind
offenbar die beiden Pferde von verschiedener Farbe. Der Induktionsschritt
ist f"ur $1\to 2$ nicht durchf"uhrbar.
\end{loesung}

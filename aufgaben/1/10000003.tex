Was ist falsch an folgendem Beweis f"ur $2=1$?

Betrachte die Gleichung $a=b$. Multipliziere beide Seiten mit
$a$ um $a^2=ab$ zu erhalten. Subtrahiere $b^2$ auf beiden
Seiten, was $a^2-b^2=ab-b^2$ ergibt. Beide Seiten kann man
faktorisieren und bekommt $(a+b)(a-b)=b(a-b)$. Teile beide
Seiten durch $a-b$, dies ergibt $a+b=b$. Jetzt setzt man
$a=1$ und $b=1$, und es ergibt sich $2=1$.

\ifthenelse{\boolean{loesungen}}{
\begin{loesung}
Im Laufe des Beweises wird durch $(a-b)$ geteilt, ohne dass man
sicherstellt, dass dies auch tats"achlich m"oglich ist.
Tats"achlich wurde sogar davon ausgegangen, dass $a=b$, dass
also in jedem Fall $a-b=0$ ist, die Division ist also unabh"angig davon,
was man sp"ater einsetzt, nicht zul"assig.
F"ur die
gew"ahlten Werte $a=1$ und $b=1$  ist erst recht $a-b=0$.
\end{loesung}
}{}


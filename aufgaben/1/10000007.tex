Zeigen Sie, dass die logischen Formeln
\[
x\vee y\qquad\text{und}\qquad (x\vee z)\wedge (y\vee \neg z)
\]
"aquivalent sind, d.\,h.~es
wenn die Variablen der einen Formel k"onnen genau dann so mit Wahrheitswerten
belegt werden, dass die Formel war wird, wenn dies f"ur die andere Formel
m"oglich ist.

\begin{loesung}
Wenn $x\vee y$ wahr ist, dann ist eine der Variablen, zum Beispiel
$x$ wahr.
Dann kann mann $z$ auf ``falsch'' setzen, was die zweite Formel wahr macht.

Ist umgekehrt die zweite Formel war, und auch $z$, dann ist $\neg z$ nicht 
wahr, also muss $y$ wahr sein. Dann ist auch $x\vee y$ wahr. Dasselbe
Argument funktioniert auch, wenn $z$ falsch ist, dann folgt, dass $x$
wahr sein muss, und wieder ist $x\vee y$ wahr.
\end{loesung}

Betrachten Sie die Behauptung:
\begin{quote}
In einem Graphen mit mindestens zwei Knoten gibt es zwei verschiedene Knoten
mit dem gleichen Grade (der gleichen Anzahl Kanten, die von diesem
Knoten ausgehen).
\end{quote}
Ist die Behauptung wahr oder falsch? Warum (Beweis
oder Gegenbeispiel)?

\begin{hinweis}
Gehen Sie von einem zusammenh"angenden Graphen
aus, und "uberlegen sie sich, welche Grade m"oglich sind.
\end{hinweis}

\begin{loesung}
Die Behauptung ist wahr.

In einem zusammenh"angenden Graphen mit $n$
Ecken kann der Grad einer Ecke nicht gr"osser als $n-1$ sein.
M"oglich sind also nur die $n-1$ verschiedenen Grade
$1,2,\dots,n-1$. Daher m"ussen zwei Ecken den gleichen Grad
haben.

Falls der Graph nicht zusammenh"angend ist, nimmt man eine
Zusammenhangskomponente mit mindestens zwei Ecken, nach dem
eben bewiesenen muss diese zwei Ecken mit dem gleichen
Grad haben.

Falls es keine Zusammenhangskomponente mit mit mindestens
zwei Ecken gibt, besteht der Graph nur aus Vertices und hat
kein Kanten, also haben alle Vertices den Grad $0$, auch
in diesem Fall ist die Aussage also richtig.
\end{loesung}

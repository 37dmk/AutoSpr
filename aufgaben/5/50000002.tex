Beschreiben Sie eine Turing-Maschine, die Stack-Automaten simuliert.

\ifthenelse{\boolean{loesungen}}{
\begin{loesung}
Wir verwenden eine nicht deterministische Turing-Maschine mit zwei B"andern.
Das erste Band
enth"alt den Input, das zweite Band dient als Stack. Eine Turing-Maschine
ver"andert in jedem Schritt die Kopf-Position, ein Stack-Automat kann
jedoch in einem "Ubergang den Inhalt des Stack unver"andert lassen,
oder keinen Input lesen. Durch Kombination von zwei "Uberg"angen
kann dieses Verhalten jedoch auch auf einer Turing-Maschine simuliert
werden.

Um den Stackautomaten zu simulieren, m"ussen wir die Operationen des
Stackautomaten abbilden:
\begin{itemize}
\item Lesen eines Input-Zeichen: Kopf auf Band 1 um ein Feld nach
rechts bewegen. Damit das Ende des Input erkannt werden kann, muss ein
zus"atzliches Bandalphabetzeichen erfunden werden, welches das Ende
des Input signalisiert.
\item Inhalt auf dem Stack ersetzen: Inhalt des Feldes unter dem Kopf
auf Band 2 ersetzen.
\item Push: Kopf auf Band 2 ein Feld nach rechts bewegen und Feld "uberschreiben.
\item Pop: Kopf auf Band 2 ein Feld nach links bewegen.
\end{itemize}
Wenn der Kopf auf Band 1 auf das Zeichen f"ur das Ende des Input zeigt,
und der Stackautomatenzustand akzeptiert, h"alt akzeptiert die Maschine.
Andernfalls verwirft sie.
\end{loesung}
}{}


Die Sprache $L$ "uber dem Alphabet $\Sigma = \{{\tt 0},\;{\tt 1}\}$
besteht aus denjenigen W"ortern, in denen jede Folge von {\tt 0}
von einer mindestens so langen Folge von {\tt 1} gefolgt wird.
Beschreiben Sie eine Standard-Turing-Maschinen, welche $L$
entscheidet.

\begin{loesung}
Eine solche Turing-Maschine muss wiederholt Sequenzen der
Form ${\tt 0}^n{\tt 1}^m$ analysieren und kontrollieren, ob $m \ge n$ ist,
bis sie auf ein Zeichen \blank\ trifft, welches das Ende des
Wortes anzeigt. Da wir das Ende solcher Sequenzen markieren 
k"onnen m"ussen, verwenden wir ein zus"atzliches Bandzeichen
{\tt x}, das Bandalphabet ist also $\Gamma
=\{{\tt 0},\,{\tt 1},\,{\tt x}\}$.

Die Idee des folgenden Algorithmus ist, in jeder Sequenz jeweils
die {\tt 1} und {\tt 0} paarweise zu markieren, bis klar ist,
ob die Sequenz die Bedingung $m\ge n$ erf"ullt, und dann zur
n"achsten Sequenz weiter zu gehen.

Zun"achst "uberlegen wir uns einen Algorithmus, der W"orter der Form
${\tt 0}^n{\tt 1}^{n+m}$ akzeptiert:
\begin{enumerate}
\item Falls das Zeichen under dem Schreibe-Lesekopf ein \blank\ ist:
$q_\text{accept}$.
\item
\label{50000015:phase1:2}
Falls das Zeichen in {\tt 0} ist ersetze es durch einen \blank\ und
"uberspringe alle folgenden {\tt 0} und {\tt x}.
Andernfalls fahre weiter bei \ref{50000015:phase1:1}.
\item
Falls das n"achste Zeichen ein \blank\ ist: $q_\text{reject}$.
\item
Falls das n"achste Zeichen ein {\tt 1}, "uberschreibe es durch ein {\tt x}
und fahre zur"uck bis ein \blank\ gefunden wird, positioniere den Kopf
rechts davon, fahre weiter bei \ref{50000015:phase1:2}.
\item
\label{50000015:phase1:1}
"Uberspringe alle folgenden {\tt 1} oder {\tt x}.
\end{enumerate}
Dieser Algorithmus ersetzt jede {\tt 0} durch einen \blank\ und sucht eine
passende {\tt 1}, die durch {\tt x} ersetzt werden.
Die {\tt x} und {\tt 1} bleiben auf dem Band stehen, am Ende des Algorithmus
steht der Schreib-Lesekopf auf dem ersten Zeichen nach einer Sequenz,
als auf dem Beginn der n"achsten Sequenz oder auf einem \blank.
Der Bandinhalt hat sich also zum Beispiel so gewandelt:
\[
\text {\tt 0000111111100011111}
\rightarrow
\text {\tt \blank\blank\blank\blank xxxx11100011111}
\]
Jetzt kann man einfach den algorithmus erneut laufen lassen, der Bandinhalt
wird dann zu
\[
\text {\tt \blank\blank\blank\blank xxxx11100011111}
\rightarrow
\text {\tt \blank\blank\blank\blank xxxx111\blank\blank\blank xxx11}
\]
und der Kopf steht auf dem Blank nach dem letzten Wort.
Der n"achste Lauf des Algorithmus wird in Schritt 1 das Wort akzeptieren.

Der Zusammengesetzte Algorithmus sieht jetzt so aus:
\begin{enumerate}
\item Falls das Zeichen under dem Schreibe-Lesekopf ein \blank\ ist:
$q_\text{accept}$.
\item
\label{50000015:phase2:2}
Falls das Zeichen in {\tt 0} ist ersetze es durch einen \blank\ und
"uberspringe alle folgenden {\tt 0} und {\tt x}.
Andernfalls fahre weiter bei \ref{50000015:phase2:1}.
\item
Falls das n"achste Zeichen ein \blank\ ist: $q_\text{reject}$.
\item
Falls das n"achste Zeichen ein {\tt 1}, "uberschreibe es durch ein {\tt x}
und fahre zur"uck bis ein \blank\ gefunden wird, positioniere den Kopf
rechts davon, fahre weiter bei \ref{50000015:phase2:2}.
\item
\label{50000015:phase2:1}
"Uberspringe alle folgenden {\tt 1} oder {\tt x}.
\item Weiter bei 1.
\qedhere
\end{enumerate}
\end{loesung}

\begin{diskussion}
Man kann eine geeignete Turing-Maschine auch wie folgt bekommen.
Zun"achst stellt man fest, dass W"orter der genannten Art eine
kontextfreie Sprache bilden. Eine m"ogliche Grammatik daf"ur ist
\begin{align*}
S&\to \varepsilon \\
 &\to S W\\
W&\to \varepsilon \\
 &\to \text{\tt 0}W\text{\tt 1} \\
 &\to W\text{\tt 1}
\end{align*}
Es gibt also einen Stackautomaten f"ur diese Sprache. Ein besonders eleganter
Stackautomat w"are
\[
\entrymodifiers={++[o][F]}
\xymatrix @+10mm{
*+\txt{}\ar[r]
	&*++[o][F=]{q_0}\ar[r]^{{\tt 0},\varepsilon\to \$}
	      \ar@(ur,ul)_{{\tt 1},\varepsilon\to\varepsilon}
		&{q_1}\ar[r]^{\varepsilon,\varepsilon\to{\tt 0}}
			&{q_2}\ar@(r,u)_{{\tt 0},\varepsilon\to{\tt 0}}
			      \ar[dl]^{{\tt 1},{\tt 0}\to\varepsilon}
\\
*+\txt{}
	&*+\txt{}
		&{q_3}\ar@(r,d)^{{\tt 1},{\tt 0}\to\varepsilon}
		      \ar[ul]_{\varepsilon,\$\to\varepsilon}
}
\]
Sobald im Input eine {\tt 0} auftaucht, wechselt er in den Zustand $q_1$ und
beginnt die {\tt 0}en auf den Stack zu schreiben. Wenn dann aber {\tt 1}en 
auftreten, wird der Stack wieder abgebaut. Nur wenn der Stack v"ollig
gelehrt werden kann, und eventuell noch weitere {\tt 1}en folgen,
wird das Wort akzeptiert. Man erkennt auch, dass der Stackautomat
f"ur akzeptable W"orter sogar determinisitisch ist.

Zu jedem Stackautomaten gibt es eine Turingmaschine mit zwei B"andern,
die den Stackautomaten implementiert, das zweite Band wird f"ur die 
Implementation des Stacks verwendet. 
Jede Turing-Maschine mit zwei B"andern kann aber in eine Turing-Maschine
mit nur einem Band umgewandelt werden.
\end{diskussion}

\begin{bewertung}
Verwendung eines Hilfszeichens zur Unterteilung in Sequenzen
und zum Z"ahlen ({\bf H}) 1 Punkt,
Analyse jeder einzelnen Sequenz ({\bf S}) 1 Punkt,
Paarweises abz"ahlen der {\tt 0} und {\tt 1} ({\bf P}) 1 Punkt,
Abbruchbedingung f"ur $q_{\text{accept}}$ ({\bf A}) 1 Punkt,
Abbruchbedingung f"ur $q_{\text{reject}}$ ({\bf R}) 1 Punkt,
Weiterfahr-Bedingungung f"ur n"achste Sequenz ({\bf W}) 1 Punkt.
\end{bewertung}



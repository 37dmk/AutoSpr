Betrachten Sie die Turingmaschine $M$ mit Inputalphabet
$\Sigma=\{\texttt{0},\texttt{1}\}$ und folgendem Zustandsdiagramm:
\[
\entrymodifiers={++[o][F]}
\xymatrix @+5mm {
*+\txt{}
	&*+\txt{}
		&{q_2}\ar[r]^{\texttt{0}\to\texttt{0},R}
		      \ar[ld]_{\texttt{1}\to\texttt{0},R}
			&*++[o][F=]{q_\text{reject}}
\\
*+\txt{} \ar[r]
	&{q_0}\ar[r]^{\texttt{1}\to\texttt{1},R}
	      \ar[dr]_{{\scriptstyle\texttt{0}\to\texttt{0},R}\atop{\scriptstyle\blank\to\blank,R}}
		&{q_1}\ar[u]_{\texttt{0}\to\texttt{1},L}
		      \ar[d]^{{\scriptstyle\texttt{1}\to\texttt{1},R}\atop{\scriptstyle\blank\to\blank,R}}
\\
*+\txt{}
	&*+\txt{}
		&*++[o][F=]{q_\text{accept}}
			&*+\txt{}
}
\]
\begin{teilaufgaben}
\item Was steht nach der Verarbeitung des Wortes \texttt{10010} auf dem Band?
\item Was steht nach der Verarbeitung des Wortes \texttt{100011} auf dem Band?
\item K"onnen Sie ein Beispiel f"ur ein Wort geben, welches von dieser
Turingmaschine verworfen wird?
\item Die Turingmaschine $M$ ist nicht deterministisch, weil im Zustand
$q_2$ kein "Ubergang f"ur das Zeichen $\blank$ definiert ist.
Kann es vorkommen, dass im Zustand $q_2$ das Zeichen $\blank$ gelesen
wird?
\item Geben Sie eine einfache Beschreibung des Wortes, das nach
Verarbeitung eines Wortes $w$ mit $M$ auf dem Band steht.
\end{teilaufgaben}

\begin{loesung}
\begin{teilaufgaben}
\item Auf dem Wort \texttt{10010} wird folgende Berechnung durchgef"uhrt,
die Kopfposition wird durch die Farbe {\color{red}rot} angezeigt:
\begin{center}
\begin{tabular}{>{$}r<{$}|cccccccc}
q_0& \color{red} 1 & 0 & 0 & 1 & 0 & \blank & \blank & \blank \\
q_1& 1 & \color{red} 0 & 0 & 1 & 0 & \blank & \blank & \blank \\
q_2& \color{red} 1 & 1 & 0 & 1 & 0 & \blank & \blank & \blank \\
q_0& 0 & \color{red} 1 & 0 & 1 & 0 & \blank & \blank & \blank \\
q_1& 0 & 1 & \color{red} 0 & 1 & 0 & \blank & \blank & \blank \\
q_2& 0 & \color{red} 1 & 1 & 1 & 0 & \blank & \blank & \blank \\
q_0& 0 & 0 & \color{red} 1 & 1 & 0 & \blank & \blank & \blank \\
q_1& 0 & 0 & 1 & \color{red} 1 & 0 & \blank & \blank & \blank \\
q_\text{accept}& 0 & 0 & 1 & 1 & \color{red} 0 & \blank & \blank & \blank \\
\end{tabular}
\end{center}
Das Resultat der Rechnung ist also \texttt{00110}.
\item Die Berechnung auf dem Wort \texttt{100011} ist
\begin{center}
\begin{tabular}{>{$}r<{$}|cccccccc}
q_0& \color{red} 1 & 0 & 0 & 0 & 1 & 1 & \blank & \blank \\
q_1& 1 & \color{red} 0 & 0 & 0 & 1 & 1 & \blank & \blank \\
q_2& \color{red} 1 & 1 & 0 & 0 & 1 & 1 & \blank & \blank \\
q_0& 0 & \color{red} 1 & 0 & 0 & 1 & 1 & \blank & \blank \\
q_1& 0 & 1 & \color{red} 0 & 0 & 1 & 1 & \blank & \blank \\
q_2& 0 & \color{red} 1 & 1 & 0 & 1 & 1 & \blank & \blank \\
q_0& 0 & 0 & \color{red} 1 & 0 & 1 & 1 & \blank & \blank \\
q_1& 0 & 0 & 1 & \color{red} 0 & 1 & 1 & \blank & \blank \\
q_2& 0 & 0 & \color{red} 1 & 1 & 1 & 1 & \blank & \blank \\
q_0& 0 & 0 & 0 & \color{red} 1 & 1 & 1 & \blank & \blank \\
q_1& 0 & 0 & 0 & 1 & \color{red} 1 & 1 & \blank & \blank \\
q_\text{accept}& 0 & 0 & 0 & 1 & \color{red} 1 & 1 & \blank & \blank \\
\end{tabular}
\end{center}
Das Resultat der Rechnung ist also \texttt{000111}.
\item Es gibt kein solches Wort.
Der Zustand $q_2$ ist nur erreichbar von $q_1$ aus, wenn auf dem
Feld rechts vom aktuellen Feld eine \text{0} Stand. 
Dieses Feld wurde aber erreicht vom Zustand $q_0$ aus, weil auf
dem Feld links davon, also dem aktuellen Feld, eine \texttt{1} stand.
Wenn die Maschine im Zustand $q_2$ ist, befindet sich also links vom
aktuellen Feld immer eine \texttt{1}, der "Ubergang zu $q_\text{reject}$
wird gar nie verwendet.
\item In Teilaufgabe c) wurde bereits gezeigt, dass im Zustand $q_2$
immer ein Zeichen \texttt{1} gelesen wird, der Fall kann also nicht
eintreten.
\item
Die Maschine schiebt eine eventuell vorhandene f"uhrende \texttt{1}
"uber alle \texttt{0} hinweg, bis links von der zweiten \texttt{1} steht.
\end{teilaufgaben}
\end{loesung}

\begin{bewertung}
Resultat in Teilaufgabe a) ({\bf A}) 1 Punkt,
Resultat in Teilaufgabe b) ({\bf B}) 1 Punkt,
Analyse warum in Zustand $q_2$ nur eine \texttt{1} auf dem Feld links
stehen kann ({\bf Q}) 2 Punkte,
Verwendung der Analyse f"ur Teilaufgabe d) ({\bf N}) 1 Punkt,
Beschreibung der Funktion der Turingmaschine ({\bf F}) 1 Punkt.
\end{bewertung}


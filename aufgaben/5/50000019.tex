Gegeben ist eine Turingmaschine $M$ "uber dem Alphabet
$\Sigma=\{\texttt{0}, \texttt{1}\}$,
Bandalphabet
$\Gamma=\{\texttt{0}, \texttt{1}, \blank\}$
und mit drei Zust"anden
(inklusive $q_{\text{accept}}$ und $q_{\text{reject}}$).
Man weiss folgendes "uber $M$:
\begin{enumerate}
\item\label{50000019:liestalles} $M$ liest alle Zeichen des Inputwortes,
bevor sie anh"alt.
\item\label{50000019:haeltaufblank} Sie h"alt im Zustand $q_{\text{accept}}$
nur, wenn sie auf ein Blank $\blank$ auf dem Band gestossen ist.
\end{enumerate}
\begin{teilaufgaben}
\item
Wieviele verschiedene Turing-Maschinen mit diesen Eigenschaften gibt es?
\item
Welche Sprache akzeptieren sie?
\end{teilaufgaben}

\begin{loesung}
\begin{teilaufgaben}
\item
Das Zustandsdiagramm einer Turingmaschine mit drei Zust"anden, die
Bedingung~\ref{50000019:haeltaufblank} erf"ullt, sieht so aus:

\[
\entrymodifiers={++[o][F]}
\xymatrix @+5mm {
*+\txt{} \ar[r]
        &{q_0} \ar[r]^{\blank\to ?,?}
		&*++[o][F=]{q_{\text{accept}}}
\\
*+\txt{}
	&*++[o][F=]{q_{\text{reject}}}
}
\]
Nat"urlich fehlen hier noch einige Pfeile. Damit alle Zeichen gelesen werden
k"onnen, m"ussen im Zustand $q_0$ "Uberg"ange sowohl f"ur \texttt{0}
wie auch f"ur \texttt{1} m"oglich sein. Damit das ganze Wort gelesen werden
kann, muss f"ur beide Zeichen die Kopfbewegung nach rechts erfolgen,
das Zustandsdiagramm muss also wie folgt aussehen:
\[
\entrymodifiers={++[o][F]}
\xymatrix @+5mm {
*+\txt{} \ar[r]
        &{q_0} \ar[r]^{\blank\to {\color{red}?},?}
		\ar@(ur,u)_{\texttt{0}\to {\color{red}?},\text{R}}
		\ar@(ul,u)^{\texttt{1}\to {\color{red}?},\text{R}}
		&*++[o][F=]{q_{\text{accept}}}
\\
*+\txt{}
	&*++[o][F=]{q_{\text{reject}}}
}
\]
Nur noch die roten Fragezeichen {\color{red}?} m"ussen festgelegt
werden.
Da hierf"ur jeweils drei m"ogliche Werte aus $\Gamma$ in Frage kommen,
gibt es $3^3=27$ verschiedene Turing-Maschinen mit den verlangen Eigenschaften.

\item
Offenbar kommen Kopfbewegungen nach links gar nicht vor.
Die Zeichen, die geschrieben werden, werden also nicht wieder gelesen, und
haben daher keinen Einfluss die Sprache.
Die Machine liest also in jedem Fall einfach nur von links nach rechts,
bis sie auf ein $\blank$ trifft, und h"alt dann an.
Die akzeptierte Sprache ist also unabh"angig von der Wahl der roten
{\color{red}?} Fragezeichen immer $L=\Sigma^*$.
\end{teilaufgaben}
\end{loesung}


Gegeben ist die Turingmaschine mit dem Zustandsdiagramm
\[
%\xymatrixrowsep{1in}
%\xymatrixcolsep{1in}
\entrymodifiers={++[o][F]}
\xymatrix{
*+\txt{}
	&*+\txt{}
		&*+\txt{}
			&{q_1} \ar[ddrrr]^{\blank\to\blank,R}
			       \ar[dd]^{\texttt{0}\to\texttt{x},L}
			       \ar@(ur,ul)_{\substack{\texttt{1}\to\texttt{1},R
						\\
						\texttt{x}\to\texttt{x},R}}
\\
*+\txt{}
\\
*+\txt{} \ar[r]
	&{q_0} \ar[dl]_{\blank\to\blank,R}
	       \ar[uurr]^{\texttt{1}\to\texttt{x},R}
	       \ar[ddrr]_{\texttt{0}\to\texttt{x},R}
	       \ar@(u,ul)_{\texttt{x}\to\texttt{x},R}
		&*+\txt{}
			&{q_3} \ar[ll]_{\blank\to\blank,R}
			       \ar@(ur,dr)^{\substack{
{\texttt{0}\to\texttt{0},L}\\
{\texttt{1}\to\texttt{1},L}\\
{\texttt{x}\to\texttt{x},L}}}
				&*+\txt{}
					&*+\txt{}
						&*++[o][F=]{q_{\text{reject}}}
\\
*++[o][F=]{q_{\text{accept}}}
\\
*+\txt{}
	&*+\txt{}
		&*+\txt{}
			&{q_2} \ar[uurrr]_{\blank\to\blank,R}
			       \ar[uu]_{\texttt{1}\to\texttt{x},L}
			       \ar@(dr,dl)^{\substack{\texttt{0}\to\texttt{0},R
					\\
					\texttt{x}\to\texttt{x},R}}
}
\]
"uber dem Alphabet
$\Sigma = \{\texttt{0},\texttt{1}\}$
und dem Bandalphabet
$\Gamma = \{\texttt{0},\texttt{1},\texttt{x},\blank\}$.

\begin{teilaufgaben}
\item Wird das Wort \texttt{101} akzeptiert?
\item Wird das Wort \texttt{0110} akzeptiert?
\item Falls das Inputwort auf dem Band ein $\blank$ enth"alt, kann der Teil
des Inputwortes 
rechts vom $\blank$ einen Einfluss darauf haben, ob das Wort akzeptiert
wird?
\item
Es wird behauptet, dass die Maschine alle W"orter
$w$ mit $|w|_{\texttt{0}}=|w|_{\texttt{1}}$
akzeptiert.
Ist dies korrekt?
\end{teilaufgaben}

\begin{loesung}
\begin{teilaufgaben}
\item
Die Verarbeitung des Wortes \texttt{101} ist (Zeichen unter dem
Schreib-/Lesekopf jeweils rot)
\begin{center}
\def\r{\color{red}}
\begin{tabular}{>{$}r<{$}|>{\tt}c>{\tt}c>{\tt}c>{\tt}c>{\tt}c>{\tt}c}
q_0              &    \blank & \r 1 &    0 &    1 &    \blank &    \blank \\
q_1              &    \blank &    x & \r 0 &    1 &    \blank &    \blank \\
q_3              &    \blank & \r x &    x &    1 &    \blank &    \blank \\
q_3              & \r \blank &    x &    x &    1 &    \blank &    \blank \\
q_0              &    \blank & \r x &    x &    1 &    \blank &    \blank \\
q_0              &    \blank &    x & \r x &    1 &    \blank &    \blank \\
q_0              &    \blank &    x &    x & \r 1 &    \blank &    \blank \\
q_1              &    \blank &    x &    x &    x & \r \blank &    \blank \\
q_{\text{reject}}&     \blank&    x &    x &    x &    \blank & \r \blank
\end{tabular}
\end{center}
Das Wort \texttt{101} wird nicht akzeptiert.
\item
Die Verarbeitung des Wortes \texttt{0110} ist (Zeichen unter dem
Schreib-/Lesekopf jeweils rot)
\begin{center}
\def\r{\color{red}}
\begin{tabular}{>{$}r<{$}|>{\tt}c>{\tt}c>{\tt}c>{\tt}c>{\tt}c>{\tt}c>{\tt}c}
q_0 &  \blank&\r 0&   1&   1&   0&  \blank&  \blank\\
q_2 &  \blank&   x&\r 1&   1&   0&  \blank&  \blank\\
q_3 &  \blank&\r x&   x&   1&   0&  \blank&  \blank\\
q_3 &\r\blank&   x&   x&   1&   0&  \blank&  \blank\\
q_0 &  \blank&\r x&   x&   1&   0&  \blank&  \blank\\
q_0 &  \blank&   x&\r x&   1&   0&  \blank&  \blank\\
q_0 &  \blank&   x&   x&\r 1&   0&  \blank&  \blank\\
q_1 &  \blank&   x&   x&   x&\r 0&  \blank&  \blank\\
q_3 &  \blank&   x&   x&\r x&   x&  \blank&  \blank\\
q_3 &  \blank&   x&\r x&   x&   x&  \blank&  \blank\\
q_3 &  \blank&\r x&   x&   x&   x&  \blank&  \blank\\
q_3 &\r\blank&   x&   x&   x&   x&  \blank&  \blank\\
q_0 &  \blank&\r x&   x&   x&   x&  \blank&  \blank\\
q_0 &  \blank&   x&\r x&   x&   x&  \blank&  \blank\\
q_0 &  \blank&   x&   x&\r x&   x&  \blank&  \blank\\
q_0 &  \blank&   x&   x&   x&\r x&  \blank&  \blank\\
q_0 &  \blank&   x&   x&   x&   x&\r\blank&  \blank\\
q_{\text{accept}} &  \blank&   x&   x&   x&   x&  \blank&\r\blank\\
\end{tabular}
\end{center}
Das Wort \texttt{0110} wird also akzeptiert.
\item
Zun"achst ist festzuhalten, dass $\Sigma$ das Zeichen $\blank$ nicht enth"alt,
dass diese hypothetische Situation also normalerweise nicht auftritt.
Leerzeichen zeigen also immer an, dass man auf einen Teil des Bandes
gestossen ist, der nicht vom Input-Wort initialisiert worden ist.
Es geht also um die hypothetische Situation, dass man im Input $\blank$-Zeichen
zul"asst, und man m"ochte wissen, ob wie die Maschine damit umgeht.

Das erste $\blank$-Zeichen rechts vom Wortanfang kann nur durch die "Uberg"ange 
von $q_1$ nach $q_{\text{reject}}$
und
von $q_2$ nach $q_{\text{reject}}$
"uberschritten werden.
Da die Maschine dann aber anh"alt, wird der nachfolgende Teil des Bandes nie
gelesen, und kann daher auch den Ausgang nicht beeinflussen.
\item
Die Turingmaschine "uberschreibt \texttt{0} und \texttt{1} paarweise mit
\texttt{x} und akzeptiert, wenn am Schluss nur \texttt{x} dastehen.
Die akzeptierte Sprache ist daher
\[
L=\{ w\in\Sigma^* \;|\; |w|_{\texttt{0}}=|w|_{\texttt{1}}\}.
\]
\end{teilaufgaben}
\end{loesung}

\begin{bewertung}
Teilaufgabe a) ({\bf A}) 1 Punkt,
Teilaufgabe b) ({\bf B}) 1 Punkt,
Teilaufgabe c): korrekte Antwort ($\textbf{C}_a$) 1 Punkt,
Begr"undung ($\textbf{C}_b$) 1 Punkt,
Teilaufgabe d): korrekte Antwort ($\textbf{D}_a$) 1 Punkt,
Begr"undung ($\textbf{D}_b$) 1 Punkt.
\end{bewertung}


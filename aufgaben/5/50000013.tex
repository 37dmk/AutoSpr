Welche der folgenden Mengen sind abz"ahlbar unendlich, welche sind
"uberabz"ahlbar.
\begin{teilaufgaben}
\item Die Menge aller kontextfreien Sprachen.
\item Die Menge aller Folgen $a_1,a_2,\dots$ von rationalen Zahlen,
die nach endlich Folgengliedern konstant sind.
\item Die Menge aller konvergenten Folgen $a_1,a_2,a_3,\dots$ von
rationalen Zahlen.
\item Die Menge aller Entscheider.
\item Die Menge aller Aufz"ahler.
\item Die Menge aller Polynome mit ganzzahligen Koeffizienten:
${\mathbb Z}[X]$
\end{teilaufgaben}

\begin{loesung}
\begin{teilaufgaben}
\item Die kontextfreien Grammatiken kann man zum Beispiel nach
Anzahl Regeln ihrer Chomsky-Normalform aufz"ahlen, also ist
die Menge der kontextfreien Grammatiken abz"ahlbar.
\item Die Menge der Folgen, die nach endlich vielen Gliedern konstant
werden, kann man zerlegen in die Mengen $F_k$, der Folgen, die ab dem
$k$-ten Glied konstant sind. Die Menge $F_k$ besteht aus folgen, die
$k$ verschiedene rationale Glieder haben, also gleich m"achtig wie
$\mathbb Q^{k+1}$. Als abz"ahlbare Vereinigung von abz"ahlbaren Mengen
ist die Menge der Folgen, die nach endlich vielen Gliedern konstant sind,
also abz"ahlbar.
\item Jede reelle Zahl l"asst sich durch eine Folge rationaler Zahlen
approximieren. Daher ist die Menge der konvergenten Folgen mindestens
so gross wie die Menge der reellen Zahlen, also "uberabz"ahlbar.
\item Entscheider sind Turing-Maschinen, davon gibt es nur abz"ahlbar viele.
\item Aufz"ahler sind Turing-Maschinen, davon gibt es nur abz"ahlbar viele.
\item Die Menge ${\mathbb Z}[X]$ l"asst sich schreiben als Vereinigung
der Mengen $P_k$, wobei $P_k$ die Polynome vom Grad $k$ enth"alt.
Ein Polynom vom Grad $k$ mit ganzzahligen Koeffizienten wird beschrieben
durch die Koeffizienten, $P_k$ ist also gleich m"achtig wie $\mathbb Z^{k+1}$.
Damit ist aber ${\mathbb Z}[X]$ als Vereinigung von abz"ahlbar vielen
abz"ahlbaren Mengen wieder abz"ahlbar.
\end{teilaufgaben}
\end{loesung}

Beschreiben Sie eine Turing-Maschine, welche Palindrome erkennt.
Zeichnen Sie das Zustands-Diagramm, wenn
$\Sigma=\{{\tt 0},{\tt 1}\}$.

\begin{loesung}
Die Turingmaschine muss herausfinden, ob die Zeichen am Anfang und
am Ende des Wortes jeweils gleich sind. Sie muss also folgenden
Algorithmus implementieren:
\begin{compactenum}
\item Merke das Zeichen unter dem Schreib-/Lesekopf, l"osche es und
bewege den Kopf nach rechts.
\item Falls das Zeichen unter dem Schreib-/Lesekopf ein $\blank$ ist:
$q_{\text{accept}}$
\item Fahre nach rechts bis zum ersten $\blank$, bewege den Kopf dann
nach links.
\item Falls das Zeichen unter dem Schreib-/Lesekopf verschieden ist
vom gespeicherten Zeichen: $q_{\text{reject}}$.
\item L"osche das Zeichen unter dem Schreib-/Lesekopf und bewege
den Kopf nach links.
\item Fahre nach links bis zum ersten $\blank$, bewege den Kopf nach
rechts.
\item Weiter bei 1.
\end{compactenum}
Die Anweisung ``merke das Zeichen unter dem Schreib-/Lesekopf'' muss
durch separate Zust"ande f"ur jedes m"ogliche Zeichen des Alphabetes
implementiert werden. Das folgende Zustandsdiagramm implementiert dies
f"ur das Alphabet $\Sigma=\{{\tt 0},{\tt 1}\}$:
\[
\entrymodifiers={++[o][F]}
\xymatrix{
*+\txt{}
	&*+\txt{}
		&{} \ar[dr]^{\blank\to\blank,\text{R}}
		    \ar[r]^{{\tt x}\to{\tt x},\text{R}}
			&{} \ar@(ul,ur)^{{\tt x}\to{\tt x},\text{R}}
		    	    \ar[r]^{\blank\to\blank,\text{L}}
				&{} \ar[d]^{{\tt 1}\to{\tt 1},\text{R}}
				    \ar[r]^{{\tt 0}\to{\tt 0},\text{L}}
					&{} \ar@(r,d)^{{\tt x}\to{\tt x},\text{L}}
					    \ar `u^l[llll]+/u1.3cm/ `^d[lllld]_{\blank\to\blank,\text{R}} [lllld]
\\
*+\txt{}\ar[r]
	&{} \ar[ur]_{{\tt 0}\to\blank,\text{R}}
	    \ar[dr]^{{\tt 1}\to\blank,\text{R}}
	    \ar[rr]^{\blank\to\blank,\text{R}}
		&*+\txt{}
			&*++[o][F=]{q_{\text{accept}}}
				&*++[o][F=]{q_{\text{reject}}}
\\
*+\txt{}
	&*+\txt{}
		&{} \ar[ur]_{\blank\to\blank,\text{R}}
		    \ar[r]_{{\tt x}\to{\tt x},\text{R}}
			&{} \ar@(dl,dr)_{{\tt x}\to{\tt x},\text{R}}
		    	    \ar[r]_{\blank\to\blank,\text{L}}
				&{} \ar[u]_{{\tt 0}\to{\tt 0},\text{R}}
				    \ar[r]_{{\tt 1}\to{\tt 1},\text{L}}
					&{} \ar@(r,u)_{{\tt x}\to{\tt x},\text{L}}
					    \ar `d[llll]+/d1.3cm/ `[llllu]^{\blank\to\blank,\text{R}} [llllu]
}
\]
Darin steht abk"urzend ${\tt x}\to{\tt x},\text{R}$ f"ur ``ein beliebiges
Alphabetzeichen unter dem Schreib-/Lesekopf wird belassen und der Kopf
nach rechts bewegt''.
\end{loesung}

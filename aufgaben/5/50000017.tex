Turing-Maschinen mit $\Sigma=\{\texttt{0}, \texttt{1}\}$ und genau
drei Zust"anden haben notwendigerweise ein Zustandsdiagramm der Form
\[
\entrymodifiers={++[o][F]}
\xymatrix @-1mm {
*+\txt{} \ar[r]
	&{q_0} \ar@(ur,ul) \ar[r] \ar[d]
		&*++[o][F=]{q_{\text{accept}}}
\\
*+\txt{}
	&*++[o][F=]{q_{\text{reject}}}
}
\]
Hier fehlen nur noch die "Uberg"ange, die von der Art
\[
\entrymodifiers={++[o][F]}
\xymatrix @+5mm {
{q_0}\ar[r]^{a\to b,d}&{q}
}
\]
sein m"ussen, wobei $q$ irgend ein Zustand ist, $a$ und $b$ Bandalphabethzeichen
und $d$ die Bewegungsrichtung des Schreibe-/Lesekopfes aus $\{\text{L},\text{R}\}$.
F"ur "Uberg"ange in einen Akzeptierzustand spielen die Werte von $b$ und $d$
keine Rolle, "Uberg"ange in einen Akzeptierzustand, die sich nur durch die
Werte von $b$ und $d$ unterscheiden, sind im Wesentlichen gleich.
\begin{teilaufgaben}
\item
Welche "Uberg"angen m"ussen Sie w"ahlen, damit die Turing-Maschine
die Sprache $\Sigma^*$ erkennt?
\item
Welche "Uberg"ange m"ussen Sie w"ahlen, damit die Turing-Maschine
die Sprache $\emptyset$ erkennt?
\item
Kann man eine Turing-Maschine mit drei Zust"anden bauen, welche die
Sprache
\[
L=\{ \texttt{0}^n\,|\,n\ge 0\}
\]
erkennt?
\item
Wieviele wesentlich verschiedene Zustandsdiagramme von solchen Turing-Maschinen 
gibt es?
\end{teilaufgaben}

\begin{loesung}
F"ur "Uberg"ange, die in einem Akzeptierzustand enden, sind die Werte
von $b$ und $d$ nicht relevant. Nur die "Uberg"ange, die in $q_0$ enden
f"uhren zu tats"achlich verschiedenen Turing-Maschinen. F"ur die drei
Bandalphabet-Zeichen $\Gamma =\{\texttt{0},\texttt{1},\blank\}$ muss 
also nur entscheiden werden, zu welchem Zustand sie f"uhren, und welche
Werte f"ur $b$ und $d$ gegebenenfalls verwendet werden sollen.
Zu jedem Zeichen f"ur $a$ gibt es also zwei "Uberg"ange, die in
einem Akzeptierzustand enden, eine Anzahl von "Uberg"angen, die in $q_0$
enden. Solche "Uberg"ange gibt es mit drei verschiedenen Zeichen auf
dem Band nach dem "Ubergang, und zwei Bewegungsrichtungen, also insgesamt
6 m"ogliche "Uberg"ange:
\[
\entrymodifiers={++[o][F]}
\xymatrix @+5mm{
{q_0} \ar[r]^{?\to ?,?}	&*++[o][F=]{q_{\text{accept}}}&
{q_0} \ar[r]^{?\to ?,?}	&*++[o][F=]{q_{\text{reject}}}\\
{q_0} \ar[r]^{?\to \texttt{0},\text{L}}	&{q_0}&
{q_0} \ar[r]^{?\to \texttt{0},\text{R}}	&{q_0}\\
{q_0} \ar[r]^{?\to \texttt{1},\text{L}}	&{q_0}&
{q_0} \ar[r]^{?\to \texttt{1},\text{R}}	&{q_0}\\
{q_0} \ar[r]^{?\to \blank,\text{L}}	&{q_0}&
{q_0} \ar[r]^{?\to \blank,\text{R}}	&{q_0}\\
}
\]
\begin{teilaufgaben}
\item
Die Turing-Maschine
\[
\entrymodifiers={++[o][F]}
\xymatrix @-1mm {
*+\txt{} \ar[r]
	&{q_0} \ar[r]
		&*++[o][F=]{q_{\text{accept}}}
\\
*+\txt{}
	&*++[o][F=]{q_{\text{reject}}}
}
\]
mit einem "Ubergang nach $q_{\text{accept}}$ f"ur jedes Zeichen in $\Gamma$
akzeptiert alle W"orter, also die Sprache $\Sigma^*$.
Alternativ k"onnte man auch alle Zeichen \texttt{0} und \texttt{1}
"uberlesen und akzeptieren, sobald man einen $\blank$ findet:
\[
\entrymodifiers={++[o][F]}
\xymatrix @-1mm {
*+\txt{} \ar[r]
	&{q_0}
		\ar@(u,ul)_{\texttt{0}\to\texttt{0},\text{R}}
		\ar@(u,ur)^{\texttt{1}\to\texttt{1},\text{R}}
		\ar[r]^{\blank\to\blank,\text{R}}
		&*++[o][F=]{q_{\text{accept}}}
\\
*+\txt{}
	&*++[o][F=]{q_{\text{reject}}}
}
\]
\item
Die Turing-Maschine
\[
\entrymodifiers={++[o][F]}
\xymatrix @-1mm {
*+\txt{} \ar[r]
	&{q_0} \ar[d]
		&*++[o][F=]{q_{\text{accept}}}
\\
*+\txt{}
	&*++[o][F=]{q_{\text{reject}}}
}
\]
mit einem "Ubergang nach $q_{\text{reject}}$ f"ur jedes Zeichen in $\Gamma$
akzeptiert nur die leere Sprache. Wie in Teilaufgabe a) kann man nat"urlich
die Maschine auch noch etwas l"anger arbeiten lassen, bevor man in den
Zustand $q_\text{reject}$ geht:
\[
\entrymodifiers={++[o][F]}
\xymatrix @-1mm {
*+\txt{} \ar[r]
	&{q_0}
		\ar@(u,ul)_{\texttt{0}\to\texttt{0},\text{R}}
		\ar@(u,ur)^{\texttt{1}\to\texttt{1},\text{R}}
		\ar[d]^{\blank\to\blank,\text{R}}
		&*++[o][F=]{q_{\text{accept}}}
\\
*+\txt{}
	&*++[o][F=]{q_{\text{reject}}}
}
\]
Alternativ k"onnte man auch beim ersten \texttt{1} in oder beim ersten \texttt{0} in
den Zustand $q_\text{reject}$ wechseln.
\item
Die Turing-Maschine
\[
\entrymodifiers={++[o][F]}
\xymatrix @+5mm {
*+\txt{} \ar[r]
	&{q_0} \ar@(ur,ul)_{\texttt{0}\to\texttt{0},\text{R}}
		\ar[d]^{\texttt{1}\to\texttt{1},\text{R}}
		\ar[r]^{\blank\to\blank,\text{R}}
		&*++[o][F=]{q_{\text{accept}}}
\\
*+\txt{}
	&*++[o][F=]{q_{\text{reject}}}
}
\]
"uberspringt alle \texttt{0}, verwirft, sobald sie ein \texttt{1} trifft,
und erkennt, wenn sie das Ende des Wortes erreicht.
Sie erkennt also genau die W"orter, die aus lauter \texttt{0} zusammengesetzt
sind, dies entspricht der Spezifikation von $L$.
\item
Damit kann man jetzt die m"oglichen Turing-Maschinen aufz"ahlen.
Da f"ur jedes Zeichen in $\Gamma$ einer von $8$ "Uberg"angen
gew"ahlt werden muss, gibt es $8^3=512$ wesentlich verschiedene
Zustandsdiagramme.
\qedhere
\end{teilaufgaben}
\end{loesung}

\begin{bewertung}
\begin{teilaufgaben}
\item Funktionierende TM f"ur $\Sigma^*$ (\textbf{A}) 1 Punkt.
\item Funktionierende TM f"ur $\emptyset$ (\textbf{E}) 1 Punkt.
\item Funktionierende TM f"ur $L$ (\texttt{L}) 2 Punkte.
\item Anzahl m"oglicher Turing-Maschinen (\textbf{N}) 1 Punkt,
Abz"ahlargument (\textbf{Z}) 1 Punkt.
\end{teilaufgaben}
\end{bewertung}


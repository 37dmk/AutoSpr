Sei $\Sigma=\{{\tt 0},{\tt 1}\}$. In der Vorlesung wurde gezeigt, dass 
die Menge aller W"orter $\Sigma^*$ abz"ahlbar ist, die Menge $P(\Sigma^*)$
aller Sprachen aber nicht. Zeigen Sie, dass die Menge aller endlichen
Sprachen abz"ahlbar ist.

\begin{loesung}
Man kann eine Aufz"ahlung der endlichen Teilmengen von $\Sigma^*$ wie
folgt konstruieren.
\begin{compactenum}
\item Die Menge der Sprachen mit genau einem Wort ist abz"ahlbar, da die
Menge $\Sigma^*$ der W"orter abz"ahlbar ist.
\item Die  Menge der Sprachen mit genau zwei W"ortern ist abz"ahlbar,
sie ist eine Menge von Paaren von W"ortern, aber $\Sigma^*\times \Sigma^*$
ist abz"ahlbar.
\item Die Menge der Sprachen mit genau drei W"ortern ist abz"ahlbar,
sie besteht aus Paaren bestehend aus einer Sprache aus dem ersten
Schritt und einer Sprache aus dem zweiten Schritt, es gibt also 
abz"ahlbar viele davon.
\item Auf die gleiche Art sieht man, dass die Menge der Sprachen mit
genau $n$ W"ortern abz"ahlbar ist.
\end{compactenum}
Die Sprachen mit $n$ W"ortern bilden also jeweils eine abz"ahlbare Menge,
sie lassen sich also mit zwei Zahlen eindeutig identifizieren: der Anzahl
W"orter $n$ und der Nummer $m$ innerhalb Sprachen mit $n$ W"ortern.
Die Menge aller Paare ist aber wieder abz"ahlbar.
\end{loesung}


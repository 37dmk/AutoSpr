Beschreiben Sie eine Turing-Maschine "uber dem Alphabet
$\Sigma=\{\text{\tt 0},\text{\tt 1}\}$, die als Input ein Wort
aus lauter {\tt 0} auf dem Band erwartet, also $w=\text{\tt 0}^n$,
und auf dem Band ein Wort aus $2^n$ Einsen, also $\text{\tt 1}^{2^n}$ hinterl"asst. Sie
produziert also zum Beispiel folgendes:
\begin{align*}
\text{\tt 0}&\rightarrow\text{\tt 11}\\
\text{\tt 00}&\rightarrow\text{\tt 1111}\\
\text{\tt 000}&\rightarrow\text{\tt 11111111}\\
\text{\tt 0000}&\rightarrow\text{\tt 1111111111111111}\\
&\dots
\end{align*}
Sie d"urfen annehmen, dass der Schreib-/Lesekopf zu Beginn auf dem
ersten Zeichen des Input-Wortes steht. Sie d"urfen auf eine
Validierung des Input verzichten. Funktioniert Ihre Maschine auch,
wenn auf dem Band der leere String steht?

\ifthenelse{\boolean{loesungen}}{
\begin{loesung}
Wir verwenden als Bandalphabet zus"atzlich zum Blank $\sqcup$ noch
das Zeichen {\tt x}. Der Algorithmus geht wie folgt vor.
\begin{enumerate}
\item Zun"achst
wird eine {\tt 1} unmittelbar hinter dem Input-Wort geschrieben.
Das Ende des Input-Wortes wird gefunden, indem man nach rechts f"ahrt,
bis ein leeres Feld unter dem Schreib-Lese-Kopf steht.
\item \label{mainloop}F"ur jede {\tt 0}, die man zu Beginn des beschriebenen
Teils des Bandes findet, verdoppelt man die Anzahl {\tt 1}, die
bereits auf dem Band sind. Man f"ahrt also zun"achst nach links,
bis man ein leeres Feld findet, dann wieder ein Feld nach rechts.
Wenn man dort ein {\tt 0} findet, gibt es noch etwas zu tun. Man
ersetzt die {\tt 0} durch $\sqcup$ und f"ahrt nach rechts bis
zur ersten {\tt 1}. Das Verdoppeln der Einsen geschieht jetzt in
folgenden Schritten:
\begin{enumerate}
\item Fahre nach rechts bis zum ersten $\sqcup$
\item \label{doubleloop}Nach links fahrend "uberspringe alle {\tt x} bis zu einer {\tt 1},
falls statt einer {\tt 1} eine {\tt 0} gefunden wurde, sind bereits alle
{\tt 1} verdoppelt worden, weiter bei \ref{xto1}.
\item Ersetze die {\tt 1} durch ein
{\tt x}, fahre dann nach rechts bis zum ersten $\sqcup$ und ersetze
ihn ebenfalls durch ein {\tt x}. Weiter bei \ref{doubleloop}.
\item \label{xto1}Nach rechts fahrend, ersetze alle {\tt x} durch {\tt 1}.
Weiter bei \ref{mainloop}
\end{enumerate}
\end{enumerate}
Falls auf dem Band der leere String steht, wird im ersten Schritt eine
{\tt 1} geschrieben. Dann sucht das Programm nach der n"achsten {\tt 0},
wird aber keine solche finden, und damit auch keine Verdoppelung der
Einsen vornehmen. Als Resultat steht also ein Folgen von $1=2^0$ Einsen
auf dem Band, die Maschine funktioniert auch in diesem Fall.
\end{loesung}

}{ }


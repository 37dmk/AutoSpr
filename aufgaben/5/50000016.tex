Beschreiben Sie ein Turing-Maschinen-Programm, welches herausfindet,
ob eine Bin"arzahl durch drei teilbar ist.

\begin{loesung}
F"ur das Problem der Teilbarkeit durch drei wurde in der Vorlesung
ein endlicher Automat gezeigt:
\[
\entrymodifiers={++[o][F]}
\xymatrix @-1mm {
*+\txt{}\ar[r]
	&*++[o][F=]{0}	\ar@(ur,ul)_{\tt 0}
			\ar@/^/[r]^{\tt 1}
		&{1}	\ar@/^/[r]^{\tt 0}
			\ar@/^/[l]^{\tt 1}
			&{2}	\ar@/^/[l]^{\tt 0}
				\ar@(ur,ul)_{\tt 1}
}
\]
Die gesuchte Turing-Maschine muss also genau diesen endlichen Automaten
implementieren, was mit dem folgenden Zustandsdiagramm m"oglich ist:
\[
\entrymodifiers={++[o][F]}
\xymatrix @-1mm {
*+\txt{}\ar[r]
	&{0}	\ar@(ur,ul)_{{\tt 0}\to{\tt 0},\text{R}}
		\ar@/^/[r]^{{\tt 1}\to{\tt 1},\text{R}}
		\ar[d]_{\blank\to\blank,\text{R}}
		&{1}	\ar@/^/[r]^{{\tt 0}\to{\tt 0},\text{R}}
			\ar@/^/[l]^{{\tt 1}\to{\tt 1},\text{R}}
			\ar[dr]_{\blank\to\blank,\text{R}}
			&{2}	\ar@/^/[l]^{{\tt 0}\to{\tt 0},\text{R}}
				\ar@(ur,ul)_{{\tt 1}\to{\tt 1},\text{R}}
				\ar[d]^{\blank\to\blank,\text{R}}
\\
*+\txt{}
	&*++[o][F=]{q_{\text{accept}}}
		&*+\txt{}
			&*++[o][F=]{q_{\text{reject}}}
}
\]

Eine andere L"osung k"onnte man dadurch erhalten, dass man die TM
auf dem Band von der Zahl 3 (bin"ar 11) subtrahieren l"asst. Dies
kann nach dem Muster der bin"aren Addition (``schriftliche'' Subtraktion,
Video aus der Vorlesung)
durch dreimaliges bin"ares R"uckw"artsz"ahlen (Bin"arz"ahler aus den
"Ubungen) erfolgen.
Wenn
das aufgeht, also auf dem Band nur Nullen stehen, dann ist die
Zahl durch drei teilbar.

Noch ein weiterer L"osungsweg besteht darin, die Zahl zuerst auf einem zweiten
Band in un"are Darstellung umzuwandeln. Dazu braucht man ein drittes Band,
auf dem man die Zweierpotenzen in un"arer Darstellung berechnet, indem
man den Inhalt des Bandes nochmals an dessen Ende kopiert. Die un"are
Darstellung des Inputs bekommt man, indem man f"ur jede Stelle 1
des Inputs die zugeh"orige un"are Darstellung des zugeh"origen Stellenwertes
ans Ende von Band~2 kopiert. Teilbarkeit durch drei kann man in der
un"aren Darstellung dadurch pr"ufen, dass man die Zeichen von Band~2
in Dreiergruppen l"oscht. Wenn das aufgeht, ist die Zahl durch drei teilbar
und kann akzeptiert werden.
\end{loesung}

\begin{bewertung}
Endlicher Automat f"ur 3er-Rest ({\bf D}) 2 Punkte,
"Ubersetzung des Automaten in eine TM-Maschine: 4 Punkte.
Falls in einem TM-Programm eine nicht weiter spezifizierte Operation
``Division durch 3'' oder ``Modulo 3'' verwendet wurde, wurden
nur 3 Punkte vergeben.
\end{bewertung}

Beschreiben Sie ein Turing-Maschinen-Programm, welches herausfindet,
ob eine Bin"arzahl durch drei teilbar ist.

\begin{loesung}
F"ur das Problem der Teilbarkeit durch drei wurde in der Vorlesung
ein endlicher Automat gezeigt:
\[
\entrymodifiers={++[o][F]}
\xymatrix @-1mm {
*+\txt{}\ar[r]
	&*++[o][F=]{0}	\ar@(ur,ul)_{\tt 0}
			\ar@/^/[r]^{\tt 1}
		&{1}	\ar@/^/[r]^{\tt 0}
			\ar@/^/[l]^{\tt 1}
			&{2}	\ar@/^/[l]^{\tt 0}
				\ar@(ur,ul)_{\tt 1}
}
\]
Die gesuchte Turing-Maschine muss also genau diesen endlichen Automaten
implementieren, was mit dem folgenden Zustandsdiagramm m"oglich ist:
\[
\entrymodifiers={++[o][F]}
\xymatrix @-1mm {
*+\txt{}\ar[r]
	&{0}	\ar@(ur,ul)_{{\tt 0}\to{\tt 0},\text{R}}
		\ar@/^/[r]^{{\tt 1}\to{\tt 1},\text{R}}
		\ar[d]_{\blank\to\blank,\text{R}}
		&{1}	\ar@/^/[r]^{{\tt 0}\to{\tt 0},\text{R}}
			\ar@/^/[l]^{{\tt 1}\to{\tt 1},\text{R}}
			\ar[dr]_{\blank\to\blank,\text{R}}
			&{2}	\ar@/^/[l]^{{\tt 0}\to{\tt 0},\text{R}}
				\ar@(ur,ul)_{{\tt 1}\to{\tt 1},\text{R}}
				\ar[d]^{\blank\to\blank,\text{R}}
\\
*+\txt{}
	&*++[o][F=]{q_{\text{accept}}}
		&*+\txt{}
			&*++[o][F=]{q_{\text{reject}}}
}
\]

Eine andere L"osung k"onnte man dadurch erhalten, dass man die TM
auf dem Band von der Zahl 3 (bin"ar 11) subtrahieren l"asst. Wenn
das aufgeht, also auf dem Band nur Nullen stehen, dann ist die
Zahl durch drei teilbar.
\end{loesung}

\begin{bewertung}
Endlicher Automat f"ur 3er-Rest ({\bf D}) 2 Punkte,
"Ubersetzung des Automaten in eine TM-Maschine: 4 Punkte.
Falls in einem TM-Programm eine nicht weiter spezifizierte Operation
``Division durch 3'' oder ``Modulo 3'' verwendet wurde, wurden
nur 3 Punkte vergeben.
\end{bewertung}

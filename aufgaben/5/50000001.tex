Beschreiben Sie eine Turing-Maschine, welche eine bin"ar auf dem
Band angegebene Zahl um $1$ erh"oht.

\ifthenelse{\boolean{loesungen}}{
\begin{loesung}
Nachdem das Wort auf das Band geschrieben und der Kopf auf den
Anfang des Wortes positioniert worden ist, beginnt die Turingmaschine
zu arbeiten. 

Falls am Ende des Programms der Schreib-Lesekopf wieder auf dem
Anfang des Wortes stehen soll (was zum Beispiel dann sinnvoll ist,
wenn man den Schreib-/Lesekopf auch dazu verwenden will das
Resultat wieder auszulesen),  m"ussen folgende
Schritte durchgef"uhrt werden:
\begin{enumerate}
\item Solange unter dem Schreib-/Lesekopf ein Zeichen aus $\Sigma$
zu lesen ist, bewege den Kopf nach rechts.
\item Bewege den Kopf um ein Zeichen nach links.
\item Lese das Zeichen vom Band. Falls das Zeichen {\tt 0} ist,
schreibe eine {\tt 1} auf das Band, und merke Dir den "Ubertrag {\tt 0}.
Falls das Zeichen {\tt 1} ist, schreibe eine {\tt 0} auf das Band und
merke Dir den "Ubertrag {\tt 1}.
Bewege den Kopf nach links.
\item Falls das Band einen Blank enth"alt, und der "Ubertrag {\tt 1}
ist, schreibe eine {\tt 1} auf das Band und halte in $q_{\text{accept}}$.
Falls der "Ubertrag aber {\tt 0} ist, halte im Zustand $q_{\text{accept}}$.
\item Falls das Band eine {\tt 0} enth"alt und der "Ubertrag {\tt 0} ist,
schreibe eine {\tt 0} auf das Band, bewege den Kopf nach Links und fahre
weiter bei 4.
\item Falls das Band eine {\tt 0} enth"alt und der "Ubertrag {\tt 1}
ist oder umgekehrt, schreibe einen {\tt 1} auf das Band, merke Dir
"Ubertrag {\tt 0} und bewege den Kopf nach links. Weiter bei 4.
\item Falls das Band eine {\tt 1} enth"alt und der "Ubertrag ebenfalls
{\tt 1} ist, schreibe eine {\tt 0} auf das Band, merke Dir "Ubertrag {\tt 1},
bewege den Kopf nach links und fahre weiter bei 4.
\end{enumerate}
Etwas schneller geht es, wenn man keine Voraussetzung "uber die
Endposition von Schreib-/Lesekopf macht:
\begin{enumerate}
\item Fahre nach rechts bis zu einem Blank, bewege den Kopf nach links.
\item Solange eine {\tt 1} unter dem Schreib-/Lesekopf steht, ersetze sie
durch {\tt 0} und bewege den Kopf nach links.
\item Wenn eine {\tt 0} oder ein $\blank$ unter dem Schreib-/Lesekopf
steht, ersetze sie durch {\tt 1} und gehe in den Zustand $q_{\text{accept}}$
\end{enumerate}
Man kann diese Turing-Maschine auch als Zustandsautomat darstellen:
\[
\entrymodifiers={++[o][F]}
\xymatrix @-1mm {
*+\txt{}\ar[r]
        &q_0\ar@(ur,ul)_{{\tt 0}\to{\tt 0},R\atop {\tt 1}\to{\tt 1},R}
            \ar[r]^{\blank\to\blank,L}
                &q_1\ar[r]^{{{\tt 0}\to{\tt 1},R}
                    \atop {\blank\to{\tt 1},R}}
                    \ar@(ur,ul)_{{\tt 1}\to{\tt 0},L}
                        &*++[o][F=]{q_{\text{accept}}}
}
\]
\end{loesung}
}{}


\begin{teilaufgaben}
\item
Beschreiben Sie einen Turing-Maschinen-Algorithmus, welcher genau die
W"orter der Sprache aus Aufgabe~\ref{40000013} erkennt.
\item
Welche Komplexit"at hat ihr Algorithmus auf einer Standard-Turing-Maschine?
\end{teilaufgaben}

\begin{hinweis}
Es darf in a) auch eine erweiterte Turing-Maschine
mit mehreren B"andern oder mehreren Spuren verwendet werden.
\end{hinweis}

\begin{loesung}
\begin{teilaufgaben}
\item
Die Sprache besteht aus W"ortern "uber dem Alphabet 
$\Sigma =\{{\tt <},{\tt >},{\tt /}\}$, die so aufgebaut sind,
dass die `spitzen' Klammern `{\tt <}' und `{\tt >}' korrekt
geschachtelt sind. Diese Sprache ist kontextfrei (es w"are leicht,
daf"ur eine Grammatik anzugeben, wir brauchen das aber nicht), 
kann also sogar mit einem Stackautomaten analysiert werden.
Das Zustandsdiagramm
\[
\entrymodifiers={++[o][F]}
\xymatrix{
*+\txt{}\ar[r]
        &{}\ar[r]^{\varepsilon,\varepsilon\to {\tt \$}}
                &{}\ar@(ul,u)^{{\tt <},\varepsilon\to{\tt <}}
                    \ar@(ur,u)_{{\tt >},{\tt <}\to\varepsilon}
                    \ar@(dl,dr)_{{\tt /},\varepsilon\to\varepsilon}
                    \ar[r]^{\varepsilon,{\tt\$}\to\varepsilon}
                        &*++[o][F=]{}
}
\]
beschreibt diesen Stackautomaten.
Wir bauen daher eine Turing-Maschine, die einen Stackautomaten
simuliert. Dazu verwenden wir eine TM mit zwei B"andern, wobei das zweite
Band die Funktion des Stack "ubernimmt.

Zu Beginn steht also das zu untersuchende Wort auf Band~1,
der Schreib-/Lesekopf steht auf dem ersten Zeichen.
Das zweite Band ist leer.
{
\renewcommand{\theenumii}{\arabic{enumii}}
\renewcommand{\labelenumii}{\theenumii.}
\begin{enumerate}
\item Falls das aktuelle Zeichen auf Band 1 ein `{\tt <}' ist,
schreibe es auf Band zwei, und bewege beide Schreib-/Lesek"opfe ein
Feld nach rechts. Fahre weiter bei 1.
\item Falls das aktuelle Zeichen auf Band 1 ein `{\tt >}' ist,
und das aktuelle Zeichen auf Band 2 ein `{\tt <}', "uberschreibe
das Zeichen auf Band 2 mit \textvisiblespace und bewege den Kopf
nach links. Der Schreib-/Lesekopf von Band 1 wird nach rechts bewegt.
Fahre weietr bei 1.
\item Falls das aktuelle Zeichen auf Band 1 ein `{\tt /}' ist, bewege
den Kopf nach rechts.
Fahre weiter bei 1.
\item Falls das aktuelle Zeichen auf Band 1 ein \textvisiblespace{} ist,
akzeptiere.
\item In allen anderen F"allen verwerfe.
\end{enumerate}
}
\item
Der oben beschriebene Algorithmus hat Komplexit"at $O(n)$ auf einer 
Turing-Maschine mit zwei B"andern. Nach einem Satz aus der Vorlesung
kann dieser Algorithmus in einen f"ur eine Standard-Turing-Maschine
umgewandelt werden, wobei die Laufzeit quadriert wird. Der f"ur 
Standardturingmaschinen konvertierte Algorithmus h"atte also die
Laufzeit $O(n^2)$.
\end{teilaufgaben}
\end{loesung}

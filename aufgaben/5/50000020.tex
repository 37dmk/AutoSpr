Sei $\Sigma=\{\texttt{a},\texttt{b},\texttt{A},\texttt{B},\blank\}$
das Alphabet, die W"orter der Sprache $L$ "uber $\Sigma$ bestehen
jeweils aus zwei gleich langen Teilen getrennt von einem Blank.
Die einzelnen Teile enthalten keine Blanks,
sie enthalten die gleichen Buchstaben, einzige die
Gross-/Kleinschreibung hat gewechselt.
Es geh"oren also zu $L$ zum Beispiel
\[
\texttt{aaa\blank AAA},\quad
\texttt{aBbA\blank AbBa},\quad
\texttt{aaaaaaaa\blank AAAAAAAAA}
\]
aber nicht
\[
\texttt{aaa\blank aab},\quad
\texttt{aba\blank ABa},\quad
\texttt{Aaa\blank Baa}
\]

Finden Sie das Zustandsdiagramm einer Turing-Maschine, die $L$ erkennt.

\begin{loesung}
Wir verwenden eine zus"atzliches Alphabet-Zeichen \texttt{x}, mit welchem
wir diejenigen Zeichen "uberschreiben, die wir schon gepr"uft haben.

Wir beginnen beim ersten Zeichen, welches wir uns merken, und fahren
dann bis zum n"achsten Blank und eventuell danach folgende \texttt{x}
bis wir einen Buchstaben finden.
Dieser muss in der anderen Schreibung sein wie der, mit dem wir
gestartet sind.
Dann fahren wir wieder zum Blank zun weiter bis zum \texttt{x}, wo alles
von vorne beginnt.

Wenn der erste Buchstabe ein \texttt{a} ist, dann suchen wir ein passends
\texttt{A} wie folgt:
\[
\entrymodifiers={++[o][F]}
\xymatrix @+5mm {
*+\txt{} \ar[r]^{\texttt{a}\to\texttt{x},R}
	&{q_1}\ar[r]^{\blank\rightarrow\blank,R}
		\ar@(ur,ul)_{.\to.,R}
		&{q_2}\ar[r]^{\texttt{A}\to\texttt{x},L}
			\ar@(ur,ul)_{\texttt{x}\to\texttt{x},R}
			&{q_3}\ar[r]^{\blank\to\blank,L}
				\ar@(ur,ul)_{\texttt{x}\to\texttt{x},L}
				&{q_4}\ar[r]^{\texttt{x}\to\texttt{x},R}
					\ar@(ur,ul)_{.\to.,L}
					&{q_5}
}
\]
Dabei sind "Uberg"ange $\texttt{.}\to\texttt{.}$ als ``f"ur alle anderen
Zeichen'' zu verstehen.
F"ur die Buchstaben \texttt{b}, \texttt{A} und \texttt{B} m"ussen wir dasselbe
machen, der hintere Teil sogar identisch. Wir bekommen
\[
\entrymodifiers={++[o][F]}
\xymatrix @+5mm {
*+\txt{}
	&*+\txt{}
		&{q_1}\ar[rr]^{\blank\rightarrow\blank,R}
			\ar@(ur,ul)_{.\to.,R}
			&*+\txt{}
			&{q_2}\ar[rrddd]^{\texttt{A}\to\texttt{x},L}
				\ar@(ur,ul)_{\texttt{x}\to\texttt{x},R}
\\
*+\txt{}
	&*+\txt{}
		&{q_1}\ar[rr]^{\blank\rightarrow\blank,R}
			\ar@(ur,ul)_{.\to.,R}
			&*+\txt{}
			&{q_2}\ar[rrdd]^{\texttt{B}\to\texttt{x},L}
				\ar@(ur,ul)_{\texttt{x}\to\texttt{x},R}
\\
*+\txt{}
\ar[d]
	&*+\txt{}
		&{q_1}\ar[rr]^{\blank\rightarrow\blank,R}
			\ar@(ur,ul)_{.\to.,R}
			&*+\txt{}
			&{q_2}\ar[rrd]^{\texttt{a}\to\texttt{x},L}
				\ar@(ur,ul)_{\texttt{x}\to\texttt{x},R}
\\
{q_0}	\ar[rr]^{\texttt{B}\to\texttt{x},R}
	\ar[rru]^{\texttt{A}\to\texttt{x},R}
	\ar[rruu]^{\texttt{b}\to\texttt{x},R}
	\ar[rruuu]^{\texttt{a}\to\texttt{x},R}
	\ar[d]^{\blank\to\blank,R}
	&*+\txt{}
		&{q_1}\ar[rr]^{\blank\rightarrow\blank,R}
			\ar@(ur,ul)_{.\to.,R}
			&*+\txt{}
			&{q_2}\ar[rr]^{\texttt{b}\to\texttt{x},L}
				\ar@(ur,ul)_{\texttt{x}\to\texttt{x},R}
				&*+\txt{}
					&{q_3}\ar[dlll]^{\blank\to\blank,L}
						\ar@(ur,r)^{\texttt{x}\to\texttt{x},L}
\\
{q_5}\ar@(ul,l)_{\texttt{x}\to\texttt{x},R}
	\ar[r]^{\blank\to\blank,R}
	&*++[o][F=]{q_{\text{accept}}}
		&*+\txt{}
		&{q_4}\ar[ulll]_{\texttt{x}\to\texttt{x},R}
			\ar@(ur,ul)_{.\to.,L}
}
\]
Darin haben wir am Ende noch die Pr"ufung angeh"angt, dass im zweiten Teil
keine weiteren Buchstaben vorkommen.
Damit haben wir eine Turing-Maschine f"ur das gestellte Problem gefunden.
\end{loesung}

\begin{bewertung}
Erweiterung des Alphabets oder der Turing-Maschine zwecks Codierung
der bereits gematchten Zeichen ({\bf E}) 1 Punkt,
Trennzeichen erkannt ({\bf T}) 1 Punkt,
Matching-Strategie von Gross- und Kleinbuchstaben ({\bf M}) 1 Punkt,
gematchte Zeichen in beiden Teilstrings werden markiert ({\bf B}) 1 Punkt,
gleichviele Zeichen in beiden Teilstrings ({\bf G}) 1 Punkt,
alle Zeichen werden gematcht ({\bf A}) 1 Punkt,
\end{bewertung}


Nurikabe ist ein japanisches Logikr"atsel, welches auf einem rechteckigen
Gitter mit $u\times v$ Zellen gespielt wird. In einigen Feldern dieses Gitters
sind Zahlen eingetragen.
der Spieler muss einige der leeren Felder ausw"ahlen und schwarz
f"arben. Die gef"arbten Felder heissen ``Fluss''. Es entstehen so
weisse Gebiete, bestehend aus weissen Feldern, die entlang einer Kante
benachbart sind. Es m"ussen folgende Regeln beachtet werden:
\begin{enumerate}
\item In jedem weissen Gebiet gibt es genau ein Zahlfeld, und die Zahl darin
gibt an, wieviele weisse Felder das Gebiet umfasst.
\item Es darf nur einen Fluss geben (das schwarze Gebiet muss zusammenh"angend
sein), und es darf keine schwarzen $2\times 2$-Gebiete geben (der Fluss hat
keine T"umpel).
\end{enumerate}
Die Abbildung zeigt ein korrekt gel"ostes Nurikabe.
\begin{center}
\includeagraphics[width=2truein]{nurikabe}
\end{center}
Ist die Sprache
\[
\text{\textsl{NURIKABE}}=\{\langle N\rangle\,|\,\text{$N$ ist ein l"osbares Nurikabe-R"atsel}\}
\]
in NP?

\ifthenelse{\boolean{loesungen}}{
\begin{loesung}
Sie ist mit Sicherheit entscheidbar, denn man kann alle $2^{uv}$
m"oglichen Belegungen des Spielfeldes mit schwarzen Feldern daraufhin
"uberpr"ufen, ob sie die Regeln einhalten. Dieser Algorithmus braucht
zwar exponentielle Zeit, zeigt aber, dass die Sprache entscheidbar ist.

Es ist zu zeigen, dass es einen polynmiellen Verifizierer gibt. Dieser
braucht ein Zertifikat $c$, f"ur welches wir eine Liste der schwarzen
Felder verlangen. Damit wird folgender Verifikationsalgorithmus
durchgef"uhrt:

\begin{tabular}{rll}
&Schritt&Aufwand\\
\hline
1&\begin{minipage}[t]{4.2truein}\strut
F"ur jedes Zahlfeld ($O(n)$) verwendet man einen Markieralgorithmus,
der alle zum Gebiet dieses Zahlfeldes geh"orenden weissen
Felder bestimmt ($O(n)$ Durchg"ange mit Aufwand
$O(n)$.\strut\end{minipage}&$O(n^3)$\\
2&\begin{minipage}[t]{4.2truein}\strut
Trifft dieser Algorithmus auf ein weiteres
Zahlfeld, ist der erste Teil von Regel 1 verletzt
($\to q_{\text{reject}}$).
\strut\end{minipage}&$O(n)$\\
3&\begin{minipage}[t]{4.2truein}\strut
Weicht die Zahl der gefundenen weissen Felder vom Inhalt des Zahlfeldes
ab, ist der zweite Teil von Regel 1 verletzt
($\to q_{\text{reject}}$).\strut\end{minipage}&$n\cdot O(n)$\\
4&\begin{minipage}[t]{4.2truein}\strut
Ebenfalls mit einem Markieralgorithmus wird "uberpr"uft, ob das
schwarze Gebiet zusammenh"angend ist.
\strut\end{minipage}&$O(n^2)$\\
5&\begin{minipage}[t]{4.2truein}\strut
F"ur jedes schwarze Feld wird "uberpr"uft, ob es Teil eines
$2\times 2$-T"umpels ist.
\strut\end{minipage}&$O(n)$\\
\hline
&Total&$O(n^3)$
\end{tabular}

In den Aufwandsch"atzungen verwenden wir f"ur $n$ die L"ange der
Beschreibung $\langle N\rangle$. Da der Aufwand f"ur die Verifikation
polynomiell in der Gr"osse des Nurikabe ist, ist
$\text{\textsl{NURIKABE}}\in\operatorname{NP}$.

In der Tat ist sogar war, dass \textsl{NURIKABE} NP-vollst"andig
ist. Dazu wurde eine "ahnlich Technik verwendet wie in dem in
der Vorlesung skizzierten Beweis
der NP-Vollst"andigkeit von \textsl{MINESWEEPER-CONSISTENCY}:
Ein \textsl{SAT}-Problem wurde polynomiell in eine Schaltung und dann in
ein Nurikabe-Problem "ubersetzt:
http://www.cs.umass.edu/\~\ mcphailb/papers/2003thesis.pdf
\end{loesung}
}{ }


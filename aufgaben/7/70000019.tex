Zeigen Sie, dass aus $P=NP$ folgen w"urde, dass es einen Algorithmus
gibt, der in einem beliebigen ungerichteten Graphen eine Clique
maximaler Gr"osse in polynomieller Zeit finden kann.

\begin{hinweis}
Das in der Vorlesung besprochene Cliquenproblem besteht
darin, eine $k$-Clique zu finden, hier muss dagegen eine Clique maximaler
Gr"osse gefunden werden, es braucht hier also noch einen weiteren
Schritt.
\end{hinweis}

\begin{loesung}
Es ist bekannt, dass das Finden einer $k$-Clique ein NP-vollst"andiges
Problem ist. Falls $P=NP$ g"abe es einen polynomiellen Algorithmus, um
eine $k$-Clique zu finden.

In der Aufgaben wird jedoch verlangt, eine Clique maximaler
Gr"osse zu finden. Wir verwenden den eben gefunden polynomiellen Algorithmus,
um einen polynomiellen Algorithmus f"ur das Problem eine maximale Clique
zu finden.
Sei $m$ die Anzahl der Knoten des Graphen $G$.
\begin{enumerate}
\item F"ur alle $k$ beginnend bei $m$ bis hinunter zu $0$ f"uhre Schritt 2 aus:
\item Verwende $M$ um eine $k$-Clique in $G$ zu finden. Falls eine
gefunden wird, gib sie als Resultat zur"uck.
\end{enumerate}
Dieser Algorithmus f"uhrt im schlimmsten Fall $m$ mal den Algorithmus zum
Finden einer $k$-Clique aus, welcher polynomielle Laufzeit hat, der neue
Algorithmus hat also ebenfalls polynomielle Laufzeit.
\end{loesung}

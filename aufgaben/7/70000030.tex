Das Spiel {\em Hitori} wird auf einem $n\times n$-Feld gespielt, in jeder
Zelle des Spielfeldes ist eine Zahl zwischen 1 und $n$ eingetragen.
Der Spieler muss nun Zellen schw"arzen, so dass zwei Regeln
eingehalten werden:
\begin{enumerate}
\item In einer Zeile darf keine (nicht geschw"arzte) Zahl mehr als
einmal vorkommen.
\item Benachbarte Zellen d"urfen nicht geschw"arzt werden.
\end{enumerate}
Die folgende Abbildung zeigt ein {\em Hitori} (links) mit L"osung (rechts).
\begin{center}
\includeagraphics[width=0.3\hsize]{Hitori.png}
\qquad
\includeagraphics[width=0.3\hsize]{Hitoricompleted.png}
\end{center}
Kann eine nichtdeterministische Turing-Maschine in polynomieller Zeit
entscheiden, ob ein {\em Hitori} eine L"osung hat?

\begin{loesung}
Das Problem ist sicher entscheidbar, man kann alle $2^{n^2}$
m"oglichen Schw"arzungen des Spielfeldes daraufhin testen, ob sie die
Regeln 1.~und 2.~erf"ullen.

Es ist f"ur eine nicht deterministische Maschine genau dann m"oglich,
die L"osbarkeit eines {\em Hitori} zu entscheiden, wenn man einen
polynomiellen Verifizierer finden kann.

Als L"osungszertifikat f"ur einen solchen Verifizierer verwendet man die
Zellen, die geschw"arzt werden m"ussen.
Darauf ist der folgende Verifikationsalgorithmus anzuwenden:
\begin{compactenum}
\item F"ur jede geschw"arzte Zelle kontrolliere, ob die maximal
vier Nachbarzellen nicht geschw"arzt sind.
\item F"ur jede nicht geschw"arzte Zelle kontrolliere, ob die anderen
Zellen in der gleichen Zeile eine andere Zahl enthalten.
\item F"ur jede nicht geschw"arzte Zelle kontrolliere, ob die anderen
Zellen in der gleichen Spalte eine andere Zahl enthalten.
\end{compactenum}
Der Rechenaufwand f"ur die einzelnen Schritte ist:
\begin{center}
\begin{tabular}{c|l|>{$}c<{$}}
1.&Nachbarzellen nicht geschw"arzt&O(n^2)\\
2.&Keine gleichen Werte in einer Zeile&O(n^3)\\
3.&Keine gleichen Werte in einer Spalte&O(n^3)\\
\hline
  &L"osung f"ur Hitori&O(n^3)
\end{tabular}
\end{center}
Daraus kann man ablesen, dass die Verifikation in polynomieller Zeit 
m"oglich ist. Das Problem zu entscheiden, ob ein {\em Hitori} eine 
L"osung hat, ist also in NP.
\end{loesung}

\begin{bewertung}
Zertifikat ({\bf Z}) 1 Punkt,
Drei wesentliche Schritte des Entscheidungsalgorithmus ({\bf E}) je 1 Punkt,
maximal 3 Punkte,
Absch"atzung der Laufzeitkomplexit"at f"ur alle Schritte ({\bf L}) 1 Punkt,
Schlussfolgerung, dass das Problem in NP ist ({\bf NP}) 1 Punkt.
\end{bewertung}


\def\spielname{Flow}
Beim Spiel \spielname{} (als App {\em Flow Free} im AppStore erh"altlich)
sind auf einem $n\times m$-Spielfeld Paare von farbigen
Punkten vorgegeben (Abbildung links).
Der Spieler muss jeweils gleichfarbige Punkte durch einen Weg verbinden,
der vollst"andig "uber freie Felder des Spielfeldes verl"auft.
Dabei darf ein Feld nicht von zwei verschiedenen Wegen benutzt werden
(Abbildung rechts).
\begin{center}
\includeagraphics[width=0.4\hsize]{\spielname-aufgabe.png}
\qquad
\includeagraphics[width=0.4\hsize]{\spielname-loesung.png}
\end{center}
Kann eine nicht deterministische Turing-Maschine ein \spielname-R"atsel in
polynomieller Zeit l"osen?

\begin{loesung}
Ein \spielname-R"atsel kann von einer nicht deterministischen Turing-Maschine
in polynomieller Zeit gel"ost werden, wenn es einen polynomiellen
Verifizierer gibt.
Als L"osungszertifikat verlangen wir die gesuchten farbigen Pfade,
wobei die Farbe eines Pfades die Farbe der zwei Vorgabepunkte ist, die er
verbindet.
Die Pfade k"onnen dadurch beschrieben werden, dass in jedem Feld die Farbe 
angegeben wird und welche Kanten des Feldes miteinander verbunden werden.
Dies kann zum Beispiel durch die sechs Symbole
$\Gamma = \{-,\vert,\urcorner,\lrcorner,\ulcorner, \llcorner\}$
bewerkstelligt werden.
Man sagt, ein Feld f"uhrt zu einem Nachbarfeld, wenn das Symbol aus
$\Gamma$ im Feld mit einem Ende zum Nachbarfeld zeigt.
Enth"alt ein Feld zum Beispiel das Symbol $\llcorner$, dann f"uhrt
das Feld auf die Nachbarfelder oben und rechts.

Der Verifikationsalgorithmus pr"uft folgendes:
\begin{enumerate}
\item Jedes Vorgabefeld hat genau ein Nachbarfeld gleicher Farbe, welches
auf das Vorgabefeld f"uhrt.
\item Jedes Feld f"uhrt ausschliesslich zu Feldern gleicher Farbe.
\item F"uhrt ein Feld auf ein Nachbarfeld, welches kein Vorgabefeld ist,
dann f"uhrt das Nachbarfeld auch zur"uck auf das Ausgangsfeld.
\end{enumerate}
Der Aufwand f"ur diese Pr"ufungen ist
\begin{center}
\begin{tabular}{rl>{$}c<{$}}
1.&Pfade enden in Vorgabepunkten          &O(nm)\\
2.&Gleichfarbigkeit innerhalb eines Pfades&O(nm)\\
3.&Zusammenhang von Pfaden                &O(nm)\\
 &Total                                  &O(nm)
\end{tabular}
\end{center}
Der Aufwand ist also polynomiell in der Problemgr"osse $nm$, der Verifizierer
hat polynomielle Laufzeit.
Damit ist gezeigt, dass das Problem in NP ist, also dass eine nicht
deterministische Turingmaschine das Problem in polynomieller Zeit l"osen
kann.

Man kann ein \spielname-Spiel auch als ein Ausf"ullr"atsel betrachten.
Jedes Feld muss mit der Farbe eines der vorgegebenen Punkte oder scharz 
und einem Zeichen aus $\Gamma$ versehen werden.
Die dabei einzuhaltenden Regeln sind die Regeln 1--3, die oben aufgelistet
wurden.
Die Aufwandsberechnung zeigt, dass dies ein polynomielles Ausf"ullr"atsel ist.
\end{loesung}

\begin{bewertung}
Polynomieller Verifizierer {\bf V} 1 Punkt,
L"osungszertifikat {\bf C} 1 Punkt,
Verifikationsalgorithmus: Pfadenden {\bf E} 1 Punkt,
Gleichfarbigkeit {\bf G} 1 Punkt,
Zusammenhang {\bf Z} 1 Punkt,
Aufwandsch"atzung ist polynomiell {\bf A} 1 Punkt.
\end{bewertung}



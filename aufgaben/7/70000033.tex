Der neue CEO einer grossen Fluggesellschaft m"ochte das Personal in allen
Flugh"afen pers"onlich besuchen, die die Fluggesellschaft anfliegt.
Er bittet seine Sekret"arin, einen optimalen Besuchsplan zusammenzustellen,
bei der er jede Destination nur genau einmal besuchen muss.
Als der Plan eine Woche sp"ater immer noch nicht fertig ist, wird er
ungeduldig.
Warum dauert es so lange, einen Besuchsplan zusammenzustellen?

\begin{loesung}
Die Fl"uge, die die Destinationen der Fluggesellschaft miteinander
verbinden, bilden einen gerichteten Graphen.
Der CEO verlangt die L"osung des Problems HAMPATH, welches einen
hamiltonschen Pfad sucht, also einen Pfad, der jede Ecke des Graphen
genau einmal besucht.
Dieses Problem ist NP-vollst"andig, es gibt also keine effizienten
Algorithmen f"ur grosse solche Probleme.
Eine Reduktion $\text{HAMPATH}\le_P\text{BESUCH}$ ist:
\begin{align*}
\text{Knoten}    & \to \text{Destination}\\
\text{Kante}     & \to \text{Flug} \\
\text{hamiltonscher Pfad}&\to\text{Besuchsplan}
\end{align*}
\end{loesung}

\begin{diskussion}
Man k"onnte auch das Travelling-Salesman-Problem (TSP) als Vergleichsproblem
heranziehen.
Dies geht leider nicht ganz, weil das "ubliche TSP von einem ungerichteten
Graphen ausgeht.
Es gibt allerdings auch das {\em asymmetrische TSP}, welches von einem 
gerichteten Graphen mit richtungsabh"angigen Kantengewichten ausgeht.
Das asymmetrische TSP unterscheidet sich von HAMPATH durch die
zus"atzlichen Kantengewichte, die f"ur die L"osung eine m"oglichst
kleine Summe ergeben.
Eine solche L"osung muss also zus"atzlich ein Mapping f"ur die
Kantengewichte spezifizieren.
\end{diskussion}

\begin{bewertung}
Reduktionsmethode ({\bf R}) 1 Punkt,
HAMPATH als Vergleichsproblem ({\bf H}) 3 Punkte,
Reduktionsabbildung f"ur Knoten ({\bf D}, 1 Punkt) und Kanten
({\bf F}, 1 Punkt).
\end{bewertung}


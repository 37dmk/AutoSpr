Das R"atselspiel {\it Fillomino} wird auf einem $n\times m$ Spielfeld 
gespielt. In einzelnen Zellen des Spielfeldes sind Zahlen eingetragen.
Der Spieler muss die leeren Felder mit Zahlen f"ullen, so dass
zusammenh"angende Gebiete soviele Felder enthalten
wie die Zahl angibt. Ein zusammenh"angendes Gebiet besteht aus Feldern,
die sich entlang einer Kante ber"uhren und alle die gleiche Zahl enthalten.
Die Abbildung zeigt ein Fillomino-R"atsel (links) mit L"osung (rechts).
\begin{center}
\includeagraphics[width=0.4\hsize]{fill1.png}
\qquad
\includeagraphics[width=0.4\hsize]{fill8.png}
\end{center}
Zeigen Sie, dass eine nichtdeterministische Maschine in polynomieller
Zeit entscheiden kann, ob ein Fillomino eine L"osung hat.

\begin{loesung}
Es gen"ugt zu zeigen, dass es einen polynomiellen Verifizierer gibt.

Ein polynomieller Verifizierer braucht ein L"osungszertifikat als 
zus"atzlichen Input. Im vorliegenden Fall verlangen wir das vollst"andig
ausgef"ullte Fillomino-Feld als Zertifikat.

Mit dem vollst"andig ausgef"ullten Feld kann der Verifizierer wie folgt
vorgehen, um die Korrektheit der L"osung zu "uberpr"ufen.
\begin{enumerate}
\item "Uberpr"ufe, ob jede Zelle mit einer Zahl gef"ullt ist.
\item F"ur jede Zelle des Spielfeldes pr"ufe wie folgt, ab die Zahl der
Zellen des zuammenh"angenden Gebietes der Zelle die richtige Anzahl
Zellen enth"alt
\begin{enumerate}
\item Markiere die Startzelle
\item \label{70000024:loop} Gehe durch das Spielfeld und markiere alle Zellen, die zu bereits
markierten Zellen benachbart sind, und die die gleiche Zahl enthalten wie
die Startzelle.
\item Wiederhole \ref{70000024:loop} bis sich nichts mehr "andert
\item Z"ahle die markierten Zellen, brich mit $q_{\text{reject}}$ ab,
wenn die Zahl der markierten Zellen nicht mit der Zahl in der Startzelle
"ubereinstimmt.
\end{enumerate}
\end{enumerate}
Es muss jetzt nur noch gezeigt werden, dass der Verifizierer tats"achlich
in polynomieller Zeit fertig wird. Die einzelnen Schritte des Verifzierers
brauchen Zeit wie folgt
\begin{enumerate}
\item Zellen pr"ufen: $O(mn)$
\item In diesem Schritt werden $mn$ mal die folgenden Schritte ausgef"uhrt:
\begin{enumerate}
\item Startzelle markieren: $O(1)$
\item Nachbarzellen markieren: $O(mn)$
\item Abbruch falls keine "Anderung: $O(1)$
\item Markierte Zellen z"ahlen: $O(mn)$
\end{enumerate}
\end{enumerate}
Insgesamt ist also die Verifikation in maximal $mnO(mn)=O(m^2n^2)$ m"oglich,
insbesondere ist die Arbeit polynomiell in der Gr"osse des Spielfeldes.
Damit ist gezeigt, dass es einen polynomiellen Verifizierer gibt,
also kann eine nichtdeterministische Maschine das Problem in polynomieller
Zeit l"osen.
\end{loesung}

\begin{diskussion}
Takayuki Yato hat 2003 in seiner Masterarbeit in Tokyo nachgewiesen,
dass Fillomino NP-vollst"andig ist:
\url{http://www-imai.is.s.u-tokyo.ac.jp/~yato/data2/MasterThesis.pdf}.
\end{diskussion}

\begin{bewertung}
Beobachtung, dass es reicht, einen polynomiellen Verifizerer ({\bf V})
zu finden 1 Punkt,
Definition des Zertifikats ({\bf Z}) 1 Punkt,
Definition des Verifikationsalgorithmus ({\bf A}) 2 Punkte,
Absch"atzung der Laufzeitkomplexit"at ({\bf L}) 2 Punkte.
\end{bewertung}


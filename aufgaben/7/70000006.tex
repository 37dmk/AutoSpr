Job-Parallelisierbarkeit:
Gegeben sei eine Menge von $n$ Jobs $J_1,\dots,J_n$, die jeweils exklusiv
auf $m$ Resourcen $r_1,\dots,r_m$ zugreifen, diese Jobs d"urfen
nicht gleichzeitig laufen. Zeigen Sie, dass das Problem zu entscheiden,
ob mit diesen Jobs zu irgend einem Zeitpunkt mehr als $k$ Prozessoren
ausgelastet werden k"onnen, NP-vollst"andig ist.

\begin{loesung}
Wir zeigen, dass das Problem "aquivalent ist mit dem $k$-Cliquen-Problem,
welches in der Vorlesung bereits als NP-vollst"andig erkannt worden ist.

Aus einem Job-Parallelisierbarkeitsproblem konstruieren wir einen Graphen $G$,
dessen Ecken die Jobs sind. Zwischen je zwei Jobs f"ugen wir eine
Kante hinzu, wenn die Jobs gleichzeitig laufen k"onnen, wenn also
keine der Resourcen von beiden Jobs beansprucht wird. Eine $k$-Clique
ist eine Auswahl von $k$ Prozessen, die gegenseitig keine Resourcen-Konflikte
haben, also gleichzeitig laufen k"onnen. Gibt es eine $k$-Clique in $G$,
lassen sich mit den Jobs der Clique $k$ Prozessoren auslasten.

Sei umgekehrt ein Graph gegeben. Wir nennen die Ecken ``Jobs''. Die
Paare $(J_i,J_j)$, f"ur die im Graph keine Kante existiert,  nennen
wir Resourcen $r_{ij}$. Die Kanten des Graphen dr"ucken also aus, dass
die beiden Jobs an den Enden der Kante gleichzeitig laufen k"onnen.
Falls sich $k$ der Jobs gleichzeitig starten lassen, enth"alt der
Graph jede Verbindung zwischen diesen Jobs, die $k$ Jobs bilden also
eine $k$-Clique des Graphen.

Somit sind das Job-Parallelit"ats-Problem und das Cliquen-Problem
"aquivalent, insbesondere ist Job-Parallelit"at ein NP-vollst"andiges
Problem.
\end{loesung}

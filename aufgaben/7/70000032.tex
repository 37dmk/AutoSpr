{\em Heyawake} ist ein japanisches R"atsel, welches auf einem $n\times m$-Feld
gespielt wird. Das Spielfeld ist durch dickere Linien unterteilt in gr"ossere rechteckige Gebiete, die {\it Zimmer} genannt werden. 
Ziel des Spieles ist es, einzelne Felder schwarz einzuf"arben, so dass die
folgenden Regeln erf"ullt sind:
\begin{enumerate}
\item
Schwarze Quadrate grenzen niemals "uber eine Kante aneinander.
\item
Alle weissen Quadrate h"angen "uber Kanten zusammen
\item
Die Zahlen geben an, wie viele schwarze Quadrate in einem Raum vorkommen
\item
Ein Raum ohne Zahl kann beliebig viele schwarze Quadrate enthalten
(solange die anderen Regeln erf"ullt sind)
\item
Eine gerade (horizontale oder vertikale) Linie aus zusammenh"angenden
weissen Quadraten kann sich h"ochstens "uber zwei
R"aume erstrecken.
\end{enumerate}
Das folgende Beispiel zeigt Aufgabe und L"osung:
\begin{center}
\includeagraphics[width=0.35\hsize]{solution-big.png}
\qquad
\includeagraphics[width=0.35\hsize]{example-big.png}
\end{center}

Kann eine nicht deterministische Maschine in polynomieller Zeit entscheiden,
ob ein Heyawake-R"atsel eine L"osung hat?

\begin{loesung}
Das Problem ist sicher entscheidbar, man kann alle $2^{nm}$ Einf"arbungen
daraufhin "uberpr"ufen, ob die Regeln eingehalten werden.
Daf"ur ist allerdings exponentielle Zeit notwendig.

Eine nicht deterministische Maschine kann die Entscheidungsfrage genau
dann in polynomieller Zeit beantworten, wenn es einen polynomiellen
Verifizierer gibt.
Als L"osungszertifikat $c$ fordern wir die Belegung der schwarzen Felder.
Der Verfikationsalgorithmus muss die Regeln 1--5 "uberpr"ufen, und braucht
daf"ur Zeit wie folgt
\begin{center}
\begin{tabular}{r|l|>{$}c<{$}}
Regel&Arbeit&\text{Aufwand}\\
\hline
1.&F"ur jedes schwarze Feld 4 Nachbarn "uberpr"ufen&O(nm)\\
2.&Zusammenhang "uberpr"ufen&O(nm)\\
3.&F"ur jede Zahl die Anzahl schwarzer Felder im Zimmer pr"ufen&O(nm)\\
4.&automatisch erf"ullt nach 3.&0\\
5.&F"ur jedes weisse Feld, "uberpr"ufe Zeilenl"ange&O(nm(n+m))\\
\hline
&$c$ ist L"osung des Heyawake-R"atsels&O(nm(n+m))
\end{tabular}
\end{center}
Nur die "Uberpr"ufung des Zusammenhangs ist etwas anspruchsvoller.
Man kann dazu einen Einf"arbe-Algorithmus verwenden:
\begin{enumerate}
\item Markiere das erste weisse Feld {\color{red}rot}.
\item F"ur jedes {\color{red}rot} markierte Feld markiere rekursiv
alle noch nicht {\color{red}rot} markierten weissen Nachbarfelder ebenfalls
{\color{red}rot}.
\end{enumerate}
Wenn am Ende dieses Algorithmus weisse Felder "ubrig bleiben, ist die
in Regel~2 formulierte Bedinung nicht erf"ullt.
Der F"arbealgorithmus besucht jedes Feld h"ochstens einmal, braucht also
Zeit $O(nm)$.
F"ur die "Uberpr"ufung, dass es keine weissen Felder gibt, wird ebenfalls
Zeit $O(nm)$ ben"otigt.

Die "Uberpr"ufung der Regel~5 muss in jedem der $nm$-Felder h"ochstens $n$
vertikale und $m$ vertikale Felder "uberpr"ufen, daf"ur ist Zeit
$O(nm(n+m))$ n"otig.

Damit ist gezeigt, dass es einen polynomiellen Verifizierer gibt, und
es folgt, dass eine nichtdeterministische Maschine ein Heyawake-R"atsel
in polynomieller Zeit l"osen kann.
\end{loesung}

\begin{diskussion}
Die Frage, ob ein Heyawake-R"atsel eine L"osung hat hat, ist NP-vollst"andig:
Markus Holzer und Oliver Ruepp, {\it The Troubles of Interior Design--A
Complexity Analysis of the Game Heyawake}, Proceedings,
4th International Conference on Fun with Algorithms, LNCS 4475, Springer,
Berlin/Heidelberg, 2007, pp.~198-212.
\end{diskussion}

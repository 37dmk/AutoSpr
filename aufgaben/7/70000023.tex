Ein aufstrebendes Film-Festival ist derart gewachsen, dass der 
Vorf"uhrsaal nicht mehr reicht. Daher m"ussen jetzt zwei gleich
grosse S"a"ale verwendet werden, und trotzdem ist das Festival
wieder ausverkauft, und zwar in einem Masse, dass "uberhaupt nur Stars und Prominente
samt Ihrer Entourage eingelassen werden k"onnen, f"ur
einzelne Besucher gibt es keine Pl"atze.

Doch die Stars st"oren sich daran, dass sie m"oglicherweise nicht
ihre ganze Entourage im gleichen Saal haben k"onnen. Daher muss
kurzfristig eine Aufteilung der Festival-G"aste gefunden werden,
so dass die beiden S"a"ale so gef"ullt werden k"onnen, dass
jede Entourage vollst"andig in einem der S"a"ale Platz nimmt.

Der Festival-Direktor ist jedoch sehr "uberrascht, dass die
Bestimmung einer solchen Aufteilung so lange dauert. Warum
sind Sie nicht "uberrascht?

\begin{loesung}
Zu jedem Star $i\in I$ gibt es eine Entourage mit $c_i$
Mitgliedern. Diese Menge muss jetzt in zwei Teilmengen
$A$ (Stars samt Entourage, die in Saal A Platz nehmen) und
$B$ (Stars samt Entourage, die in Saal B Platz nehmen) aufgeteilt
werden, so dass $I=A\cup B$. Die Aufteilung muss so sein, dass
in beiden S"a"alen gleich viele Leute Platz nehmen, also
\[
\sum_{i\in A}c_i=
\sum_{i\in B}c_i.
\]
Dies ist das Problem {\it PARTITION}. Das gestellte Problem ist also
"aquivalent zum NP-vollst"andigen Problem {\it PARTITION}, und ist
daher ebenfalls NP-vollst"andig. Man kann daher nach aktuellem
Wissen nicht erwarten, dass es daf"ur einen effizienten Algorithmus
gibt.
\end{loesung}

\begin{bewertung}
Auswahl eines geeigneten Vergleichsproblems ({\bf V}) 1 Punkt,
Reduktion: Stars ({\bf S}) 1 Punkt, Entourage ({\bf E}) 1 Punkt,
Aufteilung ({\bf A}) 1 Punkt,
Bedingungen ({\bf B}) 1 Punkt,
Schlussfolgerung ({\bf F}) 1 Punkt.
\end{bewertung}

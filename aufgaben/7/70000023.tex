Ein aufstrebendes Film-Festival ist derart gewachsen, dass der 
Vorf"uhrsaal nicht mehr reicht. Daher m"ussen jetzt zwei gleich
grosse S"a"ale verwendet werden, und trotzdem ist das Festival
wieder ausverkauft, und zwar in einem Masse, dass "uberhaupt nur Stars und Prominente
samt ihrer Entourage eingelassen werden k"onnen, f"ur
einzelne Besucher gibt es keine Pl"atze.

Doch die Stars st"oren sich daran, dass sie m"oglicherweise nicht
ihre ganze Entourage im gleichen Saal haben k"onnen. Daher muss
kurzfristig eine Aufteilung der Festival-G"aste gefunden werden,
so dass die beiden S"a"ale so gef"ullt werden k"onnen, dass
jede Entourage vollst"andig in einem der S"a"ale Platz nimmt.

Der Festival-Direktor ist jedoch sehr "uberrascht, dass die
Bestimmung einer solchen Aufteilung so lange dauert. Warum
sind Sie nicht "uberrascht?

\begin{loesung}
Zu jedem Star $i\in I$ gibt es eine Entourage mit $c_i$
Mitgliedern. Diese Menge muss jetzt in zwei Teilmengen
$A$ (Stars samt Entourage, die in Saal A Platz nehmen) und
$B$ (Stars samt Entourage, die in Saal B Platz nehmen) aufgeteilt
werden, so dass $I=A\cup B$. Die Aufteilung muss so sein, dass
in beiden S"a"alen gleich viele Leute Platz nehmen, also
\[
\sum_{i\in A}c_i=
\sum_{i\in B}c_i.
\]
Dies ist das Problem {\it PARTITION}. Das gestellte Problem ist also
"aquivalent zum NP-vollst"andigen Problem {\it PARTITION}, und ist
daher ebenfalls NP-vollst"andig. Man kann daher nach aktuellem
Wissen nicht erwarten, dass es daf"ur einen effizienten Algorithmus
gibt.
\end{loesung}

\begin{diskussion}
Man kann auch versuchen, das Problem Festival-Problem mit
{\it SUBSET-SUM} vergleichen.
Zu jedem Star $i\in I$ geh"ort die Zahl $s_i$ der Mitglieder in der
Entourage des Stars. Sei $S=\{s_i\,|\,i\in I\}$ die Menge all dieser
Mitgliederzahlen. Sei $t$ die Platzzahl eines der Vorf"uhrs"a"ale.
Man muss jetzt eine Teilmenge von $I'\subset I$ ausw"ahlen, so dass
\[
\sum_{i\in I'} s_i=t.
\]
Dies sieht auf den ersten Blick aus wie des {\it SUBSET-SUM}-Problem,
beim genaueren Hinschauen erkennt man jedoch den Unterschied: In der
Liste der $s_i$ k"onnen einzelne Zahlen mehrfach vorkommen. Es k"onnte
sogar sein, dass die Menge $S$ "uberhaupt nur eine Zahl enth"alt,
zum Beispiel wenn dass Filmfestival allen Stars die gleiche Zahl von
Freikarten f"ur die Entourage abgegeben hat. Dann ist das erzeugte
{\it SUBSET-SUM}-Problem gar nicht mehr l"osbar, w"ahrend das urspr"ungliche
Problem genau dann l"osbar ist, wenn $t$ durch die immer gleiche Gr"osse
der Entourage teilbar ist. 

Eine allgemeinere Formulierung des Rucksack-Problems hingegen w"are
durchaus ein m"ogliches Vergleichsproblem, denn oben haben wir ja
eine Reduktion des Festival-Problems auf das Problem konstruiert,
f"ur eine Familie (nicht Menge!) $(s_i)_{i\in I}$ von Zahlen eine
Teilfamilie (nicht Teilmenge!) $(s_i)_{i\in I'}$ zu finden, so dass
die Summe der Zahlen einen bestimmten Wert $t=\sum_{i\in I'}s_i$
erreicht.
\end{diskussion}

\begin{bewertung}
L"osungsansatz mit Reduktion ({\bf R}) 1 Punkt,
Auswahl eines geeigneten Vergleichsproblems ({\bf V}) 1 Punkt.
Reduktionsabbildung f"ur Stars ({\bf S}) 1 Punkt,
f"ur die Entourage ({\bf E}) 1 Punkt,
f"ur die Zuordnung zu den beiden S"a"alen ({\bf A}) 1 Punkt,
f"ur die Aufteilung in zwei gleich grosse R"aume ({\bf P}) 1 Punkt,
maximal drei von den letztgenannten vier m"oglichen Punkten.
Schlussfolgerung, dass NP-vollst"andige Probleme nach aktuellem
Wissen nicht effizient gel"ost werden k"onnen ({\bf N}) 1 Punkt.
\end{bewertung}

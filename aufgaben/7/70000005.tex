Job-Planungs-Problem:
Gegeben sei eine Menge von $n$ Jobs $J_1,\dots,J_n$, welche zu ihrer
Ausf"uhrung einige der $m$ Resourcen $r_1,\dots,r_m$ f"ur sich reservieren
m"ussen. Jeder Task hat die gleiche Laufzeit $1$, jeweils zu ganzzahligen
Zeitpunkten k"onnen Jobs gestartet werden. Zeigen Sie, dass die
Aufgabe, die Jobs ohne Resourcenkonflikte so zu planen, dass innerhalb
gegebener Zeit $N$ alle Jobs abgearbeitet werden, ein NP-vollst"andiges Problem
ist.

\ifthenelse{\boolean{loesungen}}{
\begin{loesung}
Wir zeigen, dass das Problem "aquivalent ist zum Einf"arbe-Problem f"ur
einen Graphen. Wir m"ussen f"ur jeden Job ein Zeitinterval $I_k=[k,k+1]$
ausw"ahlen, w"ahrend dem er laufen soll. Dabei d"urfen zwei Jobs nicht
gleichzeitig laufen, wenn Sie auf die gleiche Resource angewiesen sind.
Wir konstruieren daher einen Graphen $G$ mit den Jobs als Ecken. Falls
zwei Jobs die gleiche Resource beanspruchen, f"ugen wir eine Kante
zwischen diesen beiden Jobs hinzu. Gesucht ist jetzt eine Zuteilung
von Intervallen $I_k$ mit $0\le k< N$ zu jeder Ecke, so dass keine
zwei benachbarten Ecken dem gleichen Interval zugeteilt sind. Bezeichnen
wir die Intervalle $I_k$ als ``Farben'', haben wir aus dem Job-Planungs-Problem
ein F"arbeproblem hergestellt.

Ist umgekehrt ein Graph und eine Menge von Farben vorgegeben, k"onnen
wir daraus wie folgt ein Planungs-Problem konstruieren. Die Ecken
des Graphen nennen wir Jobs, die Kanten nennen wir Resourcen. Die
mit einer Ecke inzidenten Kanten stellen die Resourcen dar, die ein
Job braucht, um laufen zu k"onnen. Der Graph kann mit $N$ Farben
eingef"arbt werden, wenn das Planungsproblem in Zeit $N$ l"osbar ist.

Die beiden Probleme sind somit "aquivalent. Da das F"arbeproblem
bekanntermassen NP-vollst"andig ist, ist auch das Job-Planungs-Problem
NP-vollst"andig.
\end{loesung}
}{}


Ein etwas unerfahrener Event-Manager hat sich f"ur seinen n"achsten
Event ein Spiel einfallen lassen, f"ur das er die G"aste in
Teams mit jeweils drei Mitgliedern einteilen muss. 
Damit die Bildung der Teams schnell vonstatten gehen kann, schickt
er den Teilnehmern die Team-Nummer bereits mit der Einladung.
Dummerweise l"asst die Datenqualit"at seiner Adressdatenbank
sehr zu w"unschen "ubrig. Die meisten Teilnehmer bekommen mehrere
Einladungen, mit verschiedenen Team-Nummern. Am Event kommt es zum Eklat.
Der Manager versucht die Situation noch zu retten, indem er den
Teilnehmern sagt, sie sollen die Dreierteams einfach mit einer
der Teamnummern bilden, mit welcher spiele keine Rolle. Die Teilnehmer
finden jedoch keine L"osung, der Event platzt, einige Teilnehmer 
gehen fr"uhzeitig nach Hause, andere ers"auffen ihren Frust an der Bar.
Warum ist es so schwierig, nach Anweisung des Managers Teams zu bilden?

\begin{loesung}
Dieses Problem, welches wir \textsl{TEAMBILDUNG} nennen wollen,
ist NP-vollst"andig, wie wir durch Vergleich mit dem bekanntermassen
NP-vollst"andigen Problem \textsl{EXACT-COVER} nachweisen wollen.

Je der vom Manager verschickten Nummern $j$ definiert eine Teilmenge
$S_j$ der Menge aller Teilnehmer $U$. Gesucht ist eine Unterfamilie
Teams $S_{j_i}$, $1\le i\le m$, so dass keine zwei Teams sich "uberschneiden
(d.~h.~kein Teilnehmer ist in mehr als einem Team) und die ganze Menge $U$
"uberdecken (d.~h.~jeder Teilnehmer ist in einem Team).

\medskip
Reduktion auf \textsl{HITTING-SET}: Dazu bildet man f"ur jeden
Teilnehmer $i$ die Menge $S_i$ der Teams, f"ur die Teilnehmer $i$
vorgesehen ist. Es ist klar, dass $S_i\subset S$, wobei $S$ die Menge
aller Teams ist. Gesucht ist jetzt eine Auswahl von Teams so, dass
jeder Teilnehmer in genau einem Team ist, also $H\subset S$ so,
dass $|H\cap S_i|=1$. Das ist das Problem \textsl{HITTING-SET}.

\medskip
Es ist in diesem Fall auch m"oglich, eine Reduktion auf SAT anzugeben.
Dazu muss man beachten, dass eine Auswahl der Teams, die auf den
Einladungen definiert wurden, ja tats"achlich eine L"osung des
Problems darstellen w"urden, man muss nur noch herausfinden, welche
Teams gebildet werden m"ussen, und welche nicht. Seien also $i\in I$
die Teamnummern, und sei f"ur jedes $i$ eine boolsche Variable $x_i$
gegeben, welche angibt, ob ein Team gebildet werden soll oder nicht.
Jetzt muss aus den Daten nur noch eine Formel $\varphi$ konstruiert werden,
welche angibt, ob die Belegung der $x_i$ mit Wahrheitswerten, also die Auswahl
der Teams, eine L"osung des Problems darstellt.

Zu jedem Teilnehmer $j\in T$ k"onnen wir die Megen $I_j$ der Nummern
der Teams bilden, f"ur die er vorgesehen ist.

Die Formel $\varphi$ besteht aus mehreren Teilen, welche f"ur jeden
Teilnehmer $j$ ausdr"ucken sollen, dass 
\begin{enumerate}
\item er in h"ochstens einem der realisierten Teams ist (Formel $\varphi_{1i}$),
und dass
\item er in mindestens einem der realisierten Teams ist (Formel $\varphi_{2i}$).
\end{enumerate}
Die gesucht Formel ist dann
$\bigwedge_{j\in T} \varphi_{1i}\wedge \varphi_{2i}$.

Die Formel $\varphi_{2j}$ dr"uckt aus, dass mindestens ein Team realisiert
wird, f"ur das der Teilnehmer $j$ vorgesehen ist:
\begin{equation}
\varphi_{2j}=x_{i_1}\vee x_{i_2}\vee\dots\vee x_{i_k} = \bigvee_{i\in I_j}x_i.
\label{70000025:1}
\end{equation}

Die Formel
\begin{equation}
x_{i_1}\wedge\overline{(x_{i_2}\vee\dots\vee x_{i_k})}
\label{70000025:2}
\end{equation}
ist genau dann war, wenn Team $i_1$ das einzige Team von $j$ ist,
welches realisiert wird. Die Formel ist falsch, wenn ausser $i_1$ auch
noch ein anderes Team realisiert wird. Sie ist aber auch falsch, wenn 
gar keines der Teams realisiert wird.

Entsprechend kann man eine Formel bilden, die dann war wird, wenn $i_2$ als 
einziges Team realisiert wird. Oder allgemein ist
die Formel
\begin{equation}
x_i\wedge \overline{\biggl(\bigvee_{i'\in I_j\setminus\{i\}}x_{i'}\biggr)}
\label{70000025:3}
\end{equation}
wahr, wen Team $i\in I_j$ als einziges der Teams von $j$ realisiert wird.
Damit eine L"osung gefunden wird, muss irgend eines der Teams von $j$ als
einziges realisiert werden,
die gesuchte Formel $\varphi_{1j}$ ist daher
\begin{equation}
\varphi_{1j} = 
\bigvee_{i\in I_j} \biggl(x_i\wedge
	\overline{\biggl(\bigvee_{i'\in I_j\setminus \{i\}}x_{i'}\biggr)}\biggr)
\label{70000025:4}
\end{equation}
Damit kann man jetzt auch die Gesamtformel $\varphi$ zusammensetzen:
\begin{equation}
\varphi=
\bigwedge_{j\in T}
\biggl(
\bigvee_{i\in I_j} \biggl(x_i\wedge
	\overline{\biggl(\bigvee_{i'\in I_j\setminus \{i\}}x_{i'}\biggr)}\biggr)
\wedge
\biggl(
\bigvee_{i\in I_j}x_i
\biggr)
\biggr)
\label{70000025:5}
\end{equation}
Damit ist zwar eine Reduktion von \textsl{TEAMBILDUNG} auf \textsl{SAT}
geschafft, es ist aber noch nicht gezeigt, dass dies auch eine polynomielle
Reduktion ist. Sei $n$ die Anzahl der Teilnehmer und $m$ die Anzahl der
Teams, die gebildet werden sollten. In unserem Beispiel gilt "urlich $n=3m$,
aber obige Reduktion ist auch m"oglich wenn die Team-Gr"osse nicht $3$
ist.

Jede der Formeln (\ref{70000025:1}) hat L"ange $O(m)$.
Da es in der Schlussformel $O(n)$ Teilformeln dieser Art gibt, 
tragen diese mit L"ange $O(nm)$ zur L"ange von $\varphi$ bei.

Die Formel (\ref{70000025:3}) hat L"ange $O(m)$, davon werden $|I_j|=O(n)$
St"uck
in (\ref{70000025:4}) zu einer Formel der L"ange $O(mn)$ zusammengesetzt.
In (\ref{70000025:5}) kommt eine solche Formel f"ur jeden Teilnehmer vor,
die Gesamtl"ange von $\varphi$ ist $O(n^2m) + O(nm)=O(n^2m)$, also auf
jeden Fall polynomiell in der Gr"osse des Problems. Damit ist gezeigt,
dass die Reduktion auf \textsl{SAT} auch eine polynomielle Reduktion ist.
\end{loesung}

\begin{diskussion}
Bei oberfl"achlicher Betrachtung scheint \textsl{TEAMBILDUNG} auch
zum Problem \textsl{3D-MATCHING} "aquivalent zu sein. Da alle NP-vollst"andigen
Problem "aquivalent sind, ist das nat"urlich immer richtig, aber
es ist nicht so einfach, eine "Aquivalenz zwischen 
\textsl{TEAMBILDUNG} und 
\textsl{3D-MATCHING} zu beschreiben.

Im Problem \textsl{3D-MATCHING} geht es um eine endliche Menge $T$,
die den Teilnehmern im Problem \textsl{TEAMBILDUNG} entspricht.
Jede Team-Nummer, die der Manager verschickt hat, erzeugt ein Tripel
von jeweils drei Teilnehmern, also eine Menge $U$ von Tripeln,
$U\subset T\times T\times T$. Die Aufgabe der Teilnehmer besteht
jetzt darin, nur einen Teil der Team-Nummern zu verwenden, um
die Teams zu bilden, also eine Auswahl von Tripeln $W$ aus der Menge $U$
zu treffen. Die Bedingungen sind jedoch nicht exakt die gleichen.
In \textsl{3D-MATCHING} m"ussen gleich viele Tripel $|W|$
gew"ahlt werden, wie es Teilnehmer $|T|$ gibt, in 
\textsl{TEAMBILDUNG} werden nur $\frac13$ so viele Teams gebildet
wie es Teilnehmer gibt. \textsl{3D-MATCHING} verlangt auch nicht,
dass sich die Tripel nicht "uberschneiden d"urfen, sie d"urfen sich
nur in keiner Koordinate "uberschneiden.
\end{diskussion}

\begin{bewertung}
L"osungsprinzip Reduktion ({\bf R}) 1 Punkt,
Auswahl eines geeigneten Vergleichsproblems ({\bf V}) 1 Punkt,
Reduktionsabbildung f"ur G"aste ({\bf G}) 1 Punkt,
Teams ({\bf T}) 1 Punkt,
Ausschlussbedingung (jeder Teilnehmer in genau einem realisierten Team,
{\bf B}) 1 Punkt,
Schlussfolgerung NP-vollst"andig ({\bf S}) 1 Punkt.
\end{bewertung}

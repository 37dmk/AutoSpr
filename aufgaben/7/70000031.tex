Eine Firma konnte in einer feindlichen "Ubernahme einen wichtigen Konkurrenten
aufkaufen. Der Konkurrent ist eigentlich eine Vereinigung von sehr vielen,
sehr intensiv und effizient zusammenarbeitenden, aber im "Ubrigen weitgehend
selbst"andigen Abteilungen, alle unter dem selben Dach.
Das Ziel der "Ubernahme war, daraus die ``Rosinen'' herauszupicken, und
den Rest zu schliessen. Die Wettbewerbsbeh"orden hatten dies vorausgesehen,
und als Auflage f"ur die "Ubernahme gemacht, dass keine einzige der Abteilungen
geschlossen werden d"urfe.
Daher dachten sich die neuen Eigent"umer den folgenden b"osartigen Plan aus,
um den gleichen Zweck zu erreichen. Sie teilten die Abteilungen auf zwei 
verschiedene Standorte auf, und sorgten daf"ur, dass jede Kommunikation zwischen
den beiden Standorten so ineffizient wurde, dass die Abteilungen kaum mehr
sinnvoll zusammenarbeiten konnten.
Dadurch w"urden die einzelnen Abteilungen wirtschaftlich ruiniert, und man
m"usste sie trotz allem schliessen.
Die neuen Eigent"umer beauftragten daher eine Beratungsfirma, eine
Aufteilung zu finden, mit der die Kommunikation zwischen den Abteilungen
m"oglichst stark behindert w"urde.
Die Beratungsfirma brauchte daf"ur sehr lange. Warum ist das nicht
"uberraschend?

\begin{loesung}
Es handelt sich hier um das Problem \textsl{MAX-CUT}.
Wir beschreiben eine Reduktion des Problems auf \textsl{MAX-CUT}:
\begin{align*}
\text{Abteilung}&\leftrightarrow \text{Vertex} \\
\text{Kommunikationsbeziehung}&\leftrightarrow \text{Kante} \\
\text{Kommunikationsvolumen}&\leftrightarrow \text{Gewicht einer Kante}
\end{align*}
Die neuen Firmeneigent"umer wollen die Menge der Vertices so in zwei
Mengen aufteilen, dass die Summe der Gewichte der Kanten, die durch die
Aufteilung zerschnitten werden, m"oglichst
gross wird. Dies ist genau die Beschreibung des Problems \textsl{MAX-CUT}.

Das Problem \textsl{MAX-CUT} ist NP-vollst"andig, nach aktuellem Wissen
gibt es also keinen effizienten (polynomiellen) Algorithmus, der ein
\textsl{MAX-CUT} Problem l"osen k"onnte.
\end{loesung}

\begin{bewertung}
Reduktionsansatz ({\bf R}) 1 Punkt,
Vergleichsproblem ({\bf M}) 1 Punkt,
Reduktionsabbildung ({\bf A}) je ein Punkt, maximal 3 Punkte,
Schlussfolgerung NP-Vollst"andigkeit ({\bf N}) 1 Punkt.
\end{bewertung}


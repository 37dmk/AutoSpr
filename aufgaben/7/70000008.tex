Auf einem weit entfernten, aber technologisch sehr fortgeschrittenen
Planeten wird der den Planeten regierende "Altestenrat
nicht durch Volkswahl bestimmt, sondern durch einen Algorithmus.
In der Vergangenheit gab es eine grosse Zahl rivalisierender
St"amme, die sich auch zu gr"osseren ``Nationen'' verb"undet haben
und wiederholt blutige Kriege gegeneinander gef"uhrt haben.
Um diesem Unheil ein Ende zu setzen, kamen die St"amme "uberein,
den "Altestenrat zu bilden, in dem wenn auch nicht jeder Stamm direkt,
so doch mindestens das ``Blut'' jedes Stammes vertreten sein musste.
Damit trug man der Tatsache Rechnung, dass Mobilt"at und planetenweiter
Handel mittlerweile zu einer weitgehenden Durchmischung in der
Bev"olkerung gef"uhrt hatte. Ausgepr"agtes Traditionsbewusstsein hatten
jedoch sichergestellt, dass von jedem B"urger bekannt war, aus welchen
St"ammen er ``Blut'' in sich trug. Warum ist trotzdem eine weit entwickelte
Technologie n"otig, um herauszufinden, ob sich ein "Altestenrat
mit $k$ Mitgliedern "uberhaupt bilden l"asst?

\ifthenelse{\boolean{loesungen}}{
\begin{loesung}
Das Problem, einen "Altestenrat mit $k$  Mitgliedern zu bilden, ist
"aquivalent zum Set-Covering Problem. Zu jedem B"urger $i$
ist die Menge $S_i$ aller St"amme bekannt, von denen er ``Blut''
in sich tr"agt. Die Vereinigung aller $S_i$ ist die Menge $U$ aller
St"amme. Gefragt wird jetzt nach $k$ B"urgern $i_1,\dots,i_k$ so,
dass die $S_{i_1},\dots S_{i_k}$ bereits alle St"amme abdecken, also
\[
\bigcup_i S_i=U.
\]
Die "Ubersetzung von Set-Covering auf des "Altestenratsproblem ist also
\begin{align*}
U&\leftrightarrow\text{Menge aller St"amme}\\
i&\leftrightarrow\text{B"urger}\\
S_i&\leftrightarrow\text{Menge der St"amme, die durch $i$ vertreten werden k"onnen}\\
\text{Unterfamilie}&\leftrightarrow\text{"Altestenrat}
\end{align*}
Da Set-Covering ein NP-vollst"andiges Problem ist, ist es mit unserer
Technologie nach allem, was wir wissen, nicht effizient l"osbar. Der
Planet braucht daher eine speziell fortgeschrittene Technologie, um es
zuverl"assig l"osen zu k"onnen.
\end{loesung}
}{ }


Student Xaver Tecco soll im Rahmen einer Big-Data-Studienarbeit die 
Kunden einer grossen Shop-Website untersuchen und klassifizieren.
Es steht eine grosse Zahl von bin"aren Eigenschaften zur Verf"ugung,
zum Beispiel ob Kunden ein bestimmtes Produkt gekauft haben, oder
ob ein Kunde nur im Dezember einkauft.
Herr Tecco soll herausfinden, ob es eine Teilmenge von Kriterien derart
gibt, dass jeder Kunde genau eine der Eigenschaften hat.
Die Abgabe der Arbeit steht in zwei Tagen bevor, und er hat noch keinen
funktionierenden Algorithmus.
Muss er sich Sorgen machen?

\begin{loesung}
Dieses Problem ist "aquivalent mit dem bekanntermassen NP-vollst"andigen
Problem \textsl{EXACT-COVER}:
\begin{align*}
\text{Eigenschaft}&\leftrightarrow \text{Menge $S_j$}\\
\text{Teilmenge von Eigenschaften}&\leftrightarrow \text{Unterfamilie $S_{j_i}$}\\
\text{genau eine der Eigenschaften}&\leftrightarrow S_{j_i}\cap S_{j_k}=\emptyset\quad\forall i\ne k\\
\text{alle Kunden erfasst}&\leftrightarrow \bigcup_{j=1} nS_j=\bigcup_{i=1}^m S_{j_i}
\end{align*}
Da nach heutigem Wissen NP-vollst"andige Problem nicht mit einem polynomiellen
Algorithmus gel"ost werden k"onnen, ist nicht mit einer schnellen L"osung
des Problems zu rechnen.
\end{loesung}

\begin{diskussion}
Der Name \textsl{Xaver Tecco} ist eine Anagramm von \textsl{EXACT-COVER}.
\end{diskussion}



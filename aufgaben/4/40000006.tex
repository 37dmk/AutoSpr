Konstruieren Sie einen Stackautomaten, der die Sprache
\[
L=\{{\tt a}^i{\tt b}^j| i\ne j\}
\]
akzeptiert.

\begin{loesung}
Der PDA funktioniert nach dem gleichen Prinzip wie der in der Vorlesung
besprochene PDA, der die Sprache $\{{\tt 0}^n{\tt 1}^n|n\ge 0\}$
akzeptiert.

Wenn $i>j$ ist, dann ist es am Ende des Wortes noch m"oglich, mindestens
ein weiteres {\tt a} vom Sack zu lesen, bevor das Stackende-Zeichen
gefunden wird.

Wenn $i<j$ ist, dann kann nach dem erreichen des Stackende-Zeichens
noch ein {\tt b} vom Input gelesen werden, und weitere {\tt b}
k"onnen gelesen werden, ohne dass der Stack modifiziert wird.
\[
\entrymodifiers={++[o][F]}
\xymatrix @+5mm{
*+\txt{}
        &*+\txt{}
                &*+\txt{}
                        &*+\txt{}
                                &{q_3}\ar@(ur,ul)_{\varepsilon, {\tt a}\to\varepsilon}
                                 \ar[r]^{\varepsilon,{\tt \$}\to\varepsilon}
                                        &*++[o][F=]{q_4}
\\
*+\txt{}\ar[r]
        &{q_0}\ar[r]^{\varepsilon,\varepsilon\to{\tt \$}}
                &{q_1}\ar@(ur,ul)_{{\tt a},\varepsilon\to{\tt a}}
                   \ar[r]^{\varepsilon,\varepsilon\to\varepsilon}
                        &{q_2}\ar@(ur,ul)_{{\tt b},{\tt a}\to\varepsilon}
                           \ar[ur]_{\varepsilon,{\tt a}\to\varepsilon}
                           \ar[dr]^{{\tt b},{\tt \$}\to\varepsilon}
\\
*+\txt{}
        &*+\txt{}
                &*+\txt{}
                        &*+\txt{}
                                &{q_5}\ar@(dr,dl)^{{\tt b},\varepsilon\to\varepsilon}
                                       \ar[r]^{\varepsilon,\varepsilon\to\varepsilon}
                                        &*++[o][F=]{q_6}
}
\]

Alternativ k"onnte man den Stackautomaten dadurch finden, dass man
zuerst eine Grammatik konstruiert, und die dann mit Hilfe des Algorithmus
aus der Vorlesung in einen Stackautomaten umwandelt. Die Grammatik
muss W"orter mit einem "Uberschuss an {\tt a} oder {\tt b}
produzieren, wir schreiben die Variable $A$ f"ur W"orter mit einem
"Uberschuss an {\tt a} und $B$ f"ur W"orter mit einem "Uberschuss an
{\tt b}. W"orter mit einem "Uberschuss an {\tt a} entstehen, in dem man
solche W"orter immer im {\tt a} links wachsen l"asst oder wenn man
ein {\tt b} rechts anf"ugt auch ein {\tt links} hinzuf"ugt. Die Grammatik
wird damit
\begin{align*}
S&\to A|B\\
A&\to {\tt a}A\\
 &\to {\tt a}A{\tt b}\\
 &\to {\tt a}\\
B&\to B{\tt b}\\
 &\to {\tt a}B{\tt b}\\
 &\to {\tt b}\\
\end{align*}
Die Konstruktion aus der Vorlesung liefert jetzt zun"achst das
Ger"ust des Automaten
\[
\entrymodifiers={++[o][F]}
\xymatrix{
*+\txt{}\ar[r]
        &{}\ar[r]^{\varepsilon,\varepsilon\to{\tt \$}}
                &{}\ar[r]^{\varepsilon,\varepsilon\to S}
                        &R\ar[r]^{\varepsilon,{\tt\$}\to\varepsilon}
                                &*++[o][F=]{}
}
\]
dem Zustand $R$ f"ugt man jetzt entsprechend der Regeln folgende
"Uberg"ange hinzu.
F"ur die Regeln mit $A$:
\[
\entrymodifiers={++[o][F]}
\xymatrix{
*+\txt{}
        &\ar@/^/[d]^{\varepsilon,\varepsilon\to {\tt a}}
\\
*+\txt{}
        &{R}    \ar[r]^{\varepsilon,A\to{\tt b}}
                \ar@/^/[u]^{\varepsilon,A\to A}
                \ar@(d,dl)^{\varepsilon,A\to{\tt a}}
                \ar@(l,dl)_{\varepsilon,S\to A}
                &{}     \ar[r]^{\varepsilon,\varepsilon\to A}
                        &{}     \ar@/^10pt/[ll]^{\varepsilon,\varepsilon\to{\tt a}}
}
\]
Und f"ur die Regeln mit $B$:
\[
\entrymodifiers={++[o][F]}
\xymatrix{
*+\txt{}
        &\ar@/^/[d]^{\varepsilon,\varepsilon\to B}
\\
*+\txt{}
        &{R}    \ar[r]^{\varepsilon,B\to{\tt b}}
                \ar@/^/[u]^{\varepsilon,B\to {\tt b}}
                \ar@(d,dl)^{\varepsilon,B\to{\tt b}}
                \ar@(l,dl)_{\varepsilon,S\to B}
                &{}     \ar[r]^{\varepsilon,\varepsilon\to B}
                        &{}     \ar@/^10pt/[ll]^{\varepsilon,\varepsilon\to{\tt a}}
}
\]
Zus"atzlich braucht es noch die Regeln, die zur Verarbeitung
der Eingabezeichen n"otig sind:
\[
\entrymodifiers={++[o][F]}
\xymatrix{
{R}     \ar@(ul,dl)_{{\tt a},{\tt a}\to\varepsilon}
        \ar@(ur,dr)^{{\tt b},{\tt b}\to\varepsilon}
}
\]
\end{loesung}

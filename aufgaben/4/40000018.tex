Die Sprache Brainfuck besteht aus Strings der folgenden Zeichen:
\begin{center}
\begin{tabular}{|c|l|}
\hline
Brainfuck&C-"Aquivalent\\
\hline
{\tt >}&\verb/++ptr;/\\
{\tt <}&\verb/--ptr;/\\
{\tt +}&\verb/++*ptr;/\\
{\tt -}&\verb/--*ptr;/\\
{\tt .}&\verb/putchar(*ptr);/\\
{\tt ,}&\verb/*ptr = getchar();/\\
{\tt [}&\verb/while (*ptr) {/\\
{\tt ]}&\verb/}/\\
\hline
\end{tabular}
\end{center}
Beschreiben sie einen Stackautomaten, der korrekte Brainfuck-Programme
erkennt.

\begin{loesung}
Der Stackautomat muss ausschliesslich pr"ufen, dass jede "offnene
eckigen Klammer auch wieder geschlossen wird. Das Stackalphabet verwendet
zus"atzlich zu den Zeichen von Brainfuck das Zeichen {\tt\$}, welches
dabei hilft, zu erkennen, ob der Stack leer ist. Dann erkennt der
folgende Automat die die korrekten Brainfuck Programme:
\[
\entrymodifiers={++[o][F]}
\xymatrix{
*+\txt{}\ar[r]
        &q_0\ar[r]^{\varepsilon,\varepsilon\to{\tt\$}}
                &q_1
                        \ar@(ul,ur)^{.,\varepsilon\to\varepsilon}
                        \ar@(d,dr)_{{\tt [},\varepsilon\to{\tt [}}
                        \ar@(dl,d)_{{\tt ]},{\tt [}\to\varepsilon}
                        \ar[r]^{\varepsilon,{\tt\$}\to\varepsilon}
                        &*++[o][F=]{q_3}
}
\]
Im "Ubergang $.,\varepsilon\to\varepsilon$ steht der Punkt f"ur die
Zeichen
`{\tt >}',
`{\tt <}',
`{\tt +}',
`{\tt -}',
`{\tt .}' und
`{\tt ,}'.
\end{loesung}

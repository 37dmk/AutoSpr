Finden Sie eine Grammatik in Chomsky-Normalform f"ur die Sprache
\[
L=\{
\texttt{a}^n\texttt{b}^m\texttt{a}^n
\;|\;
n\ge 0, m\ge 1
\}.
\]

\begin{loesung}
Zun"achst stellen wir eine Grammatik auf.
W"orter in $L$ bestehen aus einem Block von Zeichen $\texttt{b}$,
welcher mit der Grammatik
\begin{align*}
R&\rightarrow \texttt{b} \\
 &\rightarrow R\texttt{b}
\end{align*}
Dieser Block kann aber noch zus"atzlich mit beliebig vielen 
\texttt{a} eingeklammert werden, was durch die Grammatik
\begin{align*}
S&\rightarrow R \\
 &\rightarrow \texttt{a} S \texttt{a}\\
R&\rightarrow \texttt{b} \\
 &\rightarrow R\texttt{b}
\end{align*}
ausgedr"uckt werden kann.

Diese Grammatik hat nicht Chomsky-Normalform: sie enth"alt zum Beispiel die
Unit-Rule $S\rightarrow R$.
Wir wenden den Standardalgorithmus an:
\begin{enumerate}
\item Startvariable: 
\begin{align*}
S_0&\rightarrow S                      \\
S  &\rightarrow R  \;|\; \texttt{a} S \texttt{a}\\
R  &\rightarrow \texttt{b} \;|\; R\texttt{b}
\end{align*}
\item $\varepsilon$-Regeln: es gibt keine $\varepsilon$-Regeln, damit
entf"allt dieser Schritt.
\item Unit-Rules: Wir beginnen mit der Regel $S_0\rightarrow S$:
\begin{align*}
S_0&\rightarrow \color{red} R  \;|\; \texttt{a} S \texttt{a}\\
S  &\rightarrow R  \;|\; \texttt{a} S \texttt{a}\\
R  &\rightarrow \texttt{b} \;|\; R\texttt{b}
\end{align*}
Regel $S_0\rightarrow R$:
\begin{align*}
S_0&\rightarrow \texttt{a} S \texttt{a} \;|\; \color{red}\texttt{b} \;|\; R\texttt{b} \\
S  &\rightarrow R  \;|\; \texttt{a} S \texttt{a}\\
R  &\rightarrow \texttt{b} \;|\; R\texttt{b}
\end{align*}
Regel $S\rightarrow R$:
\begin{align*}
S_0&\rightarrow \texttt{a} S \texttt{a} \;|\; \texttt{b} \;|\; R\texttt{b} \\
S  &\rightarrow \texttt{a} S \texttt{a} \;|\; \color{red} \texttt{b} \;|\; R\texttt{b}\\
R  &\rightarrow \texttt{b} \;|\; R\texttt{b}
\end{align*}
Damit sind alle Unit-Rules entfernt.
\item Verkettungsregeln: Zun"achst f"ugen wir zwei Regeln f"ur die
Terminalsymbole \texttt{a} und \texttt{b} hinzu:
\begin{align*}
S_0&\rightarrow {\color{red}A}S{\color{red}A} \;|\; \texttt{b} \;|\; R{\color{red}B} \\
S  &\rightarrow {\color{red}A}S{\color{red}A} \;|\; \texttt{b} \;|\; R{\color{red}B} \\
R  &\rightarrow \texttt{b} \;|\; R{\color{red}B} \\
\color{red}A  &\color{red}\rightarrow \texttt{a} \\
\color{red}B  &\color{red}\rightarrow \texttt{b}
\end{align*}
Damit entsprechen jetzt nur noch die Regeln $S\rightarrow ASA$ und
$S_0\rightarrow ASA$ nicht der Chomsky-Normalform.
Wir k"onnen das aber durch eine neue Variable $U$ erreichen:
\begin{align*}
S_0&\rightarrow {\color{red}U}A \;|\; \texttt{b} \;|\; RB \\
S  &\rightarrow {\color{red}U}A \;|\; \texttt{b} \;|\; RB \\
R  &\rightarrow \texttt{b} \;|\; RB \\
A  &\rightarrow \texttt{a} \\
B  &\rightarrow \texttt{b} \\
\color{red}U  &\color{red}\rightarrow AS
\end{align*}
In dieser Grammatik entsprechen alle Regeln den Anforderungen der
Chomsky Normalform.
\qedhere
\end{enumerate}
\end{loesung}

\begin{bewertung}
Grammatikteil $R$ ({\bf R}) 1 Punkt,
Grammatikteil $S$ ({\bf S}) 1 Punkt,
CNF Startvariable ({\bf S0}) 1 Punkt,
CNF Unit-Rules ({\bf U}) 1 Punkt,
CNF Terminalsymbolregeln ({\bf T}) 1 Punkt,
CNF Verkettungsregel ({\bf V}) 1 Punkt.
\end{bewertung}


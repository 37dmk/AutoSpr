Zeigen Sie, dass die folgenden Sprachen nicht kontextfrei sind:
\begin{teilaufgaben}
\item $L=\{{\tt 0}^n{\tt 1}^n{\tt 0}^n{\tt 1}^n|n\ge 0\}$
\item $L=\{w{\tt \#}w|w\in\{{\tt 0},{\tt 1}\}^*\}$
\end{teilaufgaben}

\begin{loesung}
\begin{teilaufgaben}
\item Sei $N$ die Pumping length.
Bilde das Wort $w={\tt 0}^N{\tt 1}^N{\tt 0}^N{\tt 1}^N$. Das Pumping Lemma
behauptet, dass es eine Unterteilung des Wortes $w=uvxyz$ gibt, wir
m"ussen zeigen, dass jede m"oglich Unterteilung, die allen vom
Pumping Lemma garantierten Rahmenbedingungen gen"ugt, bei Aufpumpen
auf ein nicht zu $L$ geh"orendes Wort f"uhrt. Im vorliegenden Fall
gibt es zwei wesentlich verschieden F"alle.

Im ersten Fall sind die
Teilw"orter $v$ und $y$ jeweils vollst"andig ein einem der aus Nullen
oder Einsen bestehenden Bl"ocke enthalten. Durch aufpumpen nimmt de
Anzahl der gleichartigen Stellen in diesem Block zu, der andere Block
mit gleichen Ziffern wird jedoch nicht ver"andert. Die Eigenschaft,
dass die Nuller- und
Einer-Bl"ocke jeweils gleich viele Stellen enthalten m"ussen, wir
beim Aufpumpen zerst"ort.

Im zweiten Fall enth"alt einer der Teile $v$ oder $y$ eine
Grenze zwischen Nullen und Einsen, also einen Wechsel von
{\tt 0} auf {\tt 1} oder umgekehrt. Ein Wort in $L$ kann
genau zwei Wechsel von {\tt 0} auf {\tt 1} und genau einen
von {\tt 1} auf {\tt 0} enthalten. Durch Aufpumpen des Teils,
der einen Wechsel enth"alt, wird die Anzahl der Wechsel erh"oht,
so dass das aufgepumpte Wort nicht mehr zu $L$ geh"ort.

Das Pumping Lemma behauptet aber, dass auch die aufgepumpten W"orter
zur Sprache geh"oren muss. Dieser Widerspruch zeigt, dass $L$
nicht kontextfrei sein kann.
\item Sei $N$ die Pumping length. Wie in a) verwenden wir als
Wort ${\tt 0}^N{\tt 1}^N{\tt\#}{\tt 0}^N{\tt 1}^N$. Bei der Unterteilung
gem"ass den Restriktionen des Pumping Lemma gibt es genau einen zus"atzlichen
m"oglichen Fall, n"amlich den, dass $v$ oder $y$ das Zeichen {\tt\#}
enth"alt. In diesem Fall wird durch Aufpumpen die Zahl der {\tt\#}
erh"oht, ein Wort in $L$ darf jedoch nur ein einziges {\tt\#}-Zeichen
enthalten. Diese Widerspruch zeigt, dass $L$ nicht kontextfrei sein kann.
\qedhere
\end{teilaufgaben}
\end{loesung}

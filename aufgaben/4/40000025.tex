F"ur arithmetische Ausdr"ucke mit den Grundoperationen
{\tt +} und {\tt -} stehen die folgenden zwei Grammatiken
zur Auswahl. Einerseits eine leicht abgewandelte Variante
der in der Vorlesung besprochenen Grammatik:
\begin{align*}
\text{Ausdruck}&\rightarrow \text{Zahl}\;\text{'{\tt +}'}\;\text{Ausdruck} \\
               &\rightarrow \text{Zahl}\;\text{'{\tt -}'}\;\text{Ausdruck} \\
               &\rightarrow \text{Zahl}
\end{align*}
Andererseits die folgende Grammatik:
\begin{align*}
\text{Ausdruck}&\rightarrow \text{Term}\;\text{Termsequenz}\\
\text{Termsequenz}&\rightarrow \text{'{\tt +}'}\; \text{Term}\; \text{Termsequenz}\\
                  &\rightarrow \text{'{\tt -}'}\; \text{Term}\; \text{Termsequenz}\\
                  &\rightarrow \varepsilon\\
 \text{Term}&\rightarrow \text{Zahl}
\end{align*}
Die beiden Grammatiken verwenden nat"urlich die gleichen Regeln
f"ur das nichtterminale Symbol ``Zahl''.
\begin{teilaufgaben}
\item Erstellen Sie f"ur jede Grammatik den parse tree f"ur den Ausdruck
\[
47-1291+1848
\]
\item Welche Grammatik ist vorzuziehen?
\end{teilaufgaben}

\begin{loesung}
\begin{teilaufgaben}
\item Die erste Grammatik erzeugt
\[
\xymatrix @-1mm {
           &\text{Ausdruck}\ar[dl]\ar[d]\ar[drr]&               &    \\
\text{Zahl}\ar[dd]&'{\tt +}'\ar[dd]      &&\text{Ausdruck}\ar[dl]\ar[dr]\ar[d]&    \\
           &&\text{Zahl}\ar[d]    &'{\tt -}'\ar[d]      &\text{Zahl}\ar[d]\\
47&-&1291&+&1848
}
\]
Bei Verwendung der zweiten Grammatik bekommt man jedoch
\[
\xymatrix @-1mm {
        &\text{Ausdruck}\ar[dl]\ar[dr]
                &       &       &       &                       \\
\text{Term}\ar[dddd]
        &       &\text{Termsequenz}\ar[dl]\ar[d]\ar[drr]
                        &       &       &                       \\
        &\text{'{\tt -}'}\ar[ddd]
                &\text{Term}\ar[ddd]
                        &       &\text{Termsequenz}\ar[dl]\ar[d]\ar[dr]
                                        &                       \\
        &       &       &\text{'{\tt +}'}\ar[dd]
                                &\text{Term}\ar[dd]
                                        &\text{Termsequenz}\ar[d]\\
        &       &       &       &       &\varepsilon            \\
47      &-      &1291   &+      &1848   &                
}
\]

\item Wird der Parse Tree zur Berechnung des Ausdrucks verwendet,
ergibt sich f"ur die erste Grammatik ein falsches Resultat. In
diesem Fall wird n"amlich zuerst die Summe $1291+1848=
3139
$
gebildet, die anschliessend von $47$ subtrahiert, was
$
47-(1291+1848)
=
-3092
$
ergibt.

Das richtige Resultat erh"alt man jedoch mit dem Parse
Tree nach der zweiten Grammatik.  Dieser verlangt, zuerst
die Differenz $47-1291 =
-1244
$
zu bilden, und dann $1848$ zu addieren, was
$
604
$
ergibt. Daher ist die zweite Grammatik vorzuziehen.
\end{teilaufgaben}
\end{loesung}

Der Sprachforscher Stuart M.~Shieber hat aus Beispielen wie dem Satz
\begin{center}
De Jan s"ait das mer d'chind em Hans es hus h"and wele laa h"alfe aastriiche.
\end{center}
abgeleitet, dass das Schweizderdeutsch grammatische Konstruktionen
zul"asst, die auf W"orter der Form
\[
wa^mb^nxc^md^ny
\]
hinaus laufen\footnote{Stuart M.~Shieber,
{\em Evidence against the context-freeness of natural language},
Linguistics and Philosophy, {\bf 8} (1985) 333--343}.
Zeigen Sie, dass diese Sprache nicht kontextfrei ist.

\begin{loesung}
Wir wenden das Pumping Lemma f"ur kontextfreie Sprachen auf die Sprache
\[
L=\{
wa^mb^nxc^md^ny
\;|\; m,n\ge 0
\}
\]
an.
\begin{enumerate}
\item Annahme: $L$ ist kontextfrei.
\item Nach dem Pumping-Lemma gibt es die Pumping-Length $N$.
\item Wir konstruieren das Beispielwort
\[
z=wa^Nb^Nx^Nd^N,
\]
es ist offensichtlich in der Sprache $L$
und es ist ausreichend lang, dass die Schlussfolgerungen des Pumping-Lemma
darauf anwendbar sind.
\item
Gem"ass Pumping Lemma gibt es eine Unterteilung
\[
z=pqrst,
\]
wobei $|qrs|\le N$ gelten muss.
Es folgt, dass $q$ und $s$ nur jeweils zwei benachbarte der
vier ``langen'' Bl"ocke $a^N$, $b^N$, $c^N$ oder $d^N$ ber"uhren k"onnen.
\item 
Beim Aufpumpen werden zwei benachbarte langen Bl"ocke ver"andert.
Die Bedingung, dass alternierende Bl"ocke, also $a^N$ und $c^N$
bzw.~$b^N$ und $d^N$ gleich lang sein m"ussen, wird nach dem Pumpen
daher nicht mehr erf"ullt sein.
Ein aufgepumptes Wort wird daher nicht mehr in $L$ sein.
\item Dieser Widerspruch zeigt, dass die Annahme, $L$ sei kontextfrei,
nicht haltbar ist.
\end{enumerate}
\end{loesung}

\begin{diskussion}
Dies bedeutet nat"urlich nicht, dass Schweizerdeutsch eine nicht
kontextfreie Sprache ist.
In der Wirklichkeit sind die S"atze n"amlich immer von endlicher L"ange,
die Exponenten $n$ und $m$ k"onnen daher nicht beliebig gross sein.
Das m"ussen Sie aber, denn der Pumping-Lemma-Beweis verlangt, dass man
$n=m=N$ setze k"onnen muss, wobei die Pumping-Length $N$ eben sehr gross
sein kann.
\end{diskussion}

\begin{bewertung}
F"ur jeden Schritt im Pumping-Lemma-Beweis 1 Punkt.
\end{bewertung}


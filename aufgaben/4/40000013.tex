Ersetzt man in einem XML-Dokument alle "offnenden Elemente durch {\tt <},
alle schliessenden Elemente durch {\tt >} und alle leeren Elemente
durch {\tt /}, und l"asst alles andere weg, bleibt ein Wort "uber
dem Alphabet $\Sigma =\{{\tt <},{\tt >},{\tt /}\}$ "ubrig.
Bilden die so konstruierten W"orter eine regul"are Sprache?

\ifthenelse{\boolean{loesungen}}{
\begin{loesung}
Nein, wie wir mit dem Pumping Lemma zeigen k"onnen.

Sei $L$ die Sprache der W"orter, die durch den genannten Prozess
entstehen k"onnen. In so einem Wort muss die Anzahl `{\tt <}' und die
Anzahl `{\tt >}' gleich gross sein. Wir zeigen, dass dies im Widerspruch
steht zu der Annahme, $L$ sei regul"ar.

Wir nehmen also an, $L$ sei regul"ar. Nach dem Pumping Lemma gibt
es also eine Zahl $N$, die pumping length, so dass sich W"orter mit
L"ange mindestens $N$ innerhalb der ersten $N$ Zeichen aufpumpen lassen.
Wir w"ahlen als Wort $w=\text{\tt <}^N\text{\tt >}^N$. Dieses Wort
kann sicher aus einem XML-File erzeugt werden. Nach dem Pumping
Lemma kann man $w=xyz$ schreiben, mit $|y|>0$ und $|xy|\le N$, so dass
alle W"orter $xy^kz\in L$ sind. Wegen $|xy|\le N$ besteht aber
$y$ aus lauter `{\tt <}'. Das Wort $xy^kz$ ent"alt also mehr
`{\tt <}' als `{\tt >}' und kann damit nicht in $L$ sein. Dieser
Widerspruch zeigt, dass die Annahme, $L$ sei regul"ar, falsch gewesen
sein muss. $L$ ist also nicht regul"ar.
\end{loesung}
{\bf Bewertung:}
Ansatz mit Pumping Lemma 1 Punkt ({\bf PL}),
Pumping Length 1 Punkt ({\bf N}),
zweckm"assiges Beispielwort 1 Punkt ({\bf W}),
Zerlegung des Wortes 1 Punkt ({\bf Z}),
Aufpumpen zerst"ort das Wort 1 Punkt ({\bf A}),
Schlussfolgerung 1 Punkt ({\bf F}).
}{ }


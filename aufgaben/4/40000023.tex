Student Heiri Muster m"ochte einen Parser f"ur Additionsprobleme
von $n$-Tupeln schreiben, wobei $n$ varieren kann. Solche Probleme
sind Zeichenketten der Form
$$
(3,47,1291,4711)+(4,1848,2010,-1)
$$
wobei die Anzahl der Zahlen in beiden Klammern identisch sein muss.
\begin{teilaufgaben}
\item Geben Sie eine kontextfreie Grammatik f"ur Ausdr"ucke
dieser Form an.
\item Beschreiben Sie einen Stack-Automaten, der solche Ausdr"ucke
akzeptiert.
\end{teilaufgaben}

\begin{loesung}
\begin{teilaufgaben}
\item Die folgende Grammatik beschreibt offenbar die Tuppel-Additionsprobleme:
\begin{align*}
S&\rightarrow
\text{'{\tt (}'}\;
\text{Ausdruck}\;
\text{'{\tt )}'}
\\
\text{Ausdruck}&\rightarrow
\text{Zahl}\;
\text{'{\tt ,}'}\;
\text{Ausdruck}\;
\text{'{\tt ,}'}\;
\text{Zahl}
\\
&\rightarrow
\text{Zahl}\;
\text{'{\tt )}'}\;
\text{'{\tt +}'}\;
\text{'{\tt (}'}\;
\text{Zahl}
\\
\text{Zahl}&\rightarrow \dots
\end{align*}
Damit ist gezeigt, dass die Sprache kontextfrei ist. Nat"urlich
ist der Parse-Tree zu dieser Grammatik kaum n"utzlich, die
Additionsprobleme zu berechnen, weil er die erste
Komponente des ersten Terms mit der letzten des zweiten
Terms in Beziehung setzt, w"ahrend die Addition die beiden
ersten Elemente addieren m"usste.
\item
W"are die Bedingung nicht gegeben, dass die Klammern
gleich viele Zahlen enthalten m"ussen, w"are die
Sprache sogar regul"ar, es w"are also gar kein Stack-Automat
erforderlich, um die W"orter der Sprache zu erkennen.

Daraus kann man schliessen, dass man den Stack genau daf"ur
verwenden muss, die Zahlen auf in jedem Ausdruck zu z"ahlen.
Man erreicht dies, indem man beim Lesen der ersten Klammer
mit jedem Komma und der schliessenden Klammer, ein Zeichen auf den Stack legt,
welches man beim Lesen der zweiten Klammer mit jedem Komma und
der schliessenden Klammer entfernt. Ist der Stack nach diesen
Operationen leer, wird das Wort akzeptiert.

Zur Vereinfachung des Zustandsdiagramms verwenden wir f"ur
Zahlen den folgenden Automaten:
\[
\entrymodifiers={++[o][F]}
\xymatrix @-1mm {
*+\txt{}\ar[r]
        &z_0\ar[r]^{\tt[1-9]} \ar[d]_{\tt -}
                &*++[o][F=]{z_1}\ar@(ur,dr)^{\tt [0-9]}
\\
*+\txt{}
        &z_2\ar[ur]_{\tt[1-9]}
                &*+\txt{}
}
\]

Wir geben den Stack-Automaten in Form eines Zustandsdiagramms
an:
\[
\entrymodifiers={++[o][F]}
\xymatrix @-1mm {
*++\txt{} \ar[r]
        &{z_{10}} \ar[r]^{\tt (}
                &Z\ar[r]^{\text{{\tt )},$\varepsilon\to x$}} \ar@(ur,ul)_{\text{{\tt ,},$\varepsilon\to x$}}
                        &z_{11}\ar[d]^{\tt +}
\\
*++\txt{}
        &*++[o][F=]{z_{13}}
                &Z\ar[l]^{\text{{\tt )},$x\to\varepsilon$}}
                        \ar@(dr,dl)^{\text{{\tt ,},$x\to\varepsilon$}}
                        &{z_{12}}\ar[l]_{\tt (}
}
\]
\end{teilaufgaben}
\end{loesung}

In der Vorlesung wurde ein Stackautomat konstruiert, welcher die
Sprache
\[
\{{\tt a}^i{\tt b}^j{\tt c}^k|\text{$i,j,k\ge 0$ und $i=j$ oder $i=k$}\}
\]
akzeptiert.
Formulieren sie eine Grammatik, die die selbe Sprache erzeugt.

\begin{loesung}
W"orter dieser Sprache bestehen entweder aus einem Wort mit gleich
vielen {\tt a} und {\tt b}, dem eine beliebige Zahl von {\tt c}
angeh"angt wurde, oder einer beliebigen Zahl von {\tt b}, um die
symmetrisch {\tt a} und {\tt c} geschachtelt wurden.
W"orter der ersten Art werden von den folgenden Regeln erzeugt:
\begin{align*}
S_1&\to P_1C\\
P_1&\to\varepsilon\\
   &\to {\tt a}P_1{\tt b}\\
C  &\to\varepsilon\\
   &\to C{\tt c}
\end{align*}
Die W"orter der zweiten Art werden dagegen von den Regeln
\begin{align*}
S_2&\to{\tt a}S_2{\tt c}\\
   &\to B\\
B  &\to B{\tt b}\\
   &\to\varepsilon
\end{align*}
erzeugt.
Gesucht ist aber die Vereinigungsmenge der beiden Sprachen, die
nat"urlich mit Hilfe einer Alternativ-Regel $S\to S_1|S_2$
konstruiert werden kann.
Dies ergibt die Grammatik f"ur die Sprache $L$:
\begin{align*}
S  &\to S_1\\
   &\to S_2\\
S_1&\to P_1C\\
P_1&\to\varepsilon\\
   &\to {\tt a}P_1{\tt b}\\
C  &\to\varepsilon\\
   &\to C{\tt c}\\
S_2&\to{\tt a}S_2{\tt b}\\
   &\to B\\
B  &\to B{\tt b}\\
   &\to\varepsilon
\qedhere
\end{align*}
\end{loesung}

Die Programmiersprachen C und Pascal erlauben, aus Anweisungen sogenannte
Bl"ocke zu bilden, aber sie verwenden verschiedene syntaktische Regeln
daf"ur.
In C wird ein Block von geschweiften Klammern \texttt{\{} und \texttt{\}}
eingefasst, und die Anweisungen m"ussen durch Semikolon beendet werden.
In Pascal beginnt ein Block mit \texttt{begin} und endet mit \texttt{end},
und zwischen den einzelnen Anweisungen steht ein Semikolon.
Das folgende Beispiel einer \texttt{for}-Schleife illustriert den
Unterschied.

\medskip
\begin{tabular}{lr}
\begin{minipage}{0.5\hsize}
\texttt{for}-Schleife in C:
\medskip
\verbatimainput{block.c}
\end{minipage}&%
\begin{minipage}{0.5\hsize}
\texttt{for}-Schleife in Pascal:
\medskip
\verbatimainput{block.pas}
\end{minipage}
\end{tabular}
\medskip

Man beachte, dass in Pascal am Ende der zweiten Anweisung kein Semikolon
n"otig ist.
Beide Sprachen erlauben, dass ein Block keine Anweisungen enth"alt.

Konstruieren Sie f"ur jede der Sprachen eine Grammatik f"ur Bl"ocke
(Verwenden Sie die Variable \texttt{block} daf"ur), die die genannten
Unterschiede ausdr"uckt.
Verwenden Sie in Ihrer Grammatik die Variable \texttt{anweisung} f"ur
Anweisungen.
Sie m"ussen keine Grammatik f"ur Anweisungen angeben.

\begin{hinweis}
Es wird {\em nicht} verlangt, eine Grammatik f"ur die
\texttt{for}-Kontrollstruktur anzugeben, sondern nur f"ur den Block.
\end{hinweis}

\begin{loesung}
Die beiden Grammatiken m"ussen die folgenden zwei Unterschiede zwischen
den Sprachen sichtbar machen: einerseits die Wahl der Zeichen,
die einen Block begrenzen, und andererseits die das Semikolon, welches
in C ein Terminator, in Pascal aber ein Separator ist.

Bl"ocke worden aus einer Anweisungsfolge konstruiert, in C zum Beispiel
mit Hilfe der Regel
\[
\texttt{block}
\rightarrow
\text{'{\tt \{}'}\;\texttt{anweisungsfolge}\; \text{'{\tt\}}'}
\]
In Pascal muss man die Blockbegrenzer aus Zeichen zusammensetzen
\begin{align*}
\texttt{block}
&\rightarrow
B\; \texttt{anweisungsfolge}\;E
\\
B&\rightarrow
\text{'}\texttt{b}\text{'}\;
\text{'}\texttt{e}\text{'}\;
\text{'}\texttt{g}\text{'}\;
\text{'}\texttt{i}\text{'}\;
\text{'}\texttt{n}\text{'}
\\
E&\rightarrow
\text{'}\texttt{e}\text{'}\;
\text{'}\texttt{n}\text{'}\;
\text{'}\texttt{d}\text{'}
\end{align*}
Die Konstruktion einer Anweisungsfolge in C verwendet das Semikolon als
Anweisungsendzeichen:
\begin{align*}
\texttt{anweisungsfolge}
&\rightarrow
\texttt{anweisungsfolge}
\;
\texttt{folgenelement}
\\
&\rightarrow\varepsilon
\\
\texttt{folgenelement}
&\rightarrow 
\texttt{anweisung}
\;\text{'{\tt ;}'}
\end{align*}
In Pascal muss das Semikolon als Trennzeichen verwendet werden:
\begin{align*}
\texttt{anweisungsfolge}
&\rightarrow
\texttt{anweisungsfolge} \; \text{'{\tt ;}'}\; \texttt{anweisung}
\\
&\rightarrow\varepsilon
\end{align*}
Zusammengefasst ist die Grammatik f"ur C-Bl"ocke:
\begin{align*}
\texttt{block}
&\rightarrow
\text{'{\tt \{}'}\;\texttt{anweisungsfolge}\; \text{'{\tt\}}'}
\\
\texttt{anweisungsfolge}
&\rightarrow
\texttt{anweisungsfolge}
\;
\texttt{folgenelement}
\\
&\rightarrow\varepsilon
\\
\texttt{folgenelement}
&\rightarrow 
\texttt{anweisung}
\;\text{'{\tt ;}'}
\end{align*}
Und f"ur Pascal-Bl"ocke ist sie
\begin{align*}
\texttt{block}
&\rightarrow
B\; \texttt{anweisungsfolge}\;E
\\
B&\rightarrow
\text{'}\texttt{b}\text{'}\;
\text{'}\texttt{e}\text{'}\;
\text{'}\texttt{g}\text{'}\;
\text{'}\texttt{i}\text{'}\;
\text{'}\texttt{n}\text{'}
\\
E&\rightarrow
\text{'}\texttt{e}\text{'}\;
\text{'}\texttt{n}\text{'}\;
\text{'}\texttt{d}\text{'}
\\
\texttt{anweisungsfolge}
&\rightarrow
\texttt{anweisungsfolge} \; \text{'{\tt ;}'}\; \texttt{anweisung}
\\
&\rightarrow\varepsilon
\end{align*}
\end{loesung}


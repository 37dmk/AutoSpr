F"ur zwei nat"urliche Zahlen $n$ und $m$ bezeichnet $n\sqcap m$ die
gr"ossere der beiden, $n\sqcup m$ die kleinere der beiden Zahlen.
Mit nat"urlichen Zahlen, den Symbolen $\sqcap$ und $\sqcup$ und Klammern
k"onnen Ausdr"ucke gebildet werden.
Wir wollen uns keine Gedanken dar"uber machen m"ussen, ob die Reihenfolge
der beiden Operatoren einen Einfluss auf das Resultat hat, und
verlangen daher
dass der Vorrang zwischen $\sqcap$ und $\sqcup$ mit Hilfe
von Klammern festgelegt wird. Beschreiben Sie eine
kontextfreie Grammatik, die genau die unzweideutigen Ausdr"ucke
erzeugt.

\begin{loesung}
Es muss sichergestellt werden, dass zwischen verschiedenen
Verkn"upfungsoperatoren immer Klammern verwendet werden. Man kann
dies erreichen, indem man f"ur Ausdr"ucke mit $\sqcap$ die Variable
$\text{max}$ schreibt, f"ur $\sqcup$ die Variable $\text{min}$, und
folgende Grammatikregeln postuliert:
\begin{align*}
\text{min}&\rightarrow \text{number}\\
&\rightarrow \text{min}\sqcup \text{min}\\
&\rightarrow\text{klammerausdruck}\\
\text{max}&\rightarrow \text{number}\\
&\rightarrow \text{max}\sqcap \text{max}\\
&\rightarrow\text{klammerausdruck}\\
\text{klammerausdruck}&\rightarrow (\;\text{min}\;)\\
&\rightarrow (\;\text{max}\;)\\
\text{number}&\rightarrow\text{digit}\\
&\rightarrow\text{number}\;\text{digit}\\
\text{digit}&\rightarrow 0
\;|\;1
\;|\;2
\;|\;3
\;|\;4
\;|\;5
\;|\;6
\;|\;7
\;|\;8
\;|\;9
\\
\end{align*}
Dazu braucht es noch eine Startvariable, von der aus man entweder einen
Minimum- oder einen Maximum-Ausdruck generieren kann, also
$$S\rightarrow \text{min}\;|\;\text{max}$$
Offenbar ist diese Grammatik kontextfrei.
(Dies ist nicht die einzige m"ogliche L"osung).
\end{loesung}

In der Vorlesung wurde gezeigt, dass die Sprache
$\{\texttt{a}^n\texttt{b}^n\texttt{c}^n\;|\;n\ge 0\}$
"uber dem Alphabet $\Sigma=\{\texttt{a},\texttt{b},\texttt{c}\}$
nicht kontextfrei ist.
Wenn man die Bedingungen etwas lockert, und nur noch verlangt, dass die
Anzahl der verschiedenen Zeichen "ubereinstimmt, die Reihenfolge aber
beliebig sein darf, erh"alt man die Sprache
\[
L=\{w\in\Sigma^*\;|\;|w|_\texttt{a}=|w|_\texttt{b}=|w|_\texttt{c}\}.
\]
Ist $L$ kontextfrei?

\begin{loesung}
Nein, auch $L$ ist nicht kontextfrei.
Man kann dies mit dem Pumping-Lemma f"ur kontextfreie Sprachen nachweisen,
wobei man sogar das gleiche Beispielwort verwenden kann wie im Falle der
Sprache $\{ \texttt{a}^n \texttt{b}^n \texttt{c}^n\;|\; n \ge 0\}$ .

Man nimmt dazu an, dass $L$ kontextfrei sei.
Gem"asst Pumping-Lemma gibt es daher die Pumping-Length $N$.
Wir konstruieren jetzt ein Wort
$w=\texttt{a}^N\texttt{b}^N\texttt{c}^N$, welches die Voraussetzungen
des Pumping-Lemma sicher erf"ullt.
Es muss also eine Unterteilung $w=uvxyz$ geben, so dass $|vxy|\le N$ ist.
Diese letzte Bedingung hat zur Folge, dass $v$ und $y$ h"ochstens in
zwei der drei Bl"ocke $\texttt{a}^N$, $\texttt{b}^N$ oder $\texttt{c}^N$ liegen k"onnen.
Insbesondere "andert sich beim Pumpen nur die Anzahl von zwei der drei
Buchstaben, nach auf- oder abpumpen erh"alt man also ein Wort, bei dem 
die Anzahl jedes der drei Buchstaben nicht mehr gleich ist, also kein
Wort mehr aus $L$, im Widerspruch zur Aussage des Pumping-Lemma.
Daher kann $L$ nicht kontextfrei sein.
\end{loesung}

\begin{diskussion}
Es gen"ugt nicht, darauf zu verweisen, dass $L$ die Sprache
$L'=\{ \texttt{a}^n \texttt{b}^n \texttt{c}^n\;|\; n\ge 0\}$ enthalte,
und dass daher auch $L'$ nicht kontextfrei sein.  Dann k"onnte man n"amlich auch
argumentieren, dass $L$ die regul"are Sprache $\emptyset$ enthalte, und daher
regul"ar sei!

Wenn man sich den Pumping-Lemma-Beweis genauer anschaut, kann man auch verstehen,
warum es nicht reicht. Der Beweis beruht ja darauf, dass aufgepumpte W"orter nicht
mehr in der Sprache sind. Wenn die Sprache aber mehr W"orter umfasst, dann ist
es auch schwieriger ``aus der Sprache herauszufallen'', es ist ja jetzt leichter,
die Bedingung der Sprache zu erf"ullen. Damit ist es m"oglich, dass ein Wort, welches
in $L'$ nicht aufgepumpt werden kann, in $L$ aufpumpbar wird.
\end{diskussion}

\begin{bewertung}
Pumping Lemma f"ur kontextfreie Sprachen ({\bf PL}) 1 Punkt,
Pumping Length ({\bf N}) 1 Punkt,
Beispielwort unter Verwendung von $N$ ({\bf W}) 1 Punkt,
Zerlegung des Wortes ({\bf Z}) 1 Punkt,
Konsequenzen beim Auf- bzw.~Abpumpen ({\bf A}) 1 Punkt,
Schlussfolgerung ({\bf S}) 1 Punkt.
\end{bewertung}


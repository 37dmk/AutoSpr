Sei $\Sigma=\{{\tt 0},{\tt 1}\}$ und
\[
L=\{w\in\Sigma^*\;|\;\text{$|w|$ ist ungerade und das mittlere Symbol in $w$ ist {\tt 0}}\}.
\]
Zeichnen Sie das Zustandsdiagramm eines Stackautomaten, welcher $L$
erkennt.

\begin{loesung}
Wir verwenden als Stackalphabet zus"atzlich die zwei Zeichen {\tt \$}
und {\tt x}. F"ur jedes gelesen Input-Zeichen legen wir ein {\tt x}
auf den Stack. Nach der zentralen {\tt 0} entfernen wir f"ur jedes
gelesen Zeichen wieder ein {\tt x}. Akzeptiert werden kann ein Wort,
wenn danach der Stack leer ist:
\[
\entrymodifiers={++[o][F]}
\xymatrix{
*+\txt{} \ar[r]
        &{} \ar[r]^{\varepsilon,\varepsilon\to{\tt \$}}
                &{}\ar@(u,ur)^{{\tt 0},\varepsilon\to{\tt x}}
                   \ar@(ur,r)^{{\tt 1},\varepsilon\to{\tt x}}
                   \ar[d]^{{\tt 0},\varepsilon\to\varepsilon}
\\
*+\txt{}
        &*++[o][F=]{}
                &{}\ar[l]^{\varepsilon,{\tt \$}\to\varepsilon}
                   \ar@(r,dr)^{{\tt 0},{\tt x}\to\varepsilon}
                   \ar@(dr,d)^{{\tt 1},{\tt x}\to\varepsilon}
}
\]
Die alternative L"osung verwendet wieder eine Grammatik, zum Beispiel
\begin{align*}
S&\rightarrow XSX\\
&\rightarrow {\tt 0}\\
X&\rightarrow {\tt 0}\,|\,{\tt 1}
\end{align*}
Daraus kann man wieder nach bekanntem Muster den ``Bl"umchenautomaten''
konstruieren:
\[
\entrymodifiers={++[o][F]}
\xymatrix{
*+\txt{}\ar[r]
	&{} \ar[r]^{\varepsilon,\varepsilon\to{\tt \$}}
		&{} \ar[rr]^{\varepsilon,\varepsilon\to S}
			&*+\txt{}
			&{R} \ar[rr]^{\varepsilon,{\tt \$}\to\varepsilon}
				\ar@(u,ul)_{x,x\to\varepsilon}
				%\ar@(ul,l)_{{\tt 1},{\tt 1}\to\varepsilon}
				\ar@(u,ur)^{\varepsilon,X\to x}
				%\ar@(ur,r)^{\varepsilon,X\to{\tt 1}}
				\ar@/_/[dl]_{\varepsilon,S\to X}
				\ar@(dl,dr)_{\varepsilon,S\to \texttt{0}}
				&*+\txt{}
				&*++[o][F=]{}
\\
*+\txt{}
	&*+\txt{}
		&*+\txt{}
			&{}\ar@/_/[rr]_{\varepsilon,\varepsilon\to S}
				&*+\txt{}
					&{} \ar@/_/[ul]_{\varepsilon,\varepsilon\to X}
}
\]
In den Regeln oberhalb des Zustandes $R$ steht das $x$ f"ur ein Terminalsymbol.
Jeder Pfeil steht also eigentlich f"ur zwei Regeln, n"amlich eine mit
$x={\tt 0}$ und eine mit $x={\tt 1}$.
\end{loesung}


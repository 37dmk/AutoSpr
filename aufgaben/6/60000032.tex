Die SlemBunk Malware f"ur Android geh"ort zu einem ausgefeilten
System, mit dem Internet Banking unterwandert werden kann.
Die App enth"alt keine ``verd"achtige'' Funktionalit"at als direkt
einsehbaren Code. 
Vielmehr ist der Code, der die endg"ultige Malware herunterl"adt,
unscheinbar verpackt, und wir von der App erst zur Laufzeit
ausgepackt und in den Speicher geladen, wo er dann auch ausgef"uhrt
werden kann.
Wegen dieses komplizierten Verfahrens sch"opft der App-Scanner von
Android keinen Verdacht.
Gibt es eine M"oglichkeit, den Scanner so zu erweitern, dass er 
von jeder App erkennen kann, ob sie je einen Internet-Download
ausf"uhren wird?

\begin{loesung}
Nein, den damit liesse sich das Halteproblem l"osen.
Ein Programm $P_1$ kann man immer in ein gr"osseres Programm $P_2$
einbettet, welches einen Internet-Download ausf"uhrt, sobald $P_1$
anh"alt.
Wendet man den Scanner auf $P_2$ an,
kann man herausfinden, ob $P_2$
je einen Internet-Download ausf"uhren wird, und damit auch, ob
$P_1$ je anhalten wird.
Damit ist das Halteproblem gel"ost.
Da das Halteproblem nicht l"osbar ist, kann es auch keinen solchen
Scanner geben.
\end{loesung}

\begin{bewertung}
Reduktion auf Halteproblem ({\bf R}) 3 Punkte,
Halteproblem ist nicht entscheidber ({\bf H}) 3 Punkte.
\end{bewertung}




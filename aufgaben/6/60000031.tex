Man findet im Internet Websites, die anbieten, Word-Dokumente in PDF
umzuwandeln.
Ein Jungunternehmer m"ochte einen "ahnlichen Dienst anbieten, um XML Files
mit Hilfe von XSLT Stylesheets in PDF Files umzuwandeln.
Die Kunden sollen einen Satz von XML-Files und XSLT Stylesheets zum
Beispiel als ZIP-File hochladen k"onnen, und als Antwort ein fertig
formattiertes PDF erhalten.
Im Apache-Projekt FOP findet er geeignete freie Software daf"ur.
Sie sind in diesem Projekt als Sicherheitsverantwortliche(r) daf"ur zust"andig,
dass alle XML-/XSLT-File-Kombinationen zur"uckgewiesen werden, die dazu f"uhren
k"onnten, dass der Server beliebig lange mit diesem Auftrag besch"aftigt ist.
Wie stehen Ihre Aussichten, dies zu realisieren?

\begin{loesung}
Die Sprache XSLT ist Turing-vollst"andig, man kann in ihr also alles
programmieren, was mit einer Turing-Maschine gemacht werden kann.
Die gestellte Aufgabe besteht also darin, herauszufinden, ob ein
gegebenes Programm (XML-Files und XSLT Stylesheets) je anhalten wird.
Dies ist das Halteproblem, welches nicht entscheidbar ist.
Es gibt also keine M"oglichkeit, einem Auftrag anzusehen, ob er dazu
f"uhren wird, dass der PDF-Renderer nicht terminieren wird.
\end{loesung}

\begin{diskussion}
Die Aussage, dass XSLT Turing-vollst"andig ist, ist ein wesentlicher Schritt.
Ohne diesen Schritt w"are es denkbar, dass, wegen des begrenzten Sprachumfanges
der Sprache XSLT, alle Programme terminieren, so wie dies bei der Sprache LOOP
festgestellt wurde.
Erst wenn eine Sprache Turing-vollst"andig ist, kann man mit dem Halte-Theorem
argumentieren, welches ja ausschliesslich f"ur Turing-Maschinen gilt.

Man k"onnte versucht sein, den Satz von Rice anwenden zu wollen.
Dieser ist jedoch nur auf Turing-erkennbare Sprachen anwendbar.
Es muss eine Eigenschaft formuliert werden, die eine solche Sprache hat. 
Der Satz von Rice w"urde uns schliessen lassen, dass es nicht m"oglich ist
zu entscheiden, ob eine Sprache die Eigenschaft hat.
Zu jeder XML/XSLT-Kombination m"usste eine Sprache geh"oren, doch
in der Problemstellung kommt eine solche Sprache gar nicht vor.
Die XML/XSLT-Programme haben ja keinen Input, der akzeptiert oder
verworfen werden k"onnte, und so eine Sprache definieren k"onnte.
\end{diskussion}

\begin{bewertung}
XSLT ist Turing-vollst"andig ({\bf V}) 2 Punkte,
Anwendung des Halte-Theorems ({\bf H}) 4 Punkte.
\end{bewertung}

Seien $L_1$ und $L_2$ zwei Turing-erkennbare Sprachen. Zeigen Sie, dass
die folgenden Sprachen Turing-erkennbar sind (mit Hilfe von TM $M_1$
bzw.~$M_2$):
\begin{teilaufgaben}
\item $L_1\cup L_2$
\item $L_1L_2$
\item $L_1^*$
\item $L_1\cap L_2$
\end{teilaufgaben}

\begin{loesung}
Wir m"ussen eine Turing Maschine angeben, die die entsprechende
Sprache erkennen kann.
Dazu k"onnen im Prinzip die gleichen Ideen verwendet werden, wie sie
zur L"osung von Aufgabe~\ref{60000005} verwendet wurden.
Allerdings ist jetzt nicht mehr sichergestellt, dass der Aufruf
der Funktionen \texttt{l1} oder \texttt{l2} je zur"uckkehrt.
Statt die Funktionen nacheinander aufzurufen, kann man aber in allen
Algorithmen alle Aufrufe parallel ausf"uhren, und die Threads abbrechen,
sobald gen"ugend Resultate f"ur eine positive Antwort gefunden worden
sind. Im schlimmsten Fall wird kein Thread je fertig, dann ist das
Inputwort aber auch nicht akzeptabel.

Im Detail kann man dies f"ur Turing-Maschinen auch wie folgt
beschreiben.
\begin{teilaufgaben}
\item
Wir k"onnen nicht einfach beide Turing Maschinen nach einander
auf dem Input laufen lassen, da die erste ja nicht terminieren k"onnte und
somit verunm"oglichen k"onnte, dass die zweite das Wort akzeptieren kann.
Wir konstruieren daher eine Turing-Maschine mit zwei B"andern, welche
beide mit dem Input-Wort initialisiert werden.
In jedem Schritt der Maschine wird auf Band 1 ein Schritt von $M_1$
ausgef"uhrt, und auf Band 2 ein Schritt von $M_2$.
Die Maschine akzeptiert, wenn eine der Teilmaschinen
akzeptiert.

In einer modernen Turingmaschine, auch bekannt unter der Bezeichnung
Computer k"onnte man die Parallelit"at auch durch zwei parallele
Prozesse realisieren, dann w"urde das Betriebssystem die Umschaltung
zwischen den beiden Prozessen "ubernehmen.

In Java k"onnte man das realisieren, indem man zwei Threads startet,
in jedem Thread wird eine der Turingmaschinen ausgef"uhrt.
Sobald einer der Threads terminiert, kann der andere ``abgeschossen'' werden,
das Resultat steht ja dann fest.
\item
Um zu verhindern, dass die Turing Maschine auf einem Wort der
Sprachen nicht h"alt, lassen wir $M_1$ und $M_2$ jeweils nur f"ur eine
beschr"ankte Anzahl Schritte $l$ laufen.
F"ur jede Berechnungsl"ange $l$ und jede Aufteilung des Wortes in zwei
Teilw"orter $w=w_1w_2$ lassen wir auf zwei Tapes parallel die
Maschinen $M_1$ und $M_2$ auf dem Input $w_1$ bzw.~$w_2$ w"ahrend
maximal $l$ Schritten laufen.
Akzeptieren beide, akzeptieren wir das Wort.
Falls nicht, probieren wir die n"achste der $|w|+1$ Aufteilungen,
oder, falls alle Aufteilungen bereits probiert wurden, die n"achste
Berechnungsl"ange $l+1$.
\item
F"ur jede Berechnungsl"ange $l$ und
jede Aufteilung eines Wortes $w$ in Teilst"ucke $x_1,\dots,x_n$
lassen
wir auf jedem Teilst"uck die Turingmaschine w"ahrend $l$ Schritten
laufen, die
$L_1$ erkennt. Akzeptiert sie auf allen Teilst"ucken, akzeptieren wir
das Wort.
Falls nicht, versuchen wir die n"achste Aufteilung, oder falls
bereits alle Aufteilungen durchprobiert worden sind, die n"achste
Berechnungsl"ange $l$.
\item Wir verwenden eine Turing-Maschine analog zu der in Teilaufgaben a)
konstruierten, akzeptieren aber erst, wenn beide Teilmaschinen
$M_1$ und $M_2$ akzeptiert haben.
\qedhere
\end{teilaufgaben}
\end{loesung}

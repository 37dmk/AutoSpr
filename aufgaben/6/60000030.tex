Primzahlzwillinge sind aufeinanderfolgende ungerade Zahlen, die beide
Primzahlen sind, zum Beispiel $(3,5)$, $(5,7)$, $(11,13)$, $(17,19)$,\dots.
Die Primzahlzwillings-Vermutung besagt, dass es unendlich viele
Primzahlzwillinge gibt, sie konnte bisher weder bewiesen noch wiederlegt
werden.

\begin{teilaufgaben}
\item Formulieren Sie das Entscheidungsproblem ``Geh"ort die Zahl $n$ zu
einem Primzahlzwillingspaar?'' als Sprachproblem.
\item Ist die in a) definierte Sprache entscheidbar?
\item Jemand hat eine Funktion \texttt{boolean zwillingstest(BigInteger n)}
geschrieben, die herausfinden kann, ob die Zahl $n$ zu einem
Primzahlzwillingspaar geh"ort.
Gibt es eine M"oglichkeit, maschinell zu pr"ufen, ob eine solche Funktion
in allen F"allen korrekt arbeitet?
\end{teilaufgaben}

\begin{loesung}
\begin{teilaufgaben}
\item Sei $\Sigma=\{\texttt{9},\texttt{1}\}$. Die Sprache
\[
L=\left\{w\in\Sigma^*\,\left|
\begin{minipage}{4truein}
$w$ ist die Bin"arcodierung einer Zahl $n$, die eine Primzahl ist
so, dass $n-2$ oder $n+2$ ebenfalls eine Primzahl ist.
\end{minipage}
\right.
\right\}
\]
Enth"alt genau die Bin"ardarstellungen von Zahlen, die zu einem
Primzahlzwillingspaar geh"oren.
\item
Ein Entscheidungsalgorithmus f"ur $L$ geht wie folgt vor:
\begin{enumerate}
\item Pr"ufe, ob $n$ ungerade ist. Falls $n$ gerade ist: $q_\text{reject}$
\item Pr"ufe, ob $n$ eine Primzahl ist (mit irgend einem der bekannten
Testalgorithmen f"ur Primzahlen). Falls $n$ keine Primzahl ist:
$q_{\text{reject}}$
\item Pr"ufe ob $n-2$ oder $n+2$ Primzahlen sind. Falls beide keine
Primzahlen sind: $q_{\text{reject}}$, andernfalls $q_{\text{accept}}$
\end{enumerate}
\item
Eine Funktion wie \texttt{zwillingstest} ist eine Turingmaschine, die
eine Turing erkennbare Sprache $L'=L(\texttt{zwillingstest})$ definiert.
Die Aufgabe fragt nach einer Entscheidungsprozedur, welche erlauben
w"urde festzustellen, ob die von \texttt{zwillingstest} erkannte
Sprache $L'$ die Eigenschaft hat, gleich $L$ zu sein. Diese
Eigenschaft ist nicht trivial: die Sprache $L$ hat diese Eigenschaft,
die Sprache $\emptyset$ hat sie nicht. Nach dem Satz von Rice ist
die Eigenschaft nicht entscheidbar.
\end{teilaufgaben}
\end{loesung}

\begin{bewertung}
"Ubersetzung ein Sprachproblem ({\bf "U}) 1 Punkt,
Drei wesentliche Schritte f"ur den Entscheidungsalgorithmus ({\bf A})
je 1 Punkt, maximal 3 Punkte,
Formulierung einer nichttrivialen Eigenschaft ({\bf E}) 1 Punkt,
Anwendung des Satzes von Rice ({\bf R}) 1 Punkt.
\end{bewertung}


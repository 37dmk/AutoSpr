Der Herausgeber eines Konferenz-Bandes muss die einzelnen Beitr"age der 
Konferenzteilnehmer zu einem PDF File zusammenstellen, welches dem Drucker
"ubergeben werden kann.
Die Beitr"age werden also \LaTeX-Sourcecode angeliefert.
Der Herausgeber sorgt sich, dass die \LaTeX-Files nicht nur den Text
der Artikel enthalten, sondern auch Code, der auf dem PC des Herausgebers
Schaden anrichten k"onnte.
Daher sucht er mit Google nach einem Scanner, der solchen Schadcode  in
\LaTeX-Files entdecken k"onnte.
Er findet zwar einige Anfragen in exotischen Foren, die Forenthreads
enthalten nichts ausser ``das w"urde mich auch interessieren'' und,
meist nach Jahren, ``hast Du daf"ur schon eine L"osung gefunden?''
Warum ist das nicht "uberraschend?

\begin{loesung}
Die Sprache \LaTeX{} ist Turing-vollst"andig.
Die Aufgabe, Schadcode zu erkennen l"auft also darauf hinaus, das Halteproblem
zu l"osen.
Da das Halteproblem nicht entscheidbar ist, kann es einen solchen Scanner
nicht geben.
\end{loesung}


Der Herausgeber eines Konferenz-Bandes muss die einzelnen Beitr"age der 
Konferenzteilnehmer zu einem PDF File zusammenstellen, welches dem Drucker
"ubergeben werden kann.
Die Beitr"age werden also \LaTeX-Sourcecode angeliefert.
Der Herausgeber sorgt sich, dass die \LaTeX-Files nicht nur den Text
der Artikel enthalten, sondern auch Code, der auf dem PC des Herausgebers
Schaden anrichten k"onnte.
Daher sucht er auf Google nach einer Scanner-Software, der solchen Schadcode  in
\LaTeX-Files entdecken k"onnte.
Er findet zwar einige Anfragen in exotischen Foren, die Forenthreads
enthalten wie so oft nichts ausser ``das w"urde mich auch interessieren'' und,
meist nach Jahren, ``hast Du daf"ur schon eine L"osung gefunden?''
Warum ist das nicht "uberraschend?

\begin{loesung}
Die Sprache \LaTeX{} ist Turing-vollst"andig.
Die Aufgabe, Schadcode zu erkennen l"auft also darauf hinaus, das Halteproblem
zu l"osen.
Da das Halteproblem nicht entscheidbar ist, kann es einen solchen Scanner
nicht geben.
\end{loesung}

\begin{bewertung}
Turing-Vollst"andigkeit von \LaTeX~{\bf L} 2 Punkte (nur ein Punkt, wenn
als Annahme formuliert),
(informelle) Reduktion auf das Halte- oder Akzeptanz-Problem ({\bf R}) 2 Punkte,
Halteproblem nicht entscheidbar ({\bf H}) 1 Punkt,
Schlussfolgerung f"ur die Aufgabe ({\bf S}) 1 Punkt.
\end{bewertung}


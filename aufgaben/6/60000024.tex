Sei $k$ eine nat"urliche Zahl.
Betrachten Sie das Problem zu entscheiden, ob eine Turing-Maschine
Laufzeit $O(n^k)$ hat. Ist es entscheidbar?

\begin{hinweis}
Satz von Rice.
\end{hinweis}

\ifthenelse{\boolean{loesungen}}{
\begin{loesung}
Die Eigenschaft einer Funktion, $O(n^k)$ Schritte f"ur die Berechnung zu
ben"otigen, ist eine nichttriviale Eigenschaft im Sinne des Satzes von Rice.
Um dies einzusehen, m"ussen zwei Funktionen definiert werden, eine,
die die Eigenschaft hat, und eine, die die Eigenschaft nicht hat.

Die Turing-Maschine, die einen Blank auf das Band schreibt und dann anh"alt,
hat Laufzeit $1$, also hat die Eigenschaft nicht. Sie braucht nur $O(1)$
f"ur die Berechnung.

Die Turing-Maschine, die $n^k$ berechnet und entsprechend viele
Einsen auf das Band schreibt, braucht mindestens Laufzeit $n^k$, nur
schon um das Ergebnis auf das Band zu schreiben. Die Funktion hat die
Eigenschaft also.

Nach dem Satz von Rice ist diese Eigenschaft also nicht entscheidbar.
\end{loesung}

}{}

